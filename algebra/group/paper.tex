\documentclass{amsart}
\input{decls}
\title{Groups}
\author{Frank Tsai}
\date{\today}
%\thanks{}
\begin{document}
\maketitle
\tableofcontents

\section{Definition of Groups}
\label{sec:definition-of-groups}

\begin{defn}
  A \emph{group} consists of the following data:
  \begin{enumerate}
  \item a set $G$;
  \item an \emph{identity element} $e \in G$;
  \item a \emph{group multiplication} $\cdot : G \times G \to G$;
  \item a \emph{group inverse} $(\blank)\inv : G \to G$
  \end{enumerate}
  such that
  \begin{enumerate}
  \item group multiplication is associative;
  \item $a\inv \cdot a = e = a \cdot a\inv$ for all $a \in G$.
  \end{enumerate}
\end{defn}
The group axioms can be stated in terms of commutative diagrams.

\begin{mathpar}
  % https://q.uiver.app/#q=WzAsNCxbMCwwLCJHIFxcdGltZXMgRyBcXHRpbWVzIEciXSxbMiwwLCJHIFxcdGltZXMgRyJdLFswLDIsIkcgXFx0aW1lcyBHIl0sWzIsMiwiRyJdLFswLDIsIlxcaWRfe0d9IFxcdGltZXMgXFxjZG90IiwyXSxbMSwzLCJcXGNkb3QiXSxbMCwxLCJcXGNkb3QgXFx0aW1lcyBcXGlkX3tHfSJdLFsyLDMsIlxcY2RvdCIsMl1d
\begin{tikzcd}
	{G \times G \times G} && {G \times G} \\
	\\
	{G \times G} && G
	\arrow["{\id_{G} \times \cdot}"', from=1-1, to=3-1]
	\arrow["\cdot", from=1-3, to=3-3]
	\arrow["{\cdot \times \id_{G}}", from=1-1, to=1-3]
	\arrow["\cdot"', from=3-1, to=3-3]
\end{tikzcd}\and
  % https://q.uiver.app/#q=WzAsOCxbMiwwLCJHIFxcdGltZXMgRyJdLFs0LDAsIkcgXFx0aW1lcyBHIl0sWzYsMCwiRyBcXHRpbWVzIEciXSxbNCwyLCJHIl0sWzAsMCwiRyJdLFs4LDAsIkciXSxbMCwyLCIxIl0sWzgsMiwiMSJdLFswLDEsIihcXGJsYW5rKVxcaW52IFxcdGltZXMgXFxpZF97R30iXSxbMiwxLCJcXGlkX3tHfSBcXHRpbWVzIChcXGJsYW5rKVxcaW52IiwyXSxbMSwzLCJcXGNkb3QiLDFdLFs0LDAsIlxcRGVsdGEiXSxbNSwyLCJcXERlbHRhIiwyXSxbNCw2XSxbNSw3XSxbNiwzLCJlIiwyXSxbNywzLCJlIl1d
\begin{tikzcd}
	G && {G \times G} && {G \times G} && {G \times G} && G \\
	\\
	1 &&&& G &&&& 1
	\arrow["{(\blank)\inv \times \id_{G}}", from=1-3, to=1-5]
	\arrow["{\id_{G} \times (\blank)\inv}"', from=1-7, to=1-5]
	\arrow["\cdot"{description}, from=1-5, to=3-5]
	\arrow["\Delta", from=1-1, to=1-3]
	\arrow["\Delta"', from=1-9, to=1-7]
	\arrow[from=1-1, to=3-1]
	\arrow[from=1-9, to=3-9]
	\arrow["e"', from=3-1, to=3-5]
	\arrow["e", from=3-9, to=3-5]
\end{tikzcd}
\end{mathpar}

\begin{eg}[Symmetry Groups]
\end{eg}

\begin{eg}[Dihedral Groups]
\end{eg}

\section{Order}
\label{sec:order}

\begin{defn}
  An element $g$ of a group $G$ has \emph{finite order} if $g^{n} = e$ for some positive integer $n$.
  In this case, the \emph{order}, denoted $|g|$, of $g$ is the smallest such $n$.
  We write $|g| = \oo$ if $g$ does not have finite order.
  The order $|G|$ of a group $G$ is the cardinality of its underlying set.
\end{defn}

\section{Group Homomorphisms}
\label{sec:group-homomorphisms}

\begin{defn}
  A \emph{group homomorphism} from $(G,m_{G})$ to $(H,m_{H})$ is a function $\varphi : G \to H$ that preserves group multiplication.
\end{defn}
This can be express in term of a commutative diagram in $\mathsf{Set}$:
  \begin{equation}
    \label{eq:group-homomorphism}
    % https://q.uiver.app/#q=WzAsNCxbMCwwLCJHIFxcdGltZXMgRyJdLFsyLDAsIkggXFx0aW1lcyBIIl0sWzAsMiwiRyJdLFsyLDIsIkgiXSxbMCwxLCJcXHZhcnBoaSBcXHRpbWVzIFxcdmFycGhpIl0sWzIsMywiXFx2YXJwaGkiLDJdLFswLDIsIm1fe0d9IiwyXSxbMSwzLCJtX3tIfSJdXQ==
\begin{tikzcd}
	{G \times G} && {H \times H} \\
	\\
	G && H
	\arrow["{\varphi \times \varphi}", from=1-1, to=1-3]
	\arrow["\varphi"', from=3-1, to=3-3]
	\arrow["{m_{G}}"', from=1-1, to=3-1]
	\arrow["{m_{H}}", from=1-3, to=3-3]
\end{tikzcd}
  \end{equation}

\begin{lem}
  Let $\varphi : G \to H$ be a group homomorphism, and let $g \in G$ be an element of finite order.
  Then $|\varphi(g)|$ divides $|g|$.
\end{lem}
\begin{proof}
  Suppose that $|g| = n$, then $g^{n} = e$.
  Thus,
  \[
    (\varphi(g))^{n} = \varphi(g^{n}) = \varphi(e) = e
  \]
  Thus, $n$ is a multiple of $|\varphi(g)|$.
\end{proof}

\begin{lem}
  Groups and their homomorphisms assemble into a category $\mathsf{Grp}$.
\end{lem}

\begin{lem}
  Trivial groups are both initial and terminal in $\mathsf{Grp}$.
\end{lem}
\begin{proof}
  It's clear that trivial groups are terminal in $\mathsf{Grp}$ since for any set $G$, the unique function from $G$ in to a singleton set is vacuously a group homomorphism.
  Dually, since group homomorphisms preserve group identity, there's a unique group homomorphism from a singleton set to $G$.
\end{proof}

Thus, trivial groups are \emph{zero objects} of $\mathsf{Grp}$.
This implies that $\mathsf{Grp}(G,H)$ is nonempty for any group $G$ and $H$ since one always has
% https://q.uiver.app/#q=WzAsMyxbMCwwLCJHIl0sWzIsMSwiXFx7KlxcfSJdLFs0LDAsIkgiXSxbMCwxLCIhX3tHfSIsMl0sWzEsMiwiIV97R30iLDJdLFswLDJdXQ==
\[\begin{tikzcd}
	G &&&& H \\
	&& {\{*\}}
	\arrow["{!_{G}}"', from=1-1, to=2-3]
	\arrow["{!_{G}}"', from=2-3, to=1-5]
	\arrow[from=1-1, to=1-5]
\end{tikzcd}\]
Concretely, this homomorphism maps every element of $G$ to the identity of $H$.

\begin{defn}
  An isomorphism of groups is a group homomorphism $\varphi : G \to H$ whose inverse is also a group homomorphism.
  This means that the underlying function of any group homomorphism is automatically a bijection.
\end{defn}
In fact the converse also holds.

\begin{lem}
  $\varphi : G \to H$ is a group isomorphism if and only if $\varphi$ is a bijection.
\end{lem}
\begin{proof}
  Let $\varphi : G \to H$ be a bijective group homomorphism, we need to show that $\varphi\inv : H \to G$ is a group homomorphism.
  That is, we need to show that the diagram
  % https://q.uiver.app/#q=WzAsNCxbMCwwLCJHIFxcdGltZXMgRyJdLFsyLDAsIkggXFx0aW1lcyBIIl0sWzAsMiwiRyJdLFsyLDIsIkgiXSxbMCwyLCJtX3tHfSIsMl0sWzEsMywibV97SH0iXSxbMSwwLCJcXHZhcnBoaVxcaW52IFxcdGltZXMgXFx2YXJwaGlcXGludiIsMl0sWzMsMiwiXFx2YXJwaGlcXGludiJdXQ==
\[\begin{tikzcd}
	{G \times G} && {H \times H} \\
	\\
	G && H
	\arrow["{m_{G}}"', from=1-1, to=3-1]
	\arrow["{m_{H}}", from=1-3, to=3-3]
	\arrow["{\varphi\inv \times \varphi\inv}"', from=1-3, to=1-1]
	\arrow["\varphi\inv", from=3-3, to=3-1]
\end{tikzcd}\]
  commutes.
  Since $\varphi$ is a bijection, it is monic.
  Thus, it suffices to show that $\varphi m_{G} (\varphi\inv \times \varphi\inv) = \varphi \varphi\inv m_{H} = m_{H}$.
  This is immediate.
  \[
    \varphi m_{G}(\varphi\inv \times \varphi\inv) = m_{H}(\varphi \times \varphi) (\varphi\inv \times \varphi\inv) = m_{H}
  \]
\end{proof}

\section{Products}
\label{sec:products}

Since the forgetful functor $U : \mathsf{Grp} \to \mathsf{Set}$ is representable by the group $\dZ$, the underlying sets of products in $\mathsf{Grp}$ are themselves products in $\mathsf{Set}$.
Thus, products in $\mathsf{Grp}$ are formed by giving group structures to products in $\mathsf{Set}$.

\begin{defn}
  Let $G$ and $H$ be groups, the \emph{direct product} $G \times H$ is the group formed by equipping the product of the underlying sets of $G$ and $H$ with the group structure
\begin{align}
  \cdot &: ((g,h),(g',h')) \mapsto (gg',hh')\\
  (\blank)\inv &: (g,h) \mapsto (g\inv,h\inv)
\end{align}
That is, the group operations are defined componentwise.
\end{defn}

\section{Coproducts}
\label{sec:coproducts}

Coproducts in $\mathsf{Grp}$ is more delicate.
In general, there is no reasonable group structure for the coproduct of two sets $G + H$.
Later we will see how to construct the coproduct of two groups using free construction.
The coproduct of two groups $G * H$ is called the \emph{free product} of $G$ and $H$.

\section{Abelian Groups}
\label{sec:abelian-groups}

The category $\mathsf{Ab}$ of abelian groups also has zero objects, but products coincide with coproducts.
The (co)product $G \times H$ of two abelian group is called the \emph{direct sum}, denoted $G \oplus H$, of $G$ and $H$.

\section{Cyclic Groups}
\label{sec:cyclic-groups}

\begin{defn}
  A group $G$ is \emph{cyclic} if it is isomorphic to $\dZ$ or to $\dZ/n\dZ$ for some $n$.
\end{defn}

\section{Free Groups}
\label{sec:free-groups}

Given a set $A$, how do we construct a group that contains $A$ in the most efficient way?
A construction is a set-theoretic function $j : A \to G$ mapping each element of $A$ to a corresponding group element of $G$.
Then, given two constructions $j_{1} : A \to G_{1}$ and $j_{2} : A \to G_{2}$, we can compare them by asking for a group homomorphism such that the diagram
% https://q.uiver.app/#q=WzAsMyxbMCwyLCJBIl0sWzAsMCwiR197MX0iXSxbMiwwLCJHX3syfSJdLFswLDEsImpfezF9Il0sWzAsMiwial97Mn0iLDJdLFsxLDIsIlxcdmFycGhpIl1d
\[\begin{tikzcd}
	{G_{1}} && {G_{2}} \\
	\\
	A
	\arrow["{j_{1}}", from=3-1, to=1-1]
	\arrow["{j_{2}}"', from=3-1, to=1-3]
	\arrow["\varphi", from=1-1, to=1-3]
\end{tikzcd}\]
commutes.
In this case, the construction $j_{1}$ is more efficient because $j_{2}$ can be recovered from $j_{1}$ via $\varphi$.

The most efficient construction is one such that every construction can be uniquely recovered:
\begin{equation}
  \label{eq:free-group}
  % https://q.uiver.app/#q=WzAsMyxbMCwwLCJGKEEpIl0sWzAsMiwiQSJdLFsyLDAsIkciXSxbMSwwLCJcXGV0YSJdLFsxLDIsImYiLDJdLFswLDIsIlxcdmFycGhpIiwwLHsic3R5bGUiOnsiYm9keSI6eyJuYW1lIjoiZGFzaGVkIn19fV1d
\begin{tikzcd}
	{F(A)} && G \\
	\\
	A
	\arrow["\eta", from=3-1, to=1-1]
	\arrow["f"', from=3-1, to=1-3]
	\arrow["\varphi", dashed, from=1-1, to=1-3]
\end{tikzcd}
\end{equation}

That is,
\[
  \mathsf{Grp}(F(A),G) \iso \mathsf{Set}(A,U(G))
\]
Concretely, the underlying set of $F(A)$ is the quotient of all formal words whose letters are $a$ and $a\inv$ for each $a \in A$.
Words are quotient by the congruence that $a\inv a = \epsilon = a a\inv$ for all $a \in A$.

Let $[w]$ and $[w']$ be elements of $F(A)$, group multiplication is defined as $[w \cdot w']$, i.e., the concatenation of $w$ and $w'$ modulo the congruence.
The inverse replaces each letter in a word with its inverse.

The construction $\eta$ maps each element $a \in A$ to the singleton word $[a] \in F(A)$.
\eqref{eq:free-group} forces us to define $\varphi([a]) = f(a)$ for any $a \in A$, while \eqref{eq:group-homomorphism} forces us to define $\varphi([w]) = f(w_{1}) \cdot \cdots \cdot f(w_{n})$.

\section{Free Abelian Groups}
\label{sec:free-abelian-groups}

Similarly, we can ask how to construct an abelian group that contains a set $A$ in the most efficient way.
For finite sets $A = \{1,\ldots,n\}$, we take $F(A) = \dZ^{\oplus n}$, i.e., the product of $\dZ$ with itself $n$ times.
\begin{equation}
  \label{eq:free-abelian-group}
  % https://q.uiver.app/#q=WzAsMyxbMCwyLCJBIl0sWzAsMCwiXFxkWl57XFxvcGx1cyBufSJdLFsyLDAsIkciXSxbMCwxLCJcXGV0YSJdLFswLDIsImYiLDJdLFsxLDIsIlxcdmFycGhpIiwwLHsic3R5bGUiOnsiYm9keSI6eyJuYW1lIjoiZGFzaGVkIn19fV1d
\begin{tikzcd}
	{\dZ^{\oplus n}} && G \\
	\\
	A
	\arrow["\eta", from=3-1, to=1-1]
	\arrow["f"', from=3-1, to=1-3]
	\arrow["\varphi", dashed, from=1-1, to=1-3]
\end{tikzcd}
\end{equation}
The set is endowed with the evident group structure, defined componentwise.
$\eta(i) = (0,\ldots,\underbrace{1}_{i\text{-th}},\ldots,0)$.
Thus, \eqref{eq:free-abelian-group} demands that $\varphi(0,\ldots,\underbrace{1}_{i\text{-th}},\ldots,0) = f(i)$, while \eqref{eq:group-homomorphism} demands $\varphi(a,b,\ldots,z) = f(a) + f(b) + \cdots + f(z)$.

One may recognize that $\eta(i)$ is the $i$-th standard basis vector of an $n$-dimensional vectorspace.

In general, let $A$ be any set.
The set $\dZ^{A}$ admits a natural abelian group structure.
Define
\[
  \dZ^{\oplus A} = \{\alpha : A \to \dZ \mid \alpha(a) \neq 0 \,\text{for finitely many}\, a \in A\}
\]
Restricting the group operation of $\dZ^{A}$ yields an abelian group structure on $\dZ^{\oplus A}$.
% https://q.uiver.app/#q=WzAsMyxbMCwyLCJBIl0sWzAsMCwiXFxkWl57XFxvcGx1cyBBfSJdLFsyLDAsIkciXSxbMCwxLCJcXGV0YSJdLFswLDIsImYiLDJdLFsxLDIsIlxcdmFycGhpIiwwLHsic3R5bGUiOnsiYm9keSI6eyJuYW1lIjoiZGFzaGVkIn19fV1d
\[\begin{tikzcd}
	{\dZ^{\oplus A}} && G \\
	\\
	A
	\arrow["\eta", from=3-1, to=1-1]
	\arrow["f"', from=3-1, to=1-3]
	\arrow["\varphi", dashed, from=1-1, to=1-3]
\end{tikzcd}\]
Analogously, $\eta(a)$ is the function $\delta_{a} : A \to \dZ$ defined by
\[
  \delta_{a}(x) =
  \begin{dcases*}
    1 & if $x = a$\\
    0 & otherwise
  \end{dcases*}
\]
Thus, $\varphi(\delta_{a}) = f(a)$.
Any element in $\dZ^{\oplus A}$ can be written as a finite formal sum
\[
  \delta = \sum_{a \in A}m_{a}\delta_{a}, \qquad m_{a} \neq 0~\text{for finitely many}~a \in A
\]
Thus, $\varphi$ can be uniquely extended to every $\delta \in \dZ^{\oplus A}$.
\[
  \varphi(\delta) = \sum_{a \in A}m_{a}f(a), \qquad m_{a} \neq 0~\text{for finitely many}~a \in A
\]

\section{Subgroups}
\label{sec:subgroups}

\begin{defn}
  A group $(H, \bullet)$ is a \emph{subgroup} of $(G, \cdot)$ if the inclusion function $i : H \into G$ is a group homomorphism.
\end{defn}

Note that the condition demands that $i(a \bullet b) = i(a) \cdot i(b) = a \cdot b$ for all $a, b \in H$.

\begin{lem}
  Let $H$ be a nonempty subset of a group $G$.
  $H$ is a subgroup of $G$ if and only if $ab\inv \in H$ for all $a, b \in H$.
\end{lem}
\begin{proof}
  If $H$ is a subgroup of $G$, then the inclusion function is a group homomorphism.
  Thus, for any $a, b \in H$,
  \[
    i(a \bullet b\inv) = i(a) \cdot i(b)\inv = a \cdot b\inv \in H
  \]
  Conversely, if for all $a, b \in H$, $ab\inv \in H$.
  We must show that the inclusion function is a group homomorphism.
  \[
    i(a \bullet b\inv) = i(a \cdot b\inv) = a \cdot b\inv = i(a) \cdot i(b)\inv
  \]
\end{proof}

Every group homomorphism $\varphi : G \to G'$ determines two subgroups
\begin{enumerate}
\item the \emph{kernel} of $\varphi$, $\ker{\varphi} \subseteq G$;
\item the \emph{image} of $\varphi$, $\img{\varphi} \subseteq G'$
\end{enumerate}

\begin{defn}
  Let $\varphi : G \to G'$ be a group homomorphism.
  The \emph{kernel} of $\varphi$ is the equalizer of $\varphi$ with the trivial homomorphism $0$.
  % https://q.uiver.app/#q=WzAsNCxbMiwwLCJHIl0sWzQsMCwiRyciXSxbMCwwLCJcXGtlcntcXHZhcnBoaX0iXSxbMCwyLCJLIl0sWzAsMSwiMCIsMCx7Im9mZnNldCI6LTJ9XSxbMCwxLCJcXHZhcnBoaSIsMix7Im9mZnNldCI6Mn1dLFsyLDAsImUiLDAseyJzdHlsZSI6eyJ0YWlsIjp7Im5hbWUiOiJtb25vIn19fV0sWzMsMF0sWzMsMiwiIiwxLHsic3R5bGUiOnsiYm9keSI6eyJuYW1lIjoiZGFzaGVkIn19fV1d
\[\begin{tikzcd}
	{\ker{\varphi}} && G && {G'} \\
	\\
	K
	\arrow["0", shift left=2, from=1-3, to=1-5]
	\arrow["\varphi"', shift right=2, from=1-3, to=1-5]
	\arrow["e", tail, from=1-1, to=1-3]
	\arrow[from=3-1, to=1-3]
	\arrow[dashed, from=3-1, to=1-1]
\end{tikzcd}\]
\end{defn}

Concretely, $\ker{\varphi}$ consists of the set
\[
  \{g \in G \mid \varphi(g) = e_{G'}\} = \varphi\inv(e_{G'})
\]
and $e$ is the inclusion map.

\begin{defn}
  Let $\varphi : G \to G'$ be a group homomorphism.
  The \emph{image} of $\varphi$ is the ``most efficient'' object with the following factorization:
  % https://q.uiver.app/#q=WzAsMyxbMCwwLCJHIl0sWzIsMCwiRyciXSxbMSwxLCJcXGltZ3tcXHZhcnBoaX0iXSxbMCwxLCJcXHZhcnBoaSJdLFswLDIsImUiLDJdLFsyLDEsIm0iLDIseyJzdHlsZSI6eyJ0YWlsIjp7Im5hbWUiOiJtb25vIn19fV1d
\[\begin{tikzcd}
	G && {G'} \\
	& {\img{\varphi}}
	\arrow["\varphi", from=1-1, to=1-3]
	\arrow["e"', from=1-1, to=2-2]
	\arrow["m"', tail, from=2-2, to=1-3]
\end{tikzcd}\]
\end{defn}

\begin{eg}[Subgroup Generated by a Subset]
  If $A \subseteq G$ is any subset, the inclusion map $i : A \to U(G)$ uniquely determines a group homomorphism
  \[
    \varphi_{A} : F(A) \to G
  \]
  The image of this homomorphism is the \emph{subgroup generated by} $A$ in $G$, denoted $\langle A \rangle$.
  % https://q.uiver.app/#q=WzAsMyxbMCwwLCJGKEEpIl0sWzIsMCwiRyciXSxbMSwxLCJcXGltZ3tcXHZhcnBoaV97QX19Il0sWzAsMSwiXFx2YXJwaGlfe0F9Il0sWzAsMiwiZSIsMl0sWzIsMSwibSIsMix7InN0eWxlIjp7InRhaWwiOnsibmFtZSI6Im1vbm8ifX19XV0=
\[\begin{tikzcd}
	{F(A)} && {G} \\
	& {\img{\varphi_{A}}}
	\arrow["{\varphi_{A}}", from=1-1, to=1-3]
	\arrow["e"', from=1-1, to=2-2]
	\arrow["m"', tail, from=2-2, to=1-3]
\end{tikzcd}\]

  Recall that $F(A)$ consists of formal words over $A$ modulo some relation.
  And $\varphi_{A}([w]) = i(w_{1}) \cdot \cdots \cdot i(w_{n}) = w_{1} \cdot \cdots \cdot w_{n}$.
  The image of this homomorphism consists of
  \[
    \{w_{1} \cdot \cdots \cdot w_{n} \mid [w] \in F(A)\}
  \]
  If $A = \{g\}$, then $F(A) \iso \dZ$ and $\varphi_{A} : \dZ \to G$ is the exponential map.
  Thus,
  \[
    \img{\varphi_{A}} = \{\ldots,g^{-1} , e, g, g^{2},\ldots\}
  \]
\end{eg}

\begin{lem}
  In $\mathsf{Grp}$, the following are equivalent:
  \begin{enumerate}
  \item $\varphi$ is a monomorphism;
  \item $\ker{\varphi} = \{e_{G}\}$;
  \item $\varphi : G \to G'$ is injective.
  \end{enumerate}
\end{lem}
\begin{proof}
  (i) $\imp$ (ii): recall that $\ker{\varphi}$ is the equalizer
  % https://q.uiver.app/#q=WzAsNCxbMiwwLCJHIl0sWzQsMCwiRyciXSxbMCwwLCJcXGtlcntcXHZhcnBoaX0iXSxbMCwyLCJLIl0sWzAsMSwiMCIsMCx7Im9mZnNldCI6LTJ9XSxbMCwxLCJcXHZhcnBoaSIsMix7Im9mZnNldCI6Mn1dLFsyLDAsImUiLDAseyJzdHlsZSI6eyJ0YWlsIjp7Im5hbWUiOiJtb25vIn19fV0sWzMsMF0sWzMsMiwiIiwxLHsic3R5bGUiOnsiYm9keSI6eyJuYW1lIjoiZGFzaGVkIn19fV1d
\[\begin{tikzcd}
	{\ker{\varphi}} && G && {G'} \\
	\\
	K
	\arrow["0", shift left=2, from=1-3, to=1-5]
	\arrow["\varphi"', shift right=2, from=1-3, to=1-5]
	\arrow["e", tail, from=1-1, to=1-3]
	\arrow[from=3-1, to=1-3]
	\arrow[dashed, from=3-1, to=1-1]
\end{tikzcd}\]
  Since $\varphi$ is monic and $e$ equalizes $\varphi$ and $0$, it follows that $e = 0$.
  Thus, $\ker{\varphi} = \{e_{G}\}$.

  (ii) $\imp$ (iii): suppose that $\varphi(a) = \varphi(b)$.
  Then $\varphi(a)\varphi(a)\inv = e_{G'} = \varphi(ba\inv)$.
  Thus, $ba\inv \in \ker{\varphi}$.
  Since $\ker{\varphi} = \{e_{G}\}$, it must be the case that $ba\inv = e_{G}$.
  Uniqueness of inverse then implies that $a = b$.

  (iii) $\imp$ (i): an injection is a monomorphism in $\mathsf{Set}$.
  The forgetful functor $U$ is faithful; and thus reflects monomorphisms.
  Then it follows that $\varphi$ is a monomorphism.
\end{proof}

\section{Quotient Groups}
\label{sec:quotient-groups}

Recall that we can quotient a set $S$ by an equivalence relation $\sim$, yielding a quotient map $S \epi S/\sim$.
We may attempt to define a quotient group $G/\sim$ by equipping the quotient set $G/\sim$ with a group structure.
To equip $G/\sim$ with a group structure, the condition that $\pi : G \to G/\sim$ is a group homomorphism demands
\[
  [a] \bullet [b] = \pi(a) \bullet \pi(b) = \pi(a \cdot b) = [a \cdot b]
\]
This imposes certain constraints on the equivalence relation $\sim$.

\begin{lem}
  The operation
  \[
    [a] \bullet [b] := [a \cdot b]
  \]
  defines a group structure if and only if
  \begin{align}
    a \sim b &\imp \forall g \in G, ag \sim bg\\
    a \sim b &\imp \forall g \in G, ga \sim gb
  \end{align}
  When these conditions hold, we say that $\sim$ is \emph{compatible} with the group structure of $G$.
\end{lem}

Given any subgroup $H$ of $G$, the subgroup $H$ induces two equivalence relations
\begin{mathpar}
  G/\sim_{L} = \{ aH \mid a \in G \}\and

  G/\sim_{R} = \{ Ha \mid a \in G \}
\end{mathpar}

These relations can be characterized as follows
\begin{lem}
  For any $a, b \in G$,
  \begin{align}
    a \sim_{L} b &\iff ab\inv \in H\\
    a \sim_{R} b &\iff ab\inv \in H\\
  \end{align}
\end{lem}

\begin{defn}[Normal Subgroups]
  A subgroup $N$ of a group $G$ is \emph{normal} if for all $g \in G$ and $n \in N$,
  \[
    gng\inv \in N
  \]
\end{defn}

\begin{lem}
  A subgroup $H$ of a group $G$ is normal if and only if $gH = Hg$ for all $g \in G$.
\end{lem}

\begin{cor}
  $\sim_{L}$ and $\sim_{R}$ agree if and only if $H$ is a normal subgroup.
\end{cor}

\begin{lem}
  If $\varphi : G \to G'$ is a group homomorphism, then $\ker{\varphi}$ is a normal subgroup of $G$.
\end{lem}
\begin{proof}
  $\ker{\varphi} = \{g \in G \mid \varphi(g) = e_{G'}\}$.
  For any $g \in G$ and $k \in \ker{\varphi}$, one has $\varphi(gkg\inv) = \varphi(g)e_{G'}\varphi(g)\inv = e_{G'}$.
  Thus, $gkg\inv \in \ker{\varphi}$.
\end{proof}


\begin{lem}
  Every normal subgroup is the kernel of some group homomorphism.
\end{lem}

\end{document}
