\documentclass{amsart}
\input{decls}
\title{Groups}
\author{Frank Tsai}
\date{\today}
%\thanks{}
\begin{document}
\maketitle
\tableofcontents

\section{Definition of Group}
\label{sec:definition-of-group}

\begin{defn}
  A \emph{group} consists of the following data:
  \begin{enumerate}
  \item a set $G$;
  \item an \emph{identity element} $e \in G$;
  \item a \emph{group multiplication} $\cdot : G \times G \to G$;
  \item a \emph{group inverse} $(\blank)\inv : G \to G$
  \end{enumerate}
  such that
  \begin{enumerate}
  \item group multiplication is associative;
  \item $a\inv \cdot a = e = a \cdot a\inv$ for all $a \in G$.
  \end{enumerate}
\end{defn}
The group axioms can be stated in terms of commutative diagrams.

\begin{mathpar}
  % https://q.uiver.app/#q=WzAsNCxbMCwwLCJHIFxcdGltZXMgRyBcXHRpbWVzIEciXSxbMiwwLCJHIFxcdGltZXMgRyJdLFswLDIsIkcgXFx0aW1lcyBHIl0sWzIsMiwiRyJdLFswLDIsIlxcaWRfe0d9IFxcdGltZXMgXFxjZG90IiwyXSxbMSwzLCJcXGNkb3QiXSxbMCwxLCJcXGNkb3QgXFx0aW1lcyBcXGlkX3tHfSJdLFsyLDMsIlxcY2RvdCIsMl1d
\begin{tikzcd}
	{G \times G \times G} && {G \times G} \\
	\\
	{G \times G} && G
	\arrow["{\id_{G} \times \cdot}"', from=1-1, to=3-1]
	\arrow["\cdot", from=1-3, to=3-3]
	\arrow["{\cdot \times \id_{G}}", from=1-1, to=1-3]
	\arrow["\cdot"', from=3-1, to=3-3]
\end{tikzcd}\and
  % https://q.uiver.app/#q=WzAsOCxbMiwwLCJHIFxcdGltZXMgRyJdLFs0LDAsIkcgXFx0aW1lcyBHIl0sWzYsMCwiRyBcXHRpbWVzIEciXSxbNCwyLCJHIl0sWzAsMCwiRyJdLFs4LDAsIkciXSxbMCwyLCIxIl0sWzgsMiwiMSJdLFswLDEsIihcXGJsYW5rKVxcaW52IFxcdGltZXMgXFxpZF97R30iXSxbMiwxLCJcXGlkX3tHfSBcXHRpbWVzIChcXGJsYW5rKVxcaW52IiwyXSxbMSwzLCJcXGNkb3QiLDFdLFs0LDAsIlxcRGVsdGEiXSxbNSwyLCJcXERlbHRhIiwyXSxbNCw2XSxbNSw3XSxbNiwzLCJlIiwyXSxbNywzLCJlIl1d
\begin{tikzcd}
	G && {G \times G} && {G \times G} && {G \times G} && G \\
	\\
	1 &&&& G &&&& 1
	\arrow["{(\blank)\inv \times \id_{G}}", from=1-3, to=1-5]
	\arrow["{\id_{G} \times (\blank)\inv}"', from=1-7, to=1-5]
	\arrow["\cdot"{description}, from=1-5, to=3-5]
	\arrow["\Delta", from=1-1, to=1-3]
	\arrow["\Delta"', from=1-9, to=1-7]
	\arrow[from=1-1, to=3-1]
	\arrow[from=1-9, to=3-9]
	\arrow["e"', from=3-1, to=3-5]
	\arrow["e", from=3-9, to=3-5]
\end{tikzcd}
\end{mathpar}

\begin{eg}[Symmetry Groups]
\end{eg}

\begin{eg}[Dihedral Groups]
\end{eg}

\section{Order}
\label{sec:order}

\begin{defn}
  An element $g$ of a group $G$ has \emph{finite order} if $g^{n} = e$ for some positive integer $n$.
  In this case, the \emph{order}, denoted $|g|$, of $g$ is the smallest such $n$.
  We write $|g| = \oo$ if $g$ does not have finite order.
\end{defn}

\section{Group Homomorphisms}
\label{sec:group-homomorphisms}

\begin{defn}
  A \emph{group homomorphism} from $(G,m_{G})$ to $(H,m_{H})$ is a function $\varphi : G \to H$ that preserves group multiplication.
  That is, the diagram commutes in $\mathsf{Set}$:
  % https://q.uiver.app/#q=WzAsNCxbMCwwLCJHIFxcdGltZXMgRyJdLFsyLDAsIkggXFx0aW1lcyBIIl0sWzAsMiwiRyJdLFsyLDIsIkgiXSxbMCwxLCJcXHZhcnBoaSBcXHRpbWVzIFxcdmFycGhpIl0sWzIsMywiXFx2YXJwaGkiLDJdLFswLDIsIm1fe0d9IiwyXSxbMSwzLCJtX3tIfSJdXQ==
\begin{tikzcd}
	{G \times G} && {H \times H} \\
	\\
	G && H
	\arrow["{\varphi \times \varphi}", from=1-1, to=1-3]
	\arrow["\varphi"', from=3-1, to=3-3]
	\arrow["{m_{G}}"', from=1-1, to=3-1]
	\arrow["{m_{H}}", from=1-3, to=3-3]
\end{tikzcd}
\end{defn}

\begin{lem}
  Groups and their homomorphisms assemble into a category $\mathsf{Grp}$.
\end{lem}

\begin{lem}
  Trivial groups are both initial and terminal in $\mathsf{Grp}$.
\end{lem}
\begin{proof}
  It's clear that trivial groups are terminal in $\mathsf{Grp}$ since for any set $G$, the unique function from $G$ in to a singleton set is vacuously a group homomorphism.
  Dually, since group homomorphisms preserve group identity, there's a unique group homomorphism from a singleton set to $G$.
\end{proof}

Thus, trivial groups are \emph{zero objects} of $\mathsf{Grp}$.
This implies that $\mathsf{Grp}(G,H)$ is nonempty for any group $G$ and $H$ since one always has
% https://q.uiver.app/#q=WzAsMyxbMCwwLCJHIl0sWzIsMSwiXFx7KlxcfSJdLFs0LDAsIkgiXSxbMCwxLCIhX3tHfSIsMl0sWzEsMiwiIV97R30iLDJdLFswLDJdXQ==
\[\begin{tikzcd}
	G &&&& H \\
	&& {\{*\}}
	\arrow["{!_{G}}"', from=1-1, to=2-3]
	\arrow["{!_{G}}"', from=2-3, to=1-5]
	\arrow[from=1-1, to=1-5]
\end{tikzcd}\]
Concretely, this homomorphism maps every element of $G$ to the identity of $H$.

\section{Products}
\label{sec:products}

Since the forgetful functor $U : \mathsf{Grp} \to \mathsf{Set}$ is representable by the group $\dZ$, the underlying sets of products in $\mathsf{Grp}$ are themselves products in $\mathsf{Set}$.
Thus, products in $\mathsf{Grp}$ are formed by giving group structures to products in $\mathsf{Set}$.

\begin{defn}
  Let $G$ and $H$ be groups, the \emph{direct product} $G \times H$ is the group formed by equipping the product of the underlying sets of $G$ and $H$ with the group structure
\begin{align}
  \cdot &: ((g,h),(g',h')) \mapsto (gg',hh')\\
  (\blank)\inv &: (g,h) \mapsto (g\inv,h\inv)
\end{align}
That is, the group operations are defined componentwise.
\end{defn}


\end{document}
