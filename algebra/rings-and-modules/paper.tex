\documentclass{amsart}
\input{decls}
\title{Rings and Modules}
\author{Frank Tsai}
\date{\today}
%\thanks{}
\begin{document}
\maketitle
\tableofcontents

\section{Rings}
\label{sec:rings}

\begin{defn}
  A \emph{ring} consists of
  \begin{enumerate}
  \item an underlying set $R$;
  \item a binary operation $+ : R \times R \to R$;
  \item a binary operation $\cdot : R \times R \to R$;
  \item an additive identity $0_{R}$;
  \item a multiplicative identity $1_{R}$
  \end{enumerate}
  such that $(R,+,0_{R})$ is an abelian group and $(R,\cdot,1_{R})$ is a monoid.
  Additionally, the two operations satisfy the distributive laws
  \begin{align}
    t \cdot (r + s) &= t \cdot r + t \cdot s\\
    (r + s) \cdot t &= r \cdot t + s \cdot t
  \end{align}
\end{defn}

\begin{defn}
  A ring $R$ is \emph{commutative} if the ring multiplication is commutative.
\end{defn}

We can equip the trivial group $\{\star\}$ with a ring multiplication.
The ring multiplication coincides with the ring addition and the two identities coincide, i.e., $0 = 1$.
\begin{defn}
  Such a ring is called a \emph{zero-ring}.
\end{defn}

\begin{defn}
  An element $a$ in a ring $R$ is a \emph{left-zero divisor} if there is an element $b \ne 0$ in $R$ for which $ab = 0$, and is a \emph{right-zero divisor} if $ba = 0$.
\end{defn}

\begin{lem}\label{lem:zero-divisor-characterization}
  Let $R$ be a ring.
  An element $a \in R$ is not a left- (resp., right-) zero divisor if and only if left (resp., right) multiplication by $a$ is an injection.
\end{lem}
\begin{proof}
  Suppose that $a \in R$ is not a left-zero divisor and that $ab = ac$.
  \[
    a(b - c) = 0
  \]
  Since $a$ is not a zero-divisor, $b - c$ must be $0$.
  Thus, $b = c$.

  Conversely, if $a \cdot \blank$ is an injection.
  Suppose that $a$ is a left-zero divisor, then there is some $b \ne 0$ so that $ab = 0$.
  Then since $ab = a0 = 0$, it follows that $b = 0$, which is a contradiction.
  (Note that this is not a proof by contradiction).
\end{proof}

Some special rings such as $\dZ, \dQ$, etc., are commutative rings \emph{without} zero-divisors.
These commutative rings have a special name:
\begin{defn}
  An \emph{integral domain} is a commutative ring \emph{without} zero-divisors.
  Namely,
  \[
    ab = 0 \iff a = 0~\text{or}~b = 0
  \]
\end{defn}

\begin{defn}
  An element $u$ of a ring $R$ is a \emph{left (resp., right) unit} if there is some $v \in R$ such that $uv = 1$ (resp., $vu = 1$).
\end{defn}

\begin{lem}
  Let $R$ be a ring.
  Then
  \begin{enumerate}
  \item $u$ is a left (resp., right) unit if and only if left (resp., right) multiplication with $u$ is a surjection;
  \item if $u$ is a left (resp., right) unit, then $u$ is not a right (resp., left) zero-divisor;
  \item two-sided units form a group under multiplication;
  \item the inverse of a two-sided unit is unique.
  \end{enumerate}
\end{lem}
\begin{proof}[Proof of (i)]
  Suppose that $u$ is a left unit.
  Then it has a right inverse $v$ so that $uv = 1$.
  Let $w \in R$ be given.
  Then
  \[
    u(vw) = 1w = w
  \]
  Thus, $u \cdot \blank$ is surjective.
  
  Conversely, if $u \cdot \blank$ is surjective, then there is some element $v \in R$ so that
  \[
    uv = 1
  \]
\end{proof}

\begin{proof}[Proof of (ii)]
  Suppose that $u$ is a left unit with a right inverse $v$.
  By \cref{lem:zero-divisor-characterization}, it suffices to prove $\blank \cdot u$ is an injection.
  Suppose that $au = bu$.
  Then $auv = buv \iff a1 = b1 \iff a = b$.
\end{proof}

\begin{proof}[Proof of (iii)]
  The multiplicative identity $1$ is a two-sided unit.
  This provides the group identity.
  Group axioms follow immediately by the definition of two-sided units.
\end{proof}

\begin{proof}[Proof of (iv)]
  This is an immediate consequence of (iii).
\end{proof}

\end{document}
