\documentclass[article,10pt,oneside]{memoir}
\usepackage{amsthm,amssymb,amsmath,stmaryrd,mathrsfs}
\usepackage[T1]{fontenc}
\usepackage{xcolor}
\definecolor{darkgreen}{rgb}{0,0.45,0} 
\usepackage[pagebackref,colorlinks,citecolor=darkgreen,linkcolor=darkgreen]{hyperref}
\usepackage{zi4}
\usepackage[capitalise]{cleveref}
\usepackage{quiver}
\usepackage{tikz,tikz-cd}
\usepackage{enumitem}
\usepackage{mathtools}
\usepackage{ifmtarg}
\usepackage{braket}
\let\setof\Set
\usepackage{url}
\usepackage{xspace}
\usepackage{mathpartir}
\usepackage{xparse}

\newcommand{\todo}[1]{{\color{red}\textbf{TODO: }#1}}

\counterwithout{section}{chapter}
\setsecnumdepth{subsection}
\setsecnumformat{\csname the#1\endcsname.\ \ }

\setsecheadstyle{\bfseries}
\setsubsecheadstyle{\bfseries}
\setsubsubsecheadstyle{\bfseries}
\setparaheadstyle{\bfseries}
\setsubparaheadstyle{\bfseries}

\pretitle{\begin{center}\LARGE\bfseries\MakeTextUppercase}
\posttitle{\par\end{center}\vskip 0.5em}

\linespread{1}

\ExplSyntaxOn

\AtEndPreamble{
  \setlrmarginsandblock{2cm}{*}{1}
  \setulmarginsandblock{2cm}{*}{1}
  \setheaderspaces{*}{\onelineskip}{*}
  \checkandfixthelayout
}

\NewDocumentCommand\ega_thesection{}{\thesection.~}
\NewDocumentCommand\ega_thesubsection{}{\thesubsection.~}
\NewDocumentCommand\ega_thesubsubsection{}{\thesubsubsection.~}


\setsecnumdepth{subsubsection}
\setsecnumformat{\csname ega_the#1\endcsname}
\setsecheadstyle{\normalsize\bfseries\MakeUppercase}
\setsubsecheadstyle{\noindent\normalfont\bfseries}


\appto\mainmatter{
  \setcounter{secnumdepth}{30}
}
\ExplSyntaxOff

%% theorem environments
\newtheorem{thm}{Theorem}[section]
\newtheorem{lem}[thm]{Lemma}
\newtheorem{cor}[thm]{Corollary}
\newtheorem{prop}[thm]{Proposition}

\theoremstyle{definition}
\newtheorem{notn}[thm]{Notation}
\newtheorem{defn}[thm]{Definition}
\newtheorem{rmk}[thm]{Remark}
\newtheorem{eg}[thm]{Example}

% magic
\makeatletter
\let\ea\expandafter

%% Defining commands that are always in math mode.
\def\mdef#1#2{\ea\ea\ea\gdef\ea\ea\noexpand#1\ea{\ea\ensuremath\ea{#2}\xspace}}
\def\alwaysmath#1{\ea\ea\ea\global\ea\ea\ea\let\ea\ea\csname your@#1\endcsname\csname #1\endcsname
  \ea\def\csname #1\endcsname{\ensuremath{\csname your@#1\endcsname}\xspace}}

%% SIMPLE COMMANDS FOR FONTS AND DECORATIONS

\newcount\foreachcount

\def\foreachletter#1#2#3{\foreachcount=#1
  \ea\loop\ea\ea\ea#3\@alph\foreachcount
  \advance\foreachcount by 1
  \ifnum\foreachcount<#2\repeat}

\def\foreachLetter#1#2#3{\foreachcount=#1
  \ea\loop\ea\ea\ea#3\@Alph\foreachcount
  \advance\foreachcount by 1
  \ifnum\foreachcount<#2\repeat}

% Script: \sA is \mathscr{A}
\def\definescr#1{\ea\gdef\csname s#1\endcsname{\ensuremath{\mathscr{#1}}\xspace}}
\foreachLetter{1}{27}{\definescr}
% Calligraphic: \cA is \mathcal{A}
\def\definecal#1{\ea\gdef\csname c#1\endcsname{\ensuremath{\mathcal{#1}}\xspace}}
\foreachLetter{1}{27}{\definecal}
% Bold: \bA is \mathbf{A}
\def\definebold#1{\ea\gdef\csname b#1\endcsname{\ensuremath{\mathbf{#1}}\xspace}}
\foreachLetter{1}{27}{\definebold}
% Blackboard Bold: \dA is \mathbb{A}
\def\definebb#1{\ea\gdef\csname d#1\endcsname{\ensuremath{\mathbb{#1}}\xspace}}
\foreachLetter{1}{27}{\definebb}
% Fraktur: \fa is \mathfrak{a}, except for \fi; \fA is \mathfrak{A}
\def\definefrak#1{\ea\gdef\csname f#1\endcsname{\ensuremath{\mathfrak{#1}}\xspace}}
\foreachletter{1}{9}{\definefrak}
\foreachletter{10}{27}{\definefrak}
\foreachLetter{1}{27}{\definefrak}
% Sans serif: \ia is \mathsf{a}, except for \if and \in and \it
\def\definesf#1{\ea\gdef\csname i#1\endcsname{\ensuremath{\mathsf{#1}}\xspace}}
\foreachletter{1}{6}{\definesf}
\foreachletter{7}{14}{\definesf}
\foreachletter{15}{20}{\definesf}
\foreachletter{21}{27}{\definesf}
\foreachLetter{1}{27}{\definesf}
% Bar: \Abar is \overline{A}, \abar is \overline{a}
\def\definebar#1{\ea\gdef\csname #1bar\endcsname{\ensuremath{\overline{#1}}\xspace}}
\foreachLetter{1}{27}{\definebar}
\foreachletter{1}{8}{\definebar} % \hbar is something else!
\foreachletter{9}{15}{\definebar} % \obar is something else!
\foreachletter{16}{27}{\definebar}
% Tilde: \Atil is \widetilde{A}, \atil is \widetilde{a}
\def\definetil#1{\ea\gdef\csname #1til\endcsname{\ensuremath{\widetilde{#1}}\xspace}}
\foreachLetter{1}{27}{\definetil}
\foreachletter{1}{27}{\definetil}
% Hats: \Ahat is \widehat{A}, \ahat is \widehat{a}
\def\definehat#1{\ea\gdef\csname #1hat\endcsname{\ensuremath{\widehat{#1}}\xspace}}
\foreachLetter{1}{27}{\definehat}
\foreachletter{1}{27}{\definehat}
% Checks: \Achk is \widecheck{A}, \achk is \widecheck{a}
\def\definechk#1{\ea\gdef\csname #1chk\endcsname{\ensuremath{\widecheck{#1}}\xspace}}
\foreachLetter{1}{27}{\definechk}
\foreachletter{1}{27}{\definechk}
% Underline: \uA is \underline{A}, \ua is \underline{a}
\def\defineul#1{\ea\gdef\csname u#1\endcsname{\ensuremath{\underline{#1}}\xspace}}
\foreachLetter{1}{27}{\defineul}
\foreachletter{1}{27}{\defineul}

% Particular commands for typefaces, sometimes with the first letter
% different.
\def\autofmt@n#1\autofmt@end{\mathrm{#1}}
\def\autofmt@b#1\autofmt@end{\mathbf{#1}}
\def\autofmt@d#1#2\autofmt@end{\mathbb{#1}\mathsf{#2}}
\def\autofmt@c#1#2\autofmt@end{\mathcal{#1}\mathit{#2}}
\def\autofmt@s#1#2\autofmt@end{\mathscr{#1}\mathit{#2}}
\def\autofmt@i#1\autofmt@end{\mathsf{#1}}
\def\autofmt@f#1\autofmt@end{\mathfrak{#1}}
% Particular commands for decorations.
\def\autofmt@u#1\autofmt@end{\underline{\smash{\mathsf{#1}}}}
\def\autofmt@U#1\autofmt@end{\underline{\underline{\smash{\mathsf{#1}}}}}
\def\autofmt@h#1\autofmt@end{\widehat{#1}}
\def\autofmt@r#1\autofmt@end{\overline{#1}}
\def\autofmt@t#1\autofmt@end{\widetilde{#1}}
\def\autofmt@k#1\autofmt@end{\check{#1}}

% Defining multi-letter commands.  Use this like so:
% \autodefs{\bSet\cCat\cCAT\kBicat\lProf}
\def\auto@drop#1{}
\def\autodef#1{\ea\ea\ea\@autodef\ea\ea\ea#1\ea\auto@drop\string#1\autodef@end}
\def\@autodef#1#2#3\autodef@end{%
  \ea\def\ea#1\ea{\ea\ensuremath\ea{\csname autofmt@#2\endcsname#3\autofmt@end}\xspace}}
\def\autodefs@end{blarg!}
\def\autodefs#1{\@autodefs#1\autodefs@end}
\def\@autodefs#1{\ifx#1\autodefs@end%
  \def\autodefs@next{}%
  \else%
  \def\autodefs@next{\autodef#1\@autodefs}%
  \fi\autodefs@next}

%% FONTS AND DECORATION FOR GREEK LETTERS

%% the package `mathbbol' gives us blackboard bold greek and numbers,
%% but it does it by redefining \mathbb to use a different font, so that
%% all the other \mathbb letters look different too.  Here we import the
%% font with bb greek and numbers, but assign it a different name,
%% \mathbbb, so as not to replace the usual one.
\DeclareSymbolFont{bbold}{U}{bbold}{m}{n}
\DeclareSymbolFontAlphabet{\mathbbb}{bbold}
\newcommand{\dDelta}{\ensuremath{\mathbbb{\Delta}}\xspace}
\newcommand{\done}{\ensuremath{\mathbbb{1}}\xspace}
\newcommand{\dtwo}{\ensuremath{\mathbbb{2}}\xspace}
\newcommand{\dthree}{\ensuremath{\mathbbb{3}}\xspace}

% greek with bars
\newcommand{\albar}{\ensuremath{\overline{\alpha}}\xspace}
\newcommand{\bebar}{\ensuremath{\overline{\beta}}\xspace}
\newcommand{\gmbar}{\ensuremath{\overline{\gamma}}\xspace}
\newcommand{\debar}{\ensuremath{\overline{\delta}}\xspace}
\newcommand{\phibar}{\ensuremath{\overline{\varphi}}\xspace}
\newcommand{\psibar}{\ensuremath{\overline{\psi}}\xspace}
\newcommand{\xibar}{\ensuremath{\overline{\xi}}\xspace}
\newcommand{\ombar}{\ensuremath{\overline{\omega}}\xspace}

% greek with tildes
\newcommand{\altil}{\ensuremath{\widetilde{\alpha}}\xspace}
\newcommand{\betil}{\ensuremath{\widetilde{\beta}}\xspace}
\newcommand{\gmtil}{\ensuremath{\widetilde{\gamma}}\xspace}
\newcommand{\phitil}{\ensuremath{\widetilde{\varphi}}\xspace}
\newcommand{\psitil}{\ensuremath{\widetilde{\psi}}\xspace}
\newcommand{\xitil}{\ensuremath{\widetilde{\xi}}\xspace}
\newcommand{\omtil}{\ensuremath{\widetilde{\omega}}\xspace}

% MISCELLANEOUS SYMBOLS
\let\del\partial
\mdef\delbar{\overline{\partial}}
\let\sm\wedge
\newcommand{\dd}[1]{\ensuremath{\frac{\partial}{\partial {#1}}}}
\newcommand{\inv}{^{-1}}
\newcommand{\dual}{^{\vee}}
\mdef\hf{\textstyle\frac12 }
\mdef\thrd{\textstyle\frac13 }
\mdef\qtr{\textstyle\frac14 }
\let\meet\wedge
\let\join\vee
\let\dn\downarrow
\newcommand{\op}{^{\mathrm{op}}}
\newcommand{\co}{^{\mathrm{co}}}
\newcommand{\coop}{^{\mathrm{coop}}}
\let\adj\dashv
\let\iso\cong
\let\eqv\simeq
\let\cng\equiv
\mdef\Id{\mathrm{Id}}
\mdef\id{\mathrm{id}}
\alwaysmath{ell}
\alwaysmath{infty}
\let\oo\infty
\def\io{\ensuremath{(\infty,1)}}
\alwaysmath{odot}
\def\frc#1/#2.{\frac{#1}{#2}}   % \frc x^2+1 / x^2-1 .
\mdef\ten{\mathrel{\otimes}}
\let\bigten\bigotimes
\mdef\sqten{\mathrel{\boxtimes}}
\def\lt{<}                      % For iTex compatibility
\def\gt{>}

%%% Blanks (shorthand for lambda abstractions)
\newcommand{\blank}{\mathord{\hspace{1pt}\text{--}\hspace{1pt}}}
%%% Nameless objects
\newcommand{\nameless}{\mathord{\hspace{1pt}\underline{\hspace{1ex}}\hspace{1pt}}}

% Hiragana "yo" for the Yoneda embedding, from https://arxiv.org/abs/1506.08870
\DeclareFontFamily{U}{min}{}
\DeclareFontShape{U}{min}{m}{n}{<-> udmj30}{}
\newcommand{\yon}{\!\text{\usefont{U}{min}{m}{n}\symbol{'210}}\!}

%% Get some new symbols from mathabx, without changing the old ones by
%% importing the package.  Font declarations copied from mathabx.sty:
\DeclareFontFamily{U}{mathb}{\hyphenchar\font45}
\DeclareFontShape{U}{mathb}{m}{n}{
      <5> <6> <7> <8> <9> <10> gen * mathb
      <10.95> mathb10 <12> <14.4> <17.28> <20.74> <24.88> mathb12
      }{}
\DeclareSymbolFont{mathb}{U}{mathb}{m}{n}
\DeclareFontSubstitution{U}{mathb}{m}{n}
\DeclareFontFamily{U}{mathx}{\hyphenchar\font45}
\DeclareFontShape{U}{mathx}{m}{n}{
      <5> <6> <7> <8> <9> <10>
      <10.95> <12> <14.4> <17.28> <20.74> <24.88>
      mathx10
      }{}
\DeclareSymbolFont{mathx}{U}{mathx}{m}{n}
\DeclareFontSubstitution{U}{mathx}{m}{n}
%% And now the symbols we want, copied from mathabx.dcl
\DeclareMathSymbol{\dotplus}       {2}{mathb}{"00}% name to be checked
\DeclareMathSymbol{\dotdiv}        {2}{mathb}{"01}% name to be checked
\DeclareMathSymbol{\dottimes}      {2}{mathb}{"02}% name to be checked
\DeclareMathSymbol{\divdot}        {2}{mathb}{"03}% name to be checked
\DeclareMathSymbol{\udot}          {2}{mathb}{"04}% name to be checked
\DeclareMathSymbol{\square}        {2}{mathb}{"05}% name to be checked
\DeclareMathSymbol{\Asterisk}      {2}{mathb}{"06}
\DeclareMathSymbol{\bigast}        {1}{mathb}{"06}
\DeclareMathSymbol{\coAsterisk}    {2}{mathb}{"07}
\DeclareMathSymbol{\bigcoast}      {1}{mathb}{"07}
\DeclareMathSymbol{\circplus}      {2}{mathb}{"08}% name to be checked
\DeclareMathSymbol{\pluscirc}      {2}{mathb}{"09}% name to be checked
\DeclareMathSymbol{\convolution}   {2}{mathb}{"0A}% name to be checked
\DeclareMathSymbol{\divideontimes} {2}{mathb}{"0B}% name to be checked
\DeclareMathSymbol{\blackdiamond}  {2}{mathb}{"0C}% name to be checked
\DeclareMathSymbol{\sqbullet}      {2}{mathb}{"0D}% name to be checked
\DeclareMathSymbol{\bigstar}       {2}{mathb}{"0E}
\DeclareMathSymbol{\bigvarstar}    {2}{mathb}{"0F}
\DeclareMathSymbol{\corresponds}   {3}{mathb}{"1D}% name to be checked
\DeclareMathSymbol{\boxleft}      {2}{mathb}{"68}
\DeclareMathSymbol{\boxright}     {2}{mathb}{"69}
\DeclareMathSymbol{\boxtop}       {2}{mathb}{"6A}
\DeclareMathSymbol{\boxbot}       {2}{mathb}{"6B}
\DeclareMathSymbol{\updownarrows}          {3}{mathb}{"D6}
\DeclareMathSymbol{\downuparrows}          {3}{mathb}{"D7}
\DeclareMathSymbol{\Lsh}                   {3}{mathb}{"E8}
\DeclareMathSymbol{\Rsh}                   {3}{mathb}{"E9}
\DeclareMathSymbol{\dlsh}                  {3}{mathb}{"EA}
\DeclareMathSymbol{\drsh}                  {3}{mathb}{"EB}
\DeclareMathSymbol{\looparrowdownleft}     {3}{mathb}{"EE}
\DeclareMathSymbol{\looparrowdownright}    {3}{mathb}{"EF}
% \DeclareMathSymbol{\curvearrowleft}        {3}{mathb}{"F0}
% \DeclareMathSymbol{\curvearrowright}       {3}{mathb}{"F1}
\DeclareMathSymbol{\curvearrowleftright}   {3}{mathb}{"F2}
\DeclareMathSymbol{\curvearrowbotleft}     {3}{mathb}{"F3}
\DeclareMathSymbol{\curvearrowbotright}    {3}{mathb}{"F4}
\DeclareMathSymbol{\curvearrowbotleftright}{3}{mathb}{"F5}
% \DeclareMathSymbol{\circlearrowleft}       {3}{mathb}{"F6}
% \DeclareMathSymbol{\circlearrowright}      {3}{mathb}{"F7}
\DeclareMathSymbol{\leftsquigarrow}        {3}{mathb}{"F8}
\DeclareMathSymbol{\rightsquigarrow}       {3}{mathb}{"F9}
\DeclareMathSymbol{\leftrightsquigarrow}   {3}{mathb}{"FA}
\DeclareMathSymbol{\lefttorightarrow}      {3}{mathb}{"FC}
\DeclareMathSymbol{\righttoleftarrow}      {3}{mathb}{"FD}
\DeclareMathSymbol{\uptodownarrow}         {3}{mathb}{"FE}
\DeclareMathSymbol{\downtouparrow}         {3}{mathb}{"FF}
\DeclareMathSymbol{\varhash}       {0}{mathb}{"23}
\DeclareMathSymbol{\sqSubset}       {3}{mathb}{"94}
\DeclareMathSymbol{\sqSupset}       {3}{mathb}{"95}
\DeclareMathSymbol{\nsqSubset}      {3}{mathb}{"96}
\DeclareMathSymbol{\nsqSupset}      {3}{mathb}{"97}
% WIDECHECK
\DeclareMathAccent{\widecheck}    {0}{mathx}{"71}


%% OPERATORS
\DeclareMathOperator\lan{Lan}
\DeclareMathOperator\ran{Ran}
\DeclareMathOperator\colim{colim}
\DeclareMathOperator\coeq{coeq}
\DeclareMathOperator\ob{ob}
\DeclareMathOperator\cod{cod}
\DeclareMathOperator\dom{dom}
\DeclareMathOperator\ev{ev}
\DeclareMathOperator\eq{eq}
\DeclareMathOperator\Tot{Tot}
\DeclareMathOperator\cosk{cosk}
\DeclareMathOperator\sk{sk}
\DeclareMathOperator\img{im}
\DeclareMathOperator\Spec{Spec}
\DeclareMathOperator\Ho{Ho}
\DeclareMathOperator\Aut{Aut}
\DeclareMathOperator\End{End}
\DeclareMathOperator\Hom{Hom}
\DeclareMathOperator\Map{Map}
\DeclareMathOperator\coker{coker}
\DeclareMathOperator\Alg{Alg}
\DeclareMathOperator\Cone{Cone}
\DeclareMathOperator\Cocone{Cocone}
\DeclareMathOperator\Idem{Idem}
\DeclareMathOperator\Cont{Cont}
\DeclareMathOperator\Pres{Pres}
\DeclareMathOperator\Psh{Psh}

%% ARROWS
% \to already exists
\newcommand{\too}[1][]{\ensuremath{\overset{#1}{\longrightarrow}}}
\newcommand{\ot}{\ensuremath{\leftarrow}}
\newcommand{\oot}[1][]{\ensuremath{\overset{#1}{\longleftarrow}}}
\let\toot\rightleftarrows
\let\otto\leftrightarrows
\let\Impl\Rightarrow
\let\imp\Rightarrow
\let\toto\rightrightarrows
\let\into\hookrightarrow
\let\xinto\xhookrightarrow
\mdef\we{\overset{\sim}{\longrightarrow}}
\mdef\leftwe{\overset{\sim}{\longleftarrow}}
\let\mono\rightarrowtail
\let\leftmono\leftarrowtail
\let\cof\rightarrowtail
\let\leftcof\leftarrowtail
\let\epi\twoheadrightarrow
\let\leftepi\twoheadleftarrow
\let\fib\twoheadrightarrow
\let\leftfib\twoheadleftarrow
\let\cohto\rightsquigarrow
\let\maps\colon
\newcommand{\spam}{\,:\!}       % \maps for left arrows
\def\acof{\mathrel{\mathrlap{\hspace{3pt}\raisebox{4pt}{$\scriptscriptstyle\sim$}}\mathord{\rightarrowtail}}}

% diagxy redefines \to, along with \toleft, \two, \epi, and \mon.

%% EXTENSIBLE ARROWS
\let\xto\xrightarrow
\let\xot\xleftarrow
% See Voss' Mathmode.tex for instructions on how to create new
% extensible arrows.
\def\rightarrowtailfill@{\arrowfill@{\Yright\joinrel\relbar}\relbar\rightarrow}
\newcommand\xrightarrowtail[2][]{\ext@arrow 0055{\rightarrowtailfill@}{#1}{#2}}
\let\xmono\xrightarrowtail
\let\xcof\xrightarrowtail
\def\twoheadrightarrowfill@{\arrowfill@{\relbar\joinrel\relbar}\relbar\twoheadrightarrow}
\newcommand\xtwoheadrightarrow[2][]{\ext@arrow 0055{\twoheadrightarrowfill@}{#1}{#2}}
\let\xepi\xtwoheadrightarrow
\let\xfib\xtwoheadrightarrow
% Let's leave the left-going ones until I need them.

%% EXTENSIBLE SLASHED ARROWS
% Making extensible slashed arrows, by modifying the underlying AMS code.
% Arguments are:
% 1 = arrowhead on the left (\relbar or \Relbar if none)
% 2 = fill character (usually \relbar or \Relbar)
% 3 = slash character (such as \mapstochar or \Mapstochar)
% 4 = arrowhead on the left (\relbar or \Relbar if none)
% 5 = display mode (\displaystyle etc)
\def\slashedarrowfill@#1#2#3#4#5{%
  $\m@th\thickmuskip0mu\medmuskip\thickmuskip\thinmuskip\thickmuskip
   \relax#5#1\mkern-7mu%
   \cleaders\hbox{$#5\mkern-2mu#2\mkern-2mu$}\hfill
   \mathclap{#3}\mathclap{#2}%
   \cleaders\hbox{$#5\mkern-2mu#2\mkern-2mu$}\hfill
   \mkern-7mu#4$%
}
% Here's the idea: \<slashed>arrowfill@ should be a box containing
% some stretchable space that is the "middle of the arrow".  This
% space is created as a "leader" using \cleader<thing>\hfill, which
% fills an \hfill of space with copies of <thing>.  Here instead of
% just one \cleader, we use two, with the slash in between (and an
% extra copy of the filler, to avoid extra space around the slash).
\def\rightslashedarrowfill@{%
  \slashedarrowfill@\relbar\relbar\mapstochar\rightarrow}
\newcommand\xslashedrightarrow[2][]{%
  \ext@arrow 0055{\rightslashedarrowfill@}{#1}{#2}}
\mdef\hto{\xslashedrightarrow{}}
\mdef\htoo{\xslashedrightarrow{\quad}}
\let\xhto\xslashedrightarrow

%% To get a slashed arrow in XYmatrix, do
% \[\xymatrix{A \ar[r]|-@{|} & B}\]
%% To get it in diagxy, do
% \morphism/{@{>}|-*@{|}}/[A`B;p]

%% Here is an \hto for diagxy:
% \def\htopppp/#1/<#2>^#3_#4{\:%
% \ifnum#2=0%
%    \setwdth{#3}{#4}\deltax=\wdth \divide \deltax by \ul%
%    \advance \deltax by \defaultmargin  \ratchet{\deltax}{100}%
% \else \deltax #2%
% \fi%
% \xy\ar@{#1}|-@{|}^{#3}_{#4}(\deltax,0) \endxy%
% \:}%
% \def\htoppp/#1/<#2>^#3{\ifnextchar_{\htopppp/#1/<#2>^{#3}}{\htopppp/#1/<#2>^{#3}_{}}}%
% \def\htopp/#1/<#2>{\ifnextchar^{\htoppp/#1/<#2>}{\htoppp/#1/<#2>^{}}}%
% \def\htoop/#1/{\ifnextchar<{\htopp/#1/}{\htopp/#1/<0>}}%
% \def\hto{\ifnextchar/{\htoop}{\htoop/>/}}%

% LABELED ISOMORPHISMS
\def\xiso#1{\mathrel{\mathrlap{\smash{\xto[\smash{\raisebox{1.3mm}{$\scriptstyle\sim$}}]{#1}}}\hphantom{\xto{#1}}}}
\def\toiso{\xto{\smash{\raisebox{-.5mm}{$\scriptstyle\sim$}}}}

% SHADOWS
\def\shvar#1#2{{\ensuremath{%
  \hspace{1mm}\makebox[-1mm]{$#1\langle$}\makebox[0mm]{$#1\langle$}\hspace{1mm}%
  {#2}%
  \makebox[1mm]{$#1\rangle$}\makebox[0mm]{$#1\rangle$}%
}}}
\def\sh{\shvar{}}
\def\scriptsh{\shvar{\scriptstyle}}
\def\bigsh{\shvar{\big}}
\def\Bigsh{\shvar{\Big}}
\def\biggsh{\shvar{\bigg}}
\def\Biggsh{\shvar{\Bigg}}

%% Paul Taylor: noncommutative diagrams
\def\puncture{
  \begingroup
  \setbox0\hbox{A}%
  \vrule height.53\ht0 depth-.47\ht0 width.35\ht0
  \kern .12\ht0
  \vrule height\ht0 depth-.65\ht0 width.06\ht0
  \kern-.06\ht0
  \vrule height.35\ht0 depth0pt width.06\ht0
  \kern .12\ht0
  \vrule height.53\ht0 depth-.47\ht0 width.35\ht0
  \endgroup
}

% TYPING JUDGMENTS
% Call this macro as \jd{x:A, y:B |- c:C}.  It adds (what I think is)
% appropriate spacing, plus auto-sized parentheses around each typing judgment.
\def\jd#1{\@jd#1\ej}
\def\@jd#1|-#2\ej{\@@jd#1,,\;\vdash\;\left(#2\right)}
\def\@@jd#1,{\@ifmtarg{#1}{\let\next=\relax}{\left(#1\right)\let\next=\@@@jd}\next}
\def\@@@jd#1,{\@ifmtarg{#1}{\let\next=\relax}{,\,\left(#1\right)\let\next=\@@@jd}\next}
% Here's a version which puts a line break before the turnstyle.
\def\jdm#1{\@jdm#1\ej}
\def\@jdm#1|-#2\ej{\@@jd#1,,\\\vdash\;\left(#2\right)}
% Make an actual comma that doesn't separate typing judgments (e.g. A,B,C : Type).
\def\cm{,}

% 2-(CO)MONAD STUFF
\def\alg{\text{-}\mathcal{A}\mathit{lg}}
\def\algs{\text{-}\mathcal{A}\mathit{lg}_s}
\def\algl{\text{-}\mathcal{A}\mathit{lg}_l}
\def\algc{\text{-}\mathcal{A}\mathit{lg}_c}
\def\algw{\text{-}\mathcal{A}\mathit{lg}_w}
\def\psalg{\text{-}\mathcal{P}\mathit{s}\mathcal{A}\mathit{lg}}
\def\dalg{\text{-}\mathbb{A}\mathsf{lg}}
\def\coalg{\text{-}\mathcal{C}\mathit{oalg}}
\def\coalgs{\text{-}\mathcal{C}\mathit{oalg}_s}
\def\coalgl{\text{-}\mathcal{C}\mathit{oalg}_l}
\def\coalgc{\text{-}\mathcal{C}\mathit{oalg}_c}
\def\coalgw{\text{-}\mathcal{C}\mathit{oalg}_w}
\def\pscoalg{\text{-}\mathcal{P}\mathit{s}\mathcal{C}\mathit{oalg}}
\def\dcoalg{\text{-}\mathbb{C}\mathsf{oalg}}
\def\Mod{\mathsf{Mod}}
\def\CSet{\mathsf{Set}}
\def\CAlg{\mathsf{ALG}}
\def\CTh{\mathsf{TH}}
\def\CFinSet{\mathsf{FinSet}}
\def\CCat{\mathsf{Cat}}
\def\CLaw{\mathsf{Law}}
\def\CSkt{\mathsf{Skt}}

%% SKIPIT in TikZ
% See http://tex.stackexchange.com/questions/3513/draw-only-some-segments-of-a-path-in-tikz
\long\def\my@drawfill#1#2;{%
\@skipfalse
\fill[#1,draw=none] #2;
\@skiptrue
\draw[#1,fill=none] #2;
}
\newif\if@skip
\newcommand{\skipit}[1]{\if@skip\else#1\fi}
\newcommand{\drawfill}[1][]{\my@drawfill{#1}}

%% TWOCELLS AND PULLBACKS in TIKZ-CD
\newcounter{nodemaker}
\setcounter{nodemaker}{0}
\newcommand{\twocell}[2][]{%
  \global\edef\mynodeone{twocell\arabic{nodemaker}}%
  \stepcounter{nodemaker}%
  \global\edef\mynodetwo{twocell\arabic{nodemaker}}%
  \stepcounter{nodemaker}%
  \ar[#2,phantom,shift left=3,""{name=\mynodeone}]%
  \ar[#2,phantom,shift right=3,""'{name=\mynodetwo}]%
  \ar[Rightarrow,from=\mynodeone,to=\mynodetwo,"{#1}"]%
}
\newcommand{\twocellop}[2][]{%
  \global\edef\mynodeone{twocell\arabic{nodemaker}}%
  \stepcounter{nodemaker}%
  \global\edef\mynodetwo{twocell\arabic{nodemaker}}%
  \stepcounter{nodemaker}%
  \ar[#2,phantom,shift left=3,""{name=\mynodeone}]%
  \ar[#2,phantom,shift right=3,""'{name=\mynodetwo}]%
  \ar[Rightarrow,from=\mynodetwo,to=\mynodeone,"{#1}"]%
}
\newcommand{\drpullback}[1][dr]{\ar[#1,phantom,near start,"\lrcorner"]}
\newcommand{\dlpullback}[1][dl]{\ar[#1,phantom,near start,"\llcorner"]}
\newcommand{\urpullback}[1][ur]{\ar[#1,phantom,near start,"\urcorner"]}
\newcommand{\ulpullback}[1][ul]{\ar[#1,phantom,near start,"\ulcorner"]}


%% Include or exclude solutions
% This code is basically copied from version.sty, except that when the
% solutions are included, we put them in a `proof' environment as
% well.  To include solutions, say \includesolutions; to exclude them
% say \excludesolutions.
% \begingroup
% 
% \catcode`{=12\relax\catcode`}=12\relax%
% \catcode`(=1\relax \catcode`)=2\relax%
% \gdef\includesolutions(\newenvironment(soln)(\begin(proof)[Solution])(\end(proof)))%
% \gdef\excludesolutions(%
%   \gdef\soln(\@bsphack\catcode`{=12\relax\catcode`}=12\relax\soln@NOTE)%
%   \long\gdef\soln@NOTE##1\end{soln}(\solnEND@NOTE)%
%   \gdef\solnEND@NOTE(\@esphack\end(soln))%
% )%
% \endgroup

\makeatother

% Local Variables:
% mode: latex
% TeX-master: ""
% End:
\title{categorical logic}
\author{Frank Tsai}
\date{\today}
%\thanks{}
\begin{document}
\maketitle
\tableofcontents

\section{Introduction}
\label{sec:introduction}

Category theory has enabled us to investigate algebraic theories in a presentation-invariant way.
Algebraic theories are precisely categories with finite products.
We think of the objects of such a category as sorts and morphisms as terms.
Equations are internalized as the diagonal subobjects.
Given two algebraic theories $\cT$ and $\cT'$, a model of $\cT$ in $\cT'$ is a finite-product preserving functor $M : \cT \to \cT'$, ensuring that products of sorts in $\cT$ are interpreted (sensibly) as products of sorts in $\cT'$.
Additionally, this also ensures that equations are preserved.
Unsurprisingly, a homomorphism from $M$ to $M'$ is precisely a natural transformation $\alpha : M \To M'$.
% https://q.uiver.app/#q=WzAsNCxbMiwwLCJNcyJdLFswLDIsIk1zIl0sWzIsMiwiTShzIFxcdGltZXMgcykiXSxbNCwyLCJNKHMpIl0sWzIsMSwiXFxwaV97MX0iXSxbMiwzLCJcXHBpX3syfSIsMl0sWzAsMSwiXFxpZCIsMl0sWzAsMywiXFxpZCJdLFswLDIsIk1cXERlbHRhIiwxLHsic3R5bGUiOnsidGFpbCI6eyJuYW1lIjoibW9ubyJ9LCJib2R5Ijp7Im5hbWUiOiJkYXNoZWQifX19XV0=
\[\begin{tikzcd}
    && Ms \\
    \\
    Ms && {M(s \times s)} && {Ms}
    \arrow["{\pi_{1}}", from=3-3, to=3-1]
    \arrow["{\pi_{2}}"', from=3-3, to=3-5]
    \arrow["\id"', from=1-3, to=3-1]
    \arrow["\id", from=1-3, to=3-5]
    \arrow["M\Delta"{description}, dashed, tail, from=1-3, to=3-3]
  \end{tikzcd}\]

Algebraic theories together with their models and homomorphisms between them assemble into a 2-category $\CAlg$.
For any (small) algebraic theory $\cT$, the hom-category $\CAlg(\cT,\CSet)$, i.e., the category of set-theoretic models, is equivalent to a cocomplete category with a strong generator comprised of perfectly presentable objects, i.e., a variety.
We can generalize this viewpoint: theories (of some kind) are categories with sufficient structures.
A model of a theory in another theory (of the same kind) is a functor preserving those structures.
The goal of this writeup is to develop this idea in more detail.

\section{Yoneda and cantor}
\label{sec:yoneda-and-cantor}

\begin{cons}[Archetypal Grothendieck construction]\label{cons:archetypal-grothendieck-construction}
  Let $X$ be a set.
  There is a bijection between the powerset $P(X)$ and the set of all truth functionals over $X$.
  \[
    \Theta : P(X) \otto 2^{X} : \Psi
  \]
  \begin{itemize}
  \item[$\Theta$:] For each subset $S \subseteq X$, $\Theta(S)$ is the characteristic function of $X$, mapping elements in $S$ to $1$ and everything else to $0$.
  \item[$\Psi$:] For each truth functional $\phi : X \to 2$, $\Psi(\phi)$ is the inverse image $\phi\inv(1)$.
  \end{itemize}
\end{cons}

\begin{rmk}
  The map $\Psi$ in \cref{cons:archetypal-grothendieck-construction} can be described as a pullback.
  % https://q.uiver.app/#q=WzAsNCxbMiwwLCIxIl0sWzIsMiwiMiJdLFswLDIsIlgiXSxbMCwwLCJcXHZhcnBoaVxcaW52KDEpIl0sWzAsMSwidCIsMCx7InN0eWxlIjp7InRhaWwiOnsibmFtZSI6Im1vbm8ifX19XSxbMiwxLCJcXHZhcnBoaSIsMl0sWzMsMiwiIiwyLHsic3R5bGUiOnsidGFpbCI6eyJuYW1lIjoibW9ubyJ9fX1dLFszLDBdLFszLDEsIiIsMSx7InN0eWxlIjp7Im5hbWUiOiJjb3JuZXIifX1dXQ==
  \[\begin{tikzcd}
      {\varphi\inv(1)} && 1 \\
      \\
      X && 2
      \arrow["t", tail, from=1-3, to=3-3]
      \arrow["\varphi"', from=3-1, to=3-3]
      \arrow[tail, from=1-1, to=3-1]
      \arrow[from=1-1, to=1-3]
      \arrow["\lrcorner"{anchor=center, pos=0.125}, draw=none, from=1-1, to=3-3]
    \end{tikzcd}\]
  Indeed, $\varphi\inv(1)$ is the largest (w.r.t. inclusion) subset of $X$ so that $\varphi$ evaluates to $1$ everywhere.
\end{rmk}

\begin{rmk}
  In $\CSet$, we make the foundational commitment that $\CSet(A, B)$ is a set and that the powerset $P(A)$ is in fact a set.
  As it turns out, assuming either one is sufficient to construct the other.
\end{rmk}

\begin{rmk}
  The set $P(X)$ has a canonical poset structure given by the set inclusion relation.
  In fact, $P(X)$ equipped with set inclusion is a complete Boolean algebra.
  Thus, the powerset functor $P : \CSet\op \to \CSet$ may be equally regarded as a functor $P : \CSet\op \to \CPos$, sending each set to its powerset ordered by inclusion.
\end{rmk}

\begin{rmk}
  The two point set has a unique nontrivial poset structure $\Set{0 \leq 1}$.
  We can define a poset structure on truth functionals by transporting the poset structure defined on the powersets along the Grothendieck construction.
  \[
    \chi_{X} \leq \chi_{Y} \text{ iff } X \subseteq Y
  \]
  This is the usual pointwise poset structure.
\end{rmk}

We can classify subsets of a given set $X$.
We can classify $X$-indexed families (i.e., functions $f : Y \to X$), too.
These families assemble into the slice construction $\CSet/X$.

\begin{cons}[Discrete Grothendieck construction]
  Let $X$ be a set construed as a discrete category.
  There is an equivalence of categories.
  \[
    \Theta : \CSet/X \otto \CSet^{X} : \Psi
  \]
  \begin{itemize}
  \item[$\Theta$:] For each function $f : Y \to X$, $\Theta(f)$ is the functor that sends each $x \in X$ to the inverse image $f\inv(x)$.
  \item[$\Psi$:] For each functor $\varphi : X \to \CSet$, $\Psi(\varphi)$ is the canonical map $\coprod_{x \in X}\varphi(x) \to X$
  \end{itemize}
\end{cons}

\begin{rmk}
  Indeed, $\Psi$ can be phrased in terms of pullback.
  % https://q.uiver.app/#q=WzAsNCxbMiwwLCJcXENTZXRfeyp9Il0sWzIsMiwiXFxDU2V0Il0sWzAsMiwiWCJdLFswLDAsIlxcY29wcm9kX3t4IFxcaW4gWH1cXHZhcnBoaSh4KSJdLFswLDFdLFsyLDEsIlxcdmFycGhpIiwyXSxbMywyXSxbMywwXSxbMywxLCIiLDEseyJzdHlsZSI6eyJuYW1lIjoiY29ybmVyIn19XV0=
  \[\begin{tikzcd}
      {\coprod_{x \in X}\varphi(x)} && {\CSet_{*}} \\
      \\
      X && \CSet
      \arrow[from=1-3, to=3-3]
      \arrow["\varphi"', from=3-1, to=3-3]
      \arrow[from=1-1, to=3-1]
      \arrow[from=1-1, to=1-3]
      \arrow["\lrcorner"{anchor=center, pos=0.125}, draw=none, from=1-1, to=3-3]
    \end{tikzcd}\]
  $\coprod_{x \in X}\varphi(x) \to X$ is the ``largest'' family so that $\varphi$ produces a witness everywhere.
\end{rmk}

\section{Hyperdoctrine}
\label{sec:hyperdoctrine}

The internal logic of a Heyting (or Boolean) algebra is intuitionistic (or classical) propositional logic.
Each sort of a given theory is equipped with a poset of propositions, in which $\varphi \leq \psi$ means $\varphi$ entails $\psi$.
For simplicity, let us assume a single sorted theory with no function symbols.
Let us recall Tarski's definition:
\begin{enumerate}
\item The unique sort is a set $S$.
\item A proposition variable is a subset of $S$.
\item $\top$ is the entire set $S$.
\item $\bot$ is the empty set $\varnothing$.
\item $\varphi \wedge \psi$ is the intersection $\varphi \cap \psi$.
\item $\varphi \vee \psi$ is the union $\varphi \cup \psi$.
\item $\varphi \To \psi$ is the exponential $\psi^{\varphi}$, i.e., the largest element so that $\psi^{\varphi} \wedge \varphi \leq \psi$.
\end{enumerate}

Essentially, propositions live in a poset with sufficient structure: various logical connectives correspond to various universal constructions.
As a model of a theory $\cT$ in another theory $\cT'$ has to preserve entailment, we expect it to be a function $M : \cT \to \cT'$ preserving relevant structures.

Generalizing this to multisorted theories is straightforward.
Each sort of the theory has a poset of propositions.
In the presence of nontrivial terms, we need to specify how propositions in context interact with substitutions.
For example, given a proposition in the context $d$, i.e., $\vdash_{d} \varphi$, substitution by a term $f : c \to d$ ought to produce a proposition in the context $c$, i.e., $\vdash_{c} \varphi[f]$.
Of course, we expect substitution to satisfy $\varphi[f][g] = \varphi[g \circ f]$ and $\varphi[\id] = \varphi$.

\begin{defn}
  A \emph{doctrine} is a functor $\cC : \iC\op \to \CPos$, where $\CPos$ is some suitable category of posets.
\end{defn}

\begin{defn}\label{def:model-of-doctrine}
  A \emph{model} of a doctrine $\cC : \iC\op \to \CPos$ in another doctrine $\cD : \iD\op \to \CPos$ consists of
  \begin{itemize}
  \item A model $M : \iC \to \iD$.
  \item A natural transformation $\lambda : \cC \To \cD \circ M\op$.
  \end{itemize}
  % https://q.uiver.app/#q=WzAsMyxbMCwwLCJcXGlDXFxvcCJdLFsyLDAsIlxcaURcXG9wIl0sWzEsMSwiXFxDUG9zIl0sWzAsMSwiTVxcb3AiXSxbMCwyLCJcXGNDIiwyXSxbMSwyLCJcXGNEIl0sWzQsMSwiXFxsYW1iZGEiLDEseyJzaG9ydGVuIjp7InNvdXJjZSI6MTAsInRhcmdldCI6MTB9fV1d
  \[\begin{tikzcd}
      \iC\op && \iD\op \\
      & \CPos
      \arrow["M\op", from=1-1, to=1-3]
      \arrow[""{name=0, anchor=center, inner sep=0}, "\cC"', from=1-1, to=2-2]
      \arrow["\cD", from=1-3, to=2-2]
      \arrow["\lambda"{description}, shorten <=4pt, shorten >=4pt, Rightarrow, from=0, to=1-3]
    \end{tikzcd}\]
\end{defn}

Let us unpack \cref{def:model-of-doctrine}.
The model $M : \iC \to \iD$ is unsurprising: we need to express $\iC$ in terms of $\iD$.
Each component $\lambda_{c}$ is a model of propositions $\cC(c)$ in $\cD M\op(c)$.
Naturality expresses that these models are compatible with substitutions. 
% https://q.uiver.app/#q=WzAsNCxbMCwwLCJcXGNDKGMpIl0sWzIsMCwiXFxjRCBNXFxvcChjKSJdLFswLDIsIlxcY0MoYycpIl0sWzIsMiwiXFxjRCBNXFxvcChjJykiXSxbMCwxLCJcXGxhbWJkYV97Y30iXSxbMiwwLCJcXGJsYW5rW2ZdIl0sWzMsMSwiXFxibGFua1tNXFxvcChmKV0iLDJdLFsyLDMsIlxcbGFtYmRhX3tjJ30iLDJdXQ==
\[\begin{tikzcd}
    {\cC(c)} && {\cD M\op(c)} \\
    \\
    {\cC(c')} && {\cD M\op(c')}
    \arrow["{\lambda_{c}}", from=1-1, to=1-3]
    \arrow["{\blank[f]}", from=3-1, to=1-1]
    \arrow["{\blank[M\op(f)]}"', from=3-3, to=1-3]
    \arrow["{\lambda_{c'}}"', from=3-1, to=3-3]
  \end{tikzcd}\]

Weighted quantifiers can be characterized as adjoints, too.
Given a term $f : c \to d$, the formula $\exists_{f}\varphi$ ought to be the smallest formula so that $\varphi \leq (\exists_{f}\varphi)[f]$.
Similarly, the formula $\forall_{f}\varphi$ ought to be the largest formula so that $(\forall_{f}\varphi)[f] \leq \varphi$.
That is, we need the following adjoints:
% https://q.uiver.app/#q=WzAsMixbMCwwLCJcXGNDKGMpIl0sWzAsMiwiXFxjQyhkKSJdLFsxLDAsIlxcYmxhbmtbZl0iLDFdLFswLDEsIlxcZXhpc3RzX3tmfSIsMix7ImN1cnZlIjo0fV0sWzAsMSwiXFxmb3JhbGxfe2Z9IiwwLHsiY3VydmUiOi00fV0sWzMsMiwiIiwxLHsibGV2ZWwiOjEsInN0eWxlIjp7Im5hbWUiOiJhZGp1bmN0aW9uIn19XSxbMiw0LCIiLDEseyJsZXZlbCI6MSwic3R5bGUiOnsibmFtZSI6ImFkanVuY3Rpb24ifX1dXQ==
\[\begin{tikzcd}
    {\cC(c)} \\
    \\
    {\cC(d)}
    \arrow[""{name=0, anchor=center, inner sep=0}, "{f^{*}}"{description}, from=3-1, to=1-1]
    \arrow[""{name=1, anchor=center, inner sep=0}, "{\exists_{f}}"', curve={height=24pt}, from=1-1, to=3-1]
    \arrow[""{name=2, anchor=center, inner sep=0}, "{\forall_{f}}", curve={height=-24pt}, from=1-1, to=3-1]
    \arrow["\dashv"{anchor=center}, draw=none, from=1, to=0]
    \arrow["\dashv"{anchor=center}, draw=none, from=0, to=2]
  \end{tikzcd}\]
Indeed, in the powerset doctrine $\cP : \CSet\op \to \CPos$, the existential quantifier and the universal quantifier are given by the direct image $f_{*}$ and the fiber image $f_{!}$, respectively.

The powerset doctrine factors through the contravariant powerset functor in a canonical way because every powerset has a canonical poset structure given by the inclusion relation.
% https://q.uiver.app/#q=WzAsNCxbMCwwLCJcXGlEIl0sWzIsMCwiXFxDU2V0XFxvcCJdLFsxLDIsIlxcQ1BvcyJdLFszLDIsIlxcQ1NldCJdLFswLDIsIlxcY0QiLDJdLFswLDEsIk0iXSxbMSwzLCJcXGNQIl0sWzIsM10sWzEsMiwiIiwxLHsic3R5bGUiOnsiYm9keSI6eyJuYW1lIjoiZGFzaGVkIn19fV0sWzQsMSwiXFxsYW1iZGEiLDAseyJzaG9ydGVuIjp7InNvdXJjZSI6MTAsInRhcmdldCI6MTB9fV1d
\[\begin{tikzcd}
    \iU\op && \CSet\op \\
    \\
    & \CPos && \CSet
    \arrow[""{name=0, anchor=center, inner sep=0}, "\cU"', from=1-1, to=3-2]
    \arrow["M", from=1-1, to=1-3]
    \arrow["\cP", from=1-3, to=3-4]
    \arrow[from=3-2, to=3-4]
    \arrow[dashed, from=1-3, to=3-2]
    \arrow["\lambda", shorten <=5pt, shorten >=5pt, Rightarrow, from=0, to=1-3]
  \end{tikzcd}\]
Thus, we may identify a model of a doctrine as a natural transformation (of some kind) between presheaves.
% https://q.uiver.app/#q=WzAsMyxbMCwwLCJcXGlEXFxvcCJdLFsyLDAsIlxcQ1NldFxcb3AiXSxbMiwyLCJcXENTZXQiXSxbMCwxXSxbMSwyXSxbMCwyLCJcXGNEIiwyXSxbNSwxLCJcXGxhbWJkYSIsMix7InNob3J0ZW4iOnsic291cmNlIjoxMCwidGFyZ2V0IjoxMH19XV0=
\[\begin{tikzcd}
    \iU\op && \CSet\op \\
    \\
    && \CSet
    \arrow[from=1-1, to=1-3]
    \arrow[from=1-3, to=3-3]
    \arrow[""{name=0, anchor=center, inner sep=0}, "\cU"', from=1-1, to=3-3]
    \arrow["\lambda"', shorten <=3pt, shorten >=3pt, Rightarrow, from=0, to=1-3]
  \end{tikzcd}\]
Given a natural transformation $\lambda : \cU \To \tilde{\cU}$ between presheaves, we may think of $\cU$ as a presheaf of ``witnesses'' and $\tilde{\cU}$ as a presheaf of ``types.''
By the Yoneda lemma, each type $A \in \tilde{\cU}(\Gamma)$ can be identified with a natural transformation $A : \iU(\blank, \Gamma) \To \cU$.
If we additionally require that the pullback of $A$ along $\lambda$ is representable, then the two projections correspond to a term $\Gamma.A \to \Gamma$ in the theory $\iU$ and a witness $t \in \cU(\Gamma.A)$, respectively.
% https://q.uiver.app/#q=WzAsNCxbMiwwLCJcXGNVIl0sWzIsMiwiXFx0aWxkZXtcXGNVfSJdLFswLDIsIlxceW9uXFxHYW1tYSJdLFswLDAsIlxceW9uXFxHYW1tYS5BIl0sWzAsMSwiXFxsYW1iZGEiXSxbMiwxLCJBIiwyXSxbMywyXSxbMywwLCJ0Il0sWzMsMSwiIiwxLHsic3R5bGUiOnsibmFtZSI6ImNvcm5lciJ9fV1d
\[\begin{tikzcd}
    {\yon\Gamma.A} && \cU \\
    \\
    \yon\Gamma && {\tilde{\cU}}
    \arrow["\lambda", from=1-3, to=3-3]
    \arrow["A"', from=3-1, to=3-3]
    \arrow[from=1-1, to=3-1]
    \arrow["t", from=1-1, to=1-3]
    \arrow["\lrcorner"{anchor=center, pos=0.125}, draw=none, from=1-1, to=3-3]
  \end{tikzcd}\]

\section{Orthogonality}
\label{sec:orthogonality}

\todo{Talk about its relationship to universal arrows.}
\todo{Ivan's notes: topics in category theory 2018}

\begin{defn}
  An object $c$ is \emph{orthogonal} to a morphism $m : d \to d'$ if for each morphism $f : d \to c$, there is a unique morphism $f' : d' \to c$ so that the following diagram commutes.
  % https://q.uiver.app/#q=WzAsMyxbMCwwLCJkIl0sWzIsMCwiZCciXSxbMSwxLCJjIl0sWzAsMSwibSJdLFswLDIsImYiLDJdLFsxLDIsImYnIiwwLHsic3R5bGUiOnsiYm9keSI6eyJuYW1lIjoiZGFzaGVkIn19fV1d
  \[\begin{tikzcd}
      d && {d'} \\
      & c
      \arrow["m", from=1-1, to=1-3]
      \arrow["f"', from=1-1, to=2-2]
      \arrow["{f'}", dashed, from=1-3, to=2-2]
    \end{tikzcd}\]
  For a class $\cM$ of morphisms in a category $\iC$, we write $\cM^{\bot}$ for the full subcategory of $\iC$ spanned by objects orthogonal to each $m \in \cM$.
\end{defn}

\section{Sheaves}
\label{sec:sheaves}

\begin{defn}
  Let $X$ be a topological space.
  A \emph{sheaf} (of sets) is a presheaf $F : \cO(X)\op \to \CSet$ satisfying the \emph{sheaf condition}: For any open cover $U = \bigcup_{i}U_{i}$, the following diagram is an equalizer.
  % https://q.uiver.app/#q=WzAsMyxbMCwwLCJGKFUpIl0sWzIsMCwiXFxwcm9kX3tpfUYoVV97aX0pIl0sWzQsMCwiXFxwcm9kX3tpLGp9RihVX3tpfSBcXGNhcCBVX3tqfSkiXSxbMCwxXSxbMSwyLCJwIiwwLHsib2Zmc2V0IjotMX1dLFsxLDIsInEiLDIseyJvZmZzZXQiOjF9XV0=
  \[\begin{tikzcd}
      {F(U)} && {\prod_{i}F(U_{i})} && {\prod_{i,j}F(U_{i} \cap U_{j})}
      \arrow[from=1-1, to=1-3]
      \arrow["p", shift left, from=1-3, to=1-5]
      \arrow["q"', shift right, from=1-3, to=1-5]
    \end{tikzcd}\]
  The morphisms $p$ and $q$ are the canonical ones specified by morphisms of the form $F(U_{i} \cap U_{j} \subseteq U_{k})$ composed with an appropriate projection.
  % https://q.uiver.app/#q=WzAsNyxbMCwyLCJGKFUpIl0sWzIsMiwiXFxwcm9kX3tpfUYoVV97aX0pIl0sWzQsMiwiXFxwcm9kX3tpLGp9RihVX3tpfSBcXGNhcCBVX3tqfSkiXSxbMiwwLCJGKFVfe2l9KSJdLFsyLDQsIkYoVV97an0pIl0sWzQsMCwiRihVX3tpfSBcXGNhcCBVX3tqfSkiXSxbNCw0LCJGKFVfe2l9IFxcY2FwIFVfe2p9KSJdLFswLDFdLFsxLDIsInAiLDAseyJvZmZzZXQiOi0xfV0sWzEsMiwicSIsMix7Im9mZnNldCI6MX1dLFsxLDMsIlxccGlfe2l9Il0sWzAsMywiRihVX3tpfSBcXHN1YnNldGVxIFUpIl0sWzEsNCwiXFxwaV97an0iLDJdLFswLDQsIkYoVV97an0gXFxzdWJzZXRlcSBVKSIsMl0sWzIsNSwiXFxwaV97aSxqfSIsMl0sWzIsNiwiXFxwaV97aSxqfSJdLFszLDUsIkYoVV97aX0gXFxjYXAgVV97an0gXFxzdWJzZXRlcSBVX3tpfSkiXSxbNCw2LCJGKFVfe2l9IFxcY2FwIFVfe2p9IFxcc3Vic2V0ZXEgVV97an0pIiwyXV0=
  \[\begin{tikzcd}
      && {F(U_{i})} && {F(U_{i} \cap U_{j})} \\
      \\
      {F(U)} && {\prod_{i}F(U_{i})} && {\prod_{i,j}F(U_{i} \cap U_{j})} \\
      \\
      && {F(U_{j})} && {F(U_{i} \cap U_{j})}
      \arrow[from=3-1, to=3-3]
      \arrow["p", shift left, from=3-3, to=3-5]
      \arrow["q"', shift right, from=3-3, to=3-5]
      \arrow["{\pi_{i}}", from=3-3, to=1-3]
      \arrow["{F(U_{i} \subseteq U)}", from=3-1, to=1-3]
      \arrow["{\pi_{j}}"', from=3-3, to=5-3]
      \arrow["{F(U_{j} \subseteq U)}"', from=3-1, to=5-3]
      \arrow["{\pi_{i,j}}"', from=3-5, to=1-5]
      \arrow["{\pi_{i,j}}", from=3-5, to=5-5]
      \arrow["{F(U_{i} \cap U_{j} \subseteq U_{i})}", from=1-3, to=1-5]
      \arrow["{F(U_{i} \cap U_{j} \subseteq U_{j})}"', from=5-3, to=5-5]
    \end{tikzcd}\]
\end{defn}

\begin{defn}
  Let $X$ be a topological space.
  We define the category of sheaves $\Sh(X)$ as the full subcategory of $\Psh(\cO(X))$ spanned by sheaves.
\end{defn}

\begin{thm}
  For any topological space $X$, there is an isomorphism
  \[
    \Theta : \cO(X) \otto \Sub_{\Sh(X)}(1) : \Psi
  \]
\end{thm}
\begin{proof}
  We construct the two order-preserving functions as follows:
  \begin{itemize}
  \item[($\Theta$)] For each open set $O$, $\Theta(O)$ is the represented subsheaf $\Hom(\blank, O)$ of the terminal sheaf $1$.
  \item[($\Psi$)] For each subsheaf $F$ of the terminal sheaf, $\Psi(F)$ is the open set $\bigcup\Set{O \in \cO(X) \mid F(O) = 1}$, i.e., it is the union of all open sets $O$ so that $F(O)$ is the singleton set.
  \end{itemize}
\end{proof}

\begin{defn}
  Let $X$ and $Y$ be topological spaces.
  A continuous map $p : Y \to X$ is called a \emph{bundle over} $X$.
  A \emph{cross section} of a bundle $p : Y \to X$ is a continuous map $s : X \to Y$ so that $ps = \id$, i.e., a morphism from the identity bundle $\id : X \to X$ to $p : Y \to X$ in the slice category $\CTop/X$.
\end{defn}

\begin{rmk}
  Let $U \subseteq X$ be an open set of the topological space $X$, and $p : Y \to X$ be a bundle over $X$.
  The pullback yields a bundle over $U$.
  % https://q.uiver.app/#q=WzAsNCxbMCwyLCJVIl0sWzIsMiwiWCJdLFsyLDAsIlkiXSxbMCwwLCJwXFxpbnYgVSJdLFsyLDEsInAiXSxbMCwxLCIiLDIseyJzdHlsZSI6eyJ0YWlsIjp7Im5hbWUiOiJtb25vIn19fV0sWzMsMCwicFxcdmVydF97VX0iLDJdLFszLDIsIiIsMCx7InN0eWxlIjp7InRhaWwiOnsibmFtZSI6Im1vbm8ifX19XSxbMywxLCIiLDEseyJzdHlsZSI6eyJuYW1lIjoiY29ybmVyIn19XSxbMCwyLCJzIiwxLHsic3R5bGUiOnsiYm9keSI6eyJuYW1lIjoiZGFzaGVkIn19fV1d
  \[\begin{tikzcd}
      {p\inv U} && Y \\
      \\
      U && X
      \arrow["p", from=1-3, to=3-3]
      \arrow[tail, from=3-1, to=3-3]
      \arrow["{p\vert_{U}}"', from=1-1, to=3-1]
      \arrow[tail, from=1-1, to=1-3]
      \arrow["\lrcorner"{anchor=center, pos=0.125}, draw=none, from=1-1, to=3-3]
      \arrow["s"{description}, dashed, from=3-1, to=1-3]
    \end{tikzcd}\]
  We call $s$ a \emph{cross section} of $p$ over $U$ if $s \circ p\vert_{U}$ is the inclusion.
\end{rmk}

\begin{rmk}
  We have an assignment $\Gamma_{p} : \cO(X) \to \CSet$ defined by sending open sets $U$ to the set of all cross sections of $p$ over $U$.
  This can be extended to a contravariant functor as for any open set $U' \subseteq U$ and any cross section $s$, we can define a cross section of $p$ over $U'$ by composition.
  % https://q.uiver.app/#q=WzAsNixbMiwyLCJVIl0sWzQsMiwiWCJdLFs0LDAsIlkiXSxbMiwwLCJwXFxpbnYgVSJdLFswLDAsInBcXGludiBVJyJdLFswLDIsIlUnIl0sWzIsMSwicCJdLFswLDEsIiIsMix7InN0eWxlIjp7InRhaWwiOnsibmFtZSI6Im1vbm8ifX19XSxbMywwLCJwXFx2ZXJ0X3tVfSIsMl0sWzMsMiwiIiwwLHsic3R5bGUiOnsidGFpbCI6eyJuYW1lIjoibW9ubyJ9fX1dLFszLDEsIiIsMSx7InN0eWxlIjp7Im5hbWUiOiJjb3JuZXIifX1dLFswLDIsInMiLDEseyJzdHlsZSI6eyJib2R5Ijp7Im5hbWUiOiJkYXNoZWQifX19XSxbNSwwLCIiLDAseyJzdHlsZSI6eyJ0YWlsIjp7Im5hbWUiOiJtb25vIn19fV0sWzQsNSwicFxcdmVydF97VSd9IiwyXSxbNCwzLCIiLDAseyJzdHlsZSI6eyJ0YWlsIjp7Im5hbWUiOiJtb25vIn19fV0sWzUsMiwiIiwyLHsiY3VydmUiOi0yLCJzdHlsZSI6eyJib2R5Ijp7Im5hbWUiOiJkYXNoZWQifX19XSxbNCwwLCIiLDIseyJzdHlsZSI6eyJuYW1lIjoiY29ybmVyIn19XV0=
  \[\begin{tikzcd}
      {p\inv U'} && {p\inv U} && Y \\
      \\
      {U'} && U && X
      \arrow["p", from=1-5, to=3-5]
      \arrow[tail, from=3-3, to=3-5]
      \arrow["{p\vert_{U}}"', from=1-3, to=3-3]
      \arrow[tail, from=1-3, to=1-5]
      \arrow["\lrcorner"{anchor=center, pos=0.125}, draw=none, from=1-3, to=3-5]
      \arrow["s"{description}, dashed, from=3-3, to=1-5]
      \arrow[tail, from=3-1, to=3-3]
      \arrow["{p\vert_{U'}}"', from=1-1, to=3-1]
      \arrow[tail, from=1-1, to=1-3]
      \arrow[curve={height=-12pt}, dashed, from=3-1, to=1-5]
      \arrow["\lrcorner"{anchor=center, pos=0.125}, draw=none, from=1-1, to=3-3]
    \end{tikzcd}\]
\end{rmk}

\section{Elementary topoi}
\label{sec:elementary-topoi}

\begin{defn}[Elementary form]\label{def:topos}
  An (elementary) \emph{topos} $\cE$ is a category with
  \begin{enumerate}
  \item\label{def:topos-pullbacks} Pullbacks;
  \item\label{def:topos-terminal} A terminal object $1$;
  \item A subobject classifier $\Omega$;
  \item\label{def:topos-exponential} To each object $a$ an object $Pa$ and a morphism $\in_{a} : Pa \times a \to \Omega$ so that for any $f : b \times a \to \Omega$, there is a unique morphism $g : b \to Pa$ so that $g \times \id$ factors $f$ through $\in_{a}$.
    % https://q.uiver.app/#q=WzAsNSxbMiwxLCJhIFxcdGltZXMgUGEiXSxbMSwyLCJcXE9tZWdhIl0sWzAsMSwiYiBcXHRpbWVzIGEiXSxbMCwwLCJiIl0sWzIsMCwiUGEiXSxbMCwxLCJcXGluX3thfSJdLFsyLDEsImYiLDJdLFsyLDAsImcgXFx0aW1lcyBcXGlkIl0sWzMsNCwiZyIsMCx7InN0eWxlIjp7ImJvZHkiOnsibmFtZSI6ImRhc2hlZCJ9fX1dXQ==
    \[\begin{tikzcd}
	b && Pa \\
	{b \times a} && {Pa \times a} \\
	& \Omega
	\arrow["{\in_{a}}", from=2-3, to=3-2]
	\arrow["f"', from=2-1, to=3-2]
	\arrow["{g \times \id}", from=2-1, to=2-3]
	\arrow["g", dashed, from=1-1, to=1-3]
      \end{tikzcd}\]
  \end{enumerate}
\end{defn}

\begin{rmk}
  We can equivalently phrase \cref{def:topos} in a nonelementary way.
  Conditions \ref{def:topos-pullbacks} and \ref{def:topos-terminal} demand that $\cE$ is finitely complete, while condition \ref{def:topos-exponential} demands that $\cE$ has power objects, and thus all exponentials (c.f., \cref{thm:topos-has-exponentials}).
  In other words, a topos is a category with finite limits, exponentials, and a subobject classifier.
\end{rmk}

\begin{rmk}
  The assignment $P : \ob\cE \to \ob\cE$ can be uniquely extended to a functor.
  Extending $P$ to a functor means that for any $c$, the following isomorphism has to be natural in $a$.
  \[
    \cE(c, Pa) \iso \cE(c \times a, \Omega)
  \]
  That is, for any $h : a \to b$, the following diagram must commute.
  % https://q.uiver.app/#q=WzAsNCxbMCwwLCJcXGNFKFBiLFBhKSJdLFsyLDAsIlxcY0UoUGIgXFx0aW1lcyBhLCBcXE9tZWdhKSJdLFsyLDIsIlxcY0UoUGIgXFx0aW1lcyBiLCBcXE9tZWdhKSJdLFswLDIsIlxcY0UoUGIsIFBiKSJdLFswLDEsIlxcaXNvIiwzLHsic3R5bGUiOnsiYm9keSI6eyJuYW1lIjoibm9uZSJ9LCJoZWFkIjp7Im5hbWUiOiJub25lIn19fV0sWzMsMiwiXFxpc28iLDMseyJzdHlsZSI6eyJib2R5Ijp7Im5hbWUiOiJub25lIn0sImhlYWQiOnsibmFtZSI6Im5vbmUifX19XSxbMiwxLCJcXGJsYW5rIFxcY2lyYyAoXFxpZCBcXHRpbWVzIGgpIiwyXSxbMywwLCJQaCBcXGNpcmMgXFxibGFuayJdXQ==
  \[\begin{tikzcd}
      {\cE(Pb,Pa)} && {\cE(Pb \times a, \Omega)} \\
      \\
      {\cE(Pb, Pb)} && {\cE(Pb \times b, \Omega)}
      \arrow["\iso"{marking, allow upside down}, draw=none, from=1-1, to=1-3]
      \arrow["\iso"{marking, allow upside down}, draw=none, from=3-1, to=3-3]
      \arrow["{\blank \circ (\id \times h)}"', from=3-3, to=1-3]
      \arrow["{Ph \circ \blank}", from=3-1, to=1-1]
    \end{tikzcd}\]
  The transpose of $\id_{Pb}$ is $\in_{b}$.
  Thus, $Ph$ must be the transpose of $\in_{b}(\id \times h)$.
  Namely, it is the unique morphism given by the universal property of power objects.
  % https://q.uiver.app/#q=WzAsMyxbMCwyLCJQYSBcXHRpbWVzIGEiXSxbMiwyLCJcXE9tZWdhIl0sWzAsMCwiUGIgXFx0aW1lcyBhIl0sWzAsMSwiXFxpbl97YX0iLDJdLFsyLDEsIlxcaW5fe2J9KFxcaWQgXFx0aW1lcyBoKSJdLFsyLDAsIlBoIFxcdGltZXMgXFxpZCIsMix7InN0eWxlIjp7ImJvZHkiOnsibmFtZSI6ImRhc2hlZCJ9fX1dXQ==
  \[\begin{tikzcd}
      {Pb \times a} \\
      \\
      {Pa \times a} && \Omega
      \arrow["{\in_{a}}"', from=3-1, to=3-3]
      \arrow["{\in_{b}(\id \times h)}", from=1-1, to=3-3]
      \arrow["{Ph \times \id}"', dashed, from=1-1, to=3-1]
    \end{tikzcd}\]
\end{rmk}

\begin{rmk}
  Indeed, the power object functor $P : \cE\op \to \cE$ is monadic.
  As monadic functors create limits and $\cE$ has finite limits by definition, $\cE\op$ has finite limits.
  Thus, $\cE$ is finitely cocomplete.
\end{rmk}

\begin{eg}
  When $\cE = \CSet$, the functor $P : \CSet\op \to \CSet$ is just the usual (contravariant) powerset functor.
\end{eg}

\begin{rmk}
  Given a topos $\cE$, we can view a subobject of $a$ as
  \begin{enumerate}
  \item An equivalence class of monos: $a' \mono a$.
  \item A characteristic morphism: $a \to \Omega$.
  \item A global element of a power object: $1 \to Pa$.
  \end{enumerate}
  This is due to the following isomorphisms.
  \[
    \Sub_{\cE}(a) \iso \cE(a, \Omega) \iso \cE(1, Pa)
  \]
\end{rmk}

\begin{defn}
  Let $\cE$ be a topos.
  Let $r : s \mono a \times b$ be a subobject with the corresponding characteristic morphism $\varphi : a \times b \to \Omega$ and global element $[\varphi] : b \to Pa$.
  We say that $s$ is an \emph{extension} of the predicate $\varphi$ and the element $[\varphi]$ \emph{names} the predicate $\varphi$.
\end{defn}

\begin{eg}
  In $\CSet$, a relation $R \subseteq X \times Y$ can be equally viewed as a truth functional (predicate) $\varphi : X \times Y \to 2$.
  The name of this predicate is a function $[\varphi] : Y \to PX$ that maps each $y \in Y$ to a subset of $X$ for which $R(x, y)$ holds for all $x$ in that subset.
\end{eg}

\begin{eg}\label{eg:singleton-predicate}
  The diagonal map $\Delta_{a} : a \to a \times a$ is always a monomorphism.
  Thus, in a topos, $\Delta_{a}$ is classified by the equality predicate $\delta_{a} : a \times a \to \Omega$, which is named by the global element $[\delta_{a}] : a \to Pa$.
  As it turns out, the name $[\delta_{a}]$ is a monomorphism itself.
  Thus, it is classified by a predicate $\sigma_{a} : Pa \to \Omega$, called the \emph{singleton} predicate.
\end{eg}

\begin{thm}\label{thm:topos-has-exponentials}
  Every topos has exponentials.
\end{thm}
\begin{proof}
  Let consider how exponentials are constructed in $\CSet$.
  The exponential $B^{A}$ is the homset $\CSet(A, B)$.
  Each function $f : A \to B$ may be presented as a graph $G_{f} \subseteq B \times A$.
  Thus, $B^{A}$ ought to be a subset of the powerset $P(B \times A)$.
  Given a subset $G \subseteq B \times A$ and $a \in A$, we define a subset of $B$ as follows.
  \[
    u(G, a) = \Set{b \mid (b, a) \in G}
  \]
  This corresponds to the name $[\in_{B \times A}] : P(B \times A) \times A \to PB$.
  $G$ is a graph whenever $u(G,a)$ is a singleton set for all $a \in A$.
  The predicate ``$u(G,a)$ is a singleton set'' is the morphism $\sigma_{B}[\in_{B \times A}] : P(B \times A) \times A \to \Omega$, whose name is $[\sigma_{B}[\in_{B \times A}]] : P(B \times A) \to PA$ (c.f., \cref{eg:singleton-predicate}).
  Now, we can construct $B^{A}$ as the following pullback.
  % https://q.uiver.app/#q=WzAsNCxbMCwyLCJQKEIgXFx0aW1lcyBBKSJdLFswLDAsIkJee0F9Il0sWzIsMiwiUEEiXSxbMiwwLCIxIl0sWzEsMCwiIiwxLHsic3R5bGUiOnsidGFpbCI6eyJuYW1lIjoibW9ubyJ9fX1dLFswLDIsIltcXHNpZ21hX3tCfVtcXGluX3tCIFxcdGltZXMgQX1dXSIsMl0sWzMsMiwiW3Rfe0F9XSIsMCx7InN0eWxlIjp7InRhaWwiOnsibmFtZSI6Im1vbm8ifX19XSxbMSwzXSxbMSwyLCIiLDEseyJzdHlsZSI6eyJuYW1lIjoiY29ybmVyIn19XV0=
  \[\begin{tikzcd}
      {B^{A}} && 1 \\
      \\
      {P(B \times A)} && PA
      \arrow[tail, from=1-1, to=3-1]
      \arrow["{[\sigma_{B}[\in_{B \times A}]]}"', from=3-1, to=3-3]
      \arrow["{[t_{A}]}", tail, from=1-3, to=3-3]
      \arrow[from=1-1, to=1-3]
      \arrow["\lrcorner"{anchor=center, pos=0.125}, draw=none, from=1-1, to=3-3]
    \end{tikzcd}\]
  Note that this construction can be done in any topos.
  \todo{Verify that $B^{A}$ is an exponential.}
\end{proof}

\begin{thm}
  In a topos, every monomorphism is an equalizer and every mono and epi morphism is an isomorphism.
\end{thm}
\begin{proof}
  Let $t_{b} := t \circ !_{b}$.
  The existence of a subobject classifier immediately implies that $m$ is an equalizer of $t_{b}$ and $\chi_{m}$.
  The second half of the theorem follows immediately.
\end{proof}

\begin{cor}
  Every topos is balanced.
\end{cor}

\begin{defn}
  Let $\cE$ and $\cF$ be topoi.
  A \emph{logical functor} $F : \cE \to \cF$ is a functor that preserves finite limits, the subobject classifier, and exponentials.
\end{defn}

\section{Grothendieck topologies}
\label{sec:grothendieck-topologies}

\begin{defn}
  A \emph{Grothendieck topology} on a category $\iC$ is a function $J$ assigning to each object $c$ a collection $J(c)$ of covering sieves on $c$ so that
  \begin{enumerate}
  \item For any object $c$, the maximal sieve is in $J(c)$.
  \item (Stability) If $S$ is a sieve that covers $c$ and $g : d \to c$ is any morphism, then the pullback sieve $g^{*}S$ covers $d$.
  \item (Local character) If $R$ is a covering sieve on $c$ and $S$ is another sieve on $c$ such that for all $f : d \to c \in R$, the pullback sieve $f^{*}S$ covers $d$, then $S$ covers $c$.
  \end{enumerate}
\end{defn}

\begin{rmk}
  Stability means the assignment $J$ extends to a functor $J : \iC\op \to \CSet$, i.e., an object in the presheaf category $\Psh(\iC)$.
  The subobject classifier $\Omega$ in $\Psh(\iC)$ is the functor $\Sub(\yon\blank)$ sending each $c \in \iC$ to the set of sieves on $c$.
  Thus, $J$ (a functor sending each $c$ to a set of covering sieves on $c$) is a subobject of $\Omega$.
\end{rmk}

\section{Lawvere-Tierney topologies}
\label{sec:lawvere-tierney-topologies}

\begin{defn}\label{def:lawvere-tierney-topology}
  A \emph{Lawvere-Tierney topology} (topology for short) on a topos $\cE$ is a morphism $j : \Omega \to \Omega$ satisfying
  % https://q.uiver.app/#q=WzAsMTAsWzAsMCwiMSJdLFsxLDAsIlxcT21lZ2EiXSxbMSwxLCJcXE9tZWdhIl0sWzIsMCwiXFxPbWVnYSJdLFszLDAsIlxcT21lZ2EiXSxbMywxLCJcXE9tZWdhIl0sWzQsMCwiXFxPbWVnYSBcXHRpbWVzIFxcT21lZ2EiXSxbNSwwLCJcXE9tZWdhIl0sWzUsMSwiXFxPbWVnYSJdLFs0LDEsIlxcT21lZ2EgXFx0aW1lcyBcXE9tZWdhIl0sWzAsMSwidCJdLFsxLDIsImoiXSxbMCwyLCJ0IiwyXSxbMyw0LCJqIl0sWzQsNSwiaiJdLFszLDUsImoiLDJdLFs2LDcsIlxcd2VkZ2UiXSxbOSw4LCJcXHdlZGdlIiwyXSxbNiw5LCJqIiwyXSxbNyw4LCJqIl1d
  \[\begin{tikzcd}
      1 & \Omega & \Omega & \Omega & {\Omega \times \Omega} & \Omega \\
      & \Omega && \Omega & {\Omega \times \Omega} & \Omega
      \arrow["t", from=1-1, to=1-2]
      \arrow["j", from=1-2, to=2-2]
      \arrow["t"', from=1-1, to=2-2]
      \arrow["j", from=1-3, to=1-4]
      \arrow["j", from=1-4, to=2-4]
      \arrow["j"', from=1-3, to=2-4]
      \arrow["\wedge", from=1-5, to=1-6]
      \arrow["\wedge"', from=2-5, to=2-6]
      \arrow["j \times j"', from=1-5, to=2-5]
      \arrow["j", from=1-6, to=2-6]
    \end{tikzcd}\]
\end{defn}

\begin{rmk}
  The morphism $j : \Omega \to \Omega$ classifies a subobject of $\Omega$.
  % https://q.uiver.app/#q=WzAsNCxbMCwwLCJKIl0sWzIsMCwiMSJdLFsyLDIsIlxcT21lZ2EiXSxbMCwyLCJcXE9tZWdhIl0sWzMsMiwiaiIsMl0sWzEsMiwidCIsMCx7InN0eWxlIjp7InRhaWwiOnsibmFtZSI6Im1vbm8ifX19XSxbMCwxXSxbMCwzLCIiLDIseyJzdHlsZSI6eyJ0YWlsIjp7Im5hbWUiOiJtb25vIn19fV0sWzAsMiwiIiwxLHsic3R5bGUiOnsibmFtZSI6ImNvcm5lciJ9fV1d
  \[\begin{tikzcd}
      J && 1 \\
      \\
      \Omega && \Omega
      \arrow["j"', from=3-1, to=3-3]
      \arrow["t", tail, from=1-3, to=3-3]
      \arrow[from=1-1, to=1-3]
      \arrow[tail, from=1-1, to=3-1]
      \arrow["\lrcorner"{anchor=center, pos=0.125}, draw=none, from=1-1, to=3-3]
    \end{tikzcd}\]
  Thus, we can phrase \cref{def:lawvere-tierney-topology} in terms of this subobject.
\end{rmk}

\begin{defn}
  Let $\cE$ be a topos equipped with a topology $j$.
  Let $A$ be a subobject of $B$ classified by $\chi_{A}$.
  The composite $j \circ \chi_{A}$ determines another subobject $\Abar$ of $B$, which we call the \emph{closure} of $A$.
\end{defn}

\begin{thm}
  Let $\cE$ be a topos, a morphism $j : \Omega \to \Omega$ induces an operator $E \mapsto \Ebar$ on the subobjects of each object $E$ of $\cE$ natural in its argument in the sense that for any $f : A \to B$ in $\cE$ and any subobject $C$ of $B$, one has
  \[
    f^{*}(\Cbar) = \overline{f^{*}C}
  \]
  Moreover, $j$ is a topology if and only if the operator satisfies
  \begin{mathpar}
    A \subseteq \Abar \and \overline{\Abar} = \Abar \and \overline{A \cap B} = \Abar \cap \Bbar
  \end{mathpar}
  for all objects.
\end{thm}
\begin{proof}
  By definition, $f^{*}(\Cbar)$ is the pullback of $\Cbar$ along $f$.
  Thus, $f^{*}(\Cbar)$ is a subobject classified by $j \circ \chi_{C} \circ f$, which by definition classifies the subobject $\overline{f^{*}C}$.
  % https://q.uiver.app/#q=WzAsNixbMiwwLCJcXENiYXIiXSxbMiwyLCJCIl0sWzQsMiwiXFxPbWVnYSJdLFs0LDAsIjEiXSxbMCwyLCJBIl0sWzAsMCwiZl57Kn0oXFxDYmFyKSJdLFswLDEsIiIsMix7InN0eWxlIjp7InRhaWwiOnsibmFtZSI6Im1vbm8ifX19XSxbMSwyLCJqIFxcY2lyYyBcXGNoaV97Q30iLDJdLFszLDIsInQiXSxbMCwzXSxbNCwxLCJmIiwyXSxbNSw0LCIiLDIseyJzdHlsZSI6eyJ0YWlsIjp7Im5hbWUiOiJtb25vIn19fV0sWzUsMF0sWzUsMSwiIiwyLHsic3R5bGUiOnsibmFtZSI6ImNvcm5lciJ9fV0sWzAsMiwiIiwyLHsic3R5bGUiOnsibmFtZSI6ImNvcm5lciJ9fV1d
  \[\begin{tikzcd}
      {f^{*}(\Cbar)} && \Cbar && 1 \\
      \\
      A && B && \Omega
      \arrow[tail, from=1-3, to=3-3]
      \arrow["{j \circ \chi_{C}}"', from=3-3, to=3-5]
      \arrow["t", from=1-5, to=3-5]
      \arrow[from=1-3, to=1-5]
      \arrow["f"', from=3-1, to=3-3]
      \arrow[tail, from=1-1, to=3-1]
      \arrow[from=1-1, to=1-3]
      \arrow["\lrcorner"{anchor=center, pos=0.125}, draw=none, from=1-1, to=3-3]
      \arrow["\lrcorner"{anchor=center, pos=0.125}, draw=none, from=1-3, to=3-5]
    \end{tikzcd}\]

  Now suppose that $j$ is a topology.
  The closure properties are can be proved by the following abstract nonsense.
  \begin{mathpar}
    % https://q.uiver.app/#q=WzAsNixbMSwyLCJCIl0sWzIsMiwiXFxPbWVnYSJdLFszLDIsIlxcT21lZ2EiXSxbMywxLCIxIl0sWzEsMSwiXFxBYmFyIl0sWzAsMCwiQSJdLFswLDEsIlxcY2hpX3tBfSIsMl0sWzEsMiwiaiIsMl0sWzMsMiwidCIsMCx7InN0eWxlIjp7InRhaWwiOnsibmFtZSI6Im1vbm8ifX19XSxbNCwwLCIiLDIseyJzdHlsZSI6eyJ0YWlsIjp7Im5hbWUiOiJtb25vIn19fV0sWzQsM10sWzUsMywiIiwwLHsiY3VydmUiOi0yfV0sWzUsMCwiIiwwLHsiY3VydmUiOjIsInN0eWxlIjp7InRhaWwiOnsibmFtZSI6Im1vbm8ifX19XSxbMywxLCJ0IiwyXSxbNSw0LCIiLDAseyJzdHlsZSI6eyJib2R5Ijp7Im5hbWUiOiJkYXNoZWQifX19XSxbNCwyLCIiLDAseyJzdHlsZSI6eyJuYW1lIjoiY29ybmVyIn19XV0=
    \begin{tikzcd}
      A \\
      & \Abar && 1 \\
      & B & \Omega & \Omega
      \arrow["{\chi_{A}}"', from=3-2, to=3-3]
      \arrow["j"', from=3-3, to=3-4]
      \arrow["t", tail, from=2-4, to=3-4]
      \arrow[tail, from=2-2, to=3-2]
      \arrow[from=2-2, to=2-4]
      \arrow[curve={height=-12pt}, from=1-1, to=2-4]
      \arrow[curve={height=12pt}, tail, from=1-1, to=3-2]
      \arrow["t"', from=2-4, to=3-3]
      \arrow[dashed, from=1-1, to=2-2]
      \arrow["\lrcorner"{anchor=center, pos=0.125}, draw=none, from=2-2, to=3-4]
    \end{tikzcd}\and
    % https://q.uiver.app/#q=WzAsNyxbMSwxLCJcXEFiYXIiXSxbMSwyLCJCIl0sWzIsMiwiXFxPbWVnYSJdLFszLDIsIlxcT21lZ2EiXSxbMywxLCIxIl0sWzQsMiwiXFxPbWVnYSJdLFswLDAsIlxcb3ZlcmxpbmV7XFxBYmFyfSJdLFswLDEsIiIsMCx7InN0eWxlIjp7InRhaWwiOnsibmFtZSI6Im1vbm8ifX19XSxbMSwyLCJcXGNoaV97QX0iLDJdLFsyLDMsImoiLDJdLFs0LDMsInQiLDAseyJzdHlsZSI6eyJ0YWlsIjp7Im5hbWUiOiJtb25vIn19fV0sWzAsNF0sWzMsNSwiaiIsMl0sWzIsNSwiaiIsMix7ImN1cnZlIjozfV0sWzYsMSwiIiwyLHsiY3VydmUiOjIsInN0eWxlIjp7InRhaWwiOnsibmFtZSI6Im1vbm8ifX19XSxbNCw1LCJ0IiwwLHsic3R5bGUiOnsidGFpbCI6eyJuYW1lIjoibW9ubyJ9fX1dLFs2LDQsIiIsMCx7ImN1cnZlIjotMn1dLFs2LDAsIlxcaXNvIiwzLHsic3R5bGUiOnsiYm9keSI6eyJuYW1lIjoibm9uZSJ9LCJoZWFkIjp7Im5hbWUiOiJub25lIn19fV1d
    \begin{tikzcd}
      {\overline{\Abar}} \\
      & \Abar && 1 \\
      & B & \Omega & \Omega & \Omega
      \arrow[tail, from=2-2, to=3-2]
      \arrow["{\chi_{A}}"', from=3-2, to=3-3]
      \arrow["j"', from=3-3, to=3-4]
      \arrow["t", tail, from=2-4, to=3-4]
      \arrow[from=2-2, to=2-4]
      \arrow["j"', from=3-4, to=3-5]
      \arrow["j"', curve={height=18pt}, from=3-3, to=3-5]
      \arrow[curve={height=12pt}, tail, from=1-1, to=3-2]
      \arrow["t", tail, from=2-4, to=3-5]
      \arrow[curve={height=-12pt}, from=1-1, to=2-4]
      \arrow["\iso"{marking, allow upside down}, draw=none, from=1-1, to=2-2]
    \end{tikzcd}\and
    % https://q.uiver.app/#q=WzAsNixbMSwxLCJcXG92ZXJsaW5le0EgXFxjYXAgQn0iXSxbMSwyLCJDIl0sWzMsMiwiXFxPbWVnYSJdLFszLDEsIjEiXSxbMCwwLCJcXEFiYXIgXFxjYXAgXFxCYmFyIl0sWzAsMiwiQyJdLFswLDEsIiIsMix7InN0eWxlIjp7InRhaWwiOnsibmFtZSI6Im1vbm8ifX19XSxbMCwzXSxbMywyLCJ0IiwwLHsic3R5bGUiOnsidGFpbCI6eyJuYW1lIjoibW9ubyJ9fX1dLFsxLDIsImpcXHdlZGdlKFxcY2hpX3tBfSxcXGNoaV97Qn0pIl0sWzQsNSwiIiwyLHsic3R5bGUiOnsidGFpbCI6eyJuYW1lIjoibW9ubyJ9fX1dLFs0LDMsIiIsMix7ImN1cnZlIjotMX1dLFs1LDIsIlxcd2VkZ2UoalxcY2hpX3tBfSxqXFxjaGlfe0J9KSIsMix7ImN1cnZlIjoyfV0sWzQsMCwiXFxpc28iLDMseyJzdHlsZSI6eyJib2R5Ijp7Im5hbWUiOiJub25lIn0sImhlYWQiOnsibmFtZSI6Im5vbmUifX19XSxbMSw1LCIiLDMseyJsZXZlbCI6Miwic3R5bGUiOnsiaGVhZCI6eyJuYW1lIjoibm9uZSJ9fX1dXQ==
    \begin{tikzcd}
      {\Abar \cap \Bbar} \\
      & {\overline{A \cap B}} && 1 \\
      C & C && \Omega
      \arrow[tail, from=2-2, to=3-2]
      \arrow[from=2-2, to=2-4]
      \arrow["t", tail, from=2-4, to=3-4]
      \arrow["{j\wedge(\chi_{A},\chi_{B})}", from=3-2, to=3-4]
      \arrow[tail, from=1-1, to=3-1]
      \arrow[curve={height=-6pt}, from=1-1, to=2-4]
      \arrow["{\wedge(j\chi_{A},j\chi_{B})}"', curve={height=12pt}, from=3-1, to=3-4]
      \arrow["\iso"{marking, allow upside down}, draw=none, from=1-1, to=2-2]
      \arrow[Rightarrow, no head, from=3-2, to=3-1]
    \end{tikzcd}
  \end{mathpar}
  Conversely, if $j$ induces a closure operation with the above properties, then the diagrams above show that $j$ must be a topology.
\end{proof}

\begin{defn}
  Let $\cE$ be a topos equipped with a topology.
  We say that a subobject $A$ of $B$ is \emph{dense} in $B$ if $\Abar = B$ and that it is \emph{closed} when $\Abar = A$.
\end{defn}

\begin{defn}
  Let $\cE$ be a topos equipped with a topology $j$.
  A \emph{$j$-sheaf} is an object $F$ so that for every dense monomorphism $m : A \mono B$, 
\end{defn}

\section{Locales}
\label{sec:locales}

\begin{defn}
  A \emph{frame} is a lattice with arbitrary joins and finite meets satisfying the infinitary distributive law.
  A \emph{frame homomorphism} is an order-preserving function that preserves infinite joins and finite meets.
\end{defn}

\begin{defn}
  The category of locales $\CLoc$ is the opposite of the category of frames $\CFrm$.
\end{defn}

\bibliographystyle{alpha}
\bibliography{all}

\end{document}
