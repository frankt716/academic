\documentclass[article,10pt,oneside]{memoir}
\usepackage{amssymb,amsmath,stmaryrd,mathrsfs}
\usepackage{quiver}
\usepackage[T1]{fontenc}
\usepackage{amsthm}
\usepackage{tikz,tikz-cd}
\usepackage{enumitem}
\usepackage{xcolor}
\definecolor{darkgreen}{rgb}{0,0.45,0} 
\usepackage[pagebackref,colorlinks,citecolor=darkgreen,linkcolor=darkgreen]{hyperref}
\usepackage{mathtools}
\usepackage{ifmtarg}
\usepackage{braket}
\let\setof\Set
\usepackage{url}
\usepackage{xspace}
\usepackage{cleveref,aliascnt}
\usepackage[status=draft,author=]{fixme}
\fxusetheme{color}
\usepackage{mathpartir}
\title{categorical logic}
\author{Frank Tsai}
\date{\today}
%\thanks{}
\begin{document}
\maketitle
\tableofcontents

\section{Warning}
\label{sec:warning}

Everything written here is based on my current (limited) knowledge of categorical logic.
Some content written herein is purely my own speculation and is by no means authoritative.
This document is expected to contain a lot of errors.

\section{Introduction}
\label{sec:introduction}

The categorification of algebraic theories allow us to study them in a presentation-independent way.
Algebraic theories are precisely categories with finite products.
We think of the objects of such a category as sorts and morphisms as terms.
Identities and composition correspond to variables and substitution, respectively.
Equations are internalized as the diagonal subobject.
Given two algebraic theories $\cT$ and $\cT'$, a model of $\cT$ in $\cT'$ is a finite-product preserving functor $M : \cT \to \cT'$.
This ensures that products of sorts in $\cT$ are interpreted (sensibly) as products of sorts in $\cT'$.
Moreover, this ensures that equations are preserved.
Then, a homomorphism from $M$ to $M'$ is precisely a natural transformation $\alpha : M \To M'$.
% https://q.uiver.app/#q=WzAsNCxbMiwwLCJNcyJdLFswLDIsIk1zIl0sWzIsMiwiTShzIFxcdGltZXMgcykiXSxbNCwyLCJNKHMpIl0sWzIsMSwiXFxwaV97MX0iXSxbMiwzLCJcXHBpX3syfSIsMl0sWzAsMSwiXFxpZCIsMl0sWzAsMywiXFxpZCJdLFswLDIsIk1cXERlbHRhIiwxLHsic3R5bGUiOnsidGFpbCI6eyJuYW1lIjoibW9ubyJ9LCJib2R5Ijp7Im5hbWUiOiJkYXNoZWQifX19XV0=
\[\begin{tikzcd}
    && Ms \\
    \\
    Ms && {M(s \times s)} && {Ms}
    \arrow["{\pi_{1}}", from=3-3, to=3-1]
    \arrow["{\pi_{2}}"', from=3-3, to=3-5]
    \arrow["\id"', from=1-3, to=3-1]
    \arrow["\id", from=1-3, to=3-5]
    \arrow["M\Delta"{description}, dashed, tail, from=1-3, to=3-3]
  \end{tikzcd}\]

Category with finite products, finite-product preserving functors, and natural transformations form the doctrine of algebraic theories $\CAlg$.
The slice $\CAlg/\CSet$ recovers the classical set-theoretic models.
Explicitly, this slice has set-theoretic models, compatible theory morphisms, and natural transformations between them.
In fact, for any algebraic theory $\cT$, the hom-category $\CAlg(\cT,\CSet)$, i.e., the category of set-theoretic models, is equivalent to a cocomplete category with a strong generator comprised of perfectly presentable objects.

We can generalize this viewpoint: theories can be internalized in a sufficiently structured category.
Then a model of a theory in another theory is a functor preserving those structures.
The goal of this writeup is to develop this idea in more detail.

\section{Propositional logic}
\label{sec:propositional-logic}



\bibliographystyle{alpha}
\bibliography{all}

\end{document}
