\documentclass[article,10pt,oneside]{memoir}
\usepackage{amsthm,amssymb,amsmath,stmaryrd,mathrsfs}
\usepackage[T1]{fontenc}
\usepackage{xcolor}
\definecolor{darkgreen}{rgb}{0,0.45,0} 
\usepackage[pagebackref,colorlinks,citecolor=darkgreen,linkcolor=darkgreen]{hyperref}
\usepackage{zi4}
\usepackage[capitalise]{cleveref}
\usepackage{quiver}
\usepackage{tikz,tikz-cd}
\usepackage{enumitem}
\usepackage{mathtools}
\usepackage{ifmtarg}
\usepackage{braket}
\let\setof\Set
\usepackage{url}
\usepackage{xspace}
\usepackage{mathpartir}
\usepackage{xparse}

\newcommand{\todo}[1]{{\color{red}\textbf{TODO: }#1}}

\counterwithout{section}{chapter}
\setsecnumdepth{subsection}
\setsecnumformat{\csname the#1\endcsname.\ \ }

\setsecheadstyle{\bfseries}
\setsubsecheadstyle{\bfseries}
\setsubsubsecheadstyle{\bfseries}
\setparaheadstyle{\bfseries}
\setsubparaheadstyle{\bfseries}

\pretitle{\begin{center}\LARGE\bfseries\MakeTextUppercase}
\posttitle{\par\end{center}\vskip 0.5em}

\linespread{1}

\ExplSyntaxOn

\AtEndPreamble{
  \setlrmarginsandblock{2cm}{*}{1}
  \setulmarginsandblock{2cm}{*}{1}
  \setheaderspaces{*}{\onelineskip}{*}
  \checkandfixthelayout
}

\NewDocumentCommand\ega_thesection{}{\thesection.~}
\NewDocumentCommand\ega_thesubsection{}{\thesubsection.~}
\NewDocumentCommand\ega_thesubsubsection{}{\thesubsubsection.~}


\setsecnumdepth{subsubsection}
\setsecnumformat{\csname ega_the#1\endcsname}
\setsecheadstyle{\normalsize\bfseries\MakeUppercase}
\setsubsecheadstyle{\noindent\normalfont\bfseries}


\appto\mainmatter{
  \setcounter{secnumdepth}{30}
}
\ExplSyntaxOff

%% theorem environments
\newtheorem{thm}{Theorem}[section]
\newtheorem{lem}[thm]{Lemma}
\newtheorem{cor}[thm]{Corollary}
\newtheorem{prop}[thm]{Proposition}

\theoremstyle{definition}
\newtheorem{notn}[thm]{Notation}
\newtheorem{defn}[thm]{Definition}
\newtheorem{rmk}[thm]{Remark}
\newtheorem{eg}[thm]{Example}

% magic
\makeatletter
\let\ea\expandafter

%% Defining commands that are always in math mode.
\def\mdef#1#2{\ea\ea\ea\gdef\ea\ea\noexpand#1\ea{\ea\ensuremath\ea{#2}\xspace}}
\def\alwaysmath#1{\ea\ea\ea\global\ea\ea\ea\let\ea\ea\csname your@#1\endcsname\csname #1\endcsname
  \ea\def\csname #1\endcsname{\ensuremath{\csname your@#1\endcsname}\xspace}}

%% SIMPLE COMMANDS FOR FONTS AND DECORATIONS

\newcount\foreachcount

\def\foreachletter#1#2#3{\foreachcount=#1
  \ea\loop\ea\ea\ea#3\@alph\foreachcount
  \advance\foreachcount by 1
  \ifnum\foreachcount<#2\repeat}

\def\foreachLetter#1#2#3{\foreachcount=#1
  \ea\loop\ea\ea\ea#3\@Alph\foreachcount
  \advance\foreachcount by 1
  \ifnum\foreachcount<#2\repeat}

% Script: \sA is \mathscr{A}
\def\definescr#1{\ea\gdef\csname s#1\endcsname{\ensuremath{\mathscr{#1}}\xspace}}
\foreachLetter{1}{27}{\definescr}
% Calligraphic: \cA is \mathcal{A}
\def\definecal#1{\ea\gdef\csname c#1\endcsname{\ensuremath{\mathcal{#1}}\xspace}}
\foreachLetter{1}{27}{\definecal}
% Bold: \bA is \mathbf{A}
\def\definebold#1{\ea\gdef\csname b#1\endcsname{\ensuremath{\mathbf{#1}}\xspace}}
\foreachLetter{1}{27}{\definebold}
% Blackboard Bold: \dA is \mathbb{A}
\def\definebb#1{\ea\gdef\csname d#1\endcsname{\ensuremath{\mathbb{#1}}\xspace}}
\foreachLetter{1}{27}{\definebb}
% Fraktur: \fa is \mathfrak{a}, except for \fi; \fA is \mathfrak{A}
\def\definefrak#1{\ea\gdef\csname f#1\endcsname{\ensuremath{\mathfrak{#1}}\xspace}}
\foreachletter{1}{9}{\definefrak}
\foreachletter{10}{27}{\definefrak}
\foreachLetter{1}{27}{\definefrak}
% Sans serif: \ia is \mathsf{a}, except for \if and \in and \it
\def\definesf#1{\ea\gdef\csname i#1\endcsname{\ensuremath{\mathsf{#1}}\xspace}}
\foreachletter{1}{6}{\definesf}
\foreachletter{7}{14}{\definesf}
\foreachletter{15}{20}{\definesf}
\foreachletter{21}{27}{\definesf}
\foreachLetter{1}{27}{\definesf}
% Bar: \Abar is \overline{A}, \abar is \overline{a}
\def\definebar#1{\ea\gdef\csname #1bar\endcsname{\ensuremath{\overline{#1}}\xspace}}
\foreachLetter{1}{27}{\definebar}
\foreachletter{1}{8}{\definebar} % \hbar is something else!
\foreachletter{9}{15}{\definebar} % \obar is something else!
\foreachletter{16}{27}{\definebar}
% Tilde: \Atil is \widetilde{A}, \atil is \widetilde{a}
\def\definetil#1{\ea\gdef\csname #1til\endcsname{\ensuremath{\widetilde{#1}}\xspace}}
\foreachLetter{1}{27}{\definetil}
\foreachletter{1}{27}{\definetil}
% Hats: \Ahat is \widehat{A}, \ahat is \widehat{a}
\def\definehat#1{\ea\gdef\csname #1hat\endcsname{\ensuremath{\widehat{#1}}\xspace}}
\foreachLetter{1}{27}{\definehat}
\foreachletter{1}{27}{\definehat}
% Checks: \Achk is \widecheck{A}, \achk is \widecheck{a}
\def\definechk#1{\ea\gdef\csname #1chk\endcsname{\ensuremath{\widecheck{#1}}\xspace}}
\foreachLetter{1}{27}{\definechk}
\foreachletter{1}{27}{\definechk}
% Underline: \uA is \underline{A}, \ua is \underline{a}
\def\defineul#1{\ea\gdef\csname u#1\endcsname{\ensuremath{\underline{#1}}\xspace}}
\foreachLetter{1}{27}{\defineul}
\foreachletter{1}{27}{\defineul}

% Particular commands for typefaces, sometimes with the first letter
% different.
\def\autofmt@n#1\autofmt@end{\mathrm{#1}}
\def\autofmt@b#1\autofmt@end{\mathbf{#1}}
\def\autofmt@d#1#2\autofmt@end{\mathbb{#1}\mathsf{#2}}
\def\autofmt@c#1#2\autofmt@end{\mathcal{#1}\mathit{#2}}
\def\autofmt@s#1#2\autofmt@end{\mathscr{#1}\mathit{#2}}
\def\autofmt@i#1\autofmt@end{\mathsf{#1}}
\def\autofmt@f#1\autofmt@end{\mathfrak{#1}}
% Particular commands for decorations.
\def\autofmt@u#1\autofmt@end{\underline{\smash{\mathsf{#1}}}}
\def\autofmt@U#1\autofmt@end{\underline{\underline{\smash{\mathsf{#1}}}}}
\def\autofmt@h#1\autofmt@end{\widehat{#1}}
\def\autofmt@r#1\autofmt@end{\overline{#1}}
\def\autofmt@t#1\autofmt@end{\widetilde{#1}}
\def\autofmt@k#1\autofmt@end{\check{#1}}

% Defining multi-letter commands.  Use this like so:
% \autodefs{\bSet\cCat\cCAT\kBicat\lProf}
\def\auto@drop#1{}
\def\autodef#1{\ea\ea\ea\@autodef\ea\ea\ea#1\ea\auto@drop\string#1\autodef@end}
\def\@autodef#1#2#3\autodef@end{%
  \ea\def\ea#1\ea{\ea\ensuremath\ea{\csname autofmt@#2\endcsname#3\autofmt@end}\xspace}}
\def\autodefs@end{blarg!}
\def\autodefs#1{\@autodefs#1\autodefs@end}
\def\@autodefs#1{\ifx#1\autodefs@end%
  \def\autodefs@next{}%
  \else%
  \def\autodefs@next{\autodef#1\@autodefs}%
  \fi\autodefs@next}

%% FONTS AND DECORATION FOR GREEK LETTERS

%% the package `mathbbol' gives us blackboard bold greek and numbers,
%% but it does it by redefining \mathbb to use a different font, so that
%% all the other \mathbb letters look different too.  Here we import the
%% font with bb greek and numbers, but assign it a different name,
%% \mathbbb, so as not to replace the usual one.
\DeclareSymbolFont{bbold}{U}{bbold}{m}{n}
\DeclareSymbolFontAlphabet{\mathbbb}{bbold}
\newcommand{\dDelta}{\ensuremath{\mathbbb{\Delta}}\xspace}
\newcommand{\done}{\ensuremath{\mathbbb{1}}\xspace}
\newcommand{\dtwo}{\ensuremath{\mathbbb{2}}\xspace}
\newcommand{\dthree}{\ensuremath{\mathbbb{3}}\xspace}

% greek with bars
\newcommand{\albar}{\ensuremath{\overline{\alpha}}\xspace}
\newcommand{\bebar}{\ensuremath{\overline{\beta}}\xspace}
\newcommand{\gmbar}{\ensuremath{\overline{\gamma}}\xspace}
\newcommand{\debar}{\ensuremath{\overline{\delta}}\xspace}
\newcommand{\phibar}{\ensuremath{\overline{\varphi}}\xspace}
\newcommand{\psibar}{\ensuremath{\overline{\psi}}\xspace}
\newcommand{\xibar}{\ensuremath{\overline{\xi}}\xspace}
\newcommand{\ombar}{\ensuremath{\overline{\omega}}\xspace}

% greek with tildes
\newcommand{\altil}{\ensuremath{\widetilde{\alpha}}\xspace}
\newcommand{\betil}{\ensuremath{\widetilde{\beta}}\xspace}
\newcommand{\gmtil}{\ensuremath{\widetilde{\gamma}}\xspace}
\newcommand{\phitil}{\ensuremath{\widetilde{\varphi}}\xspace}
\newcommand{\psitil}{\ensuremath{\widetilde{\psi}}\xspace}
\newcommand{\xitil}{\ensuremath{\widetilde{\xi}}\xspace}
\newcommand{\omtil}{\ensuremath{\widetilde{\omega}}\xspace}

% MISCELLANEOUS SYMBOLS
\let\del\partial
\mdef\delbar{\overline{\partial}}
\let\sm\wedge
\newcommand{\dd}[1]{\ensuremath{\frac{\partial}{\partial {#1}}}}
\newcommand{\inv}{^{-1}}
\newcommand{\dual}{^{\vee}}
\mdef\hf{\textstyle\frac12 }
\mdef\thrd{\textstyle\frac13 }
\mdef\qtr{\textstyle\frac14 }
\let\meet\wedge
\let\join\vee
\let\dn\downarrow
\newcommand{\op}{^{\mathrm{op}}}
\newcommand{\co}{^{\mathrm{co}}}
\newcommand{\coop}{^{\mathrm{coop}}}
\let\adj\dashv
\let\iso\cong
\let\eqv\simeq
\let\cng\equiv
\mdef\Id{\mathrm{Id}}
\mdef\id{\mathrm{id}}
\alwaysmath{ell}
\alwaysmath{infty}
\let\oo\infty
\def\io{\ensuremath{(\infty,1)}}
\alwaysmath{odot}
\def\frc#1/#2.{\frac{#1}{#2}}   % \frc x^2+1 / x^2-1 .
\mdef\ten{\mathrel{\otimes}}
\let\bigten\bigotimes
\mdef\sqten{\mathrel{\boxtimes}}
\def\lt{<}                      % For iTex compatibility
\def\gt{>}

%%% Blanks (shorthand for lambda abstractions)
\newcommand{\blank}{\mathord{\hspace{1pt}\text{--}\hspace{1pt}}}
%%% Nameless objects
\newcommand{\nameless}{\mathord{\hspace{1pt}\underline{\hspace{1ex}}\hspace{1pt}}}

% Hiragana "yo" for the Yoneda embedding, from https://arxiv.org/abs/1506.08870
\DeclareFontFamily{U}{min}{}
\DeclareFontShape{U}{min}{m}{n}{<-> udmj30}{}
\newcommand{\yon}{\!\text{\usefont{U}{min}{m}{n}\symbol{'210}}\!}

%% Get some new symbols from mathabx, without changing the old ones by
%% importing the package.  Font declarations copied from mathabx.sty:
\DeclareFontFamily{U}{mathb}{\hyphenchar\font45}
\DeclareFontShape{U}{mathb}{m}{n}{
      <5> <6> <7> <8> <9> <10> gen * mathb
      <10.95> mathb10 <12> <14.4> <17.28> <20.74> <24.88> mathb12
      }{}
\DeclareSymbolFont{mathb}{U}{mathb}{m}{n}
\DeclareFontSubstitution{U}{mathb}{m}{n}
\DeclareFontFamily{U}{mathx}{\hyphenchar\font45}
\DeclareFontShape{U}{mathx}{m}{n}{
      <5> <6> <7> <8> <9> <10>
      <10.95> <12> <14.4> <17.28> <20.74> <24.88>
      mathx10
      }{}
\DeclareSymbolFont{mathx}{U}{mathx}{m}{n}
\DeclareFontSubstitution{U}{mathx}{m}{n}
%% And now the symbols we want, copied from mathabx.dcl
\DeclareMathSymbol{\dotplus}       {2}{mathb}{"00}% name to be checked
\DeclareMathSymbol{\dotdiv}        {2}{mathb}{"01}% name to be checked
\DeclareMathSymbol{\dottimes}      {2}{mathb}{"02}% name to be checked
\DeclareMathSymbol{\divdot}        {2}{mathb}{"03}% name to be checked
\DeclareMathSymbol{\udot}          {2}{mathb}{"04}% name to be checked
\DeclareMathSymbol{\square}        {2}{mathb}{"05}% name to be checked
\DeclareMathSymbol{\Asterisk}      {2}{mathb}{"06}
\DeclareMathSymbol{\bigast}        {1}{mathb}{"06}
\DeclareMathSymbol{\coAsterisk}    {2}{mathb}{"07}
\DeclareMathSymbol{\bigcoast}      {1}{mathb}{"07}
\DeclareMathSymbol{\circplus}      {2}{mathb}{"08}% name to be checked
\DeclareMathSymbol{\pluscirc}      {2}{mathb}{"09}% name to be checked
\DeclareMathSymbol{\convolution}   {2}{mathb}{"0A}% name to be checked
\DeclareMathSymbol{\divideontimes} {2}{mathb}{"0B}% name to be checked
\DeclareMathSymbol{\blackdiamond}  {2}{mathb}{"0C}% name to be checked
\DeclareMathSymbol{\sqbullet}      {2}{mathb}{"0D}% name to be checked
\DeclareMathSymbol{\bigstar}       {2}{mathb}{"0E}
\DeclareMathSymbol{\bigvarstar}    {2}{mathb}{"0F}
\DeclareMathSymbol{\corresponds}   {3}{mathb}{"1D}% name to be checked
\DeclareMathSymbol{\boxleft}      {2}{mathb}{"68}
\DeclareMathSymbol{\boxright}     {2}{mathb}{"69}
\DeclareMathSymbol{\boxtop}       {2}{mathb}{"6A}
\DeclareMathSymbol{\boxbot}       {2}{mathb}{"6B}
\DeclareMathSymbol{\updownarrows}          {3}{mathb}{"D6}
\DeclareMathSymbol{\downuparrows}          {3}{mathb}{"D7}
\DeclareMathSymbol{\Lsh}                   {3}{mathb}{"E8}
\DeclareMathSymbol{\Rsh}                   {3}{mathb}{"E9}
\DeclareMathSymbol{\dlsh}                  {3}{mathb}{"EA}
\DeclareMathSymbol{\drsh}                  {3}{mathb}{"EB}
\DeclareMathSymbol{\looparrowdownleft}     {3}{mathb}{"EE}
\DeclareMathSymbol{\looparrowdownright}    {3}{mathb}{"EF}
% \DeclareMathSymbol{\curvearrowleft}        {3}{mathb}{"F0}
% \DeclareMathSymbol{\curvearrowright}       {3}{mathb}{"F1}
\DeclareMathSymbol{\curvearrowleftright}   {3}{mathb}{"F2}
\DeclareMathSymbol{\curvearrowbotleft}     {3}{mathb}{"F3}
\DeclareMathSymbol{\curvearrowbotright}    {3}{mathb}{"F4}
\DeclareMathSymbol{\curvearrowbotleftright}{3}{mathb}{"F5}
% \DeclareMathSymbol{\circlearrowleft}       {3}{mathb}{"F6}
% \DeclareMathSymbol{\circlearrowright}      {3}{mathb}{"F7}
\DeclareMathSymbol{\leftsquigarrow}        {3}{mathb}{"F8}
\DeclareMathSymbol{\rightsquigarrow}       {3}{mathb}{"F9}
\DeclareMathSymbol{\leftrightsquigarrow}   {3}{mathb}{"FA}
\DeclareMathSymbol{\lefttorightarrow}      {3}{mathb}{"FC}
\DeclareMathSymbol{\righttoleftarrow}      {3}{mathb}{"FD}
\DeclareMathSymbol{\uptodownarrow}         {3}{mathb}{"FE}
\DeclareMathSymbol{\downtouparrow}         {3}{mathb}{"FF}
\DeclareMathSymbol{\varhash}       {0}{mathb}{"23}
\DeclareMathSymbol{\sqSubset}       {3}{mathb}{"94}
\DeclareMathSymbol{\sqSupset}       {3}{mathb}{"95}
\DeclareMathSymbol{\nsqSubset}      {3}{mathb}{"96}
\DeclareMathSymbol{\nsqSupset}      {3}{mathb}{"97}
% WIDECHECK
\DeclareMathAccent{\widecheck}    {0}{mathx}{"71}


%% OPERATORS
\DeclareMathOperator\lan{Lan}
\DeclareMathOperator\ran{Ran}
\DeclareMathOperator\colim{colim}
\DeclareMathOperator\coeq{coeq}
\DeclareMathOperator\ob{ob}
\DeclareMathOperator\cod{cod}
\DeclareMathOperator\dom{dom}
\DeclareMathOperator\ev{ev}
\DeclareMathOperator\eq{eq}
\DeclareMathOperator\Tot{Tot}
\DeclareMathOperator\cosk{cosk}
\DeclareMathOperator\sk{sk}
\DeclareMathOperator\img{im}
\DeclareMathOperator\Spec{Spec}
\DeclareMathOperator\Ho{Ho}
\DeclareMathOperator\Aut{Aut}
\DeclareMathOperator\End{End}
\DeclareMathOperator\Hom{Hom}
\DeclareMathOperator\Map{Map}
\DeclareMathOperator\coker{coker}
\DeclareMathOperator\Alg{Alg}
\DeclareMathOperator\Cone{Cone}
\DeclareMathOperator\Cocone{Cocone}
\DeclareMathOperator\Idem{Idem}
\DeclareMathOperator\Cont{Cont}
\DeclareMathOperator\Pres{Pres}
\DeclareMathOperator\Psh{Psh}

%% ARROWS
% \to already exists
\newcommand{\too}[1][]{\ensuremath{\overset{#1}{\longrightarrow}}}
\newcommand{\ot}{\ensuremath{\leftarrow}}
\newcommand{\oot}[1][]{\ensuremath{\overset{#1}{\longleftarrow}}}
\let\toot\rightleftarrows
\let\otto\leftrightarrows
\let\Impl\Rightarrow
\let\imp\Rightarrow
\let\toto\rightrightarrows
\let\into\hookrightarrow
\let\xinto\xhookrightarrow
\mdef\we{\overset{\sim}{\longrightarrow}}
\mdef\leftwe{\overset{\sim}{\longleftarrow}}
\let\mono\rightarrowtail
\let\leftmono\leftarrowtail
\let\cof\rightarrowtail
\let\leftcof\leftarrowtail
\let\epi\twoheadrightarrow
\let\leftepi\twoheadleftarrow
\let\fib\twoheadrightarrow
\let\leftfib\twoheadleftarrow
\let\cohto\rightsquigarrow
\let\maps\colon
\newcommand{\spam}{\,:\!}       % \maps for left arrows
\def\acof{\mathrel{\mathrlap{\hspace{3pt}\raisebox{4pt}{$\scriptscriptstyle\sim$}}\mathord{\rightarrowtail}}}

% diagxy redefines \to, along with \toleft, \two, \epi, and \mon.

%% EXTENSIBLE ARROWS
\let\xto\xrightarrow
\let\xot\xleftarrow
% See Voss' Mathmode.tex for instructions on how to create new
% extensible arrows.
\def\rightarrowtailfill@{\arrowfill@{\Yright\joinrel\relbar}\relbar\rightarrow}
\newcommand\xrightarrowtail[2][]{\ext@arrow 0055{\rightarrowtailfill@}{#1}{#2}}
\let\xmono\xrightarrowtail
\let\xcof\xrightarrowtail
\def\twoheadrightarrowfill@{\arrowfill@{\relbar\joinrel\relbar}\relbar\twoheadrightarrow}
\newcommand\xtwoheadrightarrow[2][]{\ext@arrow 0055{\twoheadrightarrowfill@}{#1}{#2}}
\let\xepi\xtwoheadrightarrow
\let\xfib\xtwoheadrightarrow
% Let's leave the left-going ones until I need them.

%% EXTENSIBLE SLASHED ARROWS
% Making extensible slashed arrows, by modifying the underlying AMS code.
% Arguments are:
% 1 = arrowhead on the left (\relbar or \Relbar if none)
% 2 = fill character (usually \relbar or \Relbar)
% 3 = slash character (such as \mapstochar or \Mapstochar)
% 4 = arrowhead on the left (\relbar or \Relbar if none)
% 5 = display mode (\displaystyle etc)
\def\slashedarrowfill@#1#2#3#4#5{%
  $\m@th\thickmuskip0mu\medmuskip\thickmuskip\thinmuskip\thickmuskip
   \relax#5#1\mkern-7mu%
   \cleaders\hbox{$#5\mkern-2mu#2\mkern-2mu$}\hfill
   \mathclap{#3}\mathclap{#2}%
   \cleaders\hbox{$#5\mkern-2mu#2\mkern-2mu$}\hfill
   \mkern-7mu#4$%
}
% Here's the idea: \<slashed>arrowfill@ should be a box containing
% some stretchable space that is the "middle of the arrow".  This
% space is created as a "leader" using \cleader<thing>\hfill, which
% fills an \hfill of space with copies of <thing>.  Here instead of
% just one \cleader, we use two, with the slash in between (and an
% extra copy of the filler, to avoid extra space around the slash).
\def\rightslashedarrowfill@{%
  \slashedarrowfill@\relbar\relbar\mapstochar\rightarrow}
\newcommand\xslashedrightarrow[2][]{%
  \ext@arrow 0055{\rightslashedarrowfill@}{#1}{#2}}
\mdef\hto{\xslashedrightarrow{}}
\mdef\htoo{\xslashedrightarrow{\quad}}
\let\xhto\xslashedrightarrow

%% To get a slashed arrow in XYmatrix, do
% \[\xymatrix{A \ar[r]|-@{|} & B}\]
%% To get it in diagxy, do
% \morphism/{@{>}|-*@{|}}/[A`B;p]

%% Here is an \hto for diagxy:
% \def\htopppp/#1/<#2>^#3_#4{\:%
% \ifnum#2=0%
%    \setwdth{#3}{#4}\deltax=\wdth \divide \deltax by \ul%
%    \advance \deltax by \defaultmargin  \ratchet{\deltax}{100}%
% \else \deltax #2%
% \fi%
% \xy\ar@{#1}|-@{|}^{#3}_{#4}(\deltax,0) \endxy%
% \:}%
% \def\htoppp/#1/<#2>^#3{\ifnextchar_{\htopppp/#1/<#2>^{#3}}{\htopppp/#1/<#2>^{#3}_{}}}%
% \def\htopp/#1/<#2>{\ifnextchar^{\htoppp/#1/<#2>}{\htoppp/#1/<#2>^{}}}%
% \def\htoop/#1/{\ifnextchar<{\htopp/#1/}{\htopp/#1/<0>}}%
% \def\hto{\ifnextchar/{\htoop}{\htoop/>/}}%

% LABELED ISOMORPHISMS
\def\xiso#1{\mathrel{\mathrlap{\smash{\xto[\smash{\raisebox{1.3mm}{$\scriptstyle\sim$}}]{#1}}}\hphantom{\xto{#1}}}}
\def\toiso{\xto{\smash{\raisebox{-.5mm}{$\scriptstyle\sim$}}}}

% SHADOWS
\def\shvar#1#2{{\ensuremath{%
  \hspace{1mm}\makebox[-1mm]{$#1\langle$}\makebox[0mm]{$#1\langle$}\hspace{1mm}%
  {#2}%
  \makebox[1mm]{$#1\rangle$}\makebox[0mm]{$#1\rangle$}%
}}}
\def\sh{\shvar{}}
\def\scriptsh{\shvar{\scriptstyle}}
\def\bigsh{\shvar{\big}}
\def\Bigsh{\shvar{\Big}}
\def\biggsh{\shvar{\bigg}}
\def\Biggsh{\shvar{\Bigg}}

%% Paul Taylor: noncommutative diagrams
\def\puncture{
  \begingroup
  \setbox0\hbox{A}%
  \vrule height.53\ht0 depth-.47\ht0 width.35\ht0
  \kern .12\ht0
  \vrule height\ht0 depth-.65\ht0 width.06\ht0
  \kern-.06\ht0
  \vrule height.35\ht0 depth0pt width.06\ht0
  \kern .12\ht0
  \vrule height.53\ht0 depth-.47\ht0 width.35\ht0
  \endgroup
}

% TYPING JUDGMENTS
% Call this macro as \jd{x:A, y:B |- c:C}.  It adds (what I think is)
% appropriate spacing, plus auto-sized parentheses around each typing judgment.
\def\jd#1{\@jd#1\ej}
\def\@jd#1|-#2\ej{\@@jd#1,,\;\vdash\;\left(#2\right)}
\def\@@jd#1,{\@ifmtarg{#1}{\let\next=\relax}{\left(#1\right)\let\next=\@@@jd}\next}
\def\@@@jd#1,{\@ifmtarg{#1}{\let\next=\relax}{,\,\left(#1\right)\let\next=\@@@jd}\next}
% Here's a version which puts a line break before the turnstyle.
\def\jdm#1{\@jdm#1\ej}
\def\@jdm#1|-#2\ej{\@@jd#1,,\\\vdash\;\left(#2\right)}
% Make an actual comma that doesn't separate typing judgments (e.g. A,B,C : Type).
\def\cm{,}

% 2-(CO)MONAD STUFF
\def\alg{\text{-}\mathcal{A}\mathit{lg}}
\def\algs{\text{-}\mathcal{A}\mathit{lg}_s}
\def\algl{\text{-}\mathcal{A}\mathit{lg}_l}
\def\algc{\text{-}\mathcal{A}\mathit{lg}_c}
\def\algw{\text{-}\mathcal{A}\mathit{lg}_w}
\def\psalg{\text{-}\mathcal{P}\mathit{s}\mathcal{A}\mathit{lg}}
\def\dalg{\text{-}\mathbb{A}\mathsf{lg}}
\def\coalg{\text{-}\mathcal{C}\mathit{oalg}}
\def\coalgs{\text{-}\mathcal{C}\mathit{oalg}_s}
\def\coalgl{\text{-}\mathcal{C}\mathit{oalg}_l}
\def\coalgc{\text{-}\mathcal{C}\mathit{oalg}_c}
\def\coalgw{\text{-}\mathcal{C}\mathit{oalg}_w}
\def\pscoalg{\text{-}\mathcal{P}\mathit{s}\mathcal{C}\mathit{oalg}}
\def\dcoalg{\text{-}\mathbb{C}\mathsf{oalg}}
\def\Mod{\mathsf{Mod}}
\def\CSet{\mathsf{Set}}
\def\CAlg{\mathsf{ALG}}
\def\CTh{\mathsf{TH}}
\def\CFinSet{\mathsf{FinSet}}
\def\CCat{\mathsf{Cat}}
\def\CLaw{\mathsf{Law}}
\def\CSkt{\mathsf{Skt}}

%% SKIPIT in TikZ
% See http://tex.stackexchange.com/questions/3513/draw-only-some-segments-of-a-path-in-tikz
\long\def\my@drawfill#1#2;{%
\@skipfalse
\fill[#1,draw=none] #2;
\@skiptrue
\draw[#1,fill=none] #2;
}
\newif\if@skip
\newcommand{\skipit}[1]{\if@skip\else#1\fi}
\newcommand{\drawfill}[1][]{\my@drawfill{#1}}

%% TWOCELLS AND PULLBACKS in TIKZ-CD
\newcounter{nodemaker}
\setcounter{nodemaker}{0}
\newcommand{\twocell}[2][]{%
  \global\edef\mynodeone{twocell\arabic{nodemaker}}%
  \stepcounter{nodemaker}%
  \global\edef\mynodetwo{twocell\arabic{nodemaker}}%
  \stepcounter{nodemaker}%
  \ar[#2,phantom,shift left=3,""{name=\mynodeone}]%
  \ar[#2,phantom,shift right=3,""'{name=\mynodetwo}]%
  \ar[Rightarrow,from=\mynodeone,to=\mynodetwo,"{#1}"]%
}
\newcommand{\twocellop}[2][]{%
  \global\edef\mynodeone{twocell\arabic{nodemaker}}%
  \stepcounter{nodemaker}%
  \global\edef\mynodetwo{twocell\arabic{nodemaker}}%
  \stepcounter{nodemaker}%
  \ar[#2,phantom,shift left=3,""{name=\mynodeone}]%
  \ar[#2,phantom,shift right=3,""'{name=\mynodetwo}]%
  \ar[Rightarrow,from=\mynodetwo,to=\mynodeone,"{#1}"]%
}
\newcommand{\drpullback}[1][dr]{\ar[#1,phantom,near start,"\lrcorner"]}
\newcommand{\dlpullback}[1][dl]{\ar[#1,phantom,near start,"\llcorner"]}
\newcommand{\urpullback}[1][ur]{\ar[#1,phantom,near start,"\urcorner"]}
\newcommand{\ulpullback}[1][ul]{\ar[#1,phantom,near start,"\ulcorner"]}


%% Include or exclude solutions
% This code is basically copied from version.sty, except that when the
% solutions are included, we put them in a `proof' environment as
% well.  To include solutions, say \includesolutions; to exclude them
% say \excludesolutions.
% \begingroup
% 
% \catcode`{=12\relax\catcode`}=12\relax%
% \catcode`(=1\relax \catcode`)=2\relax%
% \gdef\includesolutions(\newenvironment(soln)(\begin(proof)[Solution])(\end(proof)))%
% \gdef\excludesolutions(%
%   \gdef\soln(\@bsphack\catcode`{=12\relax\catcode`}=12\relax\soln@NOTE)%
%   \long\gdef\soln@NOTE##1\end{soln}(\solnEND@NOTE)%
%   \gdef\solnEND@NOTE(\@esphack\end(soln))%
% )%
% \endgroup

\makeatother

% Local Variables:
% mode: latex
% TeX-master: ""
% End:
\title{categorical logic}
\author{Frank Tsai}
\date{\today}
%\thanks{}
\begin{document}
\maketitle
\tableofcontents

\section{Introduction}
\label{sec:introduction}

Any category $\iC$ with finite products has enough structure to define internal groups.
In fact, to define the group identity and group multiplication, we need a terminal object and binary products, implying the existence of all finite products.
Below is the usual presentation of the theory of groups.
\begin{mathpar}
  \vdash e : G \and x : G \vdash x\inv : G \and x, y : G \vdash m(x, y) : G \and x : G \vdash m(e, x) = x : G%
  \and x : G \vdash m(x, e) = x : G \and x : G \vdash m(x, x\inv) = e : G \and x : G \vdash m(x\inv, x) = e : G%
  \and x, y, z : G \vdash m(m(x,y),z) = m(x,m(y,z)) : G
\end{mathpar}
The axioms of the theory of groups can be expressed in terms of the following commutative diagrams.
\begin{mathpar}
  % https://q.uiver.app/#q=WzAsNCxbNCwwLCJHIl0sWzIsMCwiRyBcXHRpbWVzIEciXSxbMiwxLCJHIl0sWzAsMCwiRyJdLFswLDEsImUgXFx0aW1lcyBcXGlkX3tHfSIsMl0sWzEsMiwibSJdLFswLDIsIiIsMCx7ImxldmVsIjoyLCJzdHlsZSI6eyJoZWFkIjp7Im5hbWUiOiJub25lIn19fV0sWzMsMSwiXFxpZF97R30gXFx0aW1lcyBlIl0sWzMsMiwiIiwwLHsibGV2ZWwiOjIsInN0eWxlIjp7ImhlYWQiOnsibmFtZSI6Im5vbmUifX19XV0=
  \begin{tikzcd}
    G && {G \times G} && G \\
    && G
    \arrow["{(e,\id_{G})}"', from=1-5, to=1-3]
    \arrow["m", from=1-3, to=2-3]
    \arrow[Rightarrow, no head, from=1-5, to=2-3]
    \arrow["{(\id_{G}, e)}", from=1-1, to=1-3]
    \arrow[Rightarrow, no head, from=1-1, to=2-3]
  \end{tikzcd} \and%
  % https://q.uiver.app/#q=WzAsNCxbMCwwLCJHIl0sWzIsMCwiRyBcXHRpbWVzIEciXSxbNCwwLCJHIl0sWzIsMSwiRyJdLFswLDEsIihcXGlkX3tHfSwoKVxcaW52KSJdLFsyLDEsIigoKVxcaW52LCBcXGlkX3tHfSkiLDJdLFsxLDMsIm0iXSxbMCwzLCIiLDIseyJsZXZlbCI6Miwic3R5bGUiOnsiaGVhZCI6eyJuYW1lIjoibm9uZSJ9fX1dLFsyLDMsIiIsMix7ImxldmVsIjoyLCJzdHlsZSI6eyJoZWFkIjp7Im5hbWUiOiJub25lIn19fV1d
  \begin{tikzcd}
    G && {G \times G} && G \\
    && G
    \arrow["{(\id_{G},()\inv)}", from=1-1, to=1-3]
    \arrow["{(()\inv, \id_{G})}"', from=1-5, to=1-3]
    \arrow["m", from=1-3, to=2-3]
    \arrow[Rightarrow, no head, from=1-1, to=2-3]
    \arrow[Rightarrow, no head, from=1-5, to=2-3]
  \end{tikzcd} \and%
  % https://q.uiver.app/#q=WzAsNCxbMCwwLCJHIFxcdGltZXMgRyBcXHRpbWVzIEciXSxbMCwyLCJHIFxcdGltZXMgRyJdLFsyLDAsIkcgXFx0aW1lcyBHIl0sWzIsMiwiRyJdLFsxLDMsIm0iLDJdLFsyLDMsIm0iXSxbMCwxLCJcXGlkX3tHfSBcXHRpbWVzIG0iLDJdLFswLDIsIm0gXFx0aW1lcyBcXGlkX3tHfSJdXQ==
  \begin{tikzcd}
    {G \times G \times G} && {G \times G} \\
    \\
    {G \times G} && G
    \arrow["m"', from=3-1, to=3-3]
    \arrow["m", from=1-3, to=3-3]
    \arrow["{\id_{G} \times m}"', from=1-1, to=3-1]
    \arrow["{m \times \id_{G}}", from=1-1, to=1-3]
  \end{tikzcd}
\end{mathpar}
The first diagram, for instance, says that for any generalized element $g$ of $G$, $m(g,e) = g$.

A finite-product preserving functor $\iC \to \iD$ maps internal groups of $\iC$ to those of $\iD$.
In fact, there is a generic category $\cG$ such that finite-product preserving functors $\cG \to \iC$ correspond to internal groups of $\iC$.
When $\iC = \CSet$, an internal group is just a group in the usual sense.

The category $\CFin\op$, intuitively speaking, demands no operations and axioms, so a finite-product preserving functor $\iota : \CFin\op \to \iC$ is just an object in $\iC$.
Thus, a finite-product preserving functor $\CFin\op \to \CSet$ is just a set.
When $\iC = \cG$, the functor $\iota$ picks out the generic group object.
Then the composite $\CFin\op \xto{\iota} \cG \to \CSet$ is the underlying set of the generic group object, so precomposition with $\iota$ is the forgetful functor $U : \CGrp \to \CSet$.
% https://q.uiver.app/#q=WzAsNCxbMCwwLCJcXE1vZF97XFxDU2V0fShcXGNHKSJdLFsyLDAsIlxcTW9kX3tcXENTZXR9KFxcQ0Zpblxcb3ApIl0sWzAsMSwiXFxDR3JwIl0sWzIsMSwiXFxDU2V0Il0sWzIsMCwiXFxlcXYiLDMseyJzdHlsZSI6eyJib2R5Ijp7Im5hbWUiOiJub25lIn0sImhlYWQiOnsibmFtZSI6Im5vbmUifX19XSxbMywxLCJcXGVxdiIsMyx7InN0eWxlIjp7ImJvZHkiOnsibmFtZSI6Im5vbmUifSwiaGVhZCI6eyJuYW1lIjoibm9uZSJ9fX1dLFswLDEsIlxcaW90YV57Kn0iXV0=
\[\begin{tikzcd}
    {\Mod_{\CSet}(\cG)} && {\Mod_{\CSet}(\CFin\op)} \\
    \CGrp && \CSet
    \arrow["\eqv"{marking, allow upside down}, draw=none, from=2-1, to=1-1]
    \arrow["\eqv"{marking, allow upside down}, draw=none, from=2-3, to=1-3]
    \arrow["{\iota^{*}}", from=1-1, to=1-3]
  \end{tikzcd}\]

The forgetful functor encodes a great deal of information about groups.
Let us consider a natural transformations $\alpha : U \To U$.
For any two groups $G$ and $G'$ and any group homomorphism $f : G \to G'$, the following diagram commutes.
% https://q.uiver.app/#q=WzAsNCxbMCwwLCJVRyJdLFsyLDAsIlVHIl0sWzAsMiwiVUcnIl0sWzIsMiwiVUcnIl0sWzAsMSwiXFxhbHBoYV97R30iXSxbMiwzLCJcXGFscGhhX3tHJ30iLDJdLFswLDIsIlVmIiwyXSxbMSwzLCJVZiJdXQ==
\[\begin{tikzcd}
    UG && UG \\
    \\
    {UG'} && {UG'}
    \arrow["{\alpha_{G}}", from=1-1, to=1-3]
    \arrow["{\alpha_{G'}}"', from=3-1, to=3-3]
    \arrow["Uf"', from=1-1, to=3-1]
    \arrow["Uf", from=1-3, to=3-3]
  \end{tikzcd}\]
The means that the data of $\alpha$ is just an (implicit) $1$-ary group operation.
In general, a natural transformation $U^{m} \To U^{n}$ encodes an $n$-tuple of $m$-ary group operation.

Now, suppose that $\cT$ is a geometric theory and $f : \cE \to \cF$ is a geometric morphism between Grothendieck topoi.
Then a model of $\cT$ in $\cF$ yields a model of $\cT$ in $\cE$ via the inverse image functor $f^{*} : \cF \to \cE$ since it is cocontinuous and left exact (preserves the interpretation of $\bigvee$, $\wedge$, and other connectives).
Is there a classifying Grothendieck topos $\cG[\cT]$ such that each geometric morphism $\cE \to \cG[\cT]$ corresponds to a model of $\cT$ in $\cE$?
This has been answered positively.

\section{Yoneda and cantor}
\label{sec:yoneda-and-cantor}

\begin{cons}[Archetypal Grothendieck construction]\label{cons:archetypal-grothendieck-construction}
  Let $X$ be a set.
  There is a bijection between the powerset $P(X)$ and the set of all truth functionals over $X$.
  \[
    \Theta : P(X) \otto 2^{X} : \Psi
  \]
  \begin{itemize}
  \item[$\Theta$:] For each subset $S \subseteq X$, $\Theta(S)$ is the characteristic function of $X$, mapping elements in $S$ to $1$ and everything else to $0$.
  \item[$\Psi$:] For each truth functional $\phi : X \to 2$, $\Psi(\phi)$ is the inverse image $\phi\inv(1)$.
  \end{itemize}
\end{cons}

\begin{rmk}
  The map $\Psi$ in \cref{cons:archetypal-grothendieck-construction} can be described as a pullback.
  % https://q.uiver.app/#q=WzAsNCxbMiwwLCIxIl0sWzIsMiwiMiJdLFswLDIsIlgiXSxbMCwwLCJcXHZhcnBoaVxcaW52KDEpIl0sWzAsMSwidCIsMCx7InN0eWxlIjp7InRhaWwiOnsibmFtZSI6Im1vbm8ifX19XSxbMiwxLCJcXHZhcnBoaSIsMl0sWzMsMiwiIiwyLHsic3R5bGUiOnsidGFpbCI6eyJuYW1lIjoibW9ubyJ9fX1dLFszLDBdLFszLDEsIiIsMSx7InN0eWxlIjp7Im5hbWUiOiJjb3JuZXIifX1dXQ==
  \[\begin{tikzcd}
      {\varphi\inv(1)} && 1 \\
      \\
      X && 2
      \arrow["t", tail, from=1-3, to=3-3]
      \arrow["\varphi"', from=3-1, to=3-3]
      \arrow[tail, from=1-1, to=3-1]
      \arrow[from=1-1, to=1-3]
      \arrow["\lrcorner"{anchor=center, pos=0.125}, draw=none, from=1-1, to=3-3]
    \end{tikzcd}\]
  Indeed, $\varphi\inv(1)$ is the largest (w.r.t. inclusion) subset of $X$ so that $\varphi$ evaluates to $1$ everywhere.
\end{rmk}

\begin{rmk}
  In $\CSet$, we make the foundational commitment that $\CSet(A, B)$ is a set and that the powerset $P(A)$ is in fact a set.
  As it turns out, assuming either one is sufficient to construct the other.
\end{rmk}

\begin{rmk}
  The set $P(X)$ has a canonical poset structure given by the set inclusion relation.
  In fact, $P(X)$ equipped with set inclusion is a complete Boolean algebra.
  Thus, the powerset functor $P : \CSet\op \to \CSet$ may be equally regarded as a functor $P : \CSet\op \to \CPos$, sending each set to its powerset ordered by inclusion.
\end{rmk}

\begin{rmk}
  The two point set has a unique nontrivial poset structure $\Set{0 \leq 1}$.
  We can define a poset structure on truth functionals by transporting the poset structure defined on the powersets along the Grothendieck construction.
  \[
    \chi_{X} \leq \chi_{Y} \text{ iff } X \subseteq Y
  \]
  This is the usual pointwise poset structure.
\end{rmk}

We can classify subsets of a given set $X$.
We can classify $X$-indexed families (i.e., functions $f : Y \to X$), too.
These families assemble into the slice construction $\CSet/X$.

\begin{cons}[Discrete Grothendieck construction]
  Let $X$ be a set construed as a discrete category.
  There is an equivalence of categories.
  \[
    \Theta : \CSet/X \otto \CSet^{X} : \Psi
  \]
  \begin{itemize}
  \item[$\Theta$:] For each function $f : Y \to X$, $\Theta(f)$ is the functor that sends each $x \in X$ to the inverse image $f\inv(x)$.
  \item[$\Psi$:] For each functor $\varphi : X \to \CSet$, $\Psi(\varphi)$ is the canonical map $\coprod_{x \in X}\varphi(x) \to X$
  \end{itemize}
\end{cons}

\begin{rmk}
  Indeed, $\Psi$ can be phrased in terms of pullback.
  % https://q.uiver.app/#q=WzAsNCxbMiwwLCJcXENTZXRfeyp9Il0sWzIsMiwiXFxDU2V0Il0sWzAsMiwiWCJdLFswLDAsIlxcY29wcm9kX3t4IFxcaW4gWH1cXHZhcnBoaSh4KSJdLFswLDFdLFsyLDEsIlxcdmFycGhpIiwyXSxbMywyXSxbMywwXSxbMywxLCIiLDEseyJzdHlsZSI6eyJuYW1lIjoiY29ybmVyIn19XV0=
  \[\begin{tikzcd}
      {\coprod_{x \in X}\varphi(x)} && {\CSet_{*}} \\
      \\
      X && \CSet
      \arrow[from=1-3, to=3-3]
      \arrow["\varphi"', from=3-1, to=3-3]
      \arrow[from=1-1, to=3-1]
      \arrow[from=1-1, to=1-3]
      \arrow["\lrcorner"{anchor=center, pos=0.125}, draw=none, from=1-1, to=3-3]
    \end{tikzcd}\]
  $\coprod_{x \in X}\varphi(x) \to X$ is the ``largest'' family so that $\varphi$ produces a witness everywhere.
\end{rmk}

\section{Hyperdoctrine}
\label{sec:hyperdoctrine}

The internal logic of a Heyting (or Boolean) algebra is intuitionistic (or classical) propositional logic.
Each sort of a given theory is equipped with a poset of propositions, in which $\varphi \leq \psi$ means $\varphi$ entails $\psi$.
For simplicity, let us assume a single sorted theory with no function symbols.
Let us recall Tarski's definition:
\begin{enumerate}
\item The unique sort is a set $S$.
\item A proposition variable is a subset of $S$.
\item $\top$ is the entire set $S$.
\item $\bot$ is the empty set $\varnothing$.
\item $\varphi \wedge \psi$ is the intersection $\varphi \cap \psi$.
\item $\varphi \vee \psi$ is the union $\varphi \cup \psi$.
\item $\varphi \To \psi$ is the exponential $\psi^{\varphi}$, i.e., the largest element so that $\psi^{\varphi} \wedge \varphi \leq \psi$.
\end{enumerate}

Essentially, propositions live in a poset with sufficient structure: various logical connectives correspond to various universal constructions.
As a model of a theory $\cT$ in another theory $\cT'$ has to preserve entailment, we expect it to be a function $M : \cT \to \cT'$ preserving relevant structures.

Generalizing this to multisorted theories is straightforward.
Each sort of the theory has a poset of propositions.
In the presence of nontrivial terms, we need to specify how propositions in context interact with substitutions.
For example, given a proposition in the context $d$, i.e., $\vdash_{d} \varphi$, substitution by a term $f : c \to d$ ought to produce a proposition in the context $c$, i.e., $\vdash_{c} \varphi[f]$.
Of course, we expect substitution to satisfy $\varphi[f][g] = \varphi[g \circ f]$ and $\varphi[\id] = \varphi$.

\begin{defn}
  A \emph{doctrine} is a functor $\cC : \iC\op \to \CPos$, where $\CPos$ is some suitable category of posets.
\end{defn}

\begin{defn}\label{def:model-of-doctrine}
  A \emph{model} of a doctrine $\cC : \iC\op \to \CPos$ in another doctrine $\cD : \iD\op \to \CPos$ consists of
  \begin{itemize}
  \item A model $M : \iC \to \iD$.
  \item A natural transformation $\lambda : \cC \To \cD \circ M\op$.
  \end{itemize}
  % https://q.uiver.app/#q=WzAsMyxbMCwwLCJcXGlDXFxvcCJdLFsyLDAsIlxcaURcXG9wIl0sWzEsMSwiXFxDUG9zIl0sWzAsMSwiTVxcb3AiXSxbMCwyLCJcXGNDIiwyXSxbMSwyLCJcXGNEIl0sWzQsMSwiXFxsYW1iZGEiLDEseyJzaG9ydGVuIjp7InNvdXJjZSI6MTAsInRhcmdldCI6MTB9fV1d
  \[\begin{tikzcd}
      \iC\op && \iD\op \\
      & \CPos
      \arrow["M\op", from=1-1, to=1-3]
      \arrow[""{name=0, anchor=center, inner sep=0}, "\cC"', from=1-1, to=2-2]
      \arrow["\cD", from=1-3, to=2-2]
      \arrow["\lambda"{description}, shorten <=4pt, shorten >=4pt, Rightarrow, from=0, to=1-3]
    \end{tikzcd}\]
\end{defn}

Let us unpack \cref{def:model-of-doctrine}.
The model $M : \iC \to \iD$ is unsurprising: we need to express $\iC$ in terms of $\iD$.
Each component $\lambda_{c}$ is a model of propositions $\cC(c)$ in $\cD M\op(c)$.
Naturality expresses that these models are compatible with substitutions. 
% https://q.uiver.app/#q=WzAsNCxbMCwwLCJcXGNDKGMpIl0sWzIsMCwiXFxjRCBNXFxvcChjKSJdLFswLDIsIlxcY0MoYycpIl0sWzIsMiwiXFxjRCBNXFxvcChjJykiXSxbMCwxLCJcXGxhbWJkYV97Y30iXSxbMiwwLCJcXGJsYW5rW2ZdIl0sWzMsMSwiXFxibGFua1tNXFxvcChmKV0iLDJdLFsyLDMsIlxcbGFtYmRhX3tjJ30iLDJdXQ==
\[\begin{tikzcd}
    {\cC(c)} && {\cD M\op(c)} \\
    \\
    {\cC(c')} && {\cD M\op(c')}
    \arrow["{\lambda_{c}}", from=1-1, to=1-3]
    \arrow["{\blank[f]}", from=3-1, to=1-1]
    \arrow["{\blank[M\op(f)]}"', from=3-3, to=1-3]
    \arrow["{\lambda_{c'}}"', from=3-1, to=3-3]
  \end{tikzcd}\]

Weighted quantifiers can be characterized as adjoints, too.
Given a term $f : c \to d$, the formula $\exists_{f}\varphi$ ought to be the smallest formula so that $\varphi \leq (\exists_{f}\varphi)[f]$.
Similarly, the formula $\forall_{f}\varphi$ ought to be the largest formula so that $(\forall_{f}\varphi)[f] \leq \varphi$.
That is, we need the following adjoints:
% https://q.uiver.app/#q=WzAsMixbMCwwLCJcXGNDKGMpIl0sWzAsMiwiXFxjQyhkKSJdLFsxLDAsIlxcYmxhbmtbZl0iLDFdLFswLDEsIlxcZXhpc3RzX3tmfSIsMix7ImN1cnZlIjo0fV0sWzAsMSwiXFxmb3JhbGxfe2Z9IiwwLHsiY3VydmUiOi00fV0sWzMsMiwiIiwxLHsibGV2ZWwiOjEsInN0eWxlIjp7Im5hbWUiOiJhZGp1bmN0aW9uIn19XSxbMiw0LCIiLDEseyJsZXZlbCI6MSwic3R5bGUiOnsibmFtZSI6ImFkanVuY3Rpb24ifX1dXQ==
\[\begin{tikzcd}
    {\cC(c)} \\
    \\
    {\cC(d)}
    \arrow[""{name=0, anchor=center, inner sep=0}, "{f^{*}}"{description}, from=3-1, to=1-1]
    \arrow[""{name=1, anchor=center, inner sep=0}, "{\exists_{f}}"', curve={height=24pt}, from=1-1, to=3-1]
    \arrow[""{name=2, anchor=center, inner sep=0}, "{\forall_{f}}", curve={height=-24pt}, from=1-1, to=3-1]
    \arrow["\dashv"{anchor=center}, draw=none, from=1, to=0]
    \arrow["\dashv"{anchor=center}, draw=none, from=0, to=2]
  \end{tikzcd}\]
Indeed, in the powerset doctrine $\cP : \CSet\op \to \CPos$, the existential quantifier and the universal quantifier are given by the direct image $f_{*}$ and the fiber image $f_{!}$, respectively.

The powerset doctrine factors through the contravariant powerset functor in a canonical way because every powerset has a canonical poset structure given by the inclusion relation.
% https://q.uiver.app/#q=WzAsNCxbMCwwLCJcXGlEIl0sWzIsMCwiXFxDU2V0XFxvcCJdLFsxLDIsIlxcQ1BvcyJdLFszLDIsIlxcQ1NldCJdLFswLDIsIlxcY0QiLDJdLFswLDEsIk0iXSxbMSwzLCJcXGNQIl0sWzIsM10sWzEsMiwiIiwxLHsic3R5bGUiOnsiYm9keSI6eyJuYW1lIjoiZGFzaGVkIn19fV0sWzQsMSwiXFxsYW1iZGEiLDAseyJzaG9ydGVuIjp7InNvdXJjZSI6MTAsInRhcmdldCI6MTB9fV1d
\[\begin{tikzcd}
    \iU\op && \CSet\op \\
    \\
    & \CPos && \CSet
    \arrow[""{name=0, anchor=center, inner sep=0}, "\cU"', from=1-1, to=3-2]
    \arrow["M", from=1-1, to=1-3]
    \arrow["\cP", from=1-3, to=3-4]
    \arrow[from=3-2, to=3-4]
    \arrow[dashed, from=1-3, to=3-2]
    \arrow["\lambda", shorten <=5pt, shorten >=5pt, Rightarrow, from=0, to=1-3]
  \end{tikzcd}\]
Thus, we may identify a model of a doctrine as a natural transformation (of some kind) between presheaves.
% https://q.uiver.app/#q=WzAsMyxbMCwwLCJcXGlEXFxvcCJdLFsyLDAsIlxcQ1NldFxcb3AiXSxbMiwyLCJcXENTZXQiXSxbMCwxXSxbMSwyXSxbMCwyLCJcXGNEIiwyXSxbNSwxLCJcXGxhbWJkYSIsMix7InNob3J0ZW4iOnsic291cmNlIjoxMCwidGFyZ2V0IjoxMH19XV0=
\[\begin{tikzcd}
    \iU\op && \CSet\op \\
    \\
    && \CSet
    \arrow[from=1-1, to=1-3]
    \arrow[from=1-3, to=3-3]
    \arrow[""{name=0, anchor=center, inner sep=0}, "\cU"', from=1-1, to=3-3]
    \arrow["\lambda"', shorten <=3pt, shorten >=3pt, Rightarrow, from=0, to=1-3]
  \end{tikzcd}\]
Given a natural transformation $\lambda : \cU \To \tilde{\cU}$ between presheaves, we may think of $\cU$ as a presheaf of ``witnesses'' and $\tilde{\cU}$ as a presheaf of ``types.''
By the Yoneda lemma, each type $A \in \tilde{\cU}(\Gamma)$ can be identified with a natural transformation $A : \iU(\blank, \Gamma) \To \cU$.
If we additionally require that the pullback of $A$ along $\lambda$ is representable, then the two projections correspond to a term $\Gamma.A \to \Gamma$ in the theory $\iU$ and a witness $t \in \cU(\Gamma.A)$, respectively.
% https://q.uiver.app/#q=WzAsNCxbMiwwLCJcXGNVIl0sWzIsMiwiXFx0aWxkZXtcXGNVfSJdLFswLDIsIlxceW9uXFxHYW1tYSJdLFswLDAsIlxceW9uXFxHYW1tYS5BIl0sWzAsMSwiXFxsYW1iZGEiXSxbMiwxLCJBIiwyXSxbMywyXSxbMywwLCJ0Il0sWzMsMSwiIiwxLHsic3R5bGUiOnsibmFtZSI6ImNvcm5lciJ9fV1d
\[\begin{tikzcd}
    {\yon\Gamma.A} && \cU \\
    \\
    \yon\Gamma && {\tilde{\cU}}
    \arrow["\lambda", from=1-3, to=3-3]
    \arrow["A"', from=3-1, to=3-3]
    \arrow[from=1-1, to=3-1]
    \arrow["t", from=1-1, to=1-3]
    \arrow["\lrcorner"{anchor=center, pos=0.125}, draw=none, from=1-1, to=3-3]
  \end{tikzcd}\]

\section{Calculus of fractions and orthogonality}
\label{sec:calculus-of-fractions-and-orthogonality}
\todo{Ivan's notes: topics in category theory 2018}

\begin{defn}
  Let $\iC$ be a category and $\cW$ be a class of morphisms (to be inverted) of $\iC$.
  The \emph{localization of $\iC$ with respect to $\cW$} is a category $\iC[\cW\inv]$ equipped with a localization functor $L : \iC \to \iC[\cW\inv]$ such that
  \begin{itemize}
  \item $Lw$ is an isomorphism for all $w \in \cW$.
  \item $L$ is universal with respect to this property, i.e., if $F : \iC \to \iD$ is a functor that maps each$w \in \cW$ to an isomorphism, then there is a unique functor $i : \iC[\cW\inv] \to \iD$ so that $iL \iso F$.
  \end{itemize}
\end{defn}

\begin{thm}
  $\iC[\cW\inv]$ and $L$ exist for any category $\iC$ and any class of morphisms $\cW$.
\end{thm}
\begin{proof}
  Since $\cW$ are morphisms to be inverted, we formally add their inverses.
  Let $\iG$ be a graph whose vertices are objects of $\iC$ and morphisms are morphisms of $\iC$ and formal inverses of $w \in \cW$.
  Now, take the free category $F\iG$ generated the graph $\iG$.
  Explicitly, the morphisms of $F\iG$ are tuples of morphisms of $\iC \cup \cW\inv$.
  To enforce invertibility and universal property, we quotient the category $F\iG$ by the relation $\sim$ generated by the following equations
  \begin{itemize}
  \item $()_{c} = (\id_{c})$.
  \item $(g, f) = (gf)$ whenever the composition makes sense.
  \item $(w, w\inv) = ()_{\cod(w)}$.
  \item $(w\inv, w) = ()_{\dom(w)}$.
  \end{itemize}
  We take $\iC[\cW\inv] := F\iG/\sim$, and $L : \iC \to \iC[\cW\inv]$ is defined as the composite $\iC \into F\iG \epi \iC[\cW\inv]$.
  By construction, $Lw$ is an isomorphism for all $w \in \cW$.
  The universality of this construction is inherited from the universal property of the quotient category construction.
\end{proof}

\begin{defn}
  An object $c$ is \emph{orthogonal} to a morphism $m : d \to d'$ if for each morphism $f : d \to c$, there is a unique morphism $f' : d' \to c$ so that the following diagram commutes.
  % https://q.uiver.app/#q=WzAsMyxbMCwwLCJkIl0sWzIsMCwiZCciXSxbMSwxLCJjIl0sWzAsMSwibSJdLFswLDIsImYiLDJdLFsxLDIsImYnIiwwLHsic3R5bGUiOnsiYm9keSI6eyJuYW1lIjoiZGFzaGVkIn19fV1d
  \[\begin{tikzcd}
      d && {d'} \\
      & c
      \arrow["m", from=1-1, to=1-3]
      \arrow["f"', from=1-1, to=2-2]
      \arrow["{f'}", dashed, from=1-3, to=2-2]
    \end{tikzcd}\]
  For a class $\cM$ of morphisms in a category $\iC$, we write $\cM^{\bot}$ for the full subcategory of $\iC$ spanned by objects orthogonal to each $m \in \cM$.
\end{defn}

\section{Sheaves}
\label{sec:sheaves}

\begin{defn}
  Let $X$ be a topological space.
  A \emph{sheaf} (of sets) is a presheaf $F : \cO(X)\op \to \CSet$ satisfying the \emph{sheaf condition}: For any open cover $U = \bigcup_{i}U_{i}$, the following diagram is an equalizer.
  % https://q.uiver.app/#q=WzAsMyxbMCwwLCJGKFUpIl0sWzIsMCwiXFxwcm9kX3tpfUYoVV97aX0pIl0sWzQsMCwiXFxwcm9kX3tpLGp9RihVX3tpfSBcXGNhcCBVX3tqfSkiXSxbMCwxXSxbMSwyLCJwIiwwLHsib2Zmc2V0IjotMX1dLFsxLDIsInEiLDIseyJvZmZzZXQiOjF9XV0=
  \[\begin{tikzcd}
      {F(U)} && {\prod_{i}F(U_{i})} && {\prod_{i,j}F(U_{i} \cap U_{j})}
      \arrow[from=1-1, to=1-3]
      \arrow["p", shift left, from=1-3, to=1-5]
      \arrow["q"', shift right, from=1-3, to=1-5]
    \end{tikzcd}\]
  The morphisms $p$ and $q$ are the canonical ones specified by morphisms of the form $F(U_{i} \cap U_{j} \subseteq U_{k})$ composed with an appropriate projection.
  % https://q.uiver.app/#q=WzAsNyxbMCwyLCJGKFUpIl0sWzIsMiwiXFxwcm9kX3tpfUYoVV97aX0pIl0sWzQsMiwiXFxwcm9kX3tpLGp9RihVX3tpfSBcXGNhcCBVX3tqfSkiXSxbMiwwLCJGKFVfe2l9KSJdLFsyLDQsIkYoVV97an0pIl0sWzQsMCwiRihVX3tpfSBcXGNhcCBVX3tqfSkiXSxbNCw0LCJGKFVfe2l9IFxcY2FwIFVfe2p9KSJdLFswLDFdLFsxLDIsInAiLDAseyJvZmZzZXQiOi0xfV0sWzEsMiwicSIsMix7Im9mZnNldCI6MX1dLFsxLDMsIlxccGlfe2l9Il0sWzAsMywiRihVX3tpfSBcXHN1YnNldGVxIFUpIl0sWzEsNCwiXFxwaV97an0iLDJdLFswLDQsIkYoVV97an0gXFxzdWJzZXRlcSBVKSIsMl0sWzIsNSwiXFxwaV97aSxqfSIsMl0sWzIsNiwiXFxwaV97aSxqfSJdLFszLDUsIkYoVV97aX0gXFxjYXAgVV97an0gXFxzdWJzZXRlcSBVX3tpfSkiXSxbNCw2LCJGKFVfe2l9IFxcY2FwIFVfe2p9IFxcc3Vic2V0ZXEgVV97an0pIiwyXV0=
  \[\begin{tikzcd}
      && {F(U_{i})} && {F(U_{i} \cap U_{j})} \\
      \\
      {F(U)} && {\prod_{i}F(U_{i})} && {\prod_{i,j}F(U_{i} \cap U_{j})} \\
      \\
      && {F(U_{j})} && {F(U_{i} \cap U_{j})}
      \arrow[from=3-1, to=3-3]
      \arrow["p", shift left, from=3-3, to=3-5]
      \arrow["q"', shift right, from=3-3, to=3-5]
      \arrow["{\pi_{i}}", from=3-3, to=1-3]
      \arrow["{F(U_{i} \subseteq U)}", from=3-1, to=1-3]
      \arrow["{\pi_{j}}"', from=3-3, to=5-3]
      \arrow["{F(U_{j} \subseteq U)}"', from=3-1, to=5-3]
      \arrow["{\pi_{i,j}}"', from=3-5, to=1-5]
      \arrow["{\pi_{i,j}}", from=3-5, to=5-5]
      \arrow["{F(U_{i} \cap U_{j} \subseteq U_{i})}", from=1-3, to=1-5]
      \arrow["{F(U_{i} \cap U_{j} \subseteq U_{j})}"', from=5-3, to=5-5]
    \end{tikzcd}\]
\end{defn}

\begin{defn}
  Let $X$ be a topological space.
  We define the category of sheaves $\Sh(X)$ as the full subcategory of $\Psh(\cO(X))$ spanned by sheaves.
\end{defn}

\begin{thm}
  For any topological space $X$, there is an isomorphism
  \[
    \Theta : \cO(X) \otto \Sub_{\Sh(X)}(1) : \Psi
  \]
\end{thm}
\begin{proof}
  We construct the two order-preserving functions as follows:
  \begin{itemize}
  \item[($\Theta$)] For each open set $O$, $\Theta(O)$ is the represented subsheaf $\Hom(\blank, O)$ of the terminal sheaf $1$.
  \item[($\Psi$)] For each subsheaf $F$ of the terminal sheaf, $\Psi(F)$ is the open set $\bigcup\Set{O \in \cO(X) \mid F(O) = 1}$, i.e., it is the union of all open sets $O$ so that $F(O)$ is the singleton set.
  \end{itemize}
\end{proof}

\begin{defn}
  Let $X$ and $Y$ be topological spaces.
  A continuous map $p : Y \to X$ is called a \emph{bundle over} $X$.
  A \emph{cross section} of a bundle $p : Y \to X$ is a continuous map $s : X \to Y$ so that $ps = \id$, i.e., a morphism from the identity bundle $\id : X \to X$ to $p : Y \to X$ in the slice category $\CTop/X$.
\end{defn}

\begin{rmk}
  Let $U \subseteq X$ be an open set of the topological space $X$, and $p : Y \to X$ be a bundle over $X$.
  The pullback yields a bundle over $U$.
  % https://q.uiver.app/#q=WzAsNCxbMCwyLCJVIl0sWzIsMiwiWCJdLFsyLDAsIlkiXSxbMCwwLCJwXFxpbnYgVSJdLFsyLDEsInAiXSxbMCwxLCIiLDIseyJzdHlsZSI6eyJ0YWlsIjp7Im5hbWUiOiJtb25vIn19fV0sWzMsMCwicFxcdmVydF97VX0iLDJdLFszLDIsIiIsMCx7InN0eWxlIjp7InRhaWwiOnsibmFtZSI6Im1vbm8ifX19XSxbMywxLCIiLDEseyJzdHlsZSI6eyJuYW1lIjoiY29ybmVyIn19XSxbMCwyLCJzIiwxLHsic3R5bGUiOnsiYm9keSI6eyJuYW1lIjoiZGFzaGVkIn19fV1d
  \[\begin{tikzcd}
      {p\inv U} && Y \\
      \\
      U && X
      \arrow["p", from=1-3, to=3-3]
      \arrow[tail, from=3-1, to=3-3]
      \arrow["{p\vert_{U}}"', from=1-1, to=3-1]
      \arrow[tail, from=1-1, to=1-3]
      \arrow["\lrcorner"{anchor=center, pos=0.125}, draw=none, from=1-1, to=3-3]
      \arrow["s"{description}, dashed, from=3-1, to=1-3]
    \end{tikzcd}\]
  We call $s$ a \emph{cross section} of $p$ over $U$ if $s \circ p\vert_{U}$ is the inclusion.
\end{rmk}

\begin{rmk}
  We have an assignment $\Gamma_{p} : \cO(X) \to \CSet$ defined by sending open sets $U$ to the set of all cross sections of $p$ over $U$.
  This can be extended to a contravariant functor as for any open set $U' \subseteq U$ and any cross section $s$, we can define a cross section of $p$ over $U'$ by composition.
  % https://q.uiver.app/#q=WzAsNixbMiwyLCJVIl0sWzQsMiwiWCJdLFs0LDAsIlkiXSxbMiwwLCJwXFxpbnYgVSJdLFswLDAsInBcXGludiBVJyJdLFswLDIsIlUnIl0sWzIsMSwicCJdLFswLDEsIiIsMix7InN0eWxlIjp7InRhaWwiOnsibmFtZSI6Im1vbm8ifX19XSxbMywwLCJwXFx2ZXJ0X3tVfSIsMl0sWzMsMiwiIiwwLHsic3R5bGUiOnsidGFpbCI6eyJuYW1lIjoibW9ubyJ9fX1dLFszLDEsIiIsMSx7InN0eWxlIjp7Im5hbWUiOiJjb3JuZXIifX1dLFswLDIsInMiLDEseyJzdHlsZSI6eyJib2R5Ijp7Im5hbWUiOiJkYXNoZWQifX19XSxbNSwwLCIiLDAseyJzdHlsZSI6eyJ0YWlsIjp7Im5hbWUiOiJtb25vIn19fV0sWzQsNSwicFxcdmVydF97VSd9IiwyXSxbNCwzLCIiLDAseyJzdHlsZSI6eyJ0YWlsIjp7Im5hbWUiOiJtb25vIn19fV0sWzUsMiwiIiwyLHsiY3VydmUiOi0yLCJzdHlsZSI6eyJib2R5Ijp7Im5hbWUiOiJkYXNoZWQifX19XSxbNCwwLCIiLDIseyJzdHlsZSI6eyJuYW1lIjoiY29ybmVyIn19XV0=
  \[\begin{tikzcd}
      {p\inv U'} && {p\inv U} && Y \\
      \\
      {U'} && U && X
      \arrow["p", from=1-5, to=3-5]
      \arrow[tail, from=3-3, to=3-5]
      \arrow["{p\vert_{U}}"', from=1-3, to=3-3]
      \arrow[tail, from=1-3, to=1-5]
      \arrow["\lrcorner"{anchor=center, pos=0.125}, draw=none, from=1-3, to=3-5]
      \arrow["s"{description}, dashed, from=3-3, to=1-5]
      \arrow[tail, from=3-1, to=3-3]
      \arrow["{p\vert_{U'}}"', from=1-1, to=3-1]
      \arrow[tail, from=1-1, to=1-3]
      \arrow[curve={height=-12pt}, dashed, from=3-1, to=1-5]
      \arrow["\lrcorner"{anchor=center, pos=0.125}, draw=none, from=1-1, to=3-3]
    \end{tikzcd}\]
\end{rmk}

\section{Elementary topoi}
\label{sec:elementary-topoi}

\begin{defn}[Elementary form]\label{def:topos}
  An (elementary) \emph{topos} $\cE$ is a category with
  \begin{enumerate}
  \item\label{def:topos-pullbacks} Pullbacks;
  \item\label{def:topos-terminal} A terminal object $1$;
  \item\label{def:topos-exponential} To each object $a$ an object $Pa$ and a morphism $\in_{a} : Pa \times a \to P1$ so that for any $f : b \times a \to P1$, there is a unique morphism $g : b \to Pa$ so that $g \times \id$ factors $f$ through $\in_{a}$.
    % https://q.uiver.app/#q=WzAsNSxbMiwxLCJhIFxcdGltZXMgUGEiXSxbMSwyLCJcXE9tZWdhIl0sWzAsMSwiYiBcXHRpbWVzIGEiXSxbMCwwLCJiIl0sWzIsMCwiUGEiXSxbMCwxLCJcXGluX3thfSJdLFsyLDEsImYiLDJdLFsyLDAsImcgXFx0aW1lcyBcXGlkIl0sWzMsNCwiZyIsMCx7InN0eWxlIjp7ImJvZHkiOnsibmFtZSI6ImRhc2hlZCJ9fX1dXQ==
    \[\begin{tikzcd}
	b && Pa \\
	{b \times a} && {Pa \times a} \\
	& \Omega
	\arrow["{\in_{a}}", from=2-3, to=3-2]
	\arrow["f"', from=2-1, to=3-2]
	\arrow["{g \times \id}", from=2-1, to=2-3]
	\arrow["g", dashed, from=1-1, to=1-3]
      \end{tikzcd}\]
  \end{enumerate}
\end{defn}

\begin{rmk}
  The power object $P1$ is the subobject classifier.
\end{rmk}

\begin{rmk}
  We can phrase \cref{def:topos} in a nonelementary way.
  Conditions \ref{def:topos-pullbacks} and \ref{def:topos-terminal} demand that $\cE$ is finitely complete, while condition \ref{def:topos-exponential} demands that $\cE$ has power objects, and thus all exponentials (c.f., \cref{thm:topos-has-exponentials}).
  In other words, a topos is a category with finite limits, exponentials, and a subobject classifier.
\end{rmk}

\begin{rmk}
  The assignment $P : \ob\cE \to \ob\cE$ can be uniquely extended to a functor.
  Extending $P$ to a functor means that for any $c$, the following isomorphism has to be natural in $a$.
  \[
    \cE(c, Pa) \iso \cE(c \times a, \Omega)
  \]
  That is, for any $h : a \to b$, the following diagram must commute.
  % https://q.uiver.app/#q=WzAsNCxbMCwwLCJcXGNFKFBiLFBhKSJdLFsyLDAsIlxcY0UoUGIgXFx0aW1lcyBhLCBcXE9tZWdhKSJdLFsyLDIsIlxcY0UoUGIgXFx0aW1lcyBiLCBcXE9tZWdhKSJdLFswLDIsIlxcY0UoUGIsIFBiKSJdLFswLDEsIlxcaXNvIiwzLHsic3R5bGUiOnsiYm9keSI6eyJuYW1lIjoibm9uZSJ9LCJoZWFkIjp7Im5hbWUiOiJub25lIn19fV0sWzMsMiwiXFxpc28iLDMseyJzdHlsZSI6eyJib2R5Ijp7Im5hbWUiOiJub25lIn0sImhlYWQiOnsibmFtZSI6Im5vbmUifX19XSxbMiwxLCJcXGJsYW5rIFxcY2lyYyAoXFxpZCBcXHRpbWVzIGgpIiwyXSxbMywwLCJQaCBcXGNpcmMgXFxibGFuayJdXQ==
  \[\begin{tikzcd}
      {\cE(Pb,Pa)} && {\cE(Pb \times a, \Omega)} \\
      \\
      {\cE(Pb, Pb)} && {\cE(Pb \times b, \Omega)}
      \arrow["\iso"{marking, allow upside down}, draw=none, from=1-1, to=1-3]
      \arrow["\iso"{marking, allow upside down}, draw=none, from=3-1, to=3-3]
      \arrow["{\blank \circ (\id \times h)}"', from=3-3, to=1-3]
      \arrow["{Ph \circ \blank}", from=3-1, to=1-1]
    \end{tikzcd}\]
  The transpose of $\id_{Pb}$ is $\in_{b}$.
  Thus, $Ph$ must be the transpose of $\in_{b}(\id \times h)$.
  Namely, it is the unique morphism given by the universal property of power objects.
  % https://q.uiver.app/#q=WzAsMyxbMCwyLCJQYSBcXHRpbWVzIGEiXSxbMiwyLCJcXE9tZWdhIl0sWzAsMCwiUGIgXFx0aW1lcyBhIl0sWzAsMSwiXFxpbl97YX0iLDJdLFsyLDEsIlxcaW5fe2J9KFxcaWQgXFx0aW1lcyBoKSJdLFsyLDAsIlBoIFxcdGltZXMgXFxpZCIsMix7InN0eWxlIjp7ImJvZHkiOnsibmFtZSI6ImRhc2hlZCJ9fX1dXQ==
  \[\begin{tikzcd}
      {Pb \times a} \\
      \\
      {Pa \times a} && \Omega
      \arrow["{\in_{a}}"', from=3-1, to=3-3]
      \arrow["{\in_{b}(\id \times h)}", from=1-1, to=3-3]
      \arrow["{Ph \times \id}"', dashed, from=1-1, to=3-1]
    \end{tikzcd}\]
\end{rmk}

\begin{rmk}
  Indeed, the power object functor $P : \cE\op \to \cE$ is monadic.
  As monadic functors create limits and $\cE$ has finite limits by definition, $\cE\op$ has finite limits.
  Thus, $\cE$ is finitely cocomplete.
\end{rmk}

\begin{eg}
  When $\cE = \CSet$, the functor $P : \CSet\op \to \CSet$ is just the usual (contravariant) powerset functor.
\end{eg}

\begin{rmk}
  Given a topos $\cE$, we can view a subobject of $a$ as
  \begin{enumerate}
  \item An equivalence class of monos: $a' \mono a$.
  \item A characteristic morphism: $a \to \Omega$.
  \item A global element of a power object: $1 \to Pa$.
  \end{enumerate}
  This is due to the following isomorphisms.
  \[
    \Sub_{\cE}(a) \iso \cE(a, \Omega) \iso \cE(1, Pa)
  \]
\end{rmk}

\begin{defn}
  Let $\cE$ be a topos.
  Let $r : s \mono a \times b$ be a subobject with the corresponding characteristic morphism $\varphi : a \times b \to \Omega$ and global element $[\varphi] : b \to Pa$.
  We say that $s$ is an \emph{extension} of the predicate $\varphi$ and the element $[\varphi]$ \emph{names} the predicate $\varphi$.
\end{defn}

\begin{eg}
  In $\CSet$, a relation $R \subseteq X \times Y$ can be equally viewed as a truth functional (predicate) $\varphi : X \times Y \to 2$.
  The name of this predicate is a function $[\varphi] : Y \to PX$ that maps each $y \in Y$ to a subset of $X$ for which $R(x, y)$ holds for all $x$ in that subset.
\end{eg}

\begin{eg}\label{eg:singleton-predicate}
  The diagonal map $\Delta_{a} : a \to a \times a$ is always a monomorphism.
  Thus, in a topos, $\Delta_{a}$ is classified by the equality predicate $\delta_{a} : a \times a \to \Omega$, which is named by the global element $[\delta_{a}] : a \to Pa$.
  As it turns out, the name $[\delta_{a}]$ is a monomorphism itself.
  Thus, it is classified by a predicate $\sigma_{a} : Pa \to \Omega$, called the \emph{singleton} predicate.
\end{eg}

\begin{thm}\label{thm:topos-has-exponentials}
  Every topos has exponentials.
\end{thm}
\begin{proof}
  The main idea is that every function can be identified with a graph in $\CSet$, so a reasonable guess is to take $B^{A}$ as a subobject of $P(A \times B)$.
  Let $r : P(A \times B) \times A \to PB$ be the name of the subobject $\in_{A \times B} \mono P(A \times B) \times A \times B$.
  Take the pullback of the singleton map along $r$.
  % https://q.uiver.app/#q=WzAsNCxbMCwyLCJQKEEgXFx0aW1lcyBCKSBcXHRpbWVzIEEiXSxbMiwyLCJQQiJdLFsyLDAsIkIiXSxbMCwwLCJRIl0sWzAsMSwiciIsMl0sWzIsMSwiXFxTZXR7fSIsMCx7InN0eWxlIjp7InRhaWwiOnsibmFtZSI6Im1vbm8ifX19XSxbMywyXSxbMywwLCJxIiwyLHsic3R5bGUiOnsidGFpbCI6eyJuYW1lIjoibW9ubyJ9fX1dLFszLDEsIiIsMSx7InN0eWxlIjp7Im5hbWUiOiJjb3JuZXIifX1dXQ==
  \[\begin{tikzcd}
      Q && B \\
      \\
      {P(A \times B) \times A} && PB
      \arrow["r"', from=3-1, to=3-3]
      \arrow["{\Set{}}", tail, from=1-3, to=3-3]
      \arrow[from=1-1, to=1-3]
      \arrow[tail, from=1-1, to=3-1]
      \arrow["\lrcorner"{anchor=center, pos=0.125}, draw=none, from=1-1, to=3-3]
    \end{tikzcd}\]
  In $\CSet$, $Q$ consists of pairs $(G, a)$ so that $\Set{b \in B \mid (a, b) \in G}$ is a singleton set.
  This means the ``function'' is defined at $a$ in a unique way.
  A function must satisfy this for all $a$, so let $q : P(A \times B) \to PB$ be the name of the classifying map $\chi_{Q}$, and $t_{A} : 1 \to PA$ be the name of the maximal subobject in $\Sub(1 \times A)$.
  Take the following pullback.
  % https://q.uiver.app/#q=WzAsNCxbMiwwLCIxIl0sWzIsMiwiUEEiXSxbMCwyLCJQKEEgXFx0aW1lcyBCKSJdLFswLDAsIkJee0F9Il0sWzIsMSwicSIsMl0sWzAsMSwidF97QX0iXSxbMywyLCIiLDIseyJzdHlsZSI6eyJ0YWlsIjp7Im5hbWUiOiJtb25vIn19fV0sWzMsMF1d
  \[\begin{tikzcd}
      {B^{A}} && 1 \\
      \\
      {P(A \times B)} && PA
      \arrow["q"', from=3-1, to=3-3]
      \arrow["{t_{A}}", tail, from=1-3, to=3-3]
      \arrow[tail, from=1-1, to=3-1]
      \arrow[from=1-1, to=1-3]
      \arrow["\lrcorner"{anchor=center, pos=0.125}, draw=none, from=1-1, to=3-3]
    \end{tikzcd}\]
  In $\CSet$, $B^{A}$ is precisely the set of all graphs from $A$ to $B$.
  It remains to verify that $B^{A}$ has the universal property of exponentials.
  Note that the following diagram is a pullback square.
  % https://q.uiver.app/#q=WzAsNCxbMiwwLCJCIl0sWzIsMiwiUEIiXSxbMCwyLCJBIFxcdGltZXMgUChBIFxcdGltZXMgQikiXSxbMCwwLCJBIFxcdGltZXMgQl57QX0iXSxbMiwxLCJyIiwyXSxbMCwxLCJcXFNldHt9Il0sWzMsMiwiIiwyLHsic3R5bGUiOnsidGFpbCI6eyJuYW1lIjoibW9ubyJ9fX1dLFszLDBdLFszLDEsIiIsMSx7InN0eWxlIjp7Im5hbWUiOiJjb3JuZXIifX1dXQ==
  \[\begin{tikzcd}
      {A \times B^{A}} && B \\
      \\
      {A \times P(A \times B)} && PB
      \arrow["r"', from=3-1, to=3-3]
      \arrow["{\Set{}}", from=1-3, to=3-3]
      \arrow[tail, from=1-1, to=3-1]
      \arrow[from=1-1, to=1-3]
      \arrow["\lrcorner"{anchor=center, pos=0.125}, draw=none, from=1-1, to=3-3]
    \end{tikzcd}\]
  The projection $A \times B^{A} \to B$ defines the evaluation map.
  Now, for any morphism $A \times C \to B$, we construct a morphism $A \times C \to A \times P(A \times B)$.
  First, to construction a morphism $C \to P(A \times B)$, it suffices to construct a morphism $A \times C \to PB$.
  Take $\Set{} \circ f$, and let $g : C \to P(A \times B)$ be its transpose.
  In $\CSet$, $g$ is defined by $c \mapsto \Set{(a, f(a, c))}$.
  Then $\id_{A} \times g$ makes the outer square of the following diagram commute.
  Thus, we have proven the required universal property.
  % https://q.uiver.app/#q=WzAsNSxbMywxLCJCIl0sWzMsMywiUEIiXSxbMSwzLCJBIFxcdGltZXMgUChBIFxcdGltZXMgQikiXSxbMSwxLCJBIFxcdGltZXMgQl57QX0iXSxbMCwwLCJBIFxcdGltZXMgQyJdLFsyLDEsInIiLDJdLFswLDEsIlxcU2V0e30iXSxbMywyLCIiLDIseyJzdHlsZSI6eyJ0YWlsIjp7Im5hbWUiOiJtb25vIn19fV0sWzMsMCwiXFxldiJdLFszLDEsIiIsMSx7InN0eWxlIjp7Im5hbWUiOiJjb3JuZXIifX1dLFs0LDAsImYiLDAseyJjdXJ2ZSI6LTJ9XSxbNCwyLCJcXGlkX3tBfSBcXHRpbWVzIGciLDIseyJjdXJ2ZSI6Mn1dLFs0LDMsIlxcaWRfe0F9IFxcdGltZXMgXFxmaGF0IiwxLHsic3R5bGUiOnsiYm9keSI6eyJuYW1lIjoiZGFzaGVkIn19fV1d
  \[\begin{tikzcd}
      {A \times C} \\
      & {A \times B^{A}} && B \\
      \\
      & {A \times P(A \times B)} && PB
      \arrow["r"', from=4-2, to=4-4]
      \arrow["{\Set{}}", from=2-4, to=4-4]
      \arrow[tail, from=2-2, to=4-2]
      \arrow["\ev", from=2-2, to=2-4]
      \arrow["\lrcorner"{anchor=center, pos=0.125}, draw=none, from=2-2, to=4-4]
      \arrow["f", curve={height=-12pt}, from=1-1, to=2-4]
      \arrow["{\id_{A} \times g}"', curve={height=12pt}, from=1-1, to=4-2]
      \arrow["{\id_{A} \times \fhat}"{description}, dashed, from=1-1, to=2-2]
    \end{tikzcd}\]
\end{proof}

\begin{thm}
  In a topos, every monomorphism is an equalizer and every mono and epi morphism is an isomorphism.
\end{thm}
\begin{proof}
  The second statement follows from the first.
  Let $\Omega_{t}$ be the equalizer of $\id_{\Omega}$ and $t!_{\Omega}$.
  % https://q.uiver.app/#q=WzAsMyxbMCwwLCJcXE9tZWdhX3t0fSJdLFsyLDAsIlxcT21lZ2EiXSxbNCwwLCJcXE9tZWdhIl0sWzEsMiwidCFfe1xcT21lZ2F9IiwwLHsib2Zmc2V0IjotMX1dLFsxLDIsIlxcaWRfe1xcT21lZ2F9IiwyLHsib2Zmc2V0IjoxfV0sWzAsMSwiIiwyLHsic3R5bGUiOnsidGFpbCI6eyJuYW1lIjoibW9ubyJ9fX1dXQ==
  \[\begin{tikzcd}
      {\Omega_{t}} && \Omega && \Omega
      \arrow["{t!_{\Omega}}", shift left, from=1-3, to=1-5]
      \arrow["{\id_{\Omega}}"', shift right, from=1-3, to=1-5]
      \arrow[tail, from=1-1, to=1-3]
    \end{tikzcd}\]
  It's easy to verify that $A \mono B$ is the equalizer of $\chi_{A}$ and $t!_{\Omega}\chi_{A}$.
  Note that $!_{\Omega}\chi_{A} = !_{B}$ because $1$ is the terminal object.
\end{proof}

\begin{rmk}
  $\Omega_{t}$ classifies maximal subobjects.
  In $\CSet$, $\Omega_{t}$ coincides with the terminal object $1$, so $t_{t} : 1 \to \Omega_{t}$ can only be pulled back along the unique map $! : A \to \Omega_{t}$, giving the maximal subobject $A \subseteq A$.
\end{rmk}

\begin{cor}
  Every topos is balanced.
\end{cor}

\begin{defn}
  Let $\cE$ and $\cF$ be topoi.
  A \emph{logical functor} $F : \cE \to \cF$ is a functor that preserves finite limits, the subobject classifier, and exponentials.
\end{defn}

\begin{defn}
  Let $\cE$ and $\cF$ be topoi.
  A \emph{geometric morphism} $f : \cF \to \cE$ consists of a pair of functors
  % https://q.uiver.app/#q=WzAsMixbMCwwLCJcXGNGIl0sWzIsMCwiXFxjRSJdLFswLDEsImZfeyp9IiwyLHsiY3VydmUiOjJ9XSxbMSwwLCJmXnsqfSIsMix7ImN1cnZlIjoyfV0sWzMsMiwiIiwyLHsibGV2ZWwiOjEsInN0eWxlIjp7Im5hbWUiOiJhZGp1bmN0aW9uIn19XV0=
  \[\begin{tikzcd}
      \cF && \cE
      \arrow[""{name=0, anchor=center, inner sep=0}, "{f_{*}}"', curve={height=12pt}, from=1-1, to=1-3]
      \arrow[""{name=1, anchor=center, inner sep=0}, "{f^{*}}"', curve={height=12pt}, from=1-3, to=1-1]
      \arrow["\dashv"{anchor=center, rotate=-90}, draw=none, from=1, to=0]
    \end{tikzcd}\]
  such that $f^{*}$ preserves finite limits.
  The functor $f_{*}$ is called the \emph{direct image} of $f$ and $f^{*}$ is the \emph{inverse image} of $f$.
  If $f$ and $g$ are geometric morphism, a \emph{geometric transformation} $\alpha : f \to g$ is a natural transformation $f^{*} \To g^{*}$.
\end{defn}

\begin{defn}
  We write $\CTopoi$ for the 2-category of topoi, geometric morphisms, and geometric transformations.
\end{defn}

\begin{eg}\label{eg:geometric-morphisms-between-presheaves}
  Let $F : \iC \to \iD$ be a functor between small categories.
  The functor $f^{*}$ defined by precomposition has a left and right adjoints.
  The right Kan extension is the direct image, while $f^{*}$ is the inverse image.
  % https://q.uiver.app/#q=WzAsMixbMCwwLCJcXFBzaChcXGlEKSJdLFsyLDAsIlxcUHNoKFxcaUMpIl0sWzAsMSwiZl57Kn0iLDAseyJvZmZzZXQiOi0yfV0sWzEsMCwiXFxyYW5fe2Z9IiwwLHsib2Zmc2V0IjotMn1dLFsyLDMsIiIsMix7ImxldmVsIjoxLCJzdHlsZSI6eyJuYW1lIjoiYWRqdW5jdGlvbiJ9fV1d
  \[\begin{tikzcd}
      {\Psh(\iD)} && {\Psh(\iC)}
      \arrow[""{name=0, anchor=center, inner sep=0}, "{f^{*}}", shift left=2, from=1-1, to=1-3]
      \arrow[""{name=1, anchor=center, inner sep=0}, "{\ran_{f}}", shift left=2, from=1-3, to=1-1]
      \arrow["\dashv"{anchor=center, rotate=-90}, draw=none, from=0, to=1]
    \end{tikzcd}\]
  Thus, the assignment $\iC \mapsto \Psh(\iC)$ can be extended to a pseudofunctor $\CCat \to \CTopoi$.
\end{eg}

\begin{defn}
  A geometric morphism $f : \cF \to \cE$ is \emph{essential} if the inverse image $f^{*} : \cE \to \cF$ has a left adjoint.
  We write $f_{!}$ for the left adjoint of $f^{*}$.
\end{defn}

\todo{It appears a lot of the ``dirty work'' we did for algebraic theories can be nicely explained by examples in Elephant A.4.1.}
\todo{This appears to be a connection between geometry and algebra.}

\begin{lem}
  Let $\iC$ and $\iD$ be small categories such that $\iD$ is Cauchy-complete.
  Then every essential geometric morphism $f : \Psh(\iC) \to \Psh(\iD)$ is induced by a functor $\iC \to \iD$.
\end{lem}
\begin{proof}
  Let $g : \iC \to \iD$ be a functor.
  By \cref{eg:geometric-morphisms-between-presheaves}, this functor induces an essential geometric morphism defined by precomposition.

  Conversely, let $f$ be an essential geometric morphism.
  We want to show that $f$ is induced by a functor $g : \iC \to \iD$ by precomposition.
  Note that representable functors are tiny in the presheaf category and $f^{*}$ preserves colimits.
  Owing to the following natural isomorphism,
  \[
    \Psh(\iD)(f_{!}\iC(\blank, c), \blank) \iso \Psh(\iC)(\iC(\blank,c), f^{*}\blank)
  \]
  we see that $\Psh(\iD)(f_{!}\iC(\blank, c), \blank)$ preserves sifted colimits.
  Thus, $f_{!}\iC(\blank,c)$ is perfectly presentable and therefore it is a retract of a representable.
  Since $\iD$ is Cauchy-complete, $f_{!}\iC(\blank, c)$ is representable.
  Thus, $f_{!}$ restricts along the Yoneda embedding to a functor $f_{0} : \iC \to \iD$.
  Indeed, $f$ is induced by $f_{0}$.
  \begin{align}
    f^{*}(F)(c) &\iso \Psh(\iC)(\iC(\blank, c), f^{*}F)\\
                &\iso \Psh(\iD)(f_{!}\iC(\blank, c), F)\\
                &\iso \Psh(\iD)(\iD(\blank, f_{0}c), F)\\
                &\iso Ff_{0}c
  \end{align}
\end{proof}

\begin{lem}\label{lem:cartesian-closed-iff-pullbacks}
  A category $\iC$ is locally cartesian iff it has pullbacks.
\end{lem}
\begin{proof}
  Let $c \in \iC$ be any object.
  The forgetful functor $\Pi : \iC/c \to \iC$ preserves (and creates) connected limits.
  Thus, $\iC/c$ has pullbacks whenever $\iC$ does.
  Since $\iC/c$ has a terminal object, it is finitely complete if $\iC$ has pullbacks.
  Conversely, if $\iC$ is locally cartesian, then pullbacks in $\iC$ can be constructed as products in a slice $\iC/c$ for some $c \in \iC$.
\end{proof}

\begin{cor}\label{cor:topos-locally-cartesian}
  Every topos is locally cartesian.
\end{cor}

\begin{rmk}
  Let $\iC$ be a category and $f : a \to b$ be a morphism of $\iC$.
  By unfolding the definition of slice category, we see that $\iC/a \iso (\iC/b)/f$.
  Thus, there is a canonical functor $\Sigma_{f} : \iC/a \to \iC/b$ given by the forgetful functor $\Pi : (\iC/b)/f \iso \iC/a \to \iC/b$.
  Explicitly, $\Sigma_{f} : g \mapsto fg$.
\end{rmk}

\begin{lem}
  A category $\iC$ is locally cartesian closed iff $\Sigma_{f}$ has a right adjoint $f^{*}$ for all $f$.
\end{lem}
\begin{proof}
  By \cref{lem:cartesian-closed-iff-pullbacks}, being locally cartesian closed is the same as having pullbacks.
  If $\iC$ has pullbacks, then take $f^{*}$ to be the pullback functor.
  The universal property of pullbacks implies that $\Sigma_{f} \adj f^{*}$.
  % https://q.uiver.app/#q=WzAsNSxbMSwzLCJhIl0sWzMsMywiYiJdLFszLDEsImMiXSxbMSwxLCJcXGJ1bGxldCJdLFswLDAsIlxcYnVsbGV0Il0sWzAsMSwiZiIsMl0sWzIsMSwiaCJdLFszLDAsImZeeyp9aCIsMl0sWzMsMl0sWzQsMCwiZyIsMix7ImN1cnZlIjoyfV0sWzQsMiwiIiwyLHsiY3VydmUiOi0yfV0sWzQsMywiIiwyLHsic3R5bGUiOnsiYm9keSI6eyJuYW1lIjoiZGFzaGVkIn19fV0sWzMsMSwiIiwyLHsic3R5bGUiOnsibmFtZSI6ImNvcm5lciJ9fV1d
  \[\begin{tikzcd}
      \bullet \\
      & f \times h && c \\
      \\
      & a && b
      \arrow["f"', from=4-2, to=4-4]
      \arrow["h", from=2-4, to=4-4]
      \arrow["{f^{*}h}"', from=2-2, to=4-2]
      \arrow[from=2-2, to=2-4]
      \arrow["g"', curve={height=12pt}, from=1-1, to=4-2]
      \arrow[curve={height=-12pt}, from=1-1, to=2-4]
      \arrow[dashed, from=1-1, to=2-2]
      \arrow["\lrcorner"{anchor=center, pos=0.125}, draw=none, from=2-2, to=4-4]
    \end{tikzcd}\]
  Conversely, if $\Sigma_{f} \adj f^{*}$ for all $f$, then we can define pullbacks as follows.
  % https://q.uiver.app/#q=WzAsNCxbMCwyLCJhIl0sWzIsMiwiYiJdLFsyLDAsImMiXSxbMCwwLCJcXGJ1bGxldCJdLFswLDEsImYiLDJdLFsyLDEsImgiXSxbMywwLCJmXnsqfWgiLDJdLFszLDIsIlxcZXBzaWxvbl97aH0iXSxbMywxLCIiLDIseyJzdHlsZSI6eyJuYW1lIjoiY29ybmVyIn19XV0=
  \[\begin{tikzcd}
      f \times h && c \\
      \\
      a && b
      \arrow["f"', from=3-1, to=3-3]
      \arrow["h", from=1-3, to=3-3]
      \arrow["{f^{*}h}"', from=1-1, to=3-1]
      \arrow["{\epsilon_{h}}", from=1-1, to=1-3]
      \arrow["\lrcorner"{anchor=center, pos=0.125}, draw=none, from=1-1, to=3-3]
    \end{tikzcd}\]
\end{proof}

\begin{cor}
  Let $\cE$ be a topos.
  $\Sigma_{f}$ has a right adjoint $f^{*}$ for all morphisms $f$ in $\cE$.
\end{cor}

\begin{notn}
  In a topos, we write $\Sigma_{B}$ for $\Sigma_{!_{B}} : \cE/B \to \cE/1 \iso \cE$ and $B^{*}$ for $!_{B}^{*} : \cE/1 \iso \cE \to \cE/B$.
\end{notn}

\begin{thm}
  Let $\cE$ be a topos and $B$ be an object of $\cE$.
  Then $\cE/B$ is a topos, and the pullback functor $B^{*} : \cE \to \cE/B$ is logical.
\end{thm}
\begin{proof}
  First, for each object $A \in \cE$, we claim that $B^{*}(PA)$ equipped with the relation $B^{*}(\in_{A})$ is a power object of $B^{*}A$ in $\cE/B$.
  Let $f : C \to B$ be any object in $\cE/B$.
  The product $f \times B^{*}A$ is constructed as the following pullback in $\cE$.
  % https://q.uiver.app/#q=WzAsNCxbMCwyLCJDIl0sWzIsMiwiQiJdLFsyLDAsIkIgXFx0aW1lcyBBIl0sWzAsMCwiXFxidWxsZXQiXSxbMiwxLCJCXnsqfUEgPSBcXHBpX3sxfSJdLFswLDEsImYiLDJdLFszLDBdLFszLDJdLFszLDEsIiIsMSx7InN0eWxlIjp7Im5hbWUiOiJjb3JuZXIifX1dXQ==
  \[\begin{tikzcd}
      \bullet && {B \times A} \\
      \\
      C && B
      \arrow["{B^{*}A = \pi_{1}}", from=1-3, to=3-3]
      \arrow["f"', from=3-1, to=3-3]
      \arrow[from=1-1, to=3-1]
      \arrow[from=1-1, to=1-3]
      \arrow["\lrcorner"{anchor=center, pos=0.125}, draw=none, from=1-1, to=3-3]
    \end{tikzcd}\]
  Note that the following square is also a pullback.
  % https://q.uiver.app/#q=WzAsNCxbMCwyLCJDIl0sWzIsMiwiQiJdLFsyLDAsIkIgXFx0aW1lcyBBIl0sWzAsMCwiQyBcXHRpbWVzIEEiXSxbMiwxLCJcXHBpX3sxfSJdLFswLDEsImYiLDJdLFszLDAsIlxccGlfezF9IiwyXSxbMywyLCJmIFxcdGltZXMgXFxpZF97QX0iXSxbMywxLCIiLDEseyJzdHlsZSI6eyJuYW1lIjoiY29ybmVyIn19XV0=
  \[\begin{tikzcd}
      {C \times A} && {B \times A} \\
      \\
      C && B
      \arrow["{\pi_{1}}", from=1-3, to=3-3]
      \arrow["f"', from=3-1, to=3-3]
      \arrow["{\pi_{1}}"', from=1-1, to=3-1]
      \arrow["{f \times \id_{A}}", from=1-1, to=1-3]
      \arrow["\lrcorner"{anchor=center, pos=0.125}, draw=none, from=1-1, to=3-3]
    \end{tikzcd}\]
  Thus, $\Sigma_{B}(f \times B^{*}A) \iso C \times A$.
  This means that there is a bijection between the subobjects of $f \times B^{*}A$ in $\cE/B$ and those of $C \times A$ in $\cE$.
  % https://q.uiver.app/#q=WzAsMyxbMCwwLCJDIFxcdGltZXMgQSJdLFsxLDEsIkIiXSxbMiwwLCJcXGJ1bGxldCJdLFswLDEsImYgXFx0aW1lcyBCXnsqfUEiLDJdLFsyLDAsIiIsMCx7InN0eWxlIjp7InRhaWwiOnsibmFtZSI6Im1vbm8ifX19XSxbMiwxXV0=
  \[\begin{tikzcd}
      {C \times A} && \bullet \\
      & B
      \arrow["{f \times B^{*}A}"', from=1-1, to=2-2]
      \arrow[tail, from=1-3, to=1-1]
      \arrow[from=1-3, to=2-2]
    \end{tikzcd}\]
  Thus, every relation from $f$ to $B^{*}A$ in $\cE/B$ may be regarded as a relation from $C$ to $A$ in $\cE$.
  Let $g : C \to PA$ be the name of such a relation.
  Then $(f, g) : f \to B^{*}(PA)$ is the unique name for the same relation in $\cE/B$.
  % https://q.uiver.app/#q=WzAsNSxbMCwwLCJDIl0sWzEsMSwiQiBcXHRpbWVzIFBBIl0sWzEsMywiQiJdLFszLDMsIjEiXSxbMywxLCJQQSJdLFsxLDRdLFsxLDIsIkJeeyp9KFBBKSJdLFsyLDNdLFs0LDNdLFsxLDMsIiIsMSx7InN0eWxlIjp7Im5hbWUiOiJjb3JuZXIifX1dLFswLDIsImYiLDIseyJjdXJ2ZSI6Mn1dLFswLDQsImciLDAseyJjdXJ2ZSI6LTJ9XSxbMCwxLCIiLDAseyJzdHlsZSI6eyJib2R5Ijp7Im5hbWUiOiJkYXNoZWQifX19XV0=
  \[\begin{tikzcd}
      C \\
      & {B \times PA} && PA \\
      \\
      & B && 1
      \arrow[from=2-2, to=2-4]
      \arrow["{B^{*}(PA)}", from=2-2, to=4-2]
      \arrow[from=4-2, to=4-4]
      \arrow[from=2-4, to=4-4]
      \arrow["\lrcorner"{anchor=center, pos=0.125}, draw=none, from=2-2, to=4-4]
      \arrow["f"', curve={height=12pt}, from=1-1, to=4-2]
      \arrow["g", curve={height=-12pt}, from=1-1, to=2-4]
      \arrow[dashed, from=1-1, to=2-2]
    \end{tikzcd}\]
  Since every topos is locally cartesian (c.f., \cref{cor:topos-locally-cartesian}), it suffices to construct the subobject classifier.
  This is immediate because we have shown that $B^{*}$ preserves power objects (and therefore subobject classifier).
  \todo{I forgot to construct exponentials.}
\end{proof}

\begin{defn}
  In a category $\iC$, an object $c$ is \emph{injective} when for every monomorphism $m : d \mono d'$ in $\iC$, every morphism $f : d \to c$ can be extended to a morphism $f' : d' \to c$ so that $f = f'm$.
\end{defn}

\begin{thm}\label{thm:subobject-classifier-injective}
  In a topos $\cE$, the subobject classifier is injective.
\end{thm}
\begin{proof}
  Let $m : d \mono d'$ and $f : d \to \Omega$.
  The morphism $f$ determines a pullback square in $\cE$.
  % https://q.uiver.app/#q=WzAsNCxbMCwyLCJkIl0sWzIsMiwiXFxPbWVnYSJdLFswLDAsImMiXSxbMiwwLCIxIl0sWzAsMSwiZiIsMl0sWzIsMCwiaCIsMix7InN0eWxlIjp7InRhaWwiOnsibmFtZSI6Im1vbm8ifX19XSxbMiwzXSxbMywxLCJ0IiwwLHsic3R5bGUiOnsidGFpbCI6eyJuYW1lIjoibW9ubyJ9fX1dXQ==
  \[\begin{tikzcd}
      c && 1 \\
      \\
      d && \Omega
      \arrow["f"', from=3-1, to=3-3]
      \arrow["h"', tail, from=1-1, to=3-1]
      \arrow[from=1-1, to=1-3]
      \arrow["t", tail, from=1-3, to=3-3]
    \end{tikzcd}\]
  The composite $mh : c \mono d'$ is a subobject of $d'$ classified by a morphism $g : d' \to \Omega$.
  % https://q.uiver.app/#q=WzAsNCxbMCwyLCJkJyJdLFsyLDIsIlxcT21lZ2EiXSxbMCwwLCJjIl0sWzIsMCwiMSJdLFswLDEsImciLDJdLFsyLDAsIm1oIiwyLHsic3R5bGUiOnsidGFpbCI6eyJuYW1lIjoibW9ubyJ9fX1dLFsyLDNdLFszLDEsInQiLDAseyJzdHlsZSI6eyJ0YWlsIjp7Im5hbWUiOiJtb25vIn19fV1d
  \[\begin{tikzcd}
      c && 1 \\
      \\
      {d'} && \Omega
      \arrow["g"', from=3-1, to=3-3]
      \arrow["mh"', tail, from=1-1, to=3-1]
      \arrow[from=1-1, to=1-3]
      \arrow["t", tail, from=1-3, to=3-3]
    \end{tikzcd}\]
  As $gm$ and $f$ classify the same subobject of $d$, the universal property of subobject classifiers demands that $gm = f$.
\end{proof}

\begin{cor}\label{cor:power-objects-injective}
  In a topos $\cE$, every power object $PA$ is injective.
\end{cor}
\begin{proof}
  For any monomorphism $m : d \to d'$ and any morphism $f : d \to PA$, consider the transpose $f^{\dagger} : A \times d \to \Omega$.
  By \cref{thm:subobject-classifier-injective}, $f^{\dagger}$ extends to a morphism $g^{\dagger} : A \times d' \to \Omega$.
  % https://q.uiver.app/#q=WzAsMyxbMiwwLCJcXE9tZWdhIl0sWzAsMCwiQSBcXHRpbWVzIGQiXSxbMCwyLCJBIFxcdGltZXMgZCciXSxbMSwwLCJmXntcXGRhZ2dlcn0iXSxbMSwyLCJcXGlkX3tBfSBcXHRpbWVzIG0iLDIseyJzdHlsZSI6eyJ0YWlsIjp7Im5hbWUiOiJtb25vIn19fV0sWzIsMCwiZ157XFxkYWdnZXJ9IiwyXV0=
  \[\begin{tikzcd}
      {A \times d} && \Omega \\
      \\
      {A \times d'}
      \arrow["{f^{\dagger}}", from=1-1, to=1-3]
      \arrow["{\id_{A} \times m}"', tail, from=1-1, to=3-1]
      \arrow["{g^{\dagger}}"', from=3-1, to=1-3]
    \end{tikzcd}\]
  Thus, $g : d' \to PA$ is the required extension of $f$.
\end{proof}

\begin{cor}
  Every object $c$ in a topos $\cE$ has a monomorphism to an injective object.
\end{cor}
\begin{proof}
  The singleton morphism $\Set{}_{c} : c \mono Pc$ is a monomorphism.
  By \cref{cor:power-objects-injective}, $Pc$ is an injective object.
\end{proof}

\begin{cor}\label{cor:coproducts-disjoint}
  Let $X \xto{i} X + Y \xot{j} Y$ be a coproduct diagram in a topos $\cE$.
  Then $i$ and $j$ are monomorphisms and the following diagram is a pullback.
  % https://q.uiver.app/#q=WzAsNCxbMCwyLCJYIl0sWzIsMiwiWCtZIl0sWzIsMCwiWSJdLFswLDAsIjAiXSxbMCwxLCJpIiwyXSxbMiwxLCJqIl0sWzMsMF0sWzMsMl1d
  \[\begin{tikzcd}
      0 && Y \\
      \\
      X && {X+Y}
      \arrow["i"', from=3-1, to=3-3]
      \arrow["j", from=1-3, to=3-3]
      \arrow[from=1-1, to=3-1]
      \arrow[from=1-1, to=1-3]
    \end{tikzcd}\]
\end{cor}
\begin{proof}
  Suppose that $if = ig$.
  Then $\Set{}_{X}f = [\Set{}_{X},\Set{}_{Y}]if = [\Set{}_{X},\Set{}_{Y}]ig = \Set{}_{X}g$.
  Since $\Set{}_{X}$ is monic, $f = g$.
  The proof is completely analogous for $j$.
  % https://q.uiver.app/#q=WzAsNCxbMCwwLCJYIl0sWzIsMCwiWCtZIl0sWzIsMiwiUFgiXSxbNCwwLCJZIl0sWzAsMSwiaSJdLFswLDIsIlxcU2V0e31fe1h9IiwyLHsic3R5bGUiOnsidGFpbCI6eyJuYW1lIjoibW9ubyJ9fX1dLFsxLDIsIiIsMCx7InN0eWxlIjp7ImJvZHkiOnsibmFtZSI6ImRhc2hlZCJ9fX1dLFszLDEsImoiLDJdLFszLDIsIlxcU2V0e31fe1l9IiwwLHsic3R5bGUiOnsidGFpbCI6eyJuYW1lIjoibW9ubyJ9fX1dXQ==
  \[\begin{tikzcd}
      X && {X+Y} && Y \\
      \\
      && PX
      \arrow["i", from=1-1, to=1-3]
      \arrow["{\Set{}_{X}}"', tail, from=1-1, to=3-3]
      \arrow[dashed, from=1-3, to=3-3]
      \arrow["j"', from=1-5, to=1-3]
      \arrow["{\Set{}_{Y}}", tail, from=1-5, to=3-3]
    \end{tikzcd}\]
  Since $i$ and $j$ are monomorphism, it is easy to verify that the square is a pullback.
\end{proof}

\cref{cor:coproducts-disjoint} means that coproducts in a topos are disjoint.

\begin{defn}[Gluing construction]
  Let $F : \cE \to \cF$ be a finite-limit preserving functor between topoi.
  The \emph{glued topos}, $\Gl(F)$, is the comma category $(\cF \dn F)$.
  The comma category $(\cF \dn F)$ is indeed a topos.
\end{defn}

\begin{defn}[Filter quotient construction]
  Let $\cE$ be a topos, and $\Phi$ be a filter of subterminal objects.
  A \emph{$\Phi$-map} $A \phito B$ in $\cE$ is a morphism $A \times U \to B$ for some $U \in \Phi$.
  Two $\Phi$-maps $f : A \times U \to B$ and $g : A \times V \to B$ are \emph{equivalent} if there is $W \leq U \cap V$ in $\Phi$ so that
  % https://q.uiver.app/#q=WzAsNCxbMCwyLCJBIFxcdGltZXMgVSJdLFsyLDIsIkIiXSxbMiwwLCJBIFxcdGltZXMgViJdLFswLDAsIkEgXFx0aW1lcyBXIl0sWzAsMSwiZiIsMl0sWzIsMSwiZyJdLFszLDIsIiIsMCx7InN0eWxlIjp7InRhaWwiOnsibmFtZSI6Im1vbm8ifX19XSxbMywwLCIiLDIseyJzdHlsZSI6eyJ0YWlsIjp7Im5hbWUiOiJtb25vIn19fV1d
  \[\begin{tikzcd}
      {A \times W} && {A \times V} \\
      \\
      {A \times U} && B
      \arrow["f"', from=3-1, to=3-3]
      \arrow["g", from=1-3, to=3-3]
      \arrow[tail, from=1-1, to=1-3]
      \arrow[tail, from=1-1, to=3-1]
    \end{tikzcd}\]
  commutes.
  For two $\Phi$-maps $f : A \times U \to B$ and $g : B \times V \to C$, their composition is $g \circ (f \times \id_{V})$.
  The topos $\cE_{\Phi}$ consists of objects of $\cE$ as objects and equivalence classes of $\Phi$-maps as morphisms.
  The functor $P_{\Phi} : \cE \to \cE_{\Phi}$, which is identity on objects and sends each morphism to the equivalence class of $f$ regarded as $A \times 1 \to B$, is logical.
\end{defn}

% \section{Grothendieck topologies}
% \label{sec:grothendieck-topologies}

% \begin{defn}
%   A \emph{Grothendieck topology} on a category $\iC$ is a function $J$ assigning to each object $c$ a collection $J(c)$ of covering sieves on $c$ so that
%   \begin{enumerate}
%   \item For any object $c$, the maximal sieve is in $J(c)$.
%   \item (Stability) If $S$ is a sieve that covers $c$ and $g : d \to c$ is any morphism, then the pullback sieve $g^{*}S$ covers $d$.
%   \item (Local character) If $R$ is a covering sieve on $c$ and $S$ is another sieve on $c$ such that for all $f : d \to c \in R$, the pullback sieve $f^{*}S$ covers $d$, then $S$ covers $c$.
%   \end{enumerate}
% \end{defn}

% \begin{rmk}
%   Stability means the assignment $J$ extends to a functor $J : \iC\op \to \CSet$, i.e., an object in the presheaf category $\Psh(\iC)$.
%   The subobject classifier $\Omega$ in $\Psh(\iC)$ is the functor $\Sub(\yon\blank)$ sending each $c \in \iC$ to the set of sieves on $c$.
%   Thus, $J$ (a functor sending each $c$ to a set of covering sieves on $c$) is a subobject of $\Omega$.
% \end{rmk}

\section{Lawvere-Tierney topologies}
\label{sec:lawvere-tierney-topologies}

Let $X$ be a topological space.
A sheaf (of sets) is an assignment $\cO(X)\op \to \CSet$ so that a global section can be uniquely determined by gluing compatible local sections according to the topology of $X$.

Note that the subobject classifier in $\Psh(\cO(X))$ is the functor $\Sub\yon : \cO(X)\op \to \CSet$, assigning to each open set $U$ the set of all subfunctors $R \subseteq \cO(X)(\blank, U)$.
By definition, $\cO(X)(V,U) = 1$ if $V \subseteq U$, so $R(V) = 1$ necessarily implies that $V \subseteq U$, i.e., $\cO(X)(V, U) = 1$.
If additionally $V' \subseteq V$, then $R(V') = 1$ because we have $R(V) \to R(V')$.
Thus, we can view $R$ as a filter on $U$, i.e., a downward closed set of open subsets of $U$.

A family $\Set{U_{i}}_{i \in I}$ is an open cover of $U$ if $\bigcup_{i \in I}U_{i} = U$.
To formalize this idea, we define a natural transformation
\begin{mathpar}
  j : \Omega \to \Omega \and j_{U}(R) = \bigcup R
\end{mathpar}
For $V \subseteq U$, we have
\[
  \Omega(V \subseteq U)(j_{U}(R)) = \left(\bigcup R\right) \cap V = \bigcup(R \cap V) = j_{V}(\Omega(V \subseteq U)(R))
\]
The component $j_{U} : \Omega(U) \to \Omega(U)$ maps each filter on $U$ to the open subset that it covers.

The pullback of $t : 1 \to \Omega$ along $j$ determines a subobject $J$.
% https://q.uiver.app/#q=WzAsNCxbMCwwLCJKIl0sWzAsMiwiXFxPbWVnYSJdLFsyLDIsIlxcT21lZ2EiXSxbMiwwLCIxIl0sWzEsMiwiaiIsMl0sWzAsMSwiIiwyLHsic3R5bGUiOnsidGFpbCI6eyJuYW1lIjoibW9ubyJ9fX1dLFszLDIsInQiLDAseyJzdHlsZSI6eyJ0YWlsIjp7Im5hbWUiOiJtb25vIn19fV0sWzAsM10sWzAsMiwiIiwxLHsic3R5bGUiOnsibmFtZSI6ImNvcm5lciJ9fV1d
\[\begin{tikzcd}
    J && 1 \\
    \\
    \Omega && \Omega
    \arrow["j"', from=3-1, to=3-3]
    \arrow[tail, from=1-1, to=3-1]
    \arrow["t", tail, from=1-3, to=3-3]
    \arrow[from=1-1, to=1-3]
    \arrow["\lrcorner"{anchor=center, pos=0.125}, draw=none, from=1-1, to=3-3]
  \end{tikzcd}\]
By evaluating each functor at a given open set $U$ yields a pullback in $\CSet$, which is the fiber product of the functions $j_{U}$ and $t_{U}$, so the pullback is nothing but the set
\[
  J(U) = \Set{R \mono \Omega(U) \mid j_{U}(R) = U}
\]
By definition, $j_{U}(R) = \bigcup R$.
Thus, $J(U)$ is the set of all open covers of $U$.

\begin{defn}\label{def:lawvere-tierney-topology}
  A \emph{Lawvere-Tierney topology} (topology for short) on a topos $\cE$ is a morphism $j : \Omega \to \Omega$ satisfying
  % https://q.uiver.app/#q=WzAsMTAsWzAsMCwiMSJdLFsxLDAsIlxcT21lZ2EiXSxbMSwxLCJcXE9tZWdhIl0sWzIsMCwiXFxPbWVnYSJdLFszLDAsIlxcT21lZ2EiXSxbMywxLCJcXE9tZWdhIl0sWzQsMCwiXFxPbWVnYSBcXHRpbWVzIFxcT21lZ2EiXSxbNSwwLCJcXE9tZWdhIl0sWzUsMSwiXFxPbWVnYSJdLFs0LDEsIlxcT21lZ2EgXFx0aW1lcyBcXE9tZWdhIl0sWzAsMSwidCJdLFsxLDIsImoiXSxbMCwyLCJ0IiwyXSxbMyw0LCJqIl0sWzQsNSwiaiJdLFszLDUsImoiLDJdLFs2LDcsIlxcd2VkZ2UiXSxbOSw4LCJcXHdlZGdlIiwyXSxbNiw5LCJqIiwyXSxbNyw4LCJqIl1d
  \[\begin{tikzcd}
      1 & \Omega & \Omega & \Omega & {\Omega \times \Omega} & \Omega \\
      & \Omega && \Omega & {\Omega \times \Omega} & \Omega
      \arrow["t", from=1-1, to=1-2]
      \arrow["j", from=1-2, to=2-2]
      \arrow["t"', from=1-1, to=2-2]
      \arrow["j", from=1-3, to=1-4]
      \arrow["j", from=1-4, to=2-4]
      \arrow["j"', from=1-3, to=2-4]
      \arrow["\wedge", from=1-5, to=1-6]
      \arrow["\wedge"', from=2-5, to=2-6]
      \arrow["j \times j"', from=1-5, to=2-5]
      \arrow["j", from=1-6, to=2-6]
    \end{tikzcd}\]
\end{defn}

\begin{rmk}
  The morphism $j : \Omega \to \Omega$ classifies a subobject of $\Omega$.
  % https://q.uiver.app/#q=WzAsNCxbMCwwLCJKIl0sWzIsMCwiMSJdLFsyLDIsIlxcT21lZ2EiXSxbMCwyLCJcXE9tZWdhIl0sWzMsMiwiaiIsMl0sWzEsMiwidCIsMCx7InN0eWxlIjp7InRhaWwiOnsibmFtZSI6Im1vbm8ifX19XSxbMCwxXSxbMCwzLCIiLDIseyJzdHlsZSI6eyJ0YWlsIjp7Im5hbWUiOiJtb25vIn19fV0sWzAsMiwiIiwxLHsic3R5bGUiOnsibmFtZSI6ImNvcm5lciJ9fV1d
  \[\begin{tikzcd}
      J && 1 \\
      \\
      \Omega && \Omega
      \arrow["j"', from=3-1, to=3-3]
      \arrow["t", tail, from=1-3, to=3-3]
      \arrow[from=1-1, to=1-3]
      \arrow[tail, from=1-1, to=3-1]
      \arrow["\lrcorner"{anchor=center, pos=0.125}, draw=none, from=1-1, to=3-3]
    \end{tikzcd}\]
  Thus, we can phrase \cref{def:lawvere-tierney-topology} in terms of this subobject.
\end{rmk}

\begin{defn}
  Let $\cE$ be a topos equipped with a topology $j$.
  Let $A$ be a subobject of $B$ classified by $\chi_{A}$.
  The composite $j \circ \chi_{A}$ determines another subobject $\Abar$ of $B$, which we call the \emph{closure} of $A$.
\end{defn}

\begin{thm}\label{thm:closure-operator-in-topos}
  Let $\cE$ be a topos, a morphism $j : \Omega \to \Omega$ induces an operator $E \mapsto \Ebar$ on the subobjects of each object $E$ of $\cE$ natural in its argument in the sense that for any $f : A \to B$ in $\cE$ and any subobject $C$ of $B$, one has
  \[
    f^{*}(\Cbar) = \overline{f^{*}C}
  \]
  Moreover, $j$ is a topology if and only if the operator satisfies
  \begin{mathpar}
    A \subseteq \Abar \and \overline{\Abar} = \Abar \and \overline{A \cap B} = \Abar \cap \Bbar
  \end{mathpar}
  for all objects.
\end{thm}
\begin{proof}
  By definition, $f^{*}(\Cbar)$ is the pullback of $\Cbar$ along $f$.
  Thus, $f^{*}(\Cbar)$ is a subobject classified by $j \circ \chi_{C} \circ f$, which by definition classifies the subobject $\overline{f^{*}C}$.
  % https://q.uiver.app/#q=WzAsNixbMiwwLCJcXENiYXIiXSxbMiwyLCJCIl0sWzQsMiwiXFxPbWVnYSJdLFs0LDAsIjEiXSxbMCwyLCJBIl0sWzAsMCwiZl57Kn0oXFxDYmFyKSJdLFswLDEsIiIsMix7InN0eWxlIjp7InRhaWwiOnsibmFtZSI6Im1vbm8ifX19XSxbMSwyLCJqIFxcY2lyYyBcXGNoaV97Q30iLDJdLFszLDIsInQiXSxbMCwzXSxbNCwxLCJmIiwyXSxbNSw0LCIiLDIseyJzdHlsZSI6eyJ0YWlsIjp7Im5hbWUiOiJtb25vIn19fV0sWzUsMF0sWzUsMSwiIiwyLHsic3R5bGUiOnsibmFtZSI6ImNvcm5lciJ9fV0sWzAsMiwiIiwyLHsic3R5bGUiOnsibmFtZSI6ImNvcm5lciJ9fV1d
  \[\begin{tikzcd}
      {f^{*}(\Cbar)} && \Cbar && 1 \\
      \\
      A && B && \Omega
      \arrow[tail, from=1-3, to=3-3]
      \arrow["{j \circ \chi_{C}}"', from=3-3, to=3-5]
      \arrow["t", from=1-5, to=3-5]
      \arrow[from=1-3, to=1-5]
      \arrow["f"', from=3-1, to=3-3]
      \arrow[tail, from=1-1, to=3-1]
      \arrow[from=1-1, to=1-3]
      \arrow["\lrcorner"{anchor=center, pos=0.125}, draw=none, from=1-1, to=3-3]
      \arrow["\lrcorner"{anchor=center, pos=0.125}, draw=none, from=1-3, to=3-5]
    \end{tikzcd}\]

  Now suppose that $j$ is a topology.
  The closure properties are can be proved by the following abstract nonsense.
  \begin{mathpar}
    % https://q.uiver.app/#q=WzAsNixbMSwyLCJCIl0sWzIsMiwiXFxPbWVnYSJdLFszLDIsIlxcT21lZ2EiXSxbMywxLCIxIl0sWzEsMSwiXFxBYmFyIl0sWzAsMCwiQSJdLFswLDEsIlxcY2hpX3tBfSIsMl0sWzEsMiwiaiIsMl0sWzMsMiwidCIsMCx7InN0eWxlIjp7InRhaWwiOnsibmFtZSI6Im1vbm8ifX19XSxbNCwwLCIiLDIseyJzdHlsZSI6eyJ0YWlsIjp7Im5hbWUiOiJtb25vIn19fV0sWzQsM10sWzUsMywiIiwwLHsiY3VydmUiOi0yfV0sWzUsMCwiIiwwLHsiY3VydmUiOjIsInN0eWxlIjp7InRhaWwiOnsibmFtZSI6Im1vbm8ifX19XSxbMywxLCJ0IiwyXSxbNSw0LCIiLDAseyJzdHlsZSI6eyJib2R5Ijp7Im5hbWUiOiJkYXNoZWQifX19XSxbNCwyLCIiLDAseyJzdHlsZSI6eyJuYW1lIjoiY29ybmVyIn19XV0=
    \begin{tikzcd}
      A \\
      & \Abar && 1 \\
      & B & \Omega & \Omega
      \arrow["{\chi_{A}}"', from=3-2, to=3-3]
      \arrow["j"', from=3-3, to=3-4]
      \arrow["t", tail, from=2-4, to=3-4]
      \arrow[tail, from=2-2, to=3-2]
      \arrow[from=2-2, to=2-4]
      \arrow[curve={height=-12pt}, from=1-1, to=2-4]
      \arrow[curve={height=12pt}, tail, from=1-1, to=3-2]
      \arrow["t"', from=2-4, to=3-3]
      \arrow[dashed, from=1-1, to=2-2]
      \arrow["\lrcorner"{anchor=center, pos=0.125}, draw=none, from=2-2, to=3-4]
    \end{tikzcd}\and
    % https://q.uiver.app/#q=WzAsNyxbMSwxLCJcXEFiYXIiXSxbMSwyLCJCIl0sWzIsMiwiXFxPbWVnYSJdLFszLDIsIlxcT21lZ2EiXSxbMywxLCIxIl0sWzQsMiwiXFxPbWVnYSJdLFswLDAsIlxcb3ZlcmxpbmV7XFxBYmFyfSJdLFswLDEsIiIsMCx7InN0eWxlIjp7InRhaWwiOnsibmFtZSI6Im1vbm8ifX19XSxbMSwyLCJcXGNoaV97QX0iLDJdLFsyLDMsImoiLDJdLFs0LDMsInQiLDAseyJzdHlsZSI6eyJ0YWlsIjp7Im5hbWUiOiJtb25vIn19fV0sWzAsNF0sWzMsNSwiaiIsMl0sWzIsNSwiaiIsMix7ImN1cnZlIjozfV0sWzYsMSwiIiwyLHsiY3VydmUiOjIsInN0eWxlIjp7InRhaWwiOnsibmFtZSI6Im1vbm8ifX19XSxbNCw1LCJ0IiwwLHsic3R5bGUiOnsidGFpbCI6eyJuYW1lIjoibW9ubyJ9fX1dLFs2LDQsIiIsMCx7ImN1cnZlIjotMn1dLFs2LDAsIlxcaXNvIiwzLHsic3R5bGUiOnsiYm9keSI6eyJuYW1lIjoibm9uZSJ9LCJoZWFkIjp7Im5hbWUiOiJub25lIn19fV1d
    \begin{tikzcd}
      {\overline{\Abar}} \\
      & \Abar && 1 \\
      & B & \Omega & \Omega & \Omega
      \arrow[tail, from=2-2, to=3-2]
      \arrow["{\chi_{A}}"', from=3-2, to=3-3]
      \arrow["j"', from=3-3, to=3-4]
      \arrow["t", tail, from=2-4, to=3-4]
      \arrow[from=2-2, to=2-4]
      \arrow["j"', from=3-4, to=3-5]
      \arrow["j"', curve={height=18pt}, from=3-3, to=3-5]
      \arrow[curve={height=12pt}, tail, from=1-1, to=3-2]
      \arrow["t", tail, from=2-4, to=3-5]
      \arrow[curve={height=-12pt}, from=1-1, to=2-4]
      \arrow["\iso"{marking, allow upside down}, draw=none, from=1-1, to=2-2]
    \end{tikzcd}\and
    % https://q.uiver.app/#q=WzAsNixbMSwxLCJcXG92ZXJsaW5le0EgXFxjYXAgQn0iXSxbMSwyLCJDIl0sWzMsMiwiXFxPbWVnYSJdLFszLDEsIjEiXSxbMCwwLCJcXEFiYXIgXFxjYXAgXFxCYmFyIl0sWzAsMiwiQyJdLFswLDEsIiIsMix7InN0eWxlIjp7InRhaWwiOnsibmFtZSI6Im1vbm8ifX19XSxbMCwzXSxbMywyLCJ0IiwwLHsic3R5bGUiOnsidGFpbCI6eyJuYW1lIjoibW9ubyJ9fX1dLFsxLDIsImpcXHdlZGdlKFxcY2hpX3tBfSxcXGNoaV97Qn0pIl0sWzQsNSwiIiwyLHsic3R5bGUiOnsidGFpbCI6eyJuYW1lIjoibW9ubyJ9fX1dLFs0LDMsIiIsMix7ImN1cnZlIjotMX1dLFs1LDIsIlxcd2VkZ2UoalxcY2hpX3tBfSxqXFxjaGlfe0J9KSIsMix7ImN1cnZlIjoyfV0sWzQsMCwiXFxpc28iLDMseyJzdHlsZSI6eyJib2R5Ijp7Im5hbWUiOiJub25lIn0sImhlYWQiOnsibmFtZSI6Im5vbmUifX19XSxbMSw1LCIiLDMseyJsZXZlbCI6Miwic3R5bGUiOnsiaGVhZCI6eyJuYW1lIjoibm9uZSJ9fX1dXQ==
    \begin{tikzcd}
      {\Abar \cap \Bbar} \\
      & {\overline{A \cap B}} && 1 \\
      C & C && \Omega
      \arrow[tail, from=2-2, to=3-2]
      \arrow[from=2-2, to=2-4]
      \arrow["t", tail, from=2-4, to=3-4]
      \arrow["{j\wedge(\chi_{A},\chi_{B})}", from=3-2, to=3-4]
      \arrow[tail, from=1-1, to=3-1]
      \arrow[curve={height=-6pt}, from=1-1, to=2-4]
      \arrow["{\wedge(j\chi_{A},j\chi_{B})}"', curve={height=12pt}, from=3-1, to=3-4]
      \arrow["\iso"{marking, allow upside down}, draw=none, from=1-1, to=2-2]
      \arrow[Rightarrow, no head, from=3-2, to=3-1]
    \end{tikzcd}
  \end{mathpar}
  Conversely, if $j$ induces a closure operation with the above properties, then the diagrams above show that $j$ must be a topology.
\end{proof}

\begin{defn}
  Let $\cE$ be a topos equipped with a topology.
  We say that a subobject $A$ of $B$ is \emph{dense} in $B$ if $\Abar = B$ and that it is \emph{closed} when $\Abar = A$.
\end{defn}

\begin{defn}
  Let $\cE$ be a topos equipped with a topology $j$.
  A \emph{$j$-sheaf} is an object $F$ so that for every dense monomorphism $m : A \mono B$ and any morphism $f' : A \to F$, $m$ factors through $f'$ uniquely.
  % https://q.uiver.app/#q=WzAsMyxbMCwwLCJBIl0sWzAsMiwiQiJdLFsyLDAsIkMiXSxbMCwxLCJtIiwyLHsic3R5bGUiOnsidGFpbCI6eyJuYW1lIjoibW9ubyJ9fX1dLFswLDIsImYiXSxbMSwyLCIiLDIseyJzdHlsZSI6eyJib2R5Ijp7Im5hbWUiOiJkYXNoZWQifX19XV0=
  \[\begin{tikzcd}
      A && F \\
      \\
      B
      \arrow["m"', tail, from=1-1, to=3-1]
      \arrow["f'", from=1-1, to=1-3]
      \arrow[dashed, from=3-1, to=1-3]
    \end{tikzcd}\]
  An object is $j$-separated if $m$ factors through $f'$ in at most one way.
  The category $\Sh_{j}(\cE)$ is the full subcategory of $\cE$ spanned by $j$-sheaves.
\end{defn}

\begin{rmk}
  Clearly, $\Sh_{j}(\cE)$ is the orthogonal class $\cM^{\bot}$, where $\cM$ is the class of all dense monomorphisms.
  When $\cE$ is a locally presentable category and $\cM$ is small, $\Sh_{j}(\cE)$ is reflective in $\cE$.
  In fact, $\Sh_{j}(\cE)$ is lex reflective because $\cM$ is stable under pullbacks.
\end{rmk}

\todo{These theorems are proved for more general categories in Elephant A.}

Now we record some facts about topos equipped with a topology.

\begin{lem}\label{lem:dense-bijection-closed-subobjects}
  Let $\cE$ be a topos equipped with a topology $j$.
  If $m : A \mono B$ is dense, then there is a bijection between closed subobjects of $A$ and those of $B$.
\end{lem}
\begin{proof}
  Let $B'$ be a closed subobject of $B$, then $m^{*}(B')$ is a subobject of $A$ and moreover $\overline{m^{*}(B')} = m^{*}(\overline{B'}) = m^{*}(B')$.
  Conversely, let $A'$ be a closed subobject of $A$.
  Then $\overline{A'} \subseteq \Abar = B$, and clearly $\Abar$ is closed.
  Since dense monomorphisms are pullback stable, the projection $m^{*}(C) \mono C$ is dense for any closed subobject $C$ of $B$.
  Thus, we have
  \begin{mathpar}
    m^{*}(\overline{A'}) = \overline{m^{*}(\overline{A'})} = \overline{A'} = A'\and
    \overline{m^{*}(B')} = B'
  \end{mathpar}
  This proves that the two mappings form a bijection.
\end{proof}

\begin{lem}\label{lem:category-of-sheaves-exponential-ideal}
  Let $\cE$ be a topos equipped with a topology $j$.
  $\Sh_{j}(\cE)$ is an exponential ideal in $\cE$.
\end{lem}
\begin{proof}
  Let $B$ be a $j$-sheaf and $A$ be an object in $\cE$.
  For any dense monomorphism $m : C' \to C$, the morphism $\id_{A} \times m : A \times C' \mono A \times C$ is dense because it is a pullback.
  Then for any $f' : C' \to B^{A}$, there is a unique $A \times C \to B$ whose transpose uniquely factors $m$ through $f'$.
  % https://q.uiver.app/#q=WzAsNixbNCwwLCJBIFxcdGltZXMgQyciXSxbNCwyLCJBIFxcdGltZXMgQyJdLFs2LDAsIkIiXSxbMiwwLCJCXntBfSJdLFswLDAsIkMnIl0sWzAsMiwiQyJdLFswLDEsIlxcaWRfe0F9IFxcdGltZXMgbSIsMix7InN0eWxlIjp7InRhaWwiOnsibmFtZSI6Im1vbm8ifX19XSxbMCwyLCJcXGhhdHtmJ30iXSxbMSwyLCIiLDIseyJzdHlsZSI6eyJib2R5Ijp7Im5hbWUiOiJkYXNoZWQifX19XSxbNCw1LCJtIiwyLHsic3R5bGUiOnsidGFpbCI6eyJuYW1lIjoibW9ubyJ9fX1dLFs0LDMsImYnIl0sWzUsMywiIiwyLHsic3R5bGUiOnsiYm9keSI6eyJuYW1lIjoiZGFzaGVkIn19fV1d
  \[\begin{tikzcd}
      {C'} && {B^{A}} && {A \times C'} && B \\
      \\
      C &&&& {A \times C}
      \arrow["{\id_{A} \times m}"', tail, from=1-5, to=3-5]
      \arrow["{f'^{\dagger}}", from=1-5, to=1-7]
      \arrow[dashed, from=3-5, to=1-7]
      \arrow["m"', tail, from=1-1, to=3-1]
      \arrow["{f'}", from=1-1, to=1-3]
      \arrow[dashed, from=3-1, to=1-3]
    \end{tikzcd}\]
  Thus, $B^{A}$ is a $j$-sheaf.
\end{proof}

\begin{lem}\label{lem:closed-object-classifier-a-sheaf}
  In a topos $\cE$ equipped with a topology $j$, let $\Omega_{j}$ be the equalizer of $\id_{\Omega}$ and $j$.
  $\Omega_{j}$ is a $j$-sheaf.
\end{lem}
\begin{proof}
  Let $m : A \mono B$ be any dense monomorphism and $f' : A \to \Omega_{j}$ be any morphism.
  $f'$ corresponds to a closed subobject $A'$ of $A$ given by the composite $kf'$.
  % https://q.uiver.app/#q=WzAsNCxbMCwwLCJcXE9tZWdhX3tqfSJdLFswLDEsIkEiXSxbMSwwLCJcXE9tZWdhIl0sWzIsMCwiXFxPbWVnYSJdLFsxLDAsImYnIl0sWzAsMiwiayIsMCx7InN0eWxlIjp7InRhaWwiOnsibmFtZSI6Im1vbm8ifX19XSxbMiwzLCIiLDAseyJvZmZzZXQiOi0xfV0sWzIsMywiIiwwLHsib2Zmc2V0IjoxfV0sWzEsMiwiXFxjaGlfe0EnfSIsMl1d
  \[\begin{tikzcd}
      {\Omega_{j}} & \Omega & \Omega \\
      A
      \arrow["{f'}", from=2-1, to=1-1]
      \arrow["k", tail, from=1-1, to=1-2]
      \arrow[shift left, from=1-2, to=1-3]
      \arrow[shift right, from=1-2, to=1-3]
      \arrow["{\chi_{A'}}"', from=2-1, to=1-2]
    \end{tikzcd}\]
  The subobject $A'$ is a subobject in $B$ by the composite $mA'$.
  By \cref{lem:dense-bijection-closed-subobjects}, there is a unique closed subobject $B'$ whose pullback along $m$ is isomorphic to $A'$.
  By the universal property of equalizers, there is a unique morphism $B \to \Omega_{j}$ so that the following diagram commutes.
  % https://q.uiver.app/#q=WzAsNCxbMCwwLCJcXE9tZWdhX3tqfSJdLFswLDEsIkIiXSxbMSwwLCJcXE9tZWdhIl0sWzIsMCwiXFxPbWVnYSJdLFsxLDAsInUiLDAseyJzdHlsZSI6eyJib2R5Ijp7Im5hbWUiOiJkYXNoZWQifX19XSxbMCwyLCJrIiwwLHsic3R5bGUiOnsidGFpbCI6eyJuYW1lIjoibW9ubyJ9fX1dLFsyLDMsIiIsMCx7Im9mZnNldCI6LTF9XSxbMiwzLCIiLDAseyJvZmZzZXQiOjF9XSxbMSwyLCJcXGNoaV97Qid9IiwyXV0=
  \[\begin{tikzcd}
      {\Omega_{j}} & \Omega & \Omega \\
      B
      \arrow["u", dashed, from=2-1, to=1-1]
      \arrow["k", tail, from=1-1, to=1-2]
      \arrow[shift left, from=1-2, to=1-3]
      \arrow[shift right, from=1-2, to=1-3]
      \arrow["{\chi_{B'}}"', from=2-1, to=1-2]
    \end{tikzcd}\]
  To show that $um = f'$, it suffices to show $kum = kf'$, which is equivalent to showing $\chi_{B'}m = \chi_{A'}$.
  Since $A' \iso B'$ by construction, the desired equality follows at once.
  % https://q.uiver.app/#q=WzAsNixbMCwwLCJBJyJdLFsyLDAsIkInIl0sWzAsMiwiQSJdLFsyLDIsIkIiXSxbNCwyLCJcXE9tZWdhIl0sWzQsMCwiMSJdLFsyLDMsIm0iLDIseyJzdHlsZSI6eyJ0YWlsIjp7Im5hbWUiOiJtb25vIn19fV0sWzAsMiwiIiwyLHsic3R5bGUiOnsidGFpbCI6eyJuYW1lIjoibW9ubyJ9fX1dLFsxLDMsIiIsMCx7InN0eWxlIjp7InRhaWwiOnsibmFtZSI6Im1vbm8ifX19XSxbMCwxLCJcXGlzbyIsMyx7InN0eWxlIjp7ImJvZHkiOnsibmFtZSI6Im5vbmUifSwiaGVhZCI6eyJuYW1lIjoibm9uZSJ9fX1dLFswLDMsIiIsMyx7InN0eWxlIjp7Im5hbWUiOiJjb3JuZXIifX1dLFs1LDQsIiIsMyx7InN0eWxlIjp7InRhaWwiOnsibmFtZSI6Im1vbm8ifX19XSxbMSw1XSxbMyw0LCJcXGNoaV97Qid9IiwyXSxbMiw0LCJcXGNoaV97QSd9IiwyLHsiY3VydmUiOjN9XV0=
  \[\begin{tikzcd}
      {A'} && {B'} && 1 \\
      \\
      A && B && \Omega
      \arrow["m"', tail, from=3-1, to=3-3]
      \arrow[tail, from=1-1, to=3-1]
      \arrow[tail, from=1-3, to=3-3]
      \arrow["\iso"{marking, allow upside down}, draw=none, from=1-1, to=1-3]
      \arrow["\lrcorner"{anchor=center, pos=0.125}, draw=none, from=1-1, to=3-3]
      \arrow[tail, from=1-5, to=3-5]
      \arrow[from=1-3, to=1-5]
      \arrow["{\chi_{B'}}"', from=3-3, to=3-5]
      \arrow["{\chi_{A'}}"', curve={height=18pt}, from=3-1, to=3-5]
    \end{tikzcd}\]
\end{proof}

\begin{cor}\label{cor:power-j-objects-sheaves}
  Let $\cE$ be a topos equipped with a topology $j$.
  For any object $A \in \cE$, the exponential $\Omega_{j}^{A}$ is a $j$-sheaf.
\end{cor}
\begin{proof}
  By \cref{lem:category-of-sheaves-exponential-ideal} and \cref{lem:closed-object-classifier-a-sheaf}.
\end{proof}

\begin{lem}
  For any object $B$ in a topos $\cE$, $B$ is separated iff the diagonal $\Delta : B \mono B \times B$ is closed.
\end{lem}
\begin{proof}
  Assume that $B$ is separated.
  Consider the comparison monomorphism $B \mono \Bbar$, which is clearly dense.
  Now, consider the following commutative diagram.
  % https://q.uiver.app/#q=WzAsMyxbMCwwLCJCIl0sWzIsMCwiQiJdLFswLDIsIlxcQmJhciJdLFswLDIsIiIsMix7InN0eWxlIjp7InRhaWwiOnsibmFtZSI6Im1vbm8ifX19XSxbMCwxLCJcXHBpX3sxfVxcRGVsdGEgPSBcXHBpX3syfVxcRGVsdGEiXSxbMiwxLCJcXHBpX3syfVxcb3ZlcmxpbmV7XFxEZWx0YX0iLDIseyJjdXJ2ZSI6MX1dLFsyLDEsIlxccGlfezF9XFxvdmVybGluZXtcXERlbHRhfSIsMCx7ImN1cnZlIjotMX1dXQ==
  \[\begin{tikzcd}
      B && B \\
      \\
      \Bbar
      \arrow[tail, from=1-1, to=3-1]
      \arrow["{\pi_{1}\Delta = \pi_{2}\Delta}", from=1-1, to=1-3]
      \arrow["{\pi_{2}\overline{\Delta}}"', curve={height=6pt}, from=3-1, to=1-3]
      \arrow["{\pi_{1}\overline{\Delta}}", curve={height=-6pt}, from=3-1, to=1-3]
    \end{tikzcd}\]
  Since $B$ is separated, $\pi_{1}\overline{\Delta} = \pi_{2}\overline{\Delta}$.
  Since $\Delta$ is the equalizer of the projections, we have
  % https://q.uiver.app/#q=WzAsNCxbMCwwLCJCIl0sWzIsMCwiQiBcXHRpbWVzIEIiXSxbNCwwLCJCIl0sWzAsMiwiXFxCYmFyIl0sWzEsMiwiXFxwaV97MX0iLDAseyJvZmZzZXQiOi0xfV0sWzEsMiwiXFxwaV97Mn0iLDIseyJvZmZzZXQiOjF9XSxbMCwxLCJcXERlbHRhIiwwLHsic3R5bGUiOnsidGFpbCI6eyJuYW1lIjoibW9ubyJ9fX1dLFszLDEsIlxcb3ZlcmxpbmV7XFxEZWx0YX0iLDIseyJjdXJ2ZSI6Miwic3R5bGUiOnsidGFpbCI6eyJuYW1lIjoibW9ubyJ9fX1dLFswLDMsIiIsMix7ImN1cnZlIjoyLCJzdHlsZSI6eyJ0YWlsIjp7Im5hbWUiOiJtb25vIn19fV0sWzMsMCwiIiwxLHsiY3VydmUiOjIsInN0eWxlIjp7ImJvZHkiOnsibmFtZSI6ImRhc2hlZCJ9fX1dXQ==
  \[\begin{tikzcd}
      B && {B \times B} && B \\
      \\
      \Bbar
      \arrow["{\pi_{1}}", shift left, from=1-3, to=1-5]
      \arrow["{\pi_{2}}"', shift right, from=1-3, to=1-5]
      \arrow["\Delta", tail, from=1-1, to=1-3]
      \arrow["{\overline{\Delta}}"', curve={height=12pt}, tail, from=3-1, to=1-3]
      \arrow[curve={height=12pt}, tail, from=1-1, to=3-1]
      \arrow[curve={height=12pt}, dashed, from=3-1, to=1-1]
    \end{tikzcd}\]
  In fact, this yields an isomorphism $B \iso \Bbar$.

  Conversely, suppose that $\Delta : B \mono B \times B$ is closed.
  Then this relation is classified by $\delta_{B} : B \times B \to \Omega_{j}$ whose name is $\Set{}_{j} : B \to \Omega_{j}^{B}$.
  Note that $\Set{}_{j}$ is a monomorphism.
  By \cref{cor:power-j-objects-sheaves}, $\Omega_{j}^{B}$ is a $j$-sheaf (and therefore separated), so $B$ is separated because it is a subobject of a separated object.
\end{proof}

\begin{thm}
  Let $\cE$ be a topos equipped with a topology $j$.
  $\Sh_{j}(\cE)$ is reflective in $\cE$.
\end{thm}
\begin{proof}
  The inclusion functor $\Sh_{j}(\cE) \into \cE$ is fully faithful by construction.
  Thus, it suffices to construct the required left adjoint $L : \cE \to \Sh_{j}(\cE)$.
  Let $B$ be any object of $\cE$.
  Consider the diagonal morphism $\Delta : B \mono B \times B$.
  As the first approximation, take the closure of this subobject $\overline{\Delta} : \Bbar \mono B \times B$, which is named by the morphism $d : B \to \Omega_{j}^{B}$.
  Note that $d$ can be factorized as follows.
  % https://q.uiver.app/#q=WzAsMyxbMCwwLCJCIl0sWzIsMCwiXFxPbWVnYV97an1ee0J9Il0sWzEsMSwiTUIiXSxbMCwxLCJkIl0sWzAsMiwiIiwyLHsic3R5bGUiOnsiaGVhZCI6eyJuYW1lIjoiZXBpIn19fV0sWzIsMSwiIiwyLHsic3R5bGUiOnsidGFpbCI6eyJuYW1lIjoibW9ubyJ9fX1dXQ==
  \[\begin{tikzcd}
      B && {\Omega_{j}^{B}} \\
      & MB
      \arrow["d", from=1-1, to=1-3]
      \arrow[two heads, from=1-1, to=2-2]
      \arrow[tail, from=2-2, to=1-3]
    \end{tikzcd}\]
  Here, $MB$ is the coequalizer of the kernel pair of $d$.
  Note that $(\pi_{1}\overline{\Delta}, \pi_{2}\overline{\Delta})$ is such a kernel pair.
  Since $MB$ is a subobject of a separated object, it is separated itself.
  In fact, $MB$ is the reflection of $B$ in the full subcategory of separated objects since any morphism from $B$ into a separated object $C$ coequalizes $\pi_{1}\overline{\Delta}$ and $\pi_{2}\overline{\Delta}$.
  % https://q.uiver.app/#q=WzAsNCxbMCwwLCJcXEJiYXIiXSxbMiwwLCJCIl0sWzQsMCwiTUIiXSxbNCwyLCJDIl0sWzAsMSwiIiwyLHsib2Zmc2V0IjotMX1dLFswLDEsIiIsMCx7Im9mZnNldCI6MX1dLFsxLDIsIlxcZXRhX3tCfSciLDAseyJzdHlsZSI6eyJoZWFkIjp7Im5hbWUiOiJlcGkifX19XSxbMSwzXSxbMiwzLCIiLDEseyJzdHlsZSI6eyJib2R5Ijp7Im5hbWUiOiJkYXNoZWQifX19XV0=
  \[\begin{tikzcd}
      \Bbar && B && MB \\
      \\
      &&&& C
      \arrow[shift left, from=1-1, to=1-3]
      \arrow[shift right, from=1-1, to=1-3]
      \arrow["{\eta_{B}'}", two heads, from=1-3, to=1-5]
      \arrow[from=1-3, to=3-5]
      \arrow[dashed, from=1-5, to=3-5]
    \end{tikzcd}\]
  This establishes the universality of the unit.
  To obtain a sheaf, we take the closure of the subobject $MB \mono \Omega_{j}^{B}$.
  We claim that the composite of the coequalizer and the dense monomorphism $MB \mono LB$ is the required universal morphism.
  % https://q.uiver.app/#q=WzAsMyxbMCwwLCJCIl0sWzEsMCwiTUIiXSxbMiwwLCJMQiJdLFswLDEsIiIsMCx7InN0eWxlIjp7ImhlYWQiOnsibmFtZSI6ImVwaSJ9fX1dLFsxLDIsIiIsMCx7InN0eWxlIjp7InRhaWwiOnsibmFtZSI6Im1vbm8ifX19XV0=
  \[\begin{tikzcd}
      B & MB & LB
      \arrow[two heads, from=1-1, to=1-2]
      \arrow[tail, from=1-2, to=1-3]
    \end{tikzcd}\]
  Indeed, for any $j$-sheaf $C$ and any morphism $f : B \to C$, the coequalizer $B \epi MB$ factors through $f$ uniquely.
  Since $C$ is a sheaf, the dense monomorphism $MB \mono LB$ factors through $MB \to C$ uniquely.
  % https://q.uiver.app/#q=WzAsNCxbMCwwLCJCIl0sWzEsMCwiTUIiXSxbMiwwLCJMQiJdLFsxLDEsIkMiXSxbMCwxLCIiLDAseyJzdHlsZSI6eyJoZWFkIjp7Im5hbWUiOiJlcGkifX19XSxbMSwyLCIiLDAseyJzdHlsZSI6eyJ0YWlsIjp7Im5hbWUiOiJtb25vIn19fV0sWzAsM10sWzEsMywiIiwyLHsic3R5bGUiOnsiYm9keSI6eyJuYW1lIjoiZGFzaGVkIn19fV0sWzIsMywiIiwwLHsic3R5bGUiOnsiYm9keSI6eyJuYW1lIjoiZGFzaGVkIn19fV1d
  \[\begin{tikzcd}
      B & MB & LB \\
      & C
      \arrow[two heads, from=1-1, to=1-2]
      \arrow[tail, from=1-2, to=1-3]
      \arrow[from=1-1, to=2-2]
      \arrow[dashed, from=1-2, to=2-2]
      \arrow[dashed, from=1-3, to=2-2]
    \end{tikzcd}\]
  This proves that $LB$ is the reflection of $B$ in $\Sh_{j}(\cE)$.
\end{proof}

\begin{thm}
  Let $\cE$ be a topos equipped with a topology $j$.
  $\Sh_{j}(\cE)$ is a topos.
\end{thm}
\begin{proof}
  \todo{}
\end{proof}

\begin{lem}\label{lem:reflector-finite-product-preserving-exponential-ideal}
  Let $\cE$ be a topos and $\cL$ be a reflective subcategory of $\cE$.
  Then the reflector $L$ preserves finite products iff $\cL$ is an exponential ideal in $\cE$.
\end{lem}
\begin{proof}
  \todo{}
\end{proof}

\begin{thm}
  Let $\cE$ be a topos equipped with a topology $j$.
  The sheafification functor $L : \cE \to \Sh_{j}(\cE)$ preserves finite limits.
\end{thm}
\begin{proof}
  First, observe that $\Sh_{j}(\cE)$ is an exponential ideal in $\cE$.
  By \cref{lem:reflector-finite-product-preserving-exponential-ideal}, $L$ preserves finite products.
  It suffices to show that $L$ preserves binary intersections of subobjects. \todo{Justify this.}
  To this end, we show that $L$ preserves monomorphisms, which makes use of the following facts.
  \begin{enumerate}
  \item Every injective in $\Sh_{j}(\cE)$ is injective in $\cE$.
  \item A morphism $m : A \to B$ in a topos is monic iff every morphism from $A$ to an injective factors through $m$.
  \end{enumerate}
  For any monomorphism $A \mono B$ and any injective sheaf $I$, we have the following diagram.
  % https://q.uiver.app/#q=WzAsNSxbMCwwLCJBIl0sWzIsMCwiQiJdLFswLDIsIkxBIl0sWzIsMiwiTEIiXSxbMSwzLCJJIl0sWzAsMSwiaCIsMCx7InN0eWxlIjp7InRhaWwiOnsibmFtZSI6Im1vbm8ifX19XSxbMCwyLCJcXGV0YV97QX0iLDJdLFsxLDMsIlxcZXRhX3tCfSJdLFsyLDMsIkxoIl0sWzIsNCwiZiIsMl0sWzMsNCwiZyIsMCx7InN0eWxlIjp7ImJvZHkiOnsibmFtZSI6ImRhc2hlZCJ9fX1dLFswLDRdLFsxLDRdXQ==
  \[\begin{tikzcd}
      A && B \\
      \\
      LA && LB \\
      & I
      \arrow["h", tail, from=1-1, to=1-3]
      \arrow["{\eta_{A}}"', from=1-1, to=3-1]
      \arrow["{\eta_{B}}", from=1-3, to=3-3]
      \arrow["Lh", from=3-1, to=3-3]
      \arrow["f"', from=3-1, to=4-2]
      \arrow["g", dashed, from=3-3, to=4-2]
      \arrow[from=1-1, to=4-2]
      \arrow[from=1-3, to=4-2]
    \end{tikzcd}\]
  Universality of $\eta_{A}$ demands that $f$ factors $g$ through $Lh$.
  % https://q.uiver.app/#q=WzAsNCxbMCwwLCJBIl0sWzIsMCwiTEEiXSxbMiwyLCJJIl0sWzMsMSwiTEIiXSxbMCwxLCJcXGV0YV97QX0iXSxbMCwyLCJmXFxldGFfe0F9IiwyXSxbMSwyLCJmIiwyLHsic3R5bGUiOnsiYm9keSI6eyJuYW1lIjoiZGFzaGVkIn19fV0sWzEsMywiTGgiXSxbMywyLCJnIl0sWzYsMywiIiwyLHsic2hvcnRlbiI6eyJzb3VyY2UiOjIwfSwic3R5bGUiOnsiaGVhZCI6eyJuYW1lIjoibm9uZSJ9fX1dXQ==
  \[\begin{tikzcd}
      A && LA \\
      &&& LB \\
      && I
      \arrow["{\eta_{A}}", from=1-1, to=1-3]
      \arrow["{f\eta_{A}}"', from=1-1, to=3-3]
      \arrow[""{name=0, anchor=center, inner sep=0}, "f"', dashed, from=1-3, to=3-3]
      \arrow["Lh", from=1-3, to=2-4]
      \arrow["g", from=2-4, to=3-3]
      \arrow[shorten <=5pt, Rightarrow, no head, from=0, to=2-4]
    \end{tikzcd}\]
  Thus, $Lh$ is a monomorphism.
  Now, consider the following pushout in $\cE$.
  % https://q.uiver.app/#q=WzAsNCxbMCwwLCJBIFxcY2FwIEIiXSxbMCwyLCJBIl0sWzIsMCwiQiJdLFsyLDIsIkEgXFxjdXAgQiJdLFswLDEsIiIsMCx7InN0eWxlIjp7InRhaWwiOnsibmFtZSI6Im1vbm8ifX19XSxbMiwzLCIiLDAseyJzdHlsZSI6eyJ0YWlsIjp7Im5hbWUiOiJtb25vIn19fV0sWzAsMiwiIiwxLHsic3R5bGUiOnsidGFpbCI6eyJuYW1lIjoibW9ubyJ9fX1dLFsxLDMsIiIsMSx7InN0eWxlIjp7InRhaWwiOnsibmFtZSI6Im1vbm8ifX19XSxbMywwLCIiLDEseyJzdHlsZSI6eyJuYW1lIjoiY29ybmVyIn19XV0=
  \[\begin{tikzcd}
      {A \cap B} && B \\
      \\
      A && {A \cup B}
      \arrow[tail, from=1-1, to=3-1]
      \arrow[tail, from=1-3, to=3-3]
      \arrow[tail, from=1-1, to=1-3]
      \arrow[tail, from=3-1, to=3-3]
      \arrow["\lrcorner"{anchor=center, pos=0.125, rotate=180}, draw=none, from=3-3, to=1-1]
    \end{tikzcd}\]
  $L$ as a left adjoint preserves colimits and additionally it preserves monomorphisms so we have the following pushout in $\Sh_{j}(\cE)$.
  % https://q.uiver.app/#q=WzAsNCxbMCwwLCJMKEEgXFxjYXAgQikiXSxbMCwyLCJMQSJdLFsyLDAsIkxCIl0sWzIsMiwiTChBIFxcY3VwIEIpIl0sWzAsMSwiIiwwLHsic3R5bGUiOnsidGFpbCI6eyJuYW1lIjoibW9ubyJ9fX1dLFsyLDMsIiIsMCx7InN0eWxlIjp7InRhaWwiOnsibmFtZSI6Im1vbm8ifX19XSxbMCwyLCIiLDEseyJzdHlsZSI6eyJ0YWlsIjp7Im5hbWUiOiJtb25vIn19fV0sWzEsMywiIiwxLHsic3R5bGUiOnsidGFpbCI6eyJuYW1lIjoibW9ubyJ9fX1dLFszLDAsIiIsMSx7InN0eWxlIjp7Im5hbWUiOiJjb3JuZXIifX1dXQ==
  \[\begin{tikzcd}
      {L(A \cap B)} && LB \\
      \\
      LA && {L(A \cup B)}
      \arrow[tail, from=1-1, to=3-1]
      \arrow[tail, from=1-3, to=3-3]
      \arrow[tail, from=1-1, to=1-3]
      \arrow[tail, from=3-1, to=3-3]
      \arrow["\lrcorner"{anchor=center, pos=0.125, rotate=180}, draw=none, from=3-3, to=1-1]
    \end{tikzcd}\]
  But this pushout is also a pullback, so $L(A \cap B) \iso LA \cap LB$.
\end{proof}

\section{Classifying topoi}
\label{sec:classifying-topoi}

\todo{Elephant B4.2.}
\paragraph{Achtung.} In this section, topos means Grothendieck topos.
\paragraph{Achtung.} I don't have a coherent story for this section yet.

A geometric theory is a doctrine $T : \iC\op \to \CFrm$.
For each $\Gamma \in \iC$, $T(\Gamma)$ is a frame of formulas in the given context $\Gamma$, and $\varphi \leq \psi \in T(\Gamma)$ means $\varphi \vdash_{\Gamma} \psi$.
$T$ needs to satisfy additional conditions so that quantifiers can be interpreted in the expected way.

A model of $T$ in a topos $\cE$ consists of a lex functor $M : \iC \to \cE$ and a natural transformation $\alpha$.
% https://q.uiver.app/#q=WzAsMyxbMSwxLCJcXENGcm0iXSxbMCwwLCJcXGlDXFxvcCJdLFsyLDAsIlxcY0VcXG9wIl0sWzIsMCwiUCJdLFsxLDAsIlQiLDJdLFsxLDIsIk0iXSxbNCwyLCJcXGFscGhhIiwyLHsic2hvcnRlbiI6eyJzb3VyY2UiOjEwLCJ0YXJnZXQiOjEwfX1dXQ==
\[\begin{tikzcd}
    \iC\op && \cE\op \\
    & \CFrm
    \arrow["P", from=1-3, to=2-2]
    \arrow[""{name=0, anchor=center, inner sep=0}, "T"', from=1-1, to=2-2]
    \arrow["M", from=1-1, to=1-3]
    \arrow["\alpha"', shorten <=5pt, shorten >=5pt, Rightarrow, from=0, to=1-3]
  \end{tikzcd}\]
The subobject classifier of $\cE$ needs to be an internal frame so that the powerobject functor $P : \cE\op \to \cE$ can be lifted.

We want a topos $\iC[T]$ and a universal model $(M_{U},\alpha_{U})$ such that every $T$-model in a topos $\cE$ can be identified with a geometric morphism $\cE \to \iC[T]$.
% https://q.uiver.app/#q=WzAsNCxbMCwwLCJcXGlDXFxvcCJdLFs0LDAsIlxcY0VcXG9wIl0sWzIsMSwiXFxpQ1tUXVxcb3AiXSxbMiwzLCJcXENGcm0iXSxbMCwxLCJNIiwwLHsiY3VydmUiOi0xfV0sWzAsMiwiTV97VX0iLDIseyJjdXJ2ZSI6MX1dLFsyLDEsImZeeyp9IiwyLHsiY3VydmUiOjF9XSxbMCwzLCJUIiwyLHsiY3VydmUiOjJ9XSxbMiwzLCJQIiwyXSxbMSwzLCJQX3tcXGNFfSIsMCx7ImN1cnZlIjotMn1dLFs0LDIsIlxcaXNvIiwzLHsic3R5bGUiOnsiYm9keSI6eyJuYW1lIjoibm9uZSJ9LCJoZWFkIjp7Im5hbWUiOiJub25lIn19fV0sWzgsOSwiXFxhbHBoYSIsMix7Im9mZnNldCI6LTQsInNob3J0ZW4iOnsic291cmNlIjoyMCwidGFyZ2V0IjoyMH19XSxbNywyLCJcXGFscGhhX3tVfSIsMix7InNob3J0ZW4iOnsic291cmNlIjoxMCwidGFyZ2V0IjoxMH19XV0=
\[\begin{tikzcd}
    \iC\op &&&& \cE\op \\
    && {\iC[T]\op} \\
    \\
    && \CFrm
    \arrow[""{name=0, anchor=center, inner sep=0}, "M", curve={height=-6pt}, from=1-1, to=1-5]
    \arrow["{M_{U}}"', curve={height=6pt}, from=1-1, to=2-3]
    \arrow["{f^{*}}"', curve={height=6pt}, from=2-3, to=1-5]
    \arrow[""{name=1, anchor=center, inner sep=0}, "T"', curve={height=12pt}, from=1-1, to=4-3]
    \arrow[""{name=2, anchor=center, inner sep=0}, "P"', from=2-3, to=4-3]
    \arrow[""{name=3, anchor=center, inner sep=0}, "{P_{\cE}}", curve={height=-12pt}, from=1-5, to=4-3]
    \arrow["\iso"{marking, allow upside down}, draw=none, from=0, to=2-3]
    \arrow["\alpha"', shift left=4, shorten <=10pt, shorten >=10pt, Rightarrow, from=2, to=3]
    \arrow["{\alpha_{U}}"', shorten <=3pt, shorten >=3pt, Rightarrow, from=1, to=2-3]
  \end{tikzcd}\]

\section{Locales}
\label{sec:locales}

\begin{defn}
  A \emph{frame} is a lattice with arbitrary joins and finite meets satisfying the infinitary distributive law.
  A \emph{frame homomorphism} is an order-preserving function that preserves infinite joins and finite meets.
\end{defn}

\begin{defn}
  The category of locales $\CLoc$ is the opposite of the category of frames $\CFrm$.
\end{defn}

\bibliographystyle{alpha}
\bibliography{all}

\end{document}
