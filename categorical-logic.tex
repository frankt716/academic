\documentclass[article,10pt,oneside]{memoir}
\usepackage{amssymb,amsmath,stmaryrd,mathrsfs}
\usepackage{quiver}
\usepackage[T1]{fontenc}
\usepackage{amsthm}
\usepackage{tikz,tikz-cd}
\usepackage{enumitem}
\usepackage{xcolor}
\definecolor{darkgreen}{rgb}{0,0.45,0} 
\usepackage[pagebackref,colorlinks,citecolor=darkgreen,linkcolor=darkgreen]{hyperref}
\usepackage{mathtools}
\usepackage{ifmtarg}
\usepackage{braket}
\let\setof\Set
\usepackage{url}
\usepackage{xspace}
\usepackage{cleveref,aliascnt}
\usepackage[status=draft,author=]{fixme}
\fxusetheme{color}
\usepackage{mathpartir}
\title{categorical logic}
\author{Frank Tsai}
\date{\today}
%\thanks{}
\begin{document}
\maketitle
\tableofcontents

\section{Warning}
\label{sec:warning}

Everything written here is based on my current (limited) knowledge of categorical logic.
Some content written herein is purely my own speculation and is by no means authoritative.
This document is expected to contain a lot of errors.

\section{Introduction}
\label{sec:introduction}

The categorification of algebraic theories allow us to study them in a presentation-independent way.
Algebraic theories are precisely categories with finite products.
We think of the objects of such a category as sorts and morphisms as terms.
Identities and composition correspond to variables and substitution, respectively.
Equations are internalized as the diagonal subobject.
Given two algebraic theories $\cT$ and $\cT'$, a model of $\cT$ in $\cT'$ is a finite-product preserving functor $M : \cT \to \cT'$.
This ensures that products of sorts in $\cT$ are interpreted (sensibly) as products of sorts in $\cT'$.
Moreover, this ensures that equations are preserved.
Then, a homomorphism from $M$ to $M'$ is precisely a natural transformation $\alpha : M \To M'$.
% https://q.uiver.app/#q=WzAsNCxbMiwwLCJNcyJdLFswLDIsIk1zIl0sWzIsMiwiTShzIFxcdGltZXMgcykiXSxbNCwyLCJNKHMpIl0sWzIsMSwiXFxwaV97MX0iXSxbMiwzLCJcXHBpX3syfSIsMl0sWzAsMSwiXFxpZCIsMl0sWzAsMywiXFxpZCJdLFswLDIsIk1cXERlbHRhIiwxLHsic3R5bGUiOnsidGFpbCI6eyJuYW1lIjoibW9ubyJ9LCJib2R5Ijp7Im5hbWUiOiJkYXNoZWQifX19XV0=
\[\begin{tikzcd}
    && Ms \\
    \\
    Ms && {M(s \times s)} && {Ms}
    \arrow["{\pi_{1}}", from=3-3, to=3-1]
    \arrow["{\pi_{2}}"', from=3-3, to=3-5]
    \arrow["\id"', from=1-3, to=3-1]
    \arrow["\id", from=1-3, to=3-5]
    \arrow["M\Delta"{description}, dashed, tail, from=1-3, to=3-3]
  \end{tikzcd}\]

Category with finite products, finite-product preserving functors, and natural transformations form the doctrine of algebraic theories $\CAlg$.
The slice $\CAlg/\CSet$ recovers the classical set-theoretic models.
Explicitly, this slice has set-theoretic models, compatible theory morphisms, and natural transformations between them.
In fact, for any algebraic theory $\cT$, the hom-category $\CAlg(\cT,\CSet)$, i.e., the category of set-theoretic models, is equivalent to a cocomplete category with a strong generator comprised of perfectly presentable objects.

We can generalize this viewpoint: theories can be internalized in a sufficiently structured category.
Then a model of a theory in another theory is a functor preserving those structures.
The goal of this writeup is to develop this idea in more detail.

\section{Hyperdoctrine}
\label{sec:hyperdoctrine}

A theory in propositional logic is a Heyting (or Boolean) algebra.
Each sort of the theory is equipped with a poset of propositions, in which $\varphi \leq \psi$ means $\varphi$ entails $\psi$.
For simplicity, let us assume a single sorted theory with no function symbols.
Let us recall Tarski's definition:
\begin{enumerate}
\item The unique sort is a set $S$.
\item A proposition variable is a subset of $S$.
\item $\top$ is the entire set $S$.
\item $\bot$ is the empty set $\varnothing$.
\item $\varphi \wedge \psi$ is the intersection $\varphi \cap \psi$.
\item $\varphi \vee \psi$ is the union $\varphi \cup \psi$.
\item $\varphi \To \psi$ is the exponential $\psi^{\varphi}$, i.e., the largest element so that $\psi^{\varphi} \wedge \varphi \leq \psi$.
\end{enumerate}

Essentially, propositions live in a poset with sufficient structure: various logical connectives correspond to various universal constructions.
As a model of a theory $\cT$ in another theory $\cT'$ has to preserve entailment, we expect it to be a function $M : \cT \to \cT'$ preserving relevant structures.

Generalizing this to multisorted theories is straightforward.
Each sort of the theory has a poset of propositions.
In the presence of nontrivial terms, we need to specify how propositions in context interact with substitutions.
For example, given a proposition in the context $d$, i.e., $\vdash_{d} \varphi$, substitution by a term $f : c \to d$ ought to produce a proposition in the context $c$, i.e., $\vdash_{c} \varphi[f]$.
Of course, we expect substitution to satisfy $\varphi[f][g] = \varphi[g \circ f]$ and $\varphi[\id] = \varphi$.

\begin{defn}
  A \emph{doctrine} is a functor $\cC : \iC\op \to \CPos$, where $\CPos$ is some suitable category of posets.
\end{defn}

\begin{defn}\label{def:model-of-doctrine}
  A \emph{model} of a doctrine $\cC : \iC\op \to \CPos$ in another doctrine $\cD : \iD\op \to \CPos$ consists of
  \begin{itemize}
  \item A model $M : \iC \to \iD$.
  \item A natural transformation $\lambda : \cC \To \cD \circ M\op$.
  \end{itemize}
  % https://q.uiver.app/#q=WzAsMyxbMCwwLCJcXGlDXFxvcCJdLFsyLDAsIlxcaURcXG9wIl0sWzEsMSwiXFxDUG9zIl0sWzAsMSwiTVxcb3AiXSxbMCwyLCJcXGNDIiwyXSxbMSwyLCJcXGNEIl0sWzQsMSwiXFxsYW1iZGEiLDEseyJzaG9ydGVuIjp7InNvdXJjZSI6MTAsInRhcmdldCI6MTB9fV1d
  \[\begin{tikzcd}
      \iC\op && \iD\op \\
      & \CPos
      \arrow["M\op", from=1-1, to=1-3]
      \arrow[""{name=0, anchor=center, inner sep=0}, "\cC"', from=1-1, to=2-2]
      \arrow["\cD", from=1-3, to=2-2]
      \arrow["\lambda"{description}, shorten <=4pt, shorten >=4pt, Rightarrow, from=0, to=1-3]
    \end{tikzcd}\]
\end{defn}

Let us unpack \cref{def:model-of-doctrine}.
The model $M : \iC \to \iD$ is unsurprising: we need to express $\iC$ in terms of $\iD$.
Each component $\lambda_{c}$ is a model of propositions $\cC(c)$ in $\cD M\op(c)$.
Naturality expresses that these models are compatible with substitutions. 
% https://q.uiver.app/#q=WzAsNCxbMCwwLCJcXGNDKGMpIl0sWzIsMCwiXFxjRCBNXFxvcChjKSJdLFswLDIsIlxcY0MoYycpIl0sWzIsMiwiXFxjRCBNXFxvcChjJykiXSxbMCwxLCJcXGxhbWJkYV97Y30iXSxbMiwwLCJcXGJsYW5rW2ZdIl0sWzMsMSwiXFxibGFua1tNXFxvcChmKV0iLDJdLFsyLDMsIlxcbGFtYmRhX3tjJ30iLDJdXQ==
\[\begin{tikzcd}
    {\cC(c)} && {\cD M\op(c)} \\
    \\
    {\cC(c')} && {\cD M\op(c')}
    \arrow["{\lambda_{c}}", from=1-1, to=1-3]
    \arrow["{\blank[f]}", from=3-1, to=1-1]
    \arrow["{\blank[M\op(f)]}"', from=3-3, to=1-3]
    \arrow["{\lambda_{c'}}"', from=3-1, to=3-3]
  \end{tikzcd}\]

Quantifiers can be characterized as adjoints, too.
Given a term $f : c \to d$, the formula $\exists_{f}\varphi$ ought to be the smallest formula so that $\varphi \leq (\exists_{f}\varphi)[f]$.
Similarly, the formula $\forall_{f}\varphi$ ought to be the largest formula so that $(\forall_{f}\varphi)[f] \leq \varphi$.
That is, we need the following adjoints:
% https://q.uiver.app/#q=WzAsMixbMCwwLCJcXGNDKGMpIl0sWzAsMiwiXFxjQyhkKSJdLFsxLDAsIlxcYmxhbmtbZl0iLDFdLFswLDEsIlxcZXhpc3RzX3tmfSIsMix7ImN1cnZlIjo0fV0sWzAsMSwiXFxmb3JhbGxfe2Z9IiwwLHsiY3VydmUiOi00fV0sWzMsMiwiIiwxLHsibGV2ZWwiOjEsInN0eWxlIjp7Im5hbWUiOiJhZGp1bmN0aW9uIn19XSxbMiw0LCIiLDEseyJsZXZlbCI6MSwic3R5bGUiOnsibmFtZSI6ImFkanVuY3Rpb24ifX1dXQ==
\[\begin{tikzcd}
    {\cC(c)} \\
    \\
    {\cC(d)}
    \arrow[""{name=0, anchor=center, inner sep=0}, "{f^{*}}"{description}, from=3-1, to=1-1]
    \arrow[""{name=1, anchor=center, inner sep=0}, "{\exists_{f}}"', curve={height=24pt}, from=1-1, to=3-1]
    \arrow[""{name=2, anchor=center, inner sep=0}, "{\forall_{f}}", curve={height=-24pt}, from=1-1, to=3-1]
    \arrow["\dashv"{anchor=center}, draw=none, from=1, to=0]
    \arrow["\dashv"{anchor=center}, draw=none, from=0, to=2]
  \end{tikzcd}\]
Indeed, in the powerset doctrine $\cP : \CSet\op \to \CPos$, the existential quantifier and the universal quantifier are given by the direct image $f_{*}$ and the fiber image $f_{!}$, respectively.

\bibliographystyle{alpha}
\bibliography{all}

\end{document}
