\documentclass{zett}
\usepackage{amssymb,amsmath,stmaryrd,mathrsfs}
\usepackage{quiver}
\usepackage[T1]{fontenc}
\usepackage{amsthm}
\usepackage{tikz,tikz-cd}
\usepackage{enumitem}
\usepackage{xcolor}
\definecolor{darkgreen}{rgb}{0,0.45,0} 
\usepackage[pagebackref,colorlinks,citecolor=darkgreen,linkcolor=darkgreen]{hyperref}
\usepackage{mathtools}
\usepackage{ifmtarg}
\usepackage{braket}
\let\setof\Set
\usepackage{url}
\usepackage{xspace}
\usepackage{cleveref,aliascnt}
\usepackage[status=draft,author=]{fixme}
\fxusetheme{color}
\usepackage{mathpartir}
\title{Lawvere theories}
\author{Frank Tsai}
\date{\today}
% \thanks{}
\begin{document}
\maketitle
\tableofcontents

\section{Sources}
\label{sec:sources}

\cite{borger91:generators}, \cite{lawvere63:functsem}, \cite{adamek10:algebraic}.

\section{Lawvere Theory}
\label{sec:lawvere-theory}

\begin{node}
  We write $\dN$ for a chosen skeletal category of $\CFinSet$.
\end{node}

\begin{defn}
  The \emph{category of Lawvere theories} $\CLaw$ is the full subcategory of $\dN\op/\CCat$ spanned by those objects $A : \dN\op \to \dA$ that is identity-on-objects and preserves finite products.
  We write $A^{n}$ for the value of $A$ at $n$ and we omit the superscript when $n = 1$.
  An \emph{$n$-ary operation} of $\dA$ is a morphism $f : A^{n} \to A$.
  We write $\pi^{n}_{i} : A^{n} \to A$ for the value of $A$ at the $i$-th projection $n \to 1$ of $\dN\op$.
\end{defn}

\begin{eg}
  The identity functor $\dN\op \to \dN\op$ is a Lawvere theory, which we call ``the theory of equality''.
  This is an initial object of $\CLaw$.
\end{eg}

\begin{thm}
  The category of Lawvere theories is complete and cocomplete.
\end{thm}
\begin{proof}[Proof sketch]
  \begin{node}
    It suffices to prove that $\CLaw$ has products, equalizers, coproducts, and coequalizers.
  \end{node}
  \begin{node}
    Products in $\CLaw$ are constructed as follows.
    \[
      % https://q.uiver.app/#q=WzAsNCxbMCwyLCJcXGRBIl0sWzEsMiwiXFxkQSBcXHRpbWVzIFxcZEIiXSxbMiwyLCJcXGRCIl0sWzEsMCwiXFxkTlxcb3AiXSxbMywwLCJBIiwyXSxbMywyLCJCIl0sWzEsMCwiXFxwaV97XFxkQX0iXSxbMSwyLCJcXHBpX3tcXGRCfSIsMl0sWzMsMSwiKEEsIEIpIiwxLHsic3R5bGUiOnsiYm9keSI6eyJuYW1lIjoiZGFzaGVkIn19fV1d
      \begin{tikzcd}
	& \dN\op \\
	\\
	\dA & {\dA \times \dB} & \dB
	\arrow["A"', from=1-2, to=3-1]
	\arrow["B", from=1-2, to=3-3]
	\arrow["{\pi_{\dA}}", from=3-2, to=3-1]
	\arrow["{\pi_{\dB}}"', from=3-2, to=3-3]
	\arrow["{(A, B)}"{description}, dashed, from=1-2, to=3-2]
      \end{tikzcd}
    \]
  \end{node}
  \begin{node}
    Equalizers in $\CLaw$ are constructed as follows:
    \[
      % https://q.uiver.app/#q=WzAsNCxbMiwyLCJcXGRBIl0sWzQsMiwiXFxkQiJdLFswLDIsIlxcZEsiXSxbMiwwLCJcXGROXFxvcCJdLFswLDEsImYiLDAseyJvZmZzZXQiOi0xfV0sWzAsMSwiZyIsMix7Im9mZnNldCI6MX1dLFsyLDAsImsiLDJdLFszLDAsIkEiLDJdLFszLDEsIkIiXSxbMywyLCJLIiwyLHsic3R5bGUiOnsiYm9keSI6eyJuYW1lIjoiZGFzaGVkIn19fV1d
      \begin{tikzcd}
	&& \dN\op \\
	\\
	\dK && \dA && \dB
	\arrow["f", shift left, from=3-3, to=3-5]
	\arrow["g"', shift right, from=3-3, to=3-5]
	\arrow["k"', from=3-1, to=3-3]
	\arrow["A"', from=1-3, to=3-3]
	\arrow["B", from=1-3, to=3-5]
	\arrow["K"', dashed, from=1-3, to=3-1]
      \end{tikzcd}
    \]
  \end{node}

  \begin{node}
    Coproducts are computed as pushouts:
    \[
      % https://q.uiver.app/#q=WzAsNCxbMCwwLCJcXGROXFxvcCJdLFswLDIsIlxcZEEiXSxbMiwwLCJcXGRCIl0sWzIsMiwiXFxkQSArX3tcXGROXFxvcH0gXFxkQiJdLFswLDEsIkEiLDJdLFswLDIsIkIiXSxbMSwzXSxbMiwzXSxbMywwLCIiLDEseyJzdHlsZSI6eyJuYW1lIjoiY29ybmVyIn19XV0=
      \begin{tikzcd}
	\dN\op && \dB \\
	\\
	\dA && {\dA +_{\dN\op} \dB}
	\arrow["A"', from=1-1, to=3-1]
	\arrow["B", from=1-1, to=1-3]
	\arrow[from=3-1, to=3-3]
	\arrow[from=1-3, to=3-3]
	\arrow["\lrcorner"{anchor=center, pos=0.125, rotate=180}, draw=none, from=3-3, to=1-1]
      \end{tikzcd}
    \]
    Intuitively, when we glue a small family of Lawvere theories together, each operation of the new theory is an operation of some theory of the family.
    Each variable of each theory of the family is identified which is achieved via a pushout.
  \end{node}
  \begin{node}
    Coequalizers are computed as follows:
    \[
      % https://q.uiver.app/#q=WzAsNCxbMCwyLCJcXGRBIl0sWzIsMiwiXFxkQiJdLFs0LDIsIlxcZEsiXSxbMSwwLCJcXGROXFxvcCJdLFswLDEsImYiLDAseyJvZmZzZXQiOi0xfV0sWzAsMSwiZyIsMix7Im9mZnNldCI6MX1dLFsxLDIsImsiLDJdLFszLDEsIkIiLDFdLFszLDIsIksiXSxbMywwLCJBIiwyXV0=
      \begin{tikzcd}
	& \dN\op \\
	\\
	\dA && \dB && \dK
	\arrow["f", shift left, from=3-1, to=3-3]
	\arrow["g"', shift right, from=3-1, to=3-3]
	\arrow["k"', from=3-3, to=3-5]
	\arrow["B"{description}, from=1-2, to=3-3]
	\arrow["K", from=1-2, to=3-5]
	\arrow["A"', from=1-2, to=3-1]
      \end{tikzcd}
    \]
  \end{node}
\end{proof}

\section{Algebraic categories}
\label{sec:algebraic-categories}

\begin{defn}
  Let $A : \dN\op \to \dA$ be a Lawvere theory.
  The \emph{category of algebras of type $A$}, denoted $\Alg(A)$, is the full subcategory of $\CSet^{\dA}$ spanned by those finite-product preserving functors.
  A category $\iC$ is \emph{algebraic} if $\iC \eqv \Alg(B)$ for some Lawvere theory $B$.
\end{defn}

\begin{thm}
  Let $A : \dN\op \to \dA$ be a Lawvere theory.
  The category $\Alg(A)$ is complete.
\end{thm}
\begin{proof}
  Let $D : \iJ \to \Alg(A)$ be any small diagram.
  Consider the diagram $\iota \circ D : \iJ \to \CSet^{\dA}$, where $\iota : \Alg(A) \into \CSet^{\dA}$ is the full inclusion functor.
  Since $\iota$ reflects limits and $\CSet^{\dA}$ is complete, it suffices to show that $\lim \iota D$ is a finite-product preserving functor.
  This follows immediately from the fact that limits commute.
\end{proof}

\section{Sifted and filtered colimits}
\label{sec:sifted-and-filtered-colimits}

\begin{defn}\label{defn:sifted-categories}
  A small category $\iC$ is \emph{sifted} if finite products in $\CSet$ commute with colimits over $\iC$.
  Concretely, a small category $\iC$ is sifted if, given a diagram $D : \iC \times \iJ \to \CSet$ where $\iJ$ is a finite set, the canonical morphism
  \[
    \delta : \colim_{\iC}\prod_{\iJ}D(c,j) \to \prod_{\iJ}\colim_{\iC}D(c,j)
  \]
  is an isomorphism.
  Colimits of diagrams over sifted categories are called \emph{sifted colimits}.
\end{defn}

\begin{defn}
  A small category $\iC$ is \emph{filtered} if every finite diagram $F : \iJ \to \iC$ admits a cocone.
  Colimits of diagrams over filtered categories are called \emph{filtered colimits}.
\end{defn}

\begin{node}
  It is well-known that filtered colimits commute with finite limits in $\CSet$.
\end{node}

\begin{lem}\label{lem:sifted-empty-binary}%
  A small category $\iC$ is sifted if and only if colimits under $\iD$ commute with the empty product and binary products.
\end{lem}
\begin{proof}
  \begin{node}
    The only if direction is trivial.
    Let $D : \iD \times \iJ \to \CSet$ be an arbitrary diagram.
    Suppose the hypothesis.
    We proceed by induction on the size of $\iJ$.
    When $\iJ$ is the empty set, then the statement follows immediately from the hypothesis.
    Consider the following colimit
    \[
      \colim_{\iD}\left(\prod_{i=1}^{n}D(d, j_{i}) \times D(d, j_{i+1})\right)
    \]
    This colimit commute with binary products by the hypothesis.
    Thus,
    \[
      \colim_{\iD}\left(\prod_{i=1}^{n}D(d, j_{i}) \times D(d, j_{i+1})\right) \iso \colim_{\iD}\prod_{i=1}^{n}D(d, j_{i}) \times \colim_{\iD}D(d, j_{i+1})
    \]
    By the induction hypothesis, $\colim_{\iD}\prod_{i=1}^{n}D(d, j_{i}) \iso \prod_{i=1}^{n}\colim_{\iD}D(d, j_{i})$, so the statement follows.
  \end{node}
\end{proof}

\begin{defn}
  A functor $F : \iC \to \iD$ is \emph{final} if, for any diagram $G : \iD \to \iE$ such that $\colim_{\iC}GF$ exists, then $\colim_{\iD}G$ exists and the canonical morphism $\colim_{\iC}GF \to \colim_{\iD}G$ is an isomorphism.
\end{defn}

\begin{lem}
  A functor $F : \iC \to \iD$ is final if and only if the slice category $d \dn F$ is connected for all $d \in \iD$.
\end{lem}
\begin{proof}
  \begin{node}
    Suppose that $F$ is final.
    Consider the diagram $\iD(d, F\blank) : \iC \to \CSet$.
    By hypothesis, $\colim_{\iC}\iD(d, F\blank) \iso \colim_{\iD}\iD(d, \blank)$.
    The colimit of the represented functor is the singleton set.
    Thus, $\colim_{\iC}\iD(d, F\blank) \iso 1$, so the slice category $d \dn F$ is connected.
  \end{node}
  \begin{node}
    Conversely, suppose that $d \dn F$ is connected for all $d \in \iD$.
    Let $G : \iD \to \iE$ be any diagram such that $\colim_{\iC}GF$ exists.
    We show that $\colim_{\iD}G$ is just $\colim_{\iC}GF$.
    Let $\lambda : GF \to e$ be the colimiting cocone.
    We construct a colimiting cocone $\lambda' : G \to e$ as follows.
    Choose an arbitrary object in $d \dn F$, i.e., an object $c \in \iC$ and a morphism $f : d \to Fc$.
    Then the component $\lambda'_{d}$ is the composite
    \[
      % https://q.uiver.app/#q=WzAsMyxbMCwyLCJHZCJdLFsyLDIsIkdGYyJdLFsyLDAsImUiXSxbMCwxLCJHZiIsMl0sWzEsMiwiXFxsYW1iZGFfe2N9IiwyXSxbMCwyLCJcXGxhbWJkYSdfe2R9Il1d
      \begin{tikzcd}
	&& e \\
	\\
	Gd && GFc
	\arrow["Gf"', from=3-1, to=3-3]
	\arrow["{\lambda_{c}}"', from=3-3, to=1-3]
	\arrow["{\lambda'_{d}}", from=3-1, to=1-3]
      \end{tikzcd}
    \]
    Naturality of $\lambda'$ follows from the naturality of $\lambda$ and the fact that $d \dn F$ is connected for all $d \in \iD$.
    Connectedness means that there is a sequence of zig-zag morphisms so that every triangle in the square commutes.
    \[
      % https://q.uiver.app/#q=WzAsOCxbMCwwLCJHZCJdLFs0LDAsIkdGYyJdLFs0LDIsIkdGY197Mn0iXSxbNCwzLCJHRmNfezN9Il0sWzQsNCwiR0ZjJyJdLFswLDQsIkdkJyJdLFs0LDEsIkdGY197MX0iXSxbNiwyLCJlIl0sWzMsMl0sWzMsNF0sWzAsNSwiR2ciLDJdLFswLDEsIkdmIl0sWzUsNCwiR2YnIiwyXSxbMSw2XSxbMiw2XSxbMSw3LCJcXGxhbWJkYV97Y30iXSxbNCw3LCJcXGxhbWJkYV97Yyd9IiwyXSxbMCw2LCJHZl97MX0iLDFdLFs2LDcsIlxcbGFtYmRhX3tjX3sxfX0iLDFdLFszLDcsIlxcbGFtYmRhX3tjX3szfX0iLDFdLFswLDMsIkdmX3szfSIsMV0sWzAsNCwiRyhmJ2cpIiwyXV0=
      \begin{tikzcd}
	Gd &&&& GFc \\
	&&&& {GFc_{1}} \\
	&&&& {GFc_{2}} && e \\
	&&&& {GFc_{3}} \\
	{Gd'} &&&& {GFc'}
	\arrow[from=4-5, to=3-5]
	\arrow[from=4-5, to=5-5]
	\arrow["Gg"', from=1-1, to=5-1]
	\arrow["Gf", from=1-1, to=1-5]
	\arrow["{Gf'}"', from=5-1, to=5-5]
	\arrow[from=1-5, to=2-5]
	\arrow[from=3-5, to=2-5]
	\arrow["{\lambda_{c}}", from=1-5, to=3-7]
	\arrow["{\lambda_{c'}}"', from=5-5, to=3-7]
	\arrow["{Gf_{1}}"{description}, from=1-1, to=2-5]
	\arrow["{\lambda_{c_{1}}}"{description}, from=2-5, to=3-7]
	\arrow["{\lambda_{c_{3}}}"{description}, from=4-5, to=3-7]
	\arrow["{Gf_{3}}"{description}, from=1-1, to=4-5]
	\arrow["{G(f'g)}"', from=1-1, to=5-5]
      \end{tikzcd}
    \]
    It is easy to verify that this cocone is colimiting.
  \end{node}
\end{proof}

\begin{thm}\label{thm:characterization-of-sifted}
  A small category $\iD$ is sifted if and only if it is connected and the diagonal functor $\Delta$ is final.
\end{thm}
\begin{proof}
  \begin{node}
    By \cref{lem:sifted-empty-binary}, the sifted condition is equivalent to that colimits under $\iD$ commute with the empty product and binary products.
    Let $D : \iD \to \CSet$ be an arbitrary diagram.
    Suppose that $\iD$ is sifted.
    Then,
    \[
      \colim_{\iD}\prod_{\varnothing}D \iso 1
    \]
    The empty product $\prod_{\varnothing}D$ is the terminal object $1$ in $\CSet^{\iD}$.
    It is a functor that assigns each object in $\iD$ to the singleton set.
    A cocone $\lambda : 1 \to A$ encodes how to partition $\iD$ so that objects connected by a morphism in $\iD$ are assigned the same element in $A$.
    The colimit of $1$ is precisely the following cocone
    \[
      % https://q.uiver.app/#q=WzAsMyxbMiwwLCIxKGQnKSJdLFswLDAsIjEoZCkiXSxbMSwxLCJ8XFxpRHwvXFxzaW0iXSxbMSwyLCJbZF0iLDJdLFswLDIsIltkJ10iXSxbMSwwLCIiLDIseyJzdHlsZSI6eyJib2R5Ijp7Im5hbWUiOiJkb3R0ZWQifX19XV0=
      \begin{tikzcd}
	{1(d)} && {1(d')} \\
	& {|\iD|/\sim}
	\arrow["{[d]}"', from=1-1, to=2-2]
	\arrow["{[d']}", from=1-3, to=2-2]
	\arrow[dotted, from=1-1, to=1-3]
      \end{tikzcd}
    \]
    where $d \sim d'$ if and only if there is a zig-zag sequence of morphisms between $d$ and $d'$, i.e., the colimit of $1$ is precisely the set of connected components of $\iD$.
    Since this colimit is the singleton set, $\iD$ is connected.
    Conversely, if $\iD$ is connected, then the colimit of $1$ is the singleton set.
    This is isomorphic to the empty product, so colimits under $\iD$ commute with the empty product.
  \end{node}
  \begin{node}
    Assume that $\Delta$ is final.
    Then for any functors $F, F' : \iD \to \CSet$, the product $F \times F'$ is computed as $G \circ \Delta$, where $G : \iD \times \iD \to \CSet$ is defined by $(d, d') \mapsto Fd \times F'd'$.
    Thus,
    \[
      \colim_{\iD}F \times F' \iso \colim_{\iD}G \circ \Delta \iso \colim_{\iD \times \iD}G \iso \colim_{\iD}F \times \colim_{\iD}F'
    \]
    The last isomorphism follows from the fact that for any set $X$, the functor $X \times \blank$ is a left adjoint, and thus colimit preserving.
    Conversely, assume that colimits under $\iD$ commute with binary products.
    Then for any $d, d' \in \iD$
    \begin{align}
      \colim_{\iD}\iD \times \iD((d, d'), \Delta\blank) &= \colim_{\iD}(\iD(d, \blank) \times \iD(d', \blank))\\
                                              &\iso \colim_{\iD}\iD(d, \blank) \times \colim_{\iD}\iD(d', \blank)\\
                                              &\iso 1
    \end{align}
    Thus, the category of cospans $(d, d') \dn \Delta$ is connected.
  \end{node}
\end{proof}

\begin{cor}\label{cor:finite-coproducts-then-sifted}
  Every small category $\iD$ with finite coproducts is sifted.
\end{cor}
\begin{proof}
  By \cref{thm:characterization-of-sifted}, it suffices to show that $\iD$ is connected and that for all $d, d' \in \iD$, the category of cospans $(d, d') \dn \Delta$ is connected.
  $\iD$ has an initial object, so it is connected.
  $\iD$ has binary coproducts, so for any $d, d' \in \iD$, their coproduct in $\iD$ corresponds to the initial object in $(d, d') \dn \Delta$.
  Thus, the category of cospans is connected.
\end{proof}

\begin{lem}\label{lem:functor-category-sifted-colimits-commute-with-finite-products}
  For any small category $\iC$, sifted colimits in $\CSet^{\iC}$ commute with finite products.
\end{lem}
\begin{proof}
  Let $D : S \times J \to \CSet^{\iC}$ be a diagram where $S$ is a sifted category and $J$ is a finite set.
  The colimit $\colim_{S}\prod_{J}D(s,j)$ is computed componentwise: $\colim_{S}\prod_{J}D(s,j)(c)$ is the colimit $\colim_{S}\prod_{J}\ev_{c}D(s,j)$.
  By definition, sifted colimits commute with finite products in $\CSet$.
  These isomorphisms assemble into a natural isomorphism:
  \[
    \colim_{S}\prod_{J}D(s,j) \iso \prod_{J}\colim_{S}D(s,j)
  \]
\end{proof}

\begin{lem}
  Let $A : \dN\op \to \dA$ be a Lawvere theory.
  The full inclusion functor $\iota : \Alg(A) \to \CSet^{\dA}$ creates colimits of sifted diagrams.
\end{lem}
\begin{proof}
  Let $S$ be a sifted category and $D : S \to \Alg(A)$ be a diagram.
  Since $\CSet^{\dA}$ is cocomplete (has all sifted colimits) and $\iota$ reflects all colimits, it suffices to show that $\colim_{S}\iota \circ D$ is a finite-product preserving functor, but this is immediate from \cref{lem:functor-category-sifted-colimits-commute-with-finite-products}.
\end{proof}

\begin{thm}\label{thm:algebraic-categories-has-sifted-colimits}
  Let $A : \dN\op \to \dA$ be a Lawvere theory.
  The category $\Alg(A)$ has sifted colimits.
\end{thm}
\begin{proof}
  Since the full inclusion $\iota$ creates sifted colimits and $\CSet^{\dA}$ is cocomplete, it follows that $\Alg(A)$ has sifted colimits.
\end{proof}

\begin{cor}
  Let $A : \dN\op \to \dA$ be a Lawvere theory.
  Sifted colimits commute with finite products in $\Alg(A)$.
\end{cor}
\begin{proof}
  The full inclusion $\iota$ reflects isomorphism.
  Thus, the isomorphism in \cref{lem:functor-category-sifted-colimits-commute-with-finite-products} is reflected to an isomorphism in $\Alg(A)$.
\end{proof}

\begin{node}
  Using the same argument, one can prove that $\Alg(A)$ has filtered colimits, and filtered colimits commute with finite limits in any algebraic category.
\end{node}

\section{Reflexive coequalizers}
\label{sec:reflexive-coequalizers}

\begin{defn}
  Let $\iJ$ be the following category
  \[
    % https://q.uiver.app/#q=WzAsMixbMCwwLCJcXGJ1bGxldCJdLFsyLDAsIlxcYnVsbGV0Il0sWzAsMSwiZiIsMCx7Im9mZnNldCI6LTJ9XSxbMCwxLCJnIiwyLHsib2Zmc2V0IjoyfV0sWzEsMCwicyIsMV1d
    \begin{tikzcd}
      \bullet && \bullet
      \arrow[shift left=2, from=1-1, to=1-3]
      \arrow[shift right=2, from=1-1, to=1-3]
      \arrow[from=1-3, to=1-1]
    \end{tikzcd}
  \]
  Colimits of diagrams of shape $\iJ$ are called \emph{reflexive coequalizers}.
  These are equalizers of parallel pairs that admit a common section.
\end{defn}

\begin{node}
  It is easy to verify that $\iJ$ is a sifted category.
  Thus, reflexive coequalizers are sifted colimits.
\end{node}

\section{Algebraic categories as free cocompletions}
\label{sec:algebraic-categories-as-free-cocompletions}

\begin{node}\label{node:colimit-of-representables}
  Let $\iC$ be a small category.
  Any functor $F : \iC \to \CSet$ is naturally isomorphic to the colimit $\colim\yon\Pi$, where $\Pi : (\yon \dn F)\op \to \iC\op$ is the projection functor and $F : 1 \to \CSet^{\iC}$ is a functor that picks out $F : \iC \to \CSet$.
\end{node}

\begin{node}\label{node:colim-of-representable}%
  For any small category $\iC$ and object $c \in \iC$, the Hom functor $\iC(c, \blank)$ preserves limits.
  Thus, given a Lawvere theory $A : \dN\op \to \dA$, each object $a \in \dA$ determines an algebra $\dA(a, \blank) : \dA \to \CSet$.
  By \cref{node:colimit-of-representables}, every algebra $F : \dA \to \CSet$ is the colimit $\colim\yon\Pi$.
\end{node}

\begin{lem}\label{lem:algebras-are-sifted-colimits}
  Given an algebraic theory $A$, for any functor $F : \dA \to \CSet$ the following conditions are equivalent:
  \begin{enumerate}
  \item $F$ is an algebra.
  \item $(\yon \dn F)\op$ is a sifted category.
  \item $F$ is a sifted colimit of representable algebras.
  \end{enumerate}
\end{lem}
\begin{proof}
  \begin{node}
    (1) $\imp$ (2).
    Assume that $F$ is an algebra.
    By \cref{cor:finite-coproducts-then-sifted} and duality, it suffices to show that $(\yon \dn F)$ has finite products.
    Since $F$ is an algebra, $F1$ is the singleton set.
    Thus, $(1, *)$ is the terminal object in $(\yon \dn F)$.
    Binary products are defined pointwise: $(a, x) \times (b, y)$ is $(a \times b, (x, y))$.
    This is well-defined because $Fa \times Fb \iso F(a \times b)$.
  \end{node}
  \begin{node}
    (2) $\imp$ (3).
    Since $(\yon \dn F)\op$ is sifted, colimits of this shape are sifted colimits.
    By \cref{node:colim-of-representable}, any functor is a sifted colimit of representable algebras.
  \end{node}
  \begin{node}
    (3) $\imp$ (1).
    By \cref{thm:algebraic-categories-has-sifted-colimits}, $\Alg(A)$ is closed under sifted colimits.
  \end{node}
\end{proof}

\begin{thm}\label{thm:algebraic-categories-cocomplete}
  Every algebraic category is cocomplete.
\end{thm}
\begin{proof}
  \begin{node}
    We have already shown that algebraic categories have reflexive coequalizers and filtered colimits.
    It suffices to construct finite coproducts.
  \end{node}
  \begin{node}
    Let $F, G : \dA \to \CSet$ be algebras of type $A$.
    By \cref{lem:algebras-are-sifted-colimits}, $F$ and $G$ can be expressed as sifted colimits of representable algebras.
    Consider the category $\iD = (\yon \dn F)\op \times (\yon \dn G)\op$ and the diagram $H : \iD \to \Alg(A)$ defined by
    \[
      H : ((a \in \dA, x \in Fa), (a' \in \dA, y \in Ga)) \mapsto \yon\Pi_{A}(a, x) + \yon\Pi_{B}(a', y)
    \]
    Note that $\yon\Pi_{A}(a, x) + \yon\Pi_{B}(a', y)$ exists in $\Alg(A)$.
    Since $\iD$ is the product of two sifted categories, $\iD$ itself is also a sifted category.
    Thus, the colimit of this diagram exists in $\Alg(A)$.
    Since colimits commute with coproducts, we have
    \[
      \colim H \iso \colim_{(\yon \dn F)\op}\yon\Pi_{A} + \colim_{(\yon \dn G)\op}\yon\Pi_{B} \iso F + G
    \]
  \end{node}
\end{proof}

\section{Properties of algebras}
\label{sec:properties-of-algebras}

\begin{lem}
  Let $C' \in \iC$ be a retract of $C \in \iC$.
  The functor $\iC(C', \blank)$ preserves any colimit preserved by $\iC(C, \blank)$.
\end{lem}
\begin{proof}
  Let $r : C \to C'$ and $s : C' \to C$ be morphisms such that $rs = \id_{C'}$.
  The following diagram is an absolute coequalizer
  \[
    % https://q.uiver.app/#q=WzAsMyxbMCwwLCJcXGlDKEMsXFxibGFuaykiXSxbMiwwLCJcXGlDKEMsXFxibGFuaykiXSxbNCwwLCJcXGlDKEMnLFxcYmxhbmspIl0sWzEsMiwiXFxpQyhzLFxcYmxhbmspIl0sWzAsMSwiXFxpQyhyLFxcYmxhbmspIFxcY2lyYyBcXGlDKHMsIFxcYmxhbmspIiwwLHsib2Zmc2V0IjotMX1dLFswLDEsIlxcaWQiLDIseyJvZmZzZXQiOjF9XSxbMiwxLCJcXGlDKHIsXFxibGFuaykiLDAseyJjdXJ2ZSI6LTJ9XSxbMSwwLCJcXGlkIiwwLHsiY3VydmUiOi0yfV1d
    \begin{tikzcd}
      {\iC(C,\blank)} && {\iC(C,\blank)} && {\iC(C',\blank)}
      \arrow["{\iC(s,\blank)}", from=1-3, to=1-5]
      \arrow["{\iC(r,\blank) \circ \iC(s, \blank)}", shift left, from=1-1, to=1-3]
      \arrow["\id"', shift right, from=1-1, to=1-3]
      \arrow["{\iC(r,\blank)}", curve={height=-12pt}, from=1-5, to=1-3]
      \arrow["\id", curve={height=-12pt}, from=1-3, to=1-1]
    \end{tikzcd}
  \]
  Thus, any isomorphism $\iC(C, \colim D) \iso \colim\iC(C, D(d))$ extends to an isomorphism $\iC(C', \colim D) \iso \colim\iC(C', D(d))$.
\end{proof}

\begin{cor}
  If $C$ is regular projective (resp., finitely presentable), then its retracts are regular projective (resp., finitely presentable).
\end{cor}

\begin{thm}\label{thm:represented-functors-perfectly-presentable}
  Let $\cT$ be a small category.
  In $\CSet^{\cT}$ the representable functors are perfectly presentable.
  If $\cT$ is a Lawvere theory, the representable functors are perfectly presentable objects in $\Alg(\cT)$.
\end{thm}
\begin{proof}
  \begin{node}\label{node:represented-functors-perfectly-presentable}
    By the Yoneda Lemma,
    \[
      \CSet^{\cT}(\cT(t,\blank), \colim_{\iJ} D) \iso (\colim_{\iJ} D)(t)
    \]
    Since colimits in $\CSet^{\cT}$ are computed objectwise, we have
    \[
      (\colim_{\iJ} D)(t) = \colim_{\iJ}(D_{j}(t)) \iso \colim_{\iJ}\CSet^{\cT}(\cT(t,\blank), D_{j})
    \]
    Since $\CSet^{\cT}(\cT(t,\blank), \blank)$ preserves any colimit, it preserves sifted ones.
    Thus, $\cT(t,\blank)$ is perfectly presentable in $\CSet^{\cT}$.
  \end{node}
  \begin{node}
    If $\cT$ is a Lawvere theory, then by \cref{thm:algebraic-categories-has-sifted-colimits}, sifted colimits in $\CSet^{\cT}$ are also in $\Alg(\cT)$.
    Thus, the argument in \cref{node:represented-functors-perfectly-presentable} works for any sifted category $\iJ$.
  \end{node}
\end{proof}

\begin{lem}\label{lem:coproduct-generators-epimorphism}
  Let $\iC$ be a category with coproducts and $\iG$ be a set of objects of $\iC$.
  The following are equivalent:
  \begin{enumerate}
  \item $\iG$ is a family of generators;
  \item for every object $C \in \iC$, the canonical morphism
    \[
      e_{C} : \left(\coprod_{(G, g) \in \iG \dn C} G\right) \to C
    \]
    is an epimorphism.
  \end{enumerate}
\end{lem}
\begin{proof}
  \begin{node}
    Suppose that $\iG$ is a set of generators.
    Let $C \in \iC$ and $u, v : C \toto D$ be a parallel pair of morphisms such that $ue_{C} = ve_{C}$.
    Since $\iG$ is a set of generators, morphisms out of $\iG$ are jointly epimorphic.
    Morphisms from $\iG$ to $C$ are encoded by the canonical morphism $e_{C}$.
    The hypothesis $ue_{C} = ve_{C}$ means that $u$ and $v$ form a pair of morphisms indistinguishable by $\iG$.
    Thus, $u = v$.
  \end{node}
  \begin{node}
    Conversely, suppose that $\gamma_{C}$ is an epimorphism for each $C \in \iC$.
    The encodes precisely the jointly epimorphic condition.
    Thus, $\iG$ is a set of generators.
  \end{node}
\end{proof}

\begin{defn}
  In a category $\iA$ with coproducts, a set of objects $\iG$ is a strong generator if and only if
  \[
    e_{A} : \left(\coprod_{(G, g) \in \iG \dn A} G\right) \to A
  \]
  is an extremal epimorphism for all $A \in \iA$.
\end{defn}

\begin{cor}\label{cor:colimits-strong-generators}
  Let $\iA$ be a category with colimits.
  If every object of $\iA$ is a colimit of objects from a set $\iG$, then $\iG$ is a strong generator.
\end{cor}
\begin{proof}
  \begin{node}
    The hypothesis states that the canonical morphism
    \[
      e_{A} : \left(\coprod_{(G, g) \in \iG \dn A} G\right) \to A
    \]
    is a regular epimorphism for all $A \in \iA$.
    By \cref{lem:coproduct-generators-epimorphism}, $\iG$ is a generating set.
    Since regular epimorphisms are extremal, $\iG$ is strong.
  \end{node}
\end{proof}

\begin{lem}\label{lem:coproduct-completion-generator}
  Suppose that $\iA$ is cocomplete and $\iG$ is a set of (strong) generators of $\iA$, then the coproduct completion of $\iG$ in $\iA$ is also a set of (strong) generators.
\end{lem}
\begin{proof}
  Morphisms out of a coproduct in $\sqcup\iG$ consists of morphisms out of $\iG$.
  Thus, any indistinguishable pair of morphisms in $\iA$ remains indistinguishable.
  If $\iG$ is strong, then consider the following commutative diagram:
  \[
    % https://q.uiver.app/#q=WzAsMyxbMCwyLCJcXGNvcHJvZF97KEcsZykgXFxpbiBcXGlHIFxcZG4gXFxjb3Byb2R9RyJdLFswLDAsIlxcY29wcm9kX3soRyxnKSBcXGluIFxcc3FjdXBcXGlHIFxcZG4gQX1HIl0sWzIsMCwiQSJdLFsxLDIsImVfe0F9JyJdLFswLDIsImVfe0F9IiwyXSxbMCwxLCJlX3tcXGNvcHJvZH0iXV0=
    \begin{tikzcd}
      {\coprod_{(G,g) \in \sqcup\iG \dn A}G} && A \\
      \\
      {\coprod_{(G,g) \in \iG \dn \coprod}G}
      \arrow["{e_{A}'}", from=1-1, to=1-3]
      \arrow["{e_{A}}"', from=3-1, to=1-3]
      \arrow["{e_{\coprod}}", from=3-1, to=1-1]
    \end{tikzcd}
  \]
  Since $e_{A}$ is extremal, $e_{A}'$ is also extremal.
\end{proof}

\begin{lem}
  Let $\iA$ be a cocomplete category.
  If $\iG$ is a strong generator consisting of perfectly presentable objects and closed under coproducts, then $\iG$ is dense.
  In other words, for any $a \in \iA$, the domain functor
  \[
    \dom_{a} : \iG \dn a \to \iA, (g : G \to a) \mapsto G
  \]
  has $a$ as a colimit.
\end{lem}
\begin{proof}
  Let $\alpha : \dom_{a} \to a'$ be a colimiting cocone.
  Consider the following commutative diagram:
  \[
    % https://q.uiver.app/#q=WzAsMyxbMiwwLCJcXGNvcHJvZF97XFxpRyBcXGRuIGF9RyJdLFs0LDEsImEiXSxbMCwxLCJhJyJdLFswLDIsIlxcYWxwaGFfe2Vfe2F9fSIsMl0sWzAsMSwiZV97YX0iXSxbMiwxLCJcXGxhbWJkYSIsMix7InN0eWxlIjp7ImJvZHkiOnsibmFtZSI6ImRhc2hlZCJ9fX1dXQ==
    \begin{tikzcd}
      && {\coprod_{\iG \dn a}G} \\
      {a'} &&&& a
      \arrow["{\alpha_{e_{a}}}"', from=1-3, to=2-1]
      \arrow["{e_{a}}", from=1-3, to=2-5]
      \arrow["\lambda"', dashed, from=2-1, to=2-5]
    \end{tikzcd}
  \]
  Since $e_{a}$ is an extremal epimorphism, so is $\lambda$.
  To show that $\lambda$ is an isomorphism, it suffices to show that $\lambda$ is a monomorphism.
  Let $u, v : c \toto a'$ be a parallel pair of morphisms such that $\lambda u = \lambda v$.
  \todo{}
\end{proof}

\todo{Ivan's LN3: A variety is a cocomplete category with a dense generator consisting of ``finite projective'' objects.}
\begin{thm}[Characterization of algebraic categories]
  The following conditions are equivalent:
  \begin{enumerate}
  \item $\iA$ is algebraic.
  \item $\iA$ is cocomplete and has a set $G$ of perfectly presentable objects such that every object of $\iA$ is a sifted colimit of objects of $G$.
  \item $\iA$ is cocomplete and has a strong generator consisting of perfectly presentable objects.
  \end{enumerate}
\end{thm}
\begin{proof}
  \begin{node}
    (1) $\imp$ (2).
    By \cref{thm:algebraic-categories-cocomplete}, $\iA$ is cocomplete.
    By \cref{thm:represented-functors-perfectly-presentable} the representable functors are perfectly presentable.
    Finally, by \cref{lem:algebras-are-sifted-colimits}, every algebra in $\iA$ is a sifted colimit of representable algebras.
  \end{node}
  \begin{node}
    (2) $\imp$ (3).
    $G$ is a strong generator by \cref{cor:colimits-strong-generators}.
  \end{node}
  \begin{node}
    (3) $\imp$ (1).
    Let $\sqcup\iG$ be the coproduct completion of $\iG$ in $\iA$.
    By \cref{lem:coproduct-completion-generator}, $\sqcup\iG$ is a strong generator.
    We claim that $\Alg(\sqcup\iG\op) \eqv \iA$, where $\sqcup\iG$ is a full subcategory of $\iA$.
    \todo{}
  \end{node}
\end{proof}


\bibliographystyle{alpha}
\bibliography{lawvere}

\end{document}
