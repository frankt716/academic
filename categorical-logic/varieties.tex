\documentclass{amsart}
\input{decls}
\title{Varieties}
\author{Frank Tsai}
\date{\today}
%\thanks{}
\begin{document}
\maketitle
\tableofcontents

\section{Definitions}
\label{sec:definitions}

\begin{defn}
  An object $G$ of a category $\iC$ is \emph{regular projective} if the hom-functor
  \[
    \iC(G,\blank) : \iC \to \mathsf{Set}
  \]
  preserves regular epimorphism.
  In other words, $G$ has the left-lifting property against regular epimorphisms.
  % https://q.uiver.app/#q=WzAsMyxbMCwwLCJcXGlDIl0sWzEsMSwiXFxpRCJdLFsyLDAsIlxcaUUiXSxbMCwxLCJLIiwyXSxbMCwyLCJGIl0sWzEsMiwiXFxsYW5fe0t9RiIsMix7InN0eWxlIjp7ImJvZHkiOnsibmFtZSI6ImRhc2hlZCJ9fX1dLFs0LDEsIiIsMCx7InNob3J0ZW4iOnsic291cmNlIjoyMH19XV0=
\begin{tikzcd}
	\iC && \iE \\
	& \iD
	\arrow["K"', from=1-1, to=2-2]
	\arrow[""{name=0, anchor=center, inner sep=0}, "F", from=1-1, to=1-3]
	\arrow["{\lan_{K}F}"', dashed, from=2-2, to=1-3]
	\arrow[shorten <=3pt, Rightarrow, from=0, to=2-2]
\end{tikzcd}
\end{defn}

\begin{defn}
  An object $G$ of a category $\iC$ is \emph{finitely presentable} (resp., \emph{finitely generated}) if the hom-functor $\iC(G,\blank)$ preserves filtered colimits (resp., filtered colimits of monomorphisms).
  That is, every morphism $G \to \colim F$ factors through some injection map.
  % https://q.uiver.app/#q=WzAsNCxbMCwwLCJcXG1hdGhybXtBZmZ9X3trfShcXG1hdGhybXtBZmZ9X3trfShBKSkiXSxbMiwwLCJcXG1hdGhybXtBZmZ9X3trfShBKSJdLFswLDIsIlxcbWF0aHJte0FmZn1fe2t9KEEpIl0sWzIsMiwiQSJdLFsxLDMsIlxcbWF0aHJte2V2fV97QX0iXSxbMiwzLCJcXG1hdGhybXtldn1fe0F9IiwyXSxbMCwyLCJcXG1hdGhybXtBZmZ9X3trfShcXG1hdGhybXtldn1fe0F9KSIsMl0sWzAsMSwiXFxtdV97QX0iXV0=
\begin{tikzcd}
	{\mathrm{Aff}_{k}(\mathrm{Aff}_{k}(A))} && {\mathrm{Aff}_{k}(A)} \\
	\\
	{\mathrm{Aff}_{k}(A)} && A
	\arrow["{\mathrm{ev}_{A}}", from=1-3, to=3-3]
	\arrow["{\mathrm{ev}_{A}}"', from=3-1, to=3-3]
	\arrow["{\mathrm{Aff}_{k}(\mathrm{ev}_{A})}"', from=1-1, to=3-1]
	\arrow["{\mu_{A}}", from=1-1, to=1-3]
\end{tikzcd}
\end{defn}

\begin{defn}
  A \emph{subcopower} of a copower $\coprod_{I}G$ is the canonical morphism
  \[
    \coprod_{i}G : \coprod_{I'}G \to \coprod_{I}G
  \]
  where $i : I' \into I$ is the subset inclusion function.
\end{defn}

\begin{defn}
  An object $G$ of a category $\iC$ is \emph{abstractly finite} if it has copowers, and every morphism $f : G \to \coprod_{I}G$ has the left-lifting property against a finite subcopower.
  % https://q.uiver.app/#q=WzAsMyxbMSwwLCJGYyJdLFswLDIsIlxcbGFuX3tLfUYoZCkiXSxbMiwyLCJcXGxhbl97S31GKGQnKSJdLFswLDEsIlxcbGFtYmRhXntkfV97Zn0iLDJdLFswLDIsIlxcbGFtYmRhXntkJ31fe2dmfSJdLFsxLDIsIiIsMix7InN0eWxlIjp7ImJvZHkiOnsibmFtZSI6ImRhc2hlZCJ9fX1dXQ==
\begin{tikzcd}
	& Fc \\
	\\
	{\lan_{K}F(d)} && {\lan_{K}F(d')}
	\arrow["{\lambda^{d}_{f}}"', from=1-2, to=3-1]
	\arrow["{\lambda^{d'}_{gf}}", from=1-2, to=3-3]
	\arrow[dashed, from=3-1, to=3-3]
\end{tikzcd}
  where $I'$ is a finite subset of $I$.
\end{defn}

\begin{defn}
  For an set $S$, an \emph{$S$-sorted signature} is a collection $\Sigma$ of function symbols $\sigma$ with prescribed arities.
  The category of $S$-sorted sets is the functor category $\mathsf{Set}^{S}$.
  Let $X \in \mathsf{Set}^{S}$ be an $S$-sorted set of variables.
  The $S$-sorted set $F_{\Sigma}X$ of terms is the least set so that:
  \begin{enumerate}
  \item variables $x \in X$ are in $F_{\Sigma}X$, i.e., variables of sort $s$ are terms of sort $s$, and
  \item for each function symbol $f : s_{1} \times \cdots \times s_{n} \to s$ and terms $t_{i}$ of sort $s_{i}$, $f(t_{1},\ldots,t_{n})$ is a term of sort $s$.
  \end{enumerate}
\end{defn}

\begin{defn}
  A \emph{$\Sigma$-algebra} is a sorted set $A$ equipped with sorted functions $\sigma_{A} : A_{s_{1}} \times \cdots \times A_{s_{n}} \to A_{s}$ for every function symbol $\sigma : s_{1} \times \cdots \times s_{n} \to s$.
  Given another $\Sigma$-algebra $B$, a \emph{$\Sigma$-homomorphism} is a sorted map $f : A \to B$ preserving the operations: for each function symbol $\sigma : s_{1} \times \cdots \times s_{n} \to s$, one has
  \[
    f_{s} \circ \sigma_{A} = \sigma_{B} \circ (f_{s_{1}} \times \cdots \times f_{s_{n}})
  \]

  The category of $\Sigma$-algebras and their homomorphisms is denoted $\Sigma\mathsf{Alg}$.
\end{defn}

\begin{defn}
  An \emph{equation} in $n$ variables $x_{i}$ of sort $s_{i}$ consists of two terms $t$ and $u$ in $F_{\Sigma}\{x_{1},\ldots,x_{n}\}$ of the same sort.
  It is written as
  \[
    t = u
  \]
  A $\Sigma$-algebra $A$ \emph{satisfies} this equation if for every sorted function $f : \{x_{1},\ldots,x_{n}\} \to A$ the free homomorphism $\fbar : F_{\Sigma}\{x_{i}\} \to A$ fulfills $\fbar(t) = \fbar(u)$.
  In other words, a $\Sigma$-algebra $A$ satisfies an equation $t = u$ of $n$ variables if for every valuation of those $n$ variables, the two terms have the same interpretation.
\end{defn}

\begin{defn}
  Given a set $\cE$ of equations, we write $(\Sigma,\cE)\mathsf{Alg}$ for the full subcategory of $\Sigma$-algebras that satisfy every equation in $\cE$.
\end{defn}

\begin{thm}
  $(\Sigma,\cE)\mathsf{Alg}$ is a reflective subcategory of $\Sigma\mathsf{Alg}$.
\end{thm}
\begin{proof}
  The standard construction works: we define the localization $L : \Sigma\mathsf{Alg} \to (\Sigma,\cE)\mathsf{Alg}$ as follows:
  \begin{enumerate}
  \item Take the underlying $S$-sorted set $A$ of a given $\Sigma$-algebra and regard $A$ as an $S$-sorted set of variables.
  \item Take the $S$-sorted set of terms $F_{\Sigma}A$ and quotient this $S$-sorted set by $\cE$.
    The quotient $F_{\Sigma}A/\cE$ is the sought after $(\Sigma,\cE)$-algebra.
  \item Each homomorphism $A \to B$ extends to a morphism $F_{\Sigma}A \to F_{\Sigma}B$ by induction on the structure of those terms in $F_{\Sigma}A$.
  \item These morphisms respect the equations in $\cE$, yielding the sought after morphism $F_{\Sigma}A/\cE \to F_{\Sigma}B/\cE$
  \end{enumerate}
\end{proof}

\bibliographystyle{alpha}
\bibliography{all}

\end{document}
