\documentclass{amsart}
\input{decls}
\title{}
\author{Frank Tsai}
\date{\today}
%\thanks{}
\begin{document}
\maketitle
\tableofcontents

\section{Introduction}
\label{sec:introduction}

Recall that an adjunction consists of
\begin{itemize}
\item $F,G : \iC \toot \iD$;
\item $\iD(Fc, d) \iso \iC(c, Gd)$ natural in both $c$ and $d$.
\end{itemize}

Fix $c$, the isomorphism $\iD(Fc, \blank) \iso \iC(c, G\blank)$ implies that $Fc$ represents the functor $\iC(c, G\blank) : \iD \to \mathsf{Set}$.
Then by the Yoneda Lemma, this isomorphism determines a map $\eta_{c} : c \to GFc$, the transpose of $1_{Fc}$.
Naturality in $c$ then implies that $\eta_{c}$ assembles into a natural transformation $\eta : \id_{\iC} \to GF$, called the \emph{unit} of the adjunction.

Similarly, fix $d$, the isomorphism $\iC(\blank, Gd) \iso \iD(F\blank, d)$ implies that $Gd$ represents the functor $\iD(F\blank, d) : \iC\op \to \mathsf{Set}$.
The isomorphism then determines a map $\epsilon_{d} : FGd \to d$, which assembles into a natural transformation $\epsilon : FG \to \id_{\iC}$ called the \emph{counit} of the adjunction.

\begin{lem}
  $\eta$ is natural in $c$.
\end{lem}
\begin{proof}
  To show that the left-hand square commutes, it suffices to show the right-hand square commutes, which is evident.
  \begin{mathpar}
  % https://q.uiver.app/#q=WzAsNCxbMCwwLCJjIl0sWzAsMiwiYyciXSxbMiwwLCJHRmMiXSxbMiwyLCJHRmMnIl0sWzAsMSwiZiIsMl0sWzIsMywiR0ZmIl0sWzAsMiwiXFxldGFfe2N9Il0sWzEsMywiXFxldGFfe2MnfSIsMl1d
\begin{tikzcd}
	c && GFc \\
	\\
	{c'} && {GFc'}
	\arrow["f"', from=1-1, to=3-1]
	\arrow["GFf", from=1-3, to=3-3]
	\arrow["{\eta_{c}}", from=1-1, to=1-3]
	\arrow["{\eta_{c'}}"', from=3-1, to=3-3]
\end{tikzcd} \leftrightsquigarrow % https://q.uiver.app/#q=WzAsNCxbMCwwLCJGYyJdLFswLDIsIkZjJyJdLFsyLDAsIkZjIl0sWzIsMiwiRmMnIl0sWzAsMSwiRmYiLDJdLFsyLDMsIkZmIl0sWzAsMiwiXFxpZF97Y30iXSxbMSwzLCJcXGlkX3tjJ30iLDJdXQ==
\begin{tikzcd}
	Fc && Fc \\
	\\
	{Fc'} && {Fc'}
	\arrow["Ff"', from=1-1, to=3-1]
	\arrow["Ff", from=1-3, to=3-3]
	\arrow["{\id_{c}}", from=1-1, to=1-3]
	\arrow["{\id_{c'}}"', from=3-1, to=3-3]
\end{tikzcd}
\end{mathpar}
\end{proof}

The Yoneda Lemma also tells us how to compute the transpose of an arbitrary map $f : Fc \to d$.
$f^{\dag}$ is given by $\iC(c, Gf)(\eta_{c}) = Gf \circ \eta_{c}$.
Similarly, the transpose $g^{\dag}$ of an arbitrary map $g : c \to Gd$ is given by $\iD(Fg, d)(\epsilon_{d}) = \epsilon_{d} \circ Fg$.

We've shown that every adjunction has a unit and a counit.
Conversely, suppose that we are given two natural transformations $\eta : \id_{\iC} \to GF$ and $\epsilon : FG \to \id_{\iD}$.
When does this pair of natural transformations define the unit and the counit of an adjunction?
Recall that $\eta_{c} : c \to GFc$ is supposed to be the transpose of $\id_{Fc}$.
Thus, it's necessary that $\epsilon_{Fc} \circ F\eta_{c} = \id_{Fc}$.
Similarly, $\epsilon_{d} : FG \to d$ is supposed to be the transpose of $\id_{Gd}$.
Thus, $G\epsilon_{d} \circ \eta_{Gc} = \id_{Gd}$.

\begin{lem}
  Given a pair of functors $F,G : \iC \toot \iD$, the following data are equivalent:
  \begin{itemize}
  \item $\iD(Fc, d) \iso \iC(c, Gd)$ natural in both $c$ and $d$.
  \item A pair of natural transformations $\eta : \id_{\iC} \to GF$ and $\epsilon : FG \to \id_{\iD}$ satisfying the \emph{triangle identities}:
    \begin{mathpar}
      % https://q.uiver.app/#q=WzAsMyxbMCwwLCJGIl0sWzIsMCwiR0ZHIl0sWzIsMiwiRiJdLFswLDEsIkZcXGV0YSJdLFsxLDIsIlxcZXBzaWxvbiBGIl0sWzAsMiwiIiwyLHsibGV2ZWwiOjIsInN0eWxlIjp7ImhlYWQiOnsibmFtZSI6Im5vbmUifX19XV0=
\begin{tikzcd}
	F && GFG \\
	\\
	&& F
	\arrow["F\eta", from=1-1, to=1-3]
	\arrow["{\epsilon F}", from=1-3, to=3-3]
	\arrow[Rightarrow, no head, from=1-1, to=3-3]
\end{tikzcd} \and % https://q.uiver.app/#q=WzAsMyxbMCwwLCJHIl0sWzIsMCwiRkdGIl0sWzIsMiwiRyJdLFswLDEsIlxcZXRhIEciXSxbMSwyLCJHXFxlcHNpbG9uIl0sWzAsMiwiIiwyLHsibGV2ZWwiOjIsInN0eWxlIjp7ImhlYWQiOnsibmFtZSI6Im5vbmUifX19XV0=
\begin{tikzcd}
	G && FGF \\
	\\
	&& G
	\arrow["{\eta G}", from=1-1, to=1-3]
	\arrow["G\epsilon", from=1-3, to=3-3]
	\arrow[Rightarrow, no head, from=1-1, to=3-3]
\end{tikzcd}
    \end{mathpar}
  \end{itemize}
\end{lem}
\begin{proof}
  ($1 \implies 2$): This follows immediately by how transposes are computed.

  ($2 \implies 1$): Define the transpose by either $Gf \circ \eta_{c}$ or $\epsilon_{d} \circ Fg$ depending on where the map lives.
  The triangle identities guarantee the transpose operations to be mutual inverses.
\end{proof}

\begin{cor}
  Let $A$ and $B$ be posets and $F : A \to B$ and $G : B \to A$ form a Galois connection with $F \dashv G$, then $F$ and $G$ satisfy the following fixed point formulas:
  \begin{mathpar}
    FGF = F \and GFG = G
  \end{mathpar}
\end{cor}
\begin{proof}
  Immediate from the triangle identities.
\end{proof}

\begin{eg}
  Let $f : A \to B$ be a function.
  The direct-image $f_{*} : PA \to PB$ and the inverse-image $f\inv : PB \to PA$ form a Galois connection with $f_{*} \dashv f\inv$.
  Thus, for any $X \subseteq A$ and $Y \subseteq B$
  \begin{mathpar}
    f_{*}(X) = f_{*}(f\inv(f_{*}(A))) \and f\inv(Y) = f\inv(f_{*}(f\inv(Y)))
  \end{mathpar}
\end{eg}

\begin{defn}
  A \emph{morphism of adjunctions} from $F \dashv G$ to $F' \dashv G'$ consists of a pair of functors
  \input{img/2-adj-morphism01}
  such that the square with left adjoints and the square with right adjoints commute and one of the following equivalent conditions is satisfied:
  \begin{itemize}
  \item $H\eta = \eta'H$.
  \item $K\epsilon = \epsilon'K$.
  \item The following diagram commutes for all $c \in \iC$ and $d \in \iD$.
    % https://q.uiver.app/#q=WzAsNixbMCwwLCJcXGlEKEZjLGQpIl0sWzIsMCwiXFxpQyhjLEdkKSJdLFswLDEsIlxcaUQnKEtGYyxLZCkiXSxbMiwxLCJcXGlDJyhIYyxIR2QpIl0sWzAsMiwiXFxpRCcoRidIYyxLZCkiXSxbMiwyLCJcXGlDJyhIYyxHJ0tkKSJdLFswLDEsIlxcaXNvIl0sWzQsNSwiXFxpc28iLDJdLFswLDIsIksiLDJdLFsxLDMsIkgiXSxbMiw0LCIiLDIseyJsZXZlbCI6Miwic3R5bGUiOnsiaGVhZCI6eyJuYW1lIjoibm9uZSJ9fX1dLFszLDUsIiIsMCx7ImxldmVsIjoyLCJzdHlsZSI6eyJoZWFkIjp7Im5hbWUiOiJub25lIn19fV1d
\[\begin{tikzcd}
	{\iD(Fc,d)} && {\iC(c,Gd)} \\
	{\iD'(KFc,Kd)} && {\iC'(Hc,HGd)} \\
	{\iD'(F'Hc,Kd)} && {\iC'(Hc,G'Kd)}
	\arrow["\iso", from=1-1, to=1-3]
	\arrow["\iso"', from=3-1, to=3-3]
	\arrow["K"', from=1-1, to=2-1]
	\arrow["H", from=1-3, to=2-3]
	\arrow[Rightarrow, no head, from=2-1, to=3-1]
	\arrow[Rightarrow, no head, from=2-3, to=3-3]
\end{tikzcd}\]
  \end{itemize}
\end{defn}

\bibliographystyle{alpha}
\bibliography{all}

\end{document}
