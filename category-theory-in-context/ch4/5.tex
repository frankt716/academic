\documentclass{amsart}
\input{decls}
\title{Adjunctions, limits, and colimits}
\author{Frank Tsai}
\date{\today}
%\thanks{}
\begin{document}
\maketitle
\tableofcontents

\section{Introduction}
\label{sec:introduction}

\begin{lem}
  A category $\iC$ admits all limits of diagrams indexed by a small category $\iJ$ if and only if the constant diagram functor $\Delta : \iC \to \iC^{\iJ}$ admits a right adjoint, and admits all colimits of $\iJ$-indexed diagrams if and only if $\Delta$ admits a left adjoint:
  % https://q.uiver.app/#q=WzAsMixbMCwwLCJcXGlDIl0sWzIsMCwiXFxpQ157XFxpSn0iXSxbMCwxLCJcXERlbHRhIiwxXSxbMSwwLCJcXGNvbGltIiwyLHsiY3VydmUiOjN9XSxbMSwwLCJcXGxpbSIsMCx7ImN1cnZlIjotM31dLFsyLDQsIiIsMSx7ImxldmVsIjoxLCJzdHlsZSI6eyJuYW1lIjoiYWRqdW5jdGlvbiJ9fV0sWzMsMiwiIiwxLHsibGV2ZWwiOjEsInN0eWxlIjp7Im5hbWUiOiJhZGp1bmN0aW9uIn19XV0=
\[\begin{tikzcd}
	\iC && {\iC^{\iJ}}
	\arrow[""{name=0, anchor=center, inner sep=0}, "\Delta"{description}, from=1-1, to=1-3]
	\arrow[""{name=1, anchor=center, inner sep=0}, "\colim"', curve={height=18pt}, from=1-3, to=1-1]
	\arrow[""{name=2, anchor=center, inner sep=0}, "\lim", curve={height=-18pt}, from=1-3, to=1-1]
	\arrow["\dashv"{anchor=center, rotate=-86}, draw=none, from=0, to=2]
	\arrow["\dashv"{anchor=center, rotate=-94}, draw=none, from=1, to=0]
\end{tikzcd}\]
\end{lem}
\begin{proof}
  By definition, a limit of a diagram $D : \iJ \to \iC$ is an object $\lim D$ in $\iC$ that represents the functor $\iC^{\iJ}(\Delta\blank,D)$.
  That is,
  \[
    \iC(c,\lim D) \iso \iC^{\iJ}(\Delta c,D)
  \]
  natural in $c$.
  If for all $D$, $\lim D$ exists, then these isomorphisms assemble into an adjunction.
  Similarly, a colimit of a diagram $D$ is an object $\colim D$ in $\iC$ that represents the functor $\iC^{\iJ}(D,\Delta\blank)$.
  \[
    \iC(\colim D, c) \iso \iC^{\iJ}(D,\Delta c)
  \]
  natural in $c$.
  Thus, if $\colim D$ exists for all $D$, then this assembles into an adjunction.
\end{proof}

Given that the value of a right adjoint is characterized by a contravariant representable functor, it's expected that it interacts well with limits.

\begin{lem}
  Right adjoints preserve limits.
\end{lem}
\begin{proof}
  Suppose $F \dashv G : \iC \to \iD$ and let $K : \iJ \to \iD$ be a diagram.
  Suppose that $\lambda : \lim K \to K$, we wish to show that $G\lambda : G\lim K \to GK$ defines a limit cone.
  Recall that $F \dashv G$ defines an adjunction $F_{*} \dashv G_{*} : \iC^{\iJ} \to \iD^{\iJ}$ given by post-composition.
  Thus,
  \[
    \iC^{\iJ}(\Delta c, GK) \iso \iD^{\iJ}(F\Delta c,K) \iso \iD^{\iJ}(\Delta F c, K)
  \]
  That is, cones over $GK$ with submit $c$ correspond uniquely to natural transformations $\Delta Fc \to K$.
  Thus,
  \[
    \iD^{\iJ}(\Delta Fc, K) \iso \iD(Fc,\lim K) \iso \iC(c,G\lim K)
  \]
  Thus,
  \[
    \iC^{\iJ}(\Delta c, GK) \iso \iC(c,G\lim K) \iso \iC(c, \lim GK)
  \]
  The limit cone is then given by evaluating the identity $\id_{G\lim K} \in \iC(G\lim K, G\lim K)$ across the isomorphism, yielding $G\lambda : G\lim K \to GK$.
\end{proof}

Dually,

\begin{lem}
  Left adjoints preserve colimits.
\end{lem}

\begin{cor}
  Recall that for any function $f : A \to B$, one has
  % https://q.uiver.app/#q=WzAsMyxbMCwwLCJGVUZBIl0sWzIsMCwiRkEiXSxbNCwwLCJMKEEsXFxhbHBoYSkiXSxbMCwxLCJGXFxhbHBoYSIsMCx7Im9mZnNldCI6LTF9XSxbMCwxLCJcXGVwc2lsb25fe0ZBfSIsMix7Im9mZnNldCI6MX1dLFsxLDJdXQ==
\[\begin{tikzcd}
	FUFA && FA && {L(A,\alpha)}
	\arrow["F\alpha", shift left, from=1-1, to=1-3]
	\arrow["{\epsilon_{FA}}"', shift right, from=1-1, to=1-3]
	\arrow[from=1-3, to=1-5]
\end{tikzcd}\]
  Thus, $f_{*}$ preserves unions, $f\inv$ preserves both intersections and unions, and $f_{!}$ preserves intersections.
\end{cor}

In a Cartesian closed category where $A \times \blank \dashv (\blank)^{A}$, one immediately has
\begin{mathpar}
  A \times (B + C) \iso A \times B + A \times C \and (B \times C)^{A} \iso B^{A} \times C^{A} \and C^{A + B} \iso C^{A} \times C^{B}
\end{mathpar}

\begin{cor}
  The free group on the set $X + Y$ is the free product of the free groups on $X$ and $Y$.
\end{cor}

Note that the free product of two groups is actually a coproduct.

\begin{defn}
  A functor is \emph{right exact} if it preserves finite colimits and \emph{left exact} if it preserves finite limits.
\end{defn}

\begin{defn}
  A \emph{reflective subcategory} of a category $\iC$ is a full subcategory $\iD$ so that the inclusion functor admits a left adjoint, called the \emph{reflector} or \emph{localization}.
  % https://q.uiver.app/#q=WzAsMixbMCwwLCJcXGlEIl0sWzIsMCwiXFxpQyJdLFswLDEsIiIsMSx7ImN1cnZlIjoyLCJzdHlsZSI6eyJ0YWlsIjp7Im5hbWUiOiJob29rIiwic2lkZSI6InRvcCJ9fX1dLFsxLDAsIkwiLDIseyJjdXJ2ZSI6Mn1dLFszLDIsIiIsMSx7ImxldmVsIjoxLCJzdHlsZSI6eyJuYW1lIjoiYWRqdW5jdGlvbiJ9fV1d
\[\begin{tikzcd}
	\iD && \iC
	\arrow[""{name=0, anchor=center, inner sep=0}, curve={height=12pt}, hook, from=1-1, to=1-3]
	\arrow[""{name=1, anchor=center, inner sep=0}, "L"', curve={height=12pt}, from=1-3, to=1-1]
	\arrow["\dashv"{anchor=center, rotate=-90}, draw=none, from=1, to=0]
\end{tikzcd}\]
\end{defn}

\begin{lem}
  Given an adjunction,
  % https://q.uiver.app/#q=WzAsMixbMCwwLCJcXGlDIl0sWzIsMCwiXFxpRCJdLFswLDEsIkYiLDAseyJvZmZzZXQiOi0yfV0sWzEsMCwiRyIsMCx7Im9mZnNldCI6LTJ9XSxbMiwzLCIiLDAseyJsZXZlbCI6MSwic3R5bGUiOnsibmFtZSI6ImFkanVuY3Rpb24ifX1dXQ==
\[\begin{tikzcd}
	\iC && \iD
	\arrow[""{name=0, anchor=center, inner sep=0}, "F", shift left=2, from=1-1, to=1-3]
	\arrow[""{name=1, anchor=center, inner sep=0}, "G", shift left=2, from=1-3, to=1-1]
	\arrow["\dashv"{anchor=center, rotate=-90}, draw=none, from=0, to=1]
\end{tikzcd}\]
  with counit $\epsilon : FG \to \id_{\iD}$.
  Then:
  \begin{enumerate}
  \item $G$ is faithful if and only if each component of $\epsilon$ is an epimorphism.
  \item $G$ is full if and only if each component of $\epsilon$ is a split monomorphism.
  \item $G$ is fully faithful if and only if each component of $\epsilon$ is an isomorphism.
  \end{enumerate}
  Dually,
  \begin{enumerate}
  \item $F$ is faithful if and only if each component of $\eta$ is an monomorphism.
  \item $F$ is full if and only if each component of $\eta$ is a split epimorphism.
  \item $F$ is fully faithful if and only if each component $\eta$ is an isomorphism.
  \end{enumerate}
\end{lem}
\begin{proof}
  By definition, $G$ is faithful if and only if $G$ defines an injection $\iD(x,y) \to \iC(Gx,Gy)$ for any $x$ and $y$.
  Composition with the isomorphism $\iC(Gx,Gy) \iso \iD(FGx,y)$ yields the injection $(\epsilon_{x})^{*}$, which is equivalent to saying that $\epsilon_{x}$ is an epimorphism.
  % https://q.uiver.app/#q=WzAsMyxbMCwwLCJcXGlEKHgseSkiXSxbMiwwLCJcXGlDKEd4LEd5KSJdLFsyLDIsIlxcaUQoRkd4LHkpIl0sWzAsMSwiRyJdLFsxLDIsIlxcaXNvIiwxXSxbMCwyLCIoXFxlcHNpbG9uX3t4fSleeyp9IiwyXV0=
\[\begin{tikzcd}
	{\iD(x,y)} && {\iC(Gx,Gy)} \\
	\\
	&& {\iD(FGx,y)}
	\arrow["G", from=1-1, to=1-3]
	\arrow["\iso"{description}, from=1-3, to=3-3]
	\arrow["{(\epsilon_{x})^{*}}"', from=1-1, to=3-3]
\end{tikzcd}\]
  Clearly, if $(\epsilon_{x})^{*}$ is an injection, then $G$ must be an injection by composing $(\epsilon_{x})^{*}$ with the inverse of the isomorphism.

  Similarly, $G$ is full if and only if $(\epsilon_{x})^{*}$ is a surjection, which is equivalent to saying that $\epsilon_{x}$ is a split monomorphism.
  Thus, $G$ is fully faithful if and only if every component of $\epsilon$ is an epi split monomorphism.
  In other words, every component of $\epsilon$ is an isomorphism.

  Dually, $F$ is faithful if and only if $(\eta_{y})_{*}$ is an injection if and only if $\eta_{y}$ is a monomorphism; $F$ is full if and only if $(\eta_{y})_{*}$ is a surjection if and only if $\eta_{y}$ is a split epimorphism.
  % https://q.uiver.app/#q=WzAsMyxbMCwwLCJcXGlDKHgseSkiXSxbMiwwLCJcXGlEKEZ4LEZ5KSJdLFsyLDIsIlxcaUMoeCxHRnkpIl0sWzEsMiwiXFxpc28iLDFdLFswLDEsIkYiXSxbMCwyLCIoXFxldGFfe3l9KV97Kn0iLDJdXQ==
\[\begin{tikzcd}
	{\iC(x,y)} && {\iD(Fx,Fy)} \\
	\\
	&& {\iC(x,GFy)}
	\arrow["\iso"{description}, from=1-3, to=3-3]
	\arrow["F", from=1-1, to=1-3]
	\arrow["{(\eta_{y})_{*}}"', from=1-1, to=3-3]
\end{tikzcd}\]
\end{proof}

The following result explains why reflective subcategories are interesting:
\begin{lem}
  If $\iD \into \iC$ is a reflective subcategory, then:
  \begin{itemize}
  \item The inclusion $\iD \into \iC$ creates all limits that $\iC$ admits.
  \item $\iD$ has all colimits that $\iC$ admits, formed by applying the reflector to the colimit in $\iC$.
  \end{itemize}
\end{lem}
The proof of the first statement is deferred to a later section.
\begin{proof}
  Write $\iota : \iD \into \iC$ for the inclusion functor right adjoint to the localization $L$.
  Consider a diagram $F : \iJ \to \iD$.
  Let $\lambda : \iota F \to c$ be a limit cone in $\iC$.
  Our goal is to find a limit cone $\lambda' : F \to Lc$ in $\iD$.
  Left adjoints preserve colimits, yielding a limit cone $L\lambda : L\iota F \to Lc$ in $\iD$.
  Since the inclusion functor $\iota$ is fully faithful, the counit $\epsilon$ is an isomorphism.
  Composing with the isomorphism $\epsilon F : L\iota F \iso F$ yields the desired limit cone $\lambda' : F \iso L\iota F \to Lc$.
\end{proof}

\begin{lem}
  Consider the reflective subcategory inclusion $\iD \into \iC$ with reflector $L : \iC \to \iD$.
  \begin{itemize}
  \item $\eta L = L\eta$, and these natural transformations are isomorphisms.
  \item An object $c \in \iC$ is in the \emph{essential image} of the inclusion $\iD \into \iC$, meaning that it is isomorphic to an object in the subcategory $\iD$, if and only if $\eta_{c}$ is an isomorphism.
  \end{itemize}
\end{lem}
\begin{proof}
  The triangle identities imply that the diagram
  % https://q.uiver.app/#q=WzAsNCxbMCwwLCJMIl0sWzIsMCwiTEwiXSxbMiwyLCJMIl0sWzAsMiwiTEwiXSxbMCwxLCJcXGV0YSBMIl0sWzEsMiwiXFxlcHNpbG9uIEwiXSxbMCwyLCJcXGlkX3tMfSIsMV0sWzAsMywiTFxcZXRhIiwyXSxbMywyLCJcXGVwc2lsb24gTCIsMl1d
\[\begin{tikzcd}
	L && LL \\
	\\
	LL && L
	\arrow["{\eta L}", from=1-1, to=1-3]
	\arrow["{\epsilon L}", from=1-3, to=3-3]
	\arrow["{\id_{L}}"{description}, from=1-1, to=3-3]
	\arrow["L\eta"', from=1-1, to=3-1]
	\arrow["{\epsilon L}"', from=3-1, to=3-3]
\end{tikzcd}\]
  commutes.
  In particular, since the inclusion is fully faithful, $\epsilon$ is an isomorphism.
  Thus, $\eta L = L\eta$ and they are isomorphisms: they are the inverse of $\epsilon L$.

  Clearly, if $\eta_{c} : c \to Lc$ is an isomorphism, then $c$ is in the essential image of the inclusion.
  Conversely, if $c$ is in the essential image of the inclusion, i.e., $c \iso d$ for some $d$.
  Then $c \iso L d$.
  By naturality,
  % https://q.uiver.app/#q=WzAsNCxbMCwwLCJjIl0sWzIsMCwiTGMiXSxbMCwyLCJMZCJdLFsyLDIsIkxMZCJdLFswLDEsIlxcZXRhX3tjfSJdLFsyLDMsIlxcZXRhX3tMZH0iLDJdLFswLDIsIlxcaXNvIiwyXSxbMSwzLCJcXGlzbyJdXQ==
\[\begin{tikzcd}
	c && Lc \\
	\\
	Ld && LLd
	\arrow["{\eta_{c}}", from=1-1, to=1-3]
	\arrow["{\eta_{Ld}}"', from=3-1, to=3-3]
	\arrow["\iso"', from=1-1, to=3-1]
	\arrow["\iso", from=1-3, to=3-3]
\end{tikzcd}\]
  $\eta_{c}$ is an isomorphism.
\end{proof}

\bibliographystyle{alpha}
\bibliography{all}

\end{document}
