\documentclass{amsart}
\input{decls}
\title{Monads from Adjunctions}
\author{Frank Tsai}
\date{\today}
%\thanks{}
\begin{document}
\maketitle
\tableofcontents

\section{Introduction}
\label{sec:introduction}
\begin{defn}\label{defn:monad}
  A \emph{monad} on a category $\iC$ consists of
  \begin{enumerate}
  \item an endofunctor $T : \iC \to \iC$;
  \item a unit natural transformation $\eta : \id_{\iC} \to T$;
  \item a multiplication natural transformation $\mu : T^{2} \to T$.
  \end{enumerate}
  such that the follow diagrams commute in the functor category $\iC^{\iC}$
  \begin{mathpar}
    % https://q.uiver.app/#q=WzAsMyxbMCwxLCJSIl0sWzIsMSwiUyJdLFsxLDAsIkMiXSxbMCwyLCJmIiwwLHsic3R5bGUiOnsidGFpbCI6eyJuYW1lIjoibW9ubyJ9fX1dLFsxLDIsImciLDIseyJzdHlsZSI6eyJ0YWlsIjp7Im5hbWUiOiJtb25vIn19fV0sWzAsMSwiXFx0YXUiLDJdXQ==
\begin{tikzcd}
	& C \\
	R && S
	\arrow["f", tail, from=2-1, to=1-2]
	\arrow["g"', tail, from=2-3, to=1-2]
	\arrow["\tau"', from=2-1, to=2-3]
\end{tikzcd}\and

    % https://q.uiver.app/#q=WzAsNCxbMCwwLCJUXnszfSJdLFsyLDAsIlReezJ9Il0sWzAsMiwiVF57Mn0iXSxbMiwyLCJUIl0sWzAsMSwiVFxcbXUiXSxbMCwyLCJcXG11X3tUfSIsMl0sWzIsMywiXFxtdSIsMl0sWzEsMywiXFxtdSJdXQ==
\begin{tikzcd}
	{T^{3}} && {T^{2}} \\
	\\
	{T^{2}} && T
	\arrow["T\mu", from=1-1, to=1-3]
	\arrow["{\mu_{T}}"', from=1-1, to=3-1]
	\arrow["\mu"', from=3-1, to=3-3]
	\arrow["\mu", from=1-3, to=3-3]
\end{tikzcd}
  \end{mathpar}
\end{defn}

\begin{lem}
  Given any adjunction $F \adj U : \iC \toot \iD$ with unit $\eta$ and counit $\epsilon$.
  This triple of data defines a monad on $\iC$, with
  \begin{enumerate}
  \item the endofunctor is defined to be $UF$;
  \item the unit is just $\eta$;
  \item the whiskered counit $U\epsilon F$ is the multiplication $\mu$.
  \end{enumerate}
\end{lem}
\begin{proof}
  The triangles
  % https://q.uiver.app/#q=WzAsNCxbMCwwLCJVRiJdLFsyLDAsIlVGVUYiXSxbNCwwLCJVRiJdLFsyLDIsIlVGIl0sWzAsMywiIiwyLHsibGV2ZWwiOjIsInN0eWxlIjp7ImhlYWQiOnsibmFtZSI6Im5vbmUifX19XSxbMiwzLCIiLDAseyJsZXZlbCI6Miwic3R5bGUiOnsiaGVhZCI6eyJuYW1lIjoibm9uZSJ9fX1dLFswLDEsIlVGXFxldGEiXSxbMiwxLCJcXGV0YV97VUZ9IiwyXSxbMSwzLCJVXFxlcHNpbG9uIEYiLDFdXQ==
\[\begin{tikzcd}
	UF && UFUF && UF \\
	\\
	&& UF
	\arrow[Rightarrow, no head, from=1-1, to=3-3]
	\arrow[Rightarrow, no head, from=1-5, to=3-3]
	\arrow["UF\eta", from=1-1, to=1-3]
	\arrow["{\eta_{UF}}"', from=1-5, to=1-3]
	\arrow["{U\epsilon F}"{description}, from=1-3, to=3-3]
\end{tikzcd}\]
  commute because of the triangle identities of the adjunction.
  
  The square
  % https://q.uiver.app/#q=WzAsMyxbMCwwLCJBIl0sWzIsMCwiQiJdLFswLDIsIkMiXSxbMCwxLCJmIiwwLHsic3R5bGUiOnsiaGVhZCI6eyJuYW1lIjoiZXBpIn19fV0sWzIsMSwiaSIsMix7InN0eWxlIjp7InRhaWwiOnsibmFtZSI6Im1vbm8ifX19XSxbMCwyXV0=
\begin{tikzcd}
	A && B \\
	\\
	C
	\arrow["f", two heads, from=1-1, to=1-3]
	\arrow["i"', tail, from=3-1, to=1-3]
	\arrow[from=1-1, to=3-1]
\end{tikzcd}
  commutes by naturality of $U\epsilon : UFU \to U$.
\end{proof}

\begin{eg}
  The free and forgetful adjunction between pointed sets and sets induces the \emph{maybe monad}.
  The unit $\eta_{S} : S \to S_{+}$ is an inclusion function.
  The component $\mu_{S} : (S_{+})_{+} \to S_{+}$ is $U\epsilon FS$, the underlying function of $\epsilon FS$.
  This function is identity on the elements of $S$ and maps the two distinguished points in $(S_{+})_{+}$ to the single distinguished point in $S_{+}$.
\end{eg}

\begin{eg}
  The free and forgetful adjunction between monoids and sets induces the \emph{free monoid monad}.
  The endofunctor
  \[
    TS := \coprod_{n \in \dN}S^{n}
  \]
  maps a set $S$ to the set of all finite lists of $S$.
  The unit $\eta_{S} : S \to TS$ is the evident injection $\iota : S \to \coprod_{n \in \dN}S^{n}$ that maps $s \in S$ to the singleton list $[s] \in \coprod_{n \in \dN}S^{n}$.
  The multiplication map $\mu_{S} : T^{2}S \to TS$ is the concatenation function, sending $[[s_{1},\ldots,s_{n}],\ldots,[t_{1},\ldots,t_{m}]]$ to a single list $[s_{1},\ldots,s_{n},\ldots,t_{1},\ldots,t_{m}]$.
\end{eg}

\begin{eg}
  The contravariant powerset functor $P \adj P : \mathsf{Set} \toot \mathsf{Set}\op$ is mutually right adjoint to itself.
  The induced monad is called the \emph{double powerset monad} or the \emph{continuation monad}.
  The components of the unit are the \emph{principle ultrafilter} functions $\eta_{S} : S \to P^{2}S$, sending $s \in S$ to the set of subsets of $S$ that contains $s$.
  The components of the multiplication are the inverse image function for the map $\eta_{PA} : PA \to P^{3}A$.
\end{eg}

\begin{defn}
  A \emph{comonad} on $\iC$ is a monad on $\iC\op$.
  Explicitly, a comonad consists of an endofunctor $K : \iC \to \iC$, a counit $\epsilon : K \to \id_{\iC}$, and a comultiplication $\delta : K \to K^{2}$ such that the diagrams dual to \cref{defn:monad} commutes in $\iC^{\iC}$.
\end{defn}

\begin{eg}
  A monad $T$ on a preorder $(P, \leq)$ is an order-preserving function $T$ so that $p \leq Tp$ and $T^{2}p \leq Tp$.
  If $(P, \leq)$ is a poset, then these two conditions imply that $Tp = T^{2}p$.
  An order-preserving function $T : P \to P$ with the property that $p \leq Tp$ and $T^{2}p = Tp$ is call a \emph{closure operator}.
  Dually, a comonad on a poset $(P, \leq)$ defines a \emph{kernel operator}: an order-preserving function $K$ so that $Kp \leq p$ and $K^{2}p = Kp$.

  The poset of topological space $X$ ordered by inclusion admits a closure operator: the closure of a set $A \subseteq X$, and a kernel operator: the interior of a set $A \subseteq X$.
\end{eg}

\begin{lem}
  The adjunction associated to a reflective subcategory $\iC$ induces an \emph{idempotent monad} on $\iC$.
  The following three characterization of idempotent monads are equivalent:
  \begin{enumerate}
  \item the multiplication $\mu$ is a natural isomorphism;
  \item the natural transformations $\eta T$, $T\eta$ are equal;
  \item each component of $\mu$ is a monomorphism.
  \end{enumerate}
\end{lem}
\begin{proof}
  Let $L$ be the reflector of $\iC$.
  Then, $\eta L = L\eta$ and these are isomorphisms.

  (i) $\imp$ (ii): 

  (ii) $\imp$ (iii):

  (iii) $\imp$ (i):
\end{proof}

\bibliographystyle{alpha}
\bibliography{all}

\end{document}
