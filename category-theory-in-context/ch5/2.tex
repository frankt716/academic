\documentclass{amsart}
\input{decls}
\title{Adjunctions from Monads}
\author{Frank Tsai}
\date{\today}
%\thanks{}
\begin{document}
\maketitle
\tableofcontents

\section{Introduction}
\label{sec:introduction}
This section makes the intuition that a monad on $\iC$ is a syntactic encoding of some algebraic structure on $\iC$.

\begin{defn}
  An \emph{affine space} is a non-empty set $A$ in which affine linear combinations can be evaluated.
\end{defn}
Let $\mathrm{Aff}_{k}(A)$ be the set of finite formal linear combinations
\[
  \sum_{i=1}^{n}\lambda_{i}\mathbf{a}_{i}
\]
where $\sum_{i=1}^{n}\lambda_{i} = 1$.

To say that ``these can be evaluated'' means that there is an evaluation function
\[
  \mathrm{ev}_{A} : \mathrm{Aff}_{k}(A) \to A
\]
A reasonable evaluation function has to satisfy the following axioms
\begin{mathpar}
  % https://q.uiver.app/#q=WzAsMyxbMCwwLCJcXGlDIl0sWzEsMSwiXFxpRCJdLFsyLDAsIlxcaUUiXSxbMCwxLCJLIiwyXSxbMCwyLCJGIl0sWzEsMiwiXFxsYW5fe0t9RiIsMix7InN0eWxlIjp7ImJvZHkiOnsibmFtZSI6ImRhc2hlZCJ9fX1dLFs0LDEsIiIsMCx7InNob3J0ZW4iOnsic291cmNlIjoyMH19XV0=
\begin{tikzcd}
	\iC && \iE \\
	& \iD
	\arrow["K"', from=1-1, to=2-2]
	\arrow[""{name=0, anchor=center, inner sep=0}, "F", from=1-1, to=1-3]
	\arrow["{\lan_{K}F}"', dashed, from=2-2, to=1-3]
	\arrow[shorten <=3pt, Rightarrow, from=0, to=2-2]
\end{tikzcd}\and

  % https://q.uiver.app/#q=WzAsNCxbMCwwLCJcXG1hdGhybXtBZmZ9X3trfShcXG1hdGhybXtBZmZ9X3trfShBKSkiXSxbMiwwLCJcXG1hdGhybXtBZmZ9X3trfShBKSJdLFswLDIsIlxcbWF0aHJte0FmZn1fe2t9KEEpIl0sWzIsMiwiQSJdLFsxLDMsIlxcbWF0aHJte2V2fV97QX0iXSxbMiwzLCJcXG1hdGhybXtldn1fe0F9IiwyXSxbMCwyLCJcXG1hdGhybXtBZmZ9X3trfShcXG1hdGhybXtldn1fe0F9KSIsMl0sWzAsMSwiXFxtdV97QX0iXV0=
\begin{tikzcd}
	{\mathrm{Aff}_{k}(\mathrm{Aff}_{k}(A))} && {\mathrm{Aff}_{k}(A)} \\
	\\
	{\mathrm{Aff}_{k}(A)} && A
	\arrow["{\mathrm{ev}_{A}}", from=1-3, to=3-3]
	\arrow["{\mathrm{ev}_{A}}"', from=3-1, to=3-3]
	\arrow["{\mathrm{Aff}_{k}(\mathrm{ev}_{A})}"', from=1-1, to=3-1]
	\arrow["{\mu_{A}}", from=1-1, to=1-3]
\end{tikzcd}
\end{mathpar}

The first condition says that the singleton sum $1 \cdot \mathbf{a}$ evaluates to $\mathbf{a}$.
There are two reasonable ways to evaluate a nested expression:
\[
  \lambda_{1} \cdot (\mu_{11}\mathbf{a}_{11} + \cdots + \mu_{1n}\mathbf{a}_{1n}) + \cdots + \lambda_{m} \cdot (\mu_{m1}\mathbf{a}_{m1} + \cdots + \mu_{mk}\mathbf{a}_{mk})
\]
First, we can distribute $\lambda_{i}$'s into those sub-expressions
\[
  (\lambda_{1}\mu_{11}\mathbf{a}_{11} + \cdots + \lambda_{1}\mu_{1n}\mathbf{a}_{1n}) + \cdots + (\lambda_{m}\mu_{m1}\mathbf{a}_{m1} + \cdots + \lambda_{m}\mu_{mk}\mathbf{a}_{mk})
\]
And then evaluate this expression.

Alternatively, we can evaluate each individual expression in the parentheses.
\[
  \lambda_{1} \cdot \mathbf{a}_{1} + \cdots + \lambda_{m} \cdot \mathbf{a}_{m}
\]
And then evaluate this expression.
The second condition says that these two evaluation orders coincide.

\begin{defn}
  An \emph{affine space} is an \emph{algebra} for the affine linear combination monad $\mathrm{Aff}_{k}(\blank) : \mathsf{Set} \to \mathsf{Set}$.
\end{defn}

\begin{defn}
  Let $\iC$ be a category with a monad $(T, \eta, \mu)$.
  The \emph{Eilenberg-Moore category} for $T$ or the \emph{category of $T$-algebras} is the category $C^{T}$ whose:
  \begin{enumerate}
  \item objects are pairs $(c \in \iC, \mathrm{ev}_{c} : Tc \to c)$ so that the diagrams
    \begin{mathpar}
      % https://q.uiver.app/#q=WzAsMyxbMSwwLCJGYyJdLFswLDIsIlxcbGFuX3tLfUYoZCkiXSxbMiwyLCJcXGxhbl97S31GKGQnKSJdLFswLDEsIlxcbGFtYmRhXntkfV97Zn0iLDJdLFswLDIsIlxcbGFtYmRhXntkJ31fe2dmfSJdLFsxLDIsIiIsMix7InN0eWxlIjp7ImJvZHkiOnsibmFtZSI6ImRhc2hlZCJ9fX1dXQ==
\begin{tikzcd}
	& Fc \\
	\\
	{\lan_{K}F(d)} && {\lan_{K}F(d')}
	\arrow["{\lambda^{d}_{f}}"', from=1-2, to=3-1]
	\arrow["{\lambda^{d'}_{gf}}", from=1-2, to=3-3]
	\arrow[dashed, from=3-1, to=3-3]
\end{tikzcd}\and

      % https://q.uiver.app/#q=WzAsNCxbMCwwLCJGYyJdLFsyLDAsIlxcbGFuX3tLfUYoS2MpIl0sWzAsMiwiRmMnIl0sWzIsMiwiXFxsYW5fe0t9RihLYycpIl0sWzAsMiwiRmYiLDJdLFsxLDMsIlxcbGFuX3tLfUYoS2YpIl0sWzAsMSwiXFxldGFfe2N9Il0sWzIsMywiXFxldGFfe2MnfSIsMl0sWzAsMywiXFxsYW1iZGFee0tjJ31fe0tmfSIsMV1d
\begin{tikzcd}
	Fc && {\lan_{K}F(Kc)} \\
	\\
	{Fc'} && {\lan_{K}F(Kc')}
	\arrow["Ff"', from=1-1, to=3-1]
	\arrow["{\lan_{K}F(Kf)}", from=1-3, to=3-3]
	\arrow["{\eta_{c}}", from=1-1, to=1-3]
	\arrow["{\eta_{c'}}"', from=3-1, to=3-3]
	\arrow["{\lambda^{Kc'}_{Kf}}"{description}, from=1-1, to=3-3]
\end{tikzcd}
    \end{mathpar}
    commute in $\iC$.
  \item morphisms are $T$-algebra \emph{homomorphisms}: morphisms $f : c \to c'$ in $\iC$ so that the diagram
    % https://q.uiver.app/#q=WzAsNSxbMCwwLCJGXFxQaV57ZH0oYyxmKSJdLFsyLDAsIkZcXFBpXntkfShjJyxmJykiXSxbMCwyLCJHS2MiXSxbMiwyLCJHS2MnIl0sWzEsMywiR2QiXSxbMCwyLCJcXGdhbW1hX3tjfSIsMl0sWzEsMywiXFxnYW1tYV97Yyd9Il0sWzIsNCwiR2YiLDJdLFszLDQsIkdmJyJdLFswLDEsIkZcXFBpXntkfWciXSxbMiwzLCJHS2ciXV0=
\begin{tikzcd}
	{F\Pi^{d}(c,f)} && {F\Pi^{d}(c',f')} \\
	\\
	GKc && {GKc'} \\
	& Gd
	\arrow["{\gamma_{c}}"', from=1-1, to=3-1]
	\arrow["{\gamma_{c'}}", from=1-3, to=3-3]
	\arrow["Gf"', from=3-1, to=4-2]
	\arrow["{Gf'}", from=3-3, to=4-2]
	\arrow["{F\Pi^{d}g}", from=1-1, to=1-3]
	\arrow["GKg", from=3-1, to=3-3]
\end{tikzcd}
    commutes in $\iC$.
  \end{enumerate}
\end{defn}

An algebra consists of a generic object and an interpretation (evaluation) map.
And homomorphisms are maps that are compatible with the algebraic structure.

\begin{eg}
  An algebra for the free pointed set monad consists of a set $A$ and a function $a : A_{+} \to A$ such that the two diagrams commute.
  The triangle forces $a$ to restrict to the identity function on the $A$ component.
  A morphism $f : (A, a) \to (B, b)$ is a function $f : A \to B$ so that
  % https://q.uiver.app/#q=WzAsMyxbMCwwLCIxIl0sWzIsMCwiXFxtYXRoc2Z7U2V0fSJdLFswLDIsIkUgXFx0b3RvIFYiXSxbMCwyLCJcXG1hdGhybXtyZXN9IiwyXSxbMCwxLCJHIl0sWzIsMSwiIiwyLHsic3R5bGUiOnsiYm9keSI6eyJuYW1lIjoiZGFzaGVkIn19fV1d
\begin{tikzcd}
	1 && {\mathsf{Set}} \\
	\\
	{E \toto V}
	\arrow["{\mathrm{res}}"', from=1-1, to=3-1]
	\arrow["G", from=1-1, to=1-3]
	\arrow[dashed, from=3-1, to=1-3]
\end{tikzcd}
  commutes.
  This forces $f(a) = b \circ f_{+}(a) = b$.
  In other words, the morphisms are base-point preserving functions.
  Thus, the category of algebras is isomorphic to the category $\mathsf{Set}_{*}$.
\end{eg}

\begin{lem}\label{lem:monad-to-adjunction-eilenberg}
  For any monad $(T, \eta, \mu)$ on $\iC$, there is an adjunction
  % https://q.uiver.app/#q=WzAsMixbMiwwLCJcXGlDXntUfSJdLFswLDAsIlxcaUMiXSxbMSwwLCJGXntUfSIsMCx7Im9mZnNldCI6LTJ9XSxbMCwxLCJVXntUfSIsMCx7Im9mZnNldCI6LTJ9XSxbMiwzLCIiLDIseyJsZXZlbCI6MSwic3R5bGUiOnsibmFtZSI6ImFkanVuY3Rpb24ifX1dXQ==
\[\begin{tikzcd}
	\iC && {\iC^{T}}
	\arrow[""{name=0, anchor=center, inner sep=0}, "{F^{T}}", shift left=2, from=1-1, to=1-3]
	\arrow[""{name=1, anchor=center, inner sep=0}, "{U^{T}}", shift left=2, from=1-3, to=1-1]
	\arrow["\dashv"{anchor=center, rotate=-90}, draw=none, from=0, to=1]
\end{tikzcd}\]
  whose induced monad is $(T, \eta, \mu)$.
\end{lem}
\begin{proof}
  The functor $U^{T}$ is the evident forgetful functor that forgets the evaluation map and maps a $T$-homomorphism to its underlying morphism.

  The free functor $F^{T}$ maps an object $c$ to the \emph{free $T$-algebra}
  \[
    F^{T}(c) := (Tc \in \iC, \mu_{c} : T^{2}c \to Tc)
  \]
  And a morphism $f : c \to c'$ to the \emph{free $T$-homomorphism}
  \[
    F^{T}(f) := Tf
  \]
  Functoriality of $F^{T}$ is inherited from that of $T$.
  Note that by construction, $U^{T}F^{T} = T$.

  The unit of the adjunction is just $\eta$, while the components of the counit $\epsilon_{(c,a)} : (Tc, \mu_{c}) \to (c,a)$ are $T$-homomorphisms.
  The requirement that
  % https://q.uiver.app/#q=WzAsNCxbMCwyLCJUYyJdLFsyLDIsImMiXSxbMiwwLCJUYyJdLFswLDAsIlReezJ9YyJdLFswLDEsImEiLDJdLFsyLDEsImEiXSxbMywwLCJcXG11X3thfSIsMl0sWzMsMiwiVGEiXV0=
\[\begin{tikzcd}
	{T^{2}c} && Tc \\
	\\
	Tc && c
	\arrow["a"', from=3-1, to=3-3]
	\arrow["a", from=1-3, to=3-3]
	\arrow["{\mu_{a}}"', from=1-1, to=3-1]
	\arrow["Ta", from=1-1, to=1-3]
\end{tikzcd}\]
  commutes means that $a$ is a $T$-homomorphism.
  And we take $\epsilon_{(c,a)} := a$.
  Note that $U^{T}\epsilon F^{T}c = U^{T}\epsilon_{(c,\mu_{c})} = U^{T}\mu_{c} = \mu_{c}$.
  It remains to check the triangle identities.

  For any algebra $(c, a)$, one has
  \begin{align}
    U^{T}\epsilon_{(c,a)} \circ \eta_{U^{T}(c,a)}  &= U^{T}a \circ \eta_{c}\\
                                     &= a \circ \eta_{c}\\
                                     &= \id_{c}
  \end{align}
  The last equation follows from the commuting triangle that a given evaluation map satisfies.
  This proves the first triangle identity
  % https://q.uiver.app/#q=WzAsMyxbMCwwLCJVXntUfSJdLFsyLDAsIlVee1R9Rl57VH1VXntUfSJdLFsyLDIsIlVee1R9Il0sWzAsMSwiXFxldGEgVV57VH0iXSxbMSwyLCJVXntUfVxcZXBzaWxvbiJdLFswLDIsIiIsMix7ImxldmVsIjoyLCJzdHlsZSI6eyJoZWFkIjp7Im5hbWUiOiJub25lIn19fV1d
\[\begin{tikzcd}
	{U^{T}} && {U^{T}F^{T}U^{T}} \\
	\\
	&& {U^{T}}
	\arrow["{\eta U^{T}}", from=1-1, to=1-3]
	\arrow["{U^{T}\epsilon}", from=1-3, to=3-3]
	\arrow[Rightarrow, no head, from=1-1, to=3-3]
\end{tikzcd}\]

  Now for any object $c \in \iC$, one has
  \begin{align}
    \epsilon_{F^{T}c} \circ F^{T}\eta_{c} &= \epsilon_{(Tc, \mu_{c})} \circ T\eta_{c}\\
                            &= \mu_{c} \circ T\eta_{c}\\
                            &= \id_{Tc}
  \end{align}
  The last equation follows from the monad laws.
  This proves the second triangle identity
  % https://q.uiver.app/#q=WzAsMyxbMCwwLCJGXntUfSJdLFsyLDAsIkZee1R9VV57VH1GXntUfSJdLFsyLDIsIkZee1R9Il0sWzAsMSwiRl57VH1cXGV0YSJdLFsxLDIsIlxcZXBzaWxvbiBGXntUfSJdLFswLDIsIiIsMix7ImxldmVsIjoyLCJzdHlsZSI6eyJoZWFkIjp7Im5hbWUiOiJub25lIn19fV1d
\[\begin{tikzcd}
	{F^{T}} && {F^{T}U^{T}F^{T}} \\
	\\
	&& {F^{T}}
	\arrow["{F^{T}\eta}", from=1-1, to=1-3]
	\arrow["{\epsilon F^{T}}", from=1-3, to=3-3]
	\arrow[Rightarrow, no head, from=1-1, to=3-3]
\end{tikzcd}\]
\end{proof}

\begin{defn}
  Let $\iC$ be a category with a monad $(T, \eta, \mu)$.
  The \emph{Kleisli category} $\iC_{T}$ is defined so that
  \begin{enumerate}
  \item its objects are the objects of $\iC$, and
  \item a morphism from $c$ to $c'$ in $\iC_{T}$, depicted as $c \rightsquigarrow c'$, is a morphism $c \to Tc'$ in $\iC$.
  \item The unit $\eta_{c} : c \to Tc$ defines the identity of $C_{T}$, and
  \item the composition of $f : c \rightsquigarrow c'$ and $g : c' \rightsquigarrow c''$ is defined to be
    % https://q.uiver.app/#q=WzAsNCxbMCwwLCJjIl0sWzIsMCwiVGMnIl0sWzQsMCwiVF57Mn1jJyciXSxbNiwwLCJUYycnIl0sWzAsMSwiZiJdLFsxLDIsIlRnIl0sWzIsMywiXFxtdV97YycnfSJdXQ==
\[\begin{tikzcd}
	c && {Tc'} && {T^{2}c''} && {Tc''}
	\arrow["f", from=1-1, to=1-3]
	\arrow["Tg", from=1-3, to=1-5]
	\arrow["{\mu_{c''}}", from=1-5, to=1-7]
\end{tikzcd}\]
  \end{enumerate}
\end{defn}
For any $f : c \rightsquigarrow c'$, the composition $\id_{c'} \circ f$ is
% https://q.uiver.app/#q=WzAsNCxbMCwwLCJjIl0sWzIsMCwiVGMnIl0sWzQsMCwiVF57Mn1jJyJdLFs0LDIsIlRjJyJdLFswLDEsImYiXSxbMSwyLCJUXFxldGFfe2MnfSJdLFsyLDMsIlxcbXVfe2MnfSJdLFsxLDMsIiIsMCx7ImxldmVsIjoyLCJzdHlsZSI6eyJoZWFkIjp7Im5hbWUiOiJub25lIn19fV1d
\[\begin{tikzcd}
	c && {Tc'} && {T^{2}c'} \\
	\\
	&&&& {Tc'}
	\arrow["f", from=1-1, to=1-3]
	\arrow["{T\eta_{c'}}", from=1-3, to=1-5]
	\arrow["{\mu_{c'}}", from=1-5, to=3-5]
	\arrow[Rightarrow, no head, from=1-3, to=3-5]
\end{tikzcd}\]
and the composition $f \circ \id_{c}$ is $\mu_{c'} \circ Tf \circ \eta_{c}$.
By naturality in $\eta$, this composition is $\mu_{c'} \circ \eta_{Tc'} \circ f = f$.

For any three composite triple $f : a \rightsquigarrow b$, $g : b \rightsquigarrow c$, and $h : c \rightsquigarrow d$.
The left-associative composition is
\[
  \mu_{d} \circ Th \circ \mu_{c} \circ Tg \circ f
\]
while the right-associative composition is
\begin{align}
  \mu_{d} \circ T\mu_{d} \circ T^{2}h \circ Tg \circ f &= \mu_{d} \circ \mu_{Td} \circ T^{2}h \circ Tg \circ f\\
                                   &= \mu_{d} \circ Th \circ \mu_{c} \circ Tg \circ f
\end{align}
The first equation follows from the monad laws and the second follows from naturality in $\mu$.

\begin{eg}
  Objects in the Kleisli category for the maybe monad are sets.
  A morphism $c \rightsquigarrow d$ is a partial function $c \to d_{+}$.
  Compositions are compositions of partial functions.
  Thus, the Kleisli category is the category $\mathsf{Set}^{\partial}$ of sets and partial functions.
\end{eg}

\begin{lem}\label{lem:monad-to-adjunction-kleisli}
  For any monad $(T, \eta, \mu)$ acting on a category $\iC$, there is an adjunction $F_{T},U_{T}: C \toot C_{T}$ whose induced monad is $(T, \eta, \mu)$.
\end{lem}
\begin{proof}
  The functor $F_{T}$ is identity on object and its action on morphisms is defined by
  % https://q.uiver.app/#q=WzAsMyxbMCwwLCJjIl0sWzIsMCwiVGMiXSxbNCwwLCJUZCJdLFswLDEsIlxcZXRhX3tjfSJdLFsxLDIsIlRmIl1d
\[\begin{tikzcd}
	c && Tc && Td
	\arrow["{\eta_{c}}", from=1-1, to=1-3]
	\arrow["Tf", from=1-3, to=1-5]
\end{tikzcd}\]
  This definition is functorial since
  \begin{enumerate}
  \item $F_{T}(\id_{c}) = T\id_{c} \circ \eta_{c} = \eta_{c}$, which is the identity on $c$ in the Kleisli category.
  \item $F_{T}(f \circ g) = T(f \circ g) \circ \eta_{c} = Tf \circ Tg \circ \eta_{c}$ and
    \begin{align}
      F_{T}(f) \circ F_{T}(g) &= \mu_{e} \circ T(Tf \circ \eta_{d}) \circ Tg \circ \eta_{c}\\
                          &= \mu_{e} \circ T(\eta_{e} \circ f) \circ Tg \circ \eta_{c}\\
                          &= Tf \circ Tg \circ \eta_{c},
    \end{align}
  \end{enumerate}

  Since we want $U_{T}F_{T} = T$, this forces us to define the action on objects of $U_{T}$ by $c \mapsto Tc$, and the action on morphisms by $f \mapsto \mu_{d} \circ Tf$.
  This definition is functorial since
  \begin{enumerate}
  \item $U_{T}(\id_{c}) = \mu_{c} \circ T\eta_{c} = \id_{Tc}$.
  \item $U_{T}(f \circ g) = U_{T}(\mu_{e} \circ Tf \circ g) = \mu_{e} \circ T\mu_{e} \circ T^{2}f \circ Tg$ and
    \begin{align}
      U_{T}f \circ U_{T}g &= \mu_{e} \circ Tf \circ \mu_{d} \circ Tg\\
                      &= \mu_{e} \circ \mu_{Te} \circ T^{2}f \circ Tg\\
                      &= \mu_{e} \circ T\mu_{e} \circ T^{2}f \circ Tg
    \end{align}
  \end{enumerate}
  Note that
  \[
    \iC_{T}(F_{T}c,d) \iso \iC_{T}(c,d) \iso \iC(c,Td) \iso \iC(c,U_{T}d)
  \]
  natural in both variables.
  Thus, $F_{T} \adj U_{T}$.
  In particular, the unit of the adjunction is precisely $\eta$ because the isomorphism $\iC_{T}(F_{T}c,F_{T}c) \iso \iC(c,U_{T}F_{T}c)$ identifies the identity $\id_{F_{T}c} = \id_{c}$ with $\eta_{c}$.
  And the identity $\id_{U_{T}d} = \id_{Td} : Td \to Td$ is identified with $\epsilon$ for which $U_{T}\epsilon F_{T}c = U_{T}\epsilon_{c} = \mu_{c} \circ T(\id_{c}) = \mu_{c}$.
\end{proof}

\cref{lem:monad-to-adjunction-eilenberg,lem:monad-to-adjunction-kleisli} implies that every monad can be split into two adjunctions, one between the Eilenberg-Moore category and the other between the Kleisli category.
In fact, these two splits are universal in certain sense introduced below:

\begin{defn}
  Given a monad $(T,\eta,\mu)$ on $\iC$, the category $\mathsf{Adj}_{T}$ consists of
  \begin{enumerate}
  \item objects: fully specified adjunctions $F,U : \iC \toot \iD$.
  \item morphisms: adjoint morphisms
    % https://q.uiver.app/#q=WzAsMyxbMiwyLCJcXGlDIl0sWzAsMCwiXFxpRCJdLFs0LDAsIlxcaUQnIl0sWzEsMiwiSyJdLFswLDEsIkYiLDAseyJvZmZzZXQiOi0yfV0sWzEsMCwiVSIsMCx7Im9mZnNldCI6LTJ9XSxbMCwyLCJGJyIsMCx7Im9mZnNldCI6LTJ9XSxbMiwwLCJVJyIsMCx7Im9mZnNldCI6LTJ9XSxbNCw1LCIiLDIseyJsZXZlbCI6MSwic3R5bGUiOnsibmFtZSI6ImFkanVuY3Rpb24ifX1dLFs2LDcsIiIsMCx7ImxldmVsIjoxLCJzdHlsZSI6eyJuYW1lIjoiYWRqdW5jdGlvbiJ9fV1d
\[\begin{tikzcd}
	\iD &&&& {\iD'} \\
	\\
	&& \iC
	\arrow["K", from=1-1, to=1-5]
	\arrow[""{name=0, anchor=center, inner sep=0}, "F", shift left=2, from=3-3, to=1-1]
	\arrow[""{name=1, anchor=center, inner sep=0}, "U", shift left=2, from=1-1, to=3-3]
	\arrow[""{name=2, anchor=center, inner sep=0}, "{F'}", shift left=2, from=3-3, to=1-5]
	\arrow[""{name=3, anchor=center, inner sep=0}, "{U'}", shift left=2, from=1-5, to=3-3]
	\arrow["\dashv"{anchor=center, rotate=45}, draw=none, from=0, to=1]
	\arrow["\dashv"{anchor=center, rotate=-45}, draw=none, from=2, to=3]
\end{tikzcd}\]
  \end{enumerate}
\end{defn}

\begin{lem}
  The Kleisli category $C_{T}$ is initial in $\mathsf{Adj}_{T}$ and the Eilenberg-Moore category $C^{T}$ is terminal.
\end{lem}
\begin{proof}
  Let $J : \iC_{T} \to \iD$ and $K : \iD \to \iC^{T}$ be adjoin morphisms.
  Since $JF_{T} = F$, we are forced to define the action on objects of $J$ by $Jc = Fc$.
  $J$ has to carry the transpose of a morphism $c \to U_{T}d$ to its transpose in $\iD$.
  The isomorphism
  \[
    \iC_{T}(c,d) \iso \iC(c,U_{T}d)
  \]
  states that those transposes are precisely those morphisms $c \rightsquigarrow d$.
  Thus, for $f : c \rightsquigarrow d$, we are forced to define $Jf = \epsilon_{Fd} \circ Ff$.
  This shows that there is a unique adjoin morphism $\iC_{T} \to D$.

  To define a functor $K : \iD \to \iC^{T}$ so that $U = U^{T}K$, we need to find an appropriate structure map $\gamma : TUd \to Ud$.
  Note that for any algebra $(c,\gamma)$, $\epsilon^{T}_{(c,\gamma)} = \gamma$ is a $T$-homomorphism $\gamma : F^{T}c \to (c,\gamma)$ whose transpose along $F^{T} \adj U^{T}$ is $\id_{c} : c \to U^{T}(c,\gamma) = c$.
  Suppose that $\gamma : F^{T}Ud \to (Ud, \gamma)$ is the transpose of the identity $\id_{Ud}$.
  Then since $K$ preserves transposes, $\gamma = K\epsilon_{d}$.
  Then since $U\epsilon_{d} = U^{T}K\epsilon_{d} = U^{T}\gamma$, $\gamma = U\epsilon_{d}$.
  This suggests that $Kd = (Ud, U\epsilon_{d})$.
  And we are forced to define $Kf = Uf$, which is indeed a $T$-homomorphism.
\end{proof}

\begin{lem}\label{lem:canonical-map}
  Let $(T, \eta, \mu)$ be a monad on $\iC$, the canonical functor $K : C_{T} \to C^{T}$ is fully faithful and its image consists of the free $T$-algebras.
\end{lem}
\begin{proof}
  The isomorphism transports along $K$ to yield an isomorphism $\iC(c,Td) \iso \iC^{T}((Tc,\mu_{c}),(Td,\mu_{d}))$.
  % https://q.uiver.app/#q=WzAsMyxbMCwwLCJcXGlDX3tUfShjLGQpIl0sWzQsMCwiXFxpQ157VH0oKFRjLFxcbXVfe2N9KSwoVGQsXFxtdV97ZH0pKSJdLFsyLDIsIlxcaUMoYyxUZCkiXSxbMCwxLCJLIl0sWzIsMCwiXFxpc28iLDAseyJvZmZzZXQiOi0yfV0sWzAsMiwiIiwwLHsib2Zmc2V0IjotMn1dLFsxLDIsIiIsMix7Im9mZnNldCI6LTJ9XSxbMiwxLCIiLDIseyJvZmZzZXQiOi0yfV1d
\[\begin{tikzcd}
	{\iC_{T}(c,d)} &&&& {\iC^{T}((Tc,\mu_{c}),(Td,\mu_{d}))} \\
	\\
	&& {\iC(c,Td)}
	\arrow["K", from=1-1, to=1-5]
	\arrow["\iso", shift left=2, from=3-3, to=1-1]
	\arrow[shift left=2, from=1-1, to=3-3]
	\arrow[shift left=2, from=1-5, to=3-3]
	\arrow[shift left=2, from=3-3, to=1-5]
\end{tikzcd}\]
  In particular, $K = F^{T}U_{T}$.
  Thus, $K$ is a bijection.
\end{proof}

\cref{lem:canonical-map} tells us that the Kleisli and Eilenberg-Moore categories are equivalent precisely when every algebra (up to isomorphism) is free.
For example, all vector spaces are free modules.

\bibliographystyle{alpha}
\bibliography{all}

\end{document}
