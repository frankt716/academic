\documentclass{amsart}
\input{decls}
\title{Monadic Functors}
\author{Frank Tsai}
\date{\today}
%\thanks{}
\begin{document}
\maketitle
\tableofcontents

\section{Introduction}
\label{sec:introduction}
\begin{defn}
  Given an adjunction $F,U : \iC \toot \iD$, this adjunction induces a monad $T$ on $\iC$ and there is a canonical comparison functor $K : \iD \to \iC^{T}$.
  \begin{enumerate}
  \item This adjunction is \emph{monadic} if the canonical comparison functor $K$ is an equivalence.
  \item A functor $U : \iD \to \iC$ is \emph{monadic} if it admits a left adjoint that defines a monadic adjunction.
  \end{enumerate}
\end{defn}

\begin{defn}
  A monad whose multiplication natural transformation is an isomorphism is called an \emph{idempotent monad}.
  It gets this name since
  \[
    T^{2}c \iso Tc
  \]
\end{defn}

\begin{lem}
  The inclusion $\iD \into \iC$ of a reflective subcategory is monadic: that is, the functor $K$ is an equivalence.
  % https://q.uiver.app/#q=WzAsMyxbMCwwLCJcXGlEIl0sWzQsMCwiXFxpQ157TH0iXSxbMiwyLCJcXGlDIl0sWzAsMiwiIiwyLHsib2Zmc2V0IjoyLCJzdHlsZSI6eyJ0YWlsIjp7Im5hbWUiOiJob29rIiwic2lkZSI6InRvcCJ9fX1dLFsyLDAsIkwiLDIseyJvZmZzZXQiOjJ9XSxbMiwxLCJGXntMfSIsMCx7Im9mZnNldCI6LTJ9XSxbMSwyLCJVXntMfSIsMCx7Im9mZnNldCI6LTJ9XSxbMCwxLCJLIl0sWzQsMywiIiwwLHsibGV2ZWwiOjEsInN0eWxlIjp7Im5hbWUiOiJhZGp1bmN0aW9uIn19XSxbNSw2LCIiLDIseyJsZXZlbCI6MSwic3R5bGUiOnsibmFtZSI6ImFkanVuY3Rpb24ifX1dXQ==
\[\begin{tikzcd}
	\iD &&&& {\iC^{L}} \\
	\\
	&& \iC
	\arrow[""{name=0, anchor=center, inner sep=0}, shift right=2, hook, from=1-1, to=3-3]
	\arrow[""{name=1, anchor=center, inner sep=0}, "L"', shift right=2, from=3-3, to=1-1]
	\arrow[""{name=2, anchor=center, inner sep=0}, "{F^{L}}", shift left=2, from=3-3, to=1-5]
	\arrow[""{name=3, anchor=center, inner sep=0}, "{U^{L}}", shift left=2, from=1-5, to=3-3]
	\arrow["K", from=1-1, to=1-5]
	\arrow["\dashv"{anchor=center, rotate=-135}, draw=none, from=1, to=0]
	\arrow["\dashv"{anchor=center, rotate=-45}, draw=none, from=2, to=3]
\end{tikzcd}\]
\end{lem}
\begin{proof}
  An $L$-algebra consists of an object $c \in \iC$ and a structure map $\alpha : Lc \to c$ that is a retraction of $\eta_{c}$.
  By naturality in $\eta$, $\eta_{c} \circ \alpha = L\alpha \circ \eta_{Lc}$.
  Since $L\eta = \eta L$,
  \[
    L\alpha \circ \eta_{Lc} = L\alpha \circ L\eta_{c} = \id_{Lc}
  \]
  Thus, $\alpha$ and $\eta_{c}$ are mutual inverses.
  In other words, if $c$ can be equipped with an $L$-algebra structure, then $\eta_{c}$ must be an isomorphism.
  In fact, this condition also suffices.
  If $\eta_{c}$ is an isomorphism, then a natural candidate is $\eta_{c}\inv : Lc \to c$.
  The triangle identity is immediate and the rectangle identity is just the naturality of $\eta\inv$.
  Thus, $c \in \iC$ admits an $L$-algebra structure if and only if $\eta_{c}$ is an isomorphism.
  In conclusion, any $L$-algebra $(c,\alpha)$ is isomorphic to $(c,\eta_{c}\inv)$, which corresponds to a condition on $c \in \iC$.
  
  An object $c \in \iC$ is in the essential image of $\iD \into \iC$ if and only if $\eta_{c}$ is an isomorphism, proving that $K$ is essentially surjective.
  Naturality in $\eta$ then implies that $K$ is fully faithful.
  % https://q.uiver.app/#q=WzAsNCxbMCwwLCJhIl0sWzAsMiwiYiJdLFsyLDAsIkxhIl0sWzIsMiwiTGIiXSxbMCwyLCJcXGV0YV97YX0iXSxbMSwzLCJcXGV0YV97Yn0iLDJdLFswLDEsImYiLDJdLFsyLDMsIkxmIl1d
\[\begin{tikzcd}
	a && La \\
	\\
	b && Lb
	\arrow["{\eta_{a}}", from=1-1, to=1-3]
	\arrow["{\eta_{b}}"', from=3-1, to=3-3]
	\arrow["f"', from=1-1, to=3-1]
	\arrow["Lf", from=1-3, to=3-3]
\end{tikzcd}\]
\end{proof}

\begin{lem}
  The category of algebras for an idempotent monad on $\iC$ defines a reflective subcategory of $\iC$.
  Further, the Kleisli category also reflective.
\end{lem}
\begin{proof}
  It suffices to show that $U^{T}$ and $U_{T}$ are fully faithful.
  Note that for any $T$-algebra $(c, \alpha)$, the structure map $\alpha$ is an isomorphism with inverse $\eta_{c}$
  \[
    \eta_{c} \circ \alpha = T\alpha \circ T\eta_{c} = T(\alpha \circ \eta_{c}) = \id_{Tc}
  \]
  Thus, the component $\epsilon_{(c,\alpha)} = \alpha$ is an isomorphism, implying that $U^{T}$ is fully faithful.

  The action on morphisms of the forgetful functor $U_{T}$ can be factorized as follows:
  % https://q.uiver.app/#q=WzAsMyxbMCwwLCJcXGlDX3tUfShhLGIpIl0sWzAsMiwiXFxpQyhhLFRiKSJdLFsyLDAsIlxcaUMoVGEsVGIpIl0sWzAsMiwiVV97VH0iXSxbMCwxLCJcXGlzbyIsMSx7InN0eWxlIjp7ImhlYWQiOnsibmFtZSI6Im5vbmUifX19XSxbMSwyLCJVIiwyXV0=
\[\begin{tikzcd}
	{\iC_{T}(a,b)} && {\iC(Ta,Tb)} \\
	\\
	{\iC(a,Tb)}
	\arrow["{U_{T}}", from=1-1, to=1-3]
	\arrow["\iso"{description}, no head, from=1-1, to=3-1]
	\arrow["U"', from=3-1, to=1-3]
\end{tikzcd}\]
  where $U : f \mapsto \mu_{b} \circ Tf$.
  $U$ has an obvious inverse: $U\inv : g \mapsto g \circ \eta_{a}$.
  Indeed,
  \begin{align}
    U\inv U(f) &= \mu_{b} \circ Tf \circ \eta_{a}\\
               &= \mu_{b} \circ \eta_{Tb} \circ f\\
               &= \mu_{b} \circ T\eta_{b} \circ f\\
               &= f\\
    UU\inv(g) &= \mu_{b} \circ T(g \circ \eta_{a})\\
               &= \mu_{b} \circ Tg \circ \eta_{Ta}\\
               &= \mu_{b} \circ \eta_{b} \circ g\\
               &= g
  \end{align}
\end{proof}

\bibliographystyle{alpha}
\bibliography{all}

\end{document}
