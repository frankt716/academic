\documentclass{amsart}
\input{decls}
\title{Recognizing Categories of Algebras}
\author{Frank Tsai}
\date{\today}
%\thanks{}
\begin{document}
\maketitle
\tableofcontents

\section{Introduction}
\label{sec:introduction}

\begin{defn}
  A pair of parallel morphisms $f,g : x \toto y$ is \emph{reflexive} if both morphisms admit a common section.
  The coequalizer for a reflexive pair of morphisms is called a \emph{reflexive coequalizer}.
\end{defn}

Recall that every $T$-algebra $(A,\alpha)$ is the quotient of some free algebra:
\begin{equation}
  \label{eq:beck-coequalizer}
  % https://q.uiver.app/#q=WzAsMyxbMCwwLCJcXGlDKHgseSkiXSxbMiwwLCJcXGlEKEZ4LEZ5KSJdLFsyLDIsIlxcaUMoeCxHRnkpIl0sWzEsMiwiXFxpc28iLDFdLFswLDEsIkYiXSxbMCwyLCIoXFxldGFfe3l9KV97Kn0iLDJdXQ==
\[\begin{tikzcd}
	{\iC(x,y)} && {\iD(Fx,Fy)} \\
	\\
	&& {\iC(x,GFy)}
	\arrow["\iso"{description}, from=1-3, to=3-3]
	\arrow["F", from=1-1, to=1-3]
	\arrow["{(\eta_{y})_{*}}"', from=1-1, to=3-3]
\end{tikzcd}\]
\end{equation}

\begin{lem}
  The coequalizer in \eqref{eq:beck-coequalizer} is reflexive in $\iC^{T}$.
\end{lem}
\begin{proof}
  $T\eta_{A} : TA \to T^{2}A$ is the common section.
  The proof is by calculation:
  \begin{align}
    T\alpha \circ T\eta_{A} &= T(\alpha \circ \eta_{A})\\
                &= T(\id_{A})\\
                &= \id_{TA}\\
    \mu_{A} \circ T\eta_{A} &= \id_{TA}
  \end{align}
\end{proof}

\begin{lem}\label{lem:beck-coequalizer-left-adjoint}
  Let $K : \iD \to \iC^{T}$ be the canonical comparison functor, where $T$ is induced by the adjunction $F,D : \iC \toot \iD$.
  If the pair $F\alpha,\epsilon_{FA} : FTA \toto FA$ has a coequalizer in $\iD$, then $K$ admits a left adjoint.
\end{lem}
\begin{proof}
  The hypothesis asserts that the parallel pair $F\alpha,\epsilon_{FA} : FTA \toto FA$ in $\iD$ has a coequalizer, so we can define $L(A,\alpha)$ to be that coequalizer.
  % https://q.uiver.app/#q=WzAsNCxbMCwwLCJMIl0sWzIsMCwiTEwiXSxbMiwyLCJMIl0sWzAsMiwiTEwiXSxbMCwxLCJcXGV0YSBMIl0sWzEsMiwiXFxlcHNpbG9uIEwiXSxbMCwyLCJcXGlkX3tMfSIsMV0sWzAsMywiTFxcZXRhIiwyXSxbMywyLCJcXGVwc2lsb24gTCIsMl1d
\[\begin{tikzcd}
	L && LL \\
	\\
	LL && L
	\arrow["{\eta L}", from=1-1, to=1-3]
	\arrow["{\epsilon L}", from=1-3, to=3-3]
	\arrow["{\id_{L}}"{description}, from=1-1, to=3-3]
	\arrow["L\eta"', from=1-1, to=3-1]
	\arrow["{\epsilon L}"', from=3-1, to=3-3]
\end{tikzcd}\]
  Given any $T$-homomorphism $f : (A,\alpha) \to (B,\beta)$, there are two commutative squares on the left-hand side.
  Commutativity of each square follows from algebra law and naturality, respectively.
  % https://q.uiver.app/#q=WzAsNCxbMCwwLCJjIl0sWzIsMCwiTGMiXSxbMCwyLCJMZCJdLFsyLDIsIkxMZCJdLFswLDEsIlxcZXRhX3tjfSJdLFsyLDMsIlxcZXRhX3tMZH0iLDJdLFswLDIsIlxcaXNvIiwyXSxbMSwzLCJcXGlzbyJdXQ==
\[\begin{tikzcd}
	c && Lc \\
	\\
	Ld && LLd
	\arrow["{\eta_{c}}", from=1-1, to=1-3]
	\arrow["{\eta_{Ld}}"', from=3-1, to=3-3]
	\arrow["\iso"', from=1-1, to=3-1]
	\arrow["\iso", from=1-3, to=3-3]
\end{tikzcd}\]
  In particular, $e \circ Ff$ defines a cocone over $F\alpha,\epsilon_{FA} : FUFA \toto FA$, so we define $Lf$ to be that unique map.
  Uniqueness implies functoriality.

  It remains to check that $L \adj K$.
  The universal property of coequalizer asserts that every morphism $L(A,\alpha) \to B$ corresponds to a morphism $FA \to B$.
  This transports to a morphism $A \to UB$ through the adjunction $F \adj U$.
  In particular, these morphisms are $T$-homomorphisms, proving the adjunction $L \adj K$.
  \begin{align}
    \iD(L(A,\alpha),B) &\iso \iD(FA,B)\\
                  &\iso \iC(A,UB)\\
                  &\iso \iC^{T}((A,\alpha),KB)
  \end{align}
\end{proof}

\begin{thm}[Monadicity Theorem]\label{thm:monadicity-theorem}
  A right adjoint $U : \iD \to \iC$ is monadic if and only if it creates coequalizers of $U$-split pairs.
\end{thm}
\begin{proof}
  We have already proved the ``only if'' direction.
  For the ``if'' direction, assume that $U$ creates coequalizers of $U$-split pairs.
  This hypothesis asserts that the pair $F\alpha,\epsilon_{FA} : FUFA \toto FA$ has a coequalizer in $\iD$.
  Thus, by \cref{lem:beck-coequalizer-left-adjoint}, $L \adj K$.

  It remains to show that $L$ is an inverse equivalence.
  The unit of the adjunction $L \adj K$ is given by
  \begin{align}
    \id_{L(A,\alpha)} \in \iD(L(A,\alpha), L(A,\alpha)) &\iso e \in \iD(FA, L(A,\alpha))\\
                                       &\iso Ue \circ \eta_{A} \in \iC(A,U(A,\alpha))\\
                                       &\iso Ue \circ \eta_{A} \in \iC^{T}((A,\alpha),KB)
  \end{align}
  Since $U$ creates $U$-split coequalizers (and thus preserves $U$-split coequalizers), $Ue : UFA \epi UL(A,\alpha)$ is a coequalizer in $\iC$.
  Since $\alpha : UFA \to A$ is a coequalizer for the same pair of morphisms, $UL(A,\alpha) \iso A$.
  Thus, $\alpha = i \circ Ue$, where $i$ is the isomorphism mentioned above.
  In particular, $\eta_{A}$ is a section for $\alpha$.
  Thus,
  \[
    Ue \circ \eta_{A} = i\inv
  \]
  That is, the unit is a natural isomorphism.

  The counit is given by $\epsilon'_{B} \in \iD(LKB, B) \iso \epsilon_{B} \in \iD(FUB,B)$.
  Note that $\epsilon_{B}$ is a coequalizer for the same pair as $e : FUB \to LKB$.
  Thus, it follows that $\epsilon'_{B}$ is an isomorphism.
\end{proof}

\begin{defn}
  A functor is \emph{finitary} if it preserves filtered colimits.
  Any finitary right adjoint induces a \emph{finitary monad} since left adjoints preserve any colimit.
\end{defn}

\begin{thm}[Reflexive Tripleability Theorem]
  If $U : \iD \to \iC$ admits a left adjoint and if
  \begin{enumerate}
  \item $\iD$ has coequalizers of reflexive pairs;
  \item $U$ preserves coequalizers of reflexive pairs;
  \item $U$ reflects isomorphisms,
  \end{enumerate}
  then $U$ is monadic.
\end{thm}
\begin{proof}
  By \cref{thm:monadicity-theorem}, it suffices to prove that $U$ creates coequalizers of $U$-split pairs.
\end{proof}

\bibliographystyle{alpha}
\bibliography{all}

\end{document}
