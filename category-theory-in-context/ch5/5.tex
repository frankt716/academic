\documentclass{amsart}
\input{decls}
\title{Recognizing Categories of Algebras}
\author{Frank Tsai}
\date{\today}
%\thanks{}
\begin{document}
\maketitle
\tableofcontents

\section{Introduction}
\label{sec:introduction}

\begin{thm}[Monadicity Theorem]
  A right adjoint $U : \iD \to \iC$ is monadic if and only if it creates coequalizers of $U$-split pairs.
\end{thm}
\begin{proof}
  We have already proved the ``if'' direction.
  For the ``only if'' direction, assume that $U$ creates coequalizers of $U$-split pairs.
  Our goal is to construct an inverse equivalence $L$ for the canonical comparison functor $K : \iD \to \iC^{T}$.

  For free algebras, we define $L(TA,\mu_{A}) = FA$ and $Lf = Ff$.
  For any algebra $(A, \alpha)$, note that
  % https://q.uiver.app/#q=WzAsMixbMCwwLCJcXGlDIl0sWzIsMCwiXFxpQ157XFxpSn0iXSxbMCwxLCJcXERlbHRhIiwxXSxbMSwwLCJcXGNvbGltIiwyLHsiY3VydmUiOjN9XSxbMSwwLCJcXGxpbSIsMCx7ImN1cnZlIjotM31dLFsyLDQsIiIsMSx7ImxldmVsIjoxLCJzdHlsZSI6eyJuYW1lIjoiYWRqdW5jdGlvbiJ9fV0sWzMsMiwiIiwxLHsibGV2ZWwiOjEsInN0eWxlIjp7Im5hbWUiOiJhZGp1bmN0aW9uIn19XV0=
\[\begin{tikzcd}
	\iC && {\iC^{\iJ}}
	\arrow[""{name=0, anchor=center, inner sep=0}, "\Delta"{description}, from=1-1, to=1-3]
	\arrow[""{name=1, anchor=center, inner sep=0}, "\colim"', curve={height=18pt}, from=1-3, to=1-1]
	\arrow[""{name=2, anchor=center, inner sep=0}, "\lim", curve={height=-18pt}, from=1-3, to=1-1]
	\arrow["\dashv"{anchor=center, rotate=-86}, draw=none, from=0, to=2]
	\arrow["\dashv"{anchor=center, rotate=-94}, draw=none, from=1, to=0]
\end{tikzcd}\]
  is a $U$-split pair in $\iC$.
  Since $U$ creates coequalizers of this kind, define $L(A,\alpha)$ to be the coequalizer in $\iD$:
  % https://q.uiver.app/#q=WzAsMyxbMCwwLCJGVUZBIl0sWzIsMCwiRkEiXSxbNCwwLCJMKEEsXFxhbHBoYSkiXSxbMCwxLCJGXFxhbHBoYSIsMCx7Im9mZnNldCI6LTF9XSxbMCwxLCJcXGVwc2lsb25fe0ZBfSIsMix7Im9mZnNldCI6MX1dLFsxLDJdXQ==
\[\begin{tikzcd}
	FUFA && FA && {L(A,\alpha)}
	\arrow["F\alpha", shift left, from=1-1, to=1-3]
	\arrow["{\epsilon_{FA}}"', shift right, from=1-1, to=1-3]
	\arrow[from=1-3, to=1-5]
\end{tikzcd}\]
  For any $T$-homomorphism $f : (A, \alpha) \to (B, \beta)$, define $Lf$ as the unique map given by the universal property of coequalizer:
  % https://q.uiver.app/#q=WzAsMixbMCwwLCJcXGlEIl0sWzIsMCwiXFxpQyJdLFswLDEsIiIsMSx7ImN1cnZlIjoyLCJzdHlsZSI6eyJ0YWlsIjp7Im5hbWUiOiJob29rIiwic2lkZSI6InRvcCJ9fX1dLFsxLDAsIkwiLDIseyJjdXJ2ZSI6Mn1dLFszLDIsIiIsMSx7ImxldmVsIjoxLCJzdHlsZSI6eyJuYW1lIjoiYWRqdW5jdGlvbiJ9fV1d
\[\begin{tikzcd}
	\iD && \iC
	\arrow[""{name=0, anchor=center, inner sep=0}, curve={height=12pt}, hook, from=1-1, to=1-3]
	\arrow[""{name=1, anchor=center, inner sep=0}, "L"', curve={height=12pt}, from=1-3, to=1-1]
	\arrow["\dashv"{anchor=center, rotate=-90}, draw=none, from=1, to=0]
\end{tikzcd}\]
  Uniqueness implies functoriality.

  It remains to check that $L$ defines an inverse equivalence of $K$.
  % https://q.uiver.app/#q=WzAsMixbMCwwLCJcXGlDIl0sWzIsMCwiXFxpRCJdLFswLDEsIkYiLDAseyJvZmZzZXQiOi0yfV0sWzEsMCwiRyIsMCx7Im9mZnNldCI6LTJ9XSxbMiwzLCIiLDAseyJsZXZlbCI6MSwic3R5bGUiOnsibmFtZSI6ImFkanVuY3Rpb24ifX1dXQ==
\[\begin{tikzcd}
	\iC && \iD
	\arrow[""{name=0, anchor=center, inner sep=0}, "F", shift left=2, from=1-1, to=1-3]
	\arrow[""{name=1, anchor=center, inner sep=0}, "G", shift left=2, from=1-3, to=1-1]
	\arrow["\dashv"{anchor=center, rotate=-90}, draw=none, from=0, to=1]
\end{tikzcd}\]
  $L$ carries the right-hand parallel pair to the parallel pair on the left.
  Then $K$ carries this parallel pair to the same parallel pair on the right.
  Since $U^{T}$ strictly creates coequalizers of $U^{T}$-split pairs, $KL \iso \id_{\iC^{T}}$.

  Now, since $U$ creates coequalizers of $U$-split pairs, the top diagram is a coequalizer diagram.
  $LKD$ is defined to be the coequalizer of the parallel pair in the diagram.
  % https://q.uiver.app/#q=WzAsMyxbMCwwLCJcXGlEKHgseSkiXSxbMiwwLCJcXGlDKEd4LEd5KSJdLFsyLDIsIlxcaUQoRkd4LHkpIl0sWzAsMSwiRyJdLFsxLDIsIlxcaXNvIiwxXSxbMCwyLCIoXFxlcHNpbG9uX3t4fSleeyp9IiwyXV0=
\[\begin{tikzcd}
	{\iD(x,y)} && {\iC(Gx,Gy)} \\
	\\
	&& {\iD(FGx,y)}
	\arrow["G", from=1-1, to=1-3]
	\arrow["\iso"{description}, from=1-3, to=3-3]
	\arrow["{(\epsilon_{x})^{*}}"', from=1-1, to=3-3]
\end{tikzcd}\]
  The isomorphism $LKD \iso D$ for each $D$ assembles into a natural isomorphism $LK \iso \id_{\iD}$.
\end{proof}

\bibliographystyle{alpha}
\bibliography{all}

\end{document}
