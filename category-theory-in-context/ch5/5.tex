\documentclass{amsart}
\input{decls}
\title{Recognizing Categories of Algebras}
\author{Frank Tsai}
\date{\today}
%\thanks{}
\begin{document}
\maketitle
\tableofcontents

\section{Introduction}
\label{sec:introduction}

\begin{defn}
  A pair of parallel morphisms $f,g : x \toto y$ is \emph{reflexive} if both morphisms admit a common section.
  The coequalizer for a reflexive pair of morphisms is called a \emph{reflexive coequalizer}.
\end{defn}

Recall that every $T$-algebra $(A,\alpha)$ is the quotient of some free algebra:
\begin{equation}
  \label{eq:beck-coequalizer}
  % https://q.uiver.app/#q=WzAsMyxbMCwwLCJcXGlDKHgseSkiXSxbMiwwLCJcXGlEKEZ4LEZ5KSJdLFsyLDIsIlxcaUMoeCxHRnkpIl0sWzEsMiwiXFxpc28iLDFdLFswLDEsIkYiXSxbMCwyLCIoXFxldGFfe3l9KV97Kn0iLDJdXQ==
\[\begin{tikzcd}
	{\iC(x,y)} && {\iD(Fx,Fy)} \\
	\\
	&& {\iC(x,GFy)}
	\arrow["\iso"{description}, from=1-3, to=3-3]
	\arrow["F", from=1-1, to=1-3]
	\arrow["{(\eta_{y})_{*}}"', from=1-1, to=3-3]
\end{tikzcd}\]
\end{equation}

\begin{lem}
  The coequalizer in \eqref{eq:beck-coequalizer} is reflexive in $\iC^{T}$.
\end{lem}
\begin{proof}
  $T\eta_{A} : TA \to T^{2}A$ is the common section.
  The proof is by calculation:
  \begin{align}
    T\alpha \circ T\eta_{A} &= T(\alpha \circ \eta_{A})\\
                &= T(\id_{A})\\
                &= \id_{TA}\\
    \mu_{A} \circ T\eta_{A} &= \id_{TA}
  \end{align}
\end{proof}

\begin{lem}
  Let $K : \iD \to \iC^{T}$ be the canonical comparison functor, where $T$ is induced by the adjunction $F,D : \iC \toot \iD$.
  If the pair $F\alpha,\epsilon_{FA} : FTA \toto FA$ has a coequalizer in $\iD$, then $K$ admits a left adjoint.
\end{lem}
\begin{proof}
  The hypothesis asserts that the parallel pair $F\alpha,\epsilon_{FA} : FTA \toto FA$ in $\iD$ has a coequalizer, so we can define $L(A,\alpha)$ to be that coequalizer.
  % https://q.uiver.app/#q=WzAsNCxbMCwwLCJMIl0sWzIsMCwiTEwiXSxbMiwyLCJMIl0sWzAsMiwiTEwiXSxbMCwxLCJcXGV0YSBMIl0sWzEsMiwiXFxlcHNpbG9uIEwiXSxbMCwyLCJcXGlkX3tMfSIsMV0sWzAsMywiTFxcZXRhIiwyXSxbMywyLCJcXGVwc2lsb24gTCIsMl1d
\[\begin{tikzcd}
	L && LL \\
	\\
	LL && L
	\arrow["{\eta L}", from=1-1, to=1-3]
	\arrow["{\epsilon L}", from=1-3, to=3-3]
	\arrow["{\id_{L}}"{description}, from=1-1, to=3-3]
	\arrow["L\eta"', from=1-1, to=3-1]
	\arrow["{\epsilon L}"', from=3-1, to=3-3]
\end{tikzcd}\]
  Given any $T$-homomorphism $f : (A,\alpha) \to (B,\beta)$, there are two commutative squares on the left-hand side.
  Commutativity of each square follows from algebra law and naturality, respectively.
  % https://q.uiver.app/#q=WzAsNCxbMCwwLCJjIl0sWzIsMCwiTGMiXSxbMCwyLCJMZCJdLFsyLDIsIkxMZCJdLFswLDEsIlxcZXRhX3tjfSJdLFsyLDMsIlxcZXRhX3tMZH0iLDJdLFswLDIsIlxcaXNvIiwyXSxbMSwzLCJcXGlzbyJdXQ==
\[\begin{tikzcd}
	c && Lc \\
	\\
	Ld && LLd
	\arrow["{\eta_{c}}", from=1-1, to=1-3]
	\arrow["{\eta_{Ld}}"', from=3-1, to=3-3]
	\arrow["\iso"', from=1-1, to=3-1]
	\arrow["\iso", from=1-3, to=3-3]
\end{tikzcd}\]
  In particular, $e \circ Ff$ defines a cocone over $F\alpha,\epsilon_{FA} : FUFA \toto FA$, so we define $Lf$ to be that unique map.
  Uniqueness implies functoriality.

  It remains to check that $L \adj K$.
  The universal property of coequalizer asserts that every morphism $L(A,\alpha) \to B$ corresponds to a morphism $FA \to B$.
  This transports to a morphism $A \to UB$ through the adjunction $F \adj U$.
  In particular, these morphisms are $T$-homomorphisms, proving the adjunction $L \adj K$.
  \begin{align}
    \iD(L(A,\alpha),B) &\iso \iD(FA,B)\\
                  &\iso \iC(A,UB)\\
                  &\iso \iC^{T}((A,\alpha),KB)
  \end{align}
\end{proof}

\begin{thm}[Monadicity Theorem]\label{thm:monadicity-theorem}
  A right adjoint $U : \iD \to \iC$ is monadic if and only if it creates coequalizers of $U$-split pairs.
\end{thm}
\begin{proof}
  We have already proved the ``only if'' direction.
  For the ``if'' direction, assume that $U$ creates coequalizers of $U$-split pairs.
  Our goal is to construct an inverse equivalence $L$ for the canonical comparison functor $K : \iD \to \iC^{T}$.

  For free algebras, we define $L(TA,\mu_{A}) = FA$ and $Lf = Ff$.
  For any algebra $(A, \alpha)$, note that
  % https://q.uiver.app/#q=WzAsMixbMCwwLCJcXGlDIl0sWzIsMCwiXFxpQ157XFxpSn0iXSxbMCwxLCJcXERlbHRhIiwxXSxbMSwwLCJcXGNvbGltIiwyLHsiY3VydmUiOjN9XSxbMSwwLCJcXGxpbSIsMCx7ImN1cnZlIjotM31dLFsyLDQsIiIsMSx7ImxldmVsIjoxLCJzdHlsZSI6eyJuYW1lIjoiYWRqdW5jdGlvbiJ9fV0sWzMsMiwiIiwxLHsibGV2ZWwiOjEsInN0eWxlIjp7Im5hbWUiOiJhZGp1bmN0aW9uIn19XV0=
\[\begin{tikzcd}
	\iC && {\iC^{\iJ}}
	\arrow[""{name=0, anchor=center, inner sep=0}, "\Delta"{description}, from=1-1, to=1-3]
	\arrow[""{name=1, anchor=center, inner sep=0}, "\colim"', curve={height=18pt}, from=1-3, to=1-1]
	\arrow[""{name=2, anchor=center, inner sep=0}, "\lim", curve={height=-18pt}, from=1-3, to=1-1]
	\arrow["\dashv"{anchor=center, rotate=-86}, draw=none, from=0, to=2]
	\arrow["\dashv"{anchor=center, rotate=-94}, draw=none, from=1, to=0]
\end{tikzcd}\]
  is a $U$-split pair in $\iC$.
  Since $U$ creates coequalizers of this kind, define $L(A,\alpha)$ to be the coequalizer in $\iD$:
  % https://q.uiver.app/#q=WzAsMyxbMCwwLCJGVUZBIl0sWzIsMCwiRkEiXSxbNCwwLCJMKEEsXFxhbHBoYSkiXSxbMCwxLCJGXFxhbHBoYSIsMCx7Im9mZnNldCI6LTF9XSxbMCwxLCJcXGVwc2lsb25fe0ZBfSIsMix7Im9mZnNldCI6MX1dLFsxLDJdXQ==
\[\begin{tikzcd}
	FUFA && FA && {L(A,\alpha)}
	\arrow["F\alpha", shift left, from=1-1, to=1-3]
	\arrow["{\epsilon_{FA}}"', shift right, from=1-1, to=1-3]
	\arrow[from=1-3, to=1-5]
\end{tikzcd}\]
  For any $T$-homomorphism $f : (A, \alpha) \to (B, \beta)$, define $Lf$ as the unique map given by the universal property of coequalizer:
  % https://q.uiver.app/#q=WzAsMixbMCwwLCJcXGlEIl0sWzIsMCwiXFxpQyJdLFswLDEsIiIsMSx7ImN1cnZlIjoyLCJzdHlsZSI6eyJ0YWlsIjp7Im5hbWUiOiJob29rIiwic2lkZSI6InRvcCJ9fX1dLFsxLDAsIkwiLDIseyJjdXJ2ZSI6Mn1dLFszLDIsIiIsMSx7ImxldmVsIjoxLCJzdHlsZSI6eyJuYW1lIjoiYWRqdW5jdGlvbiJ9fV1d
\[\begin{tikzcd}
	\iD && \iC
	\arrow[""{name=0, anchor=center, inner sep=0}, curve={height=12pt}, hook, from=1-1, to=1-3]
	\arrow[""{name=1, anchor=center, inner sep=0}, "L"', curve={height=12pt}, from=1-3, to=1-1]
	\arrow["\dashv"{anchor=center, rotate=-90}, draw=none, from=1, to=0]
\end{tikzcd}\]
  Uniqueness implies functoriality.

  It remains to check that $L$ defines an inverse equivalence of $K$.
  % https://q.uiver.app/#q=WzAsMixbMCwwLCJcXGlDIl0sWzIsMCwiXFxpRCJdLFswLDEsIkYiLDAseyJvZmZzZXQiOi0yfV0sWzEsMCwiRyIsMCx7Im9mZnNldCI6LTJ9XSxbMiwzLCIiLDAseyJsZXZlbCI6MSwic3R5bGUiOnsibmFtZSI6ImFkanVuY3Rpb24ifX1dXQ==
\[\begin{tikzcd}
	\iC && \iD
	\arrow[""{name=0, anchor=center, inner sep=0}, "F", shift left=2, from=1-1, to=1-3]
	\arrow[""{name=1, anchor=center, inner sep=0}, "G", shift left=2, from=1-3, to=1-1]
	\arrow["\dashv"{anchor=center, rotate=-90}, draw=none, from=0, to=1]
\end{tikzcd}\]
  $L$ carries the right-hand parallel pair to the parallel pair on the left.
  Then $K$ carries this parallel pair to the same parallel pair on the right.
  Since $U^{T}$ strictly creates coequalizers of $U^{T}$-split pairs, $KL \iso \id_{\iC^{T}}$.

  Now, since $U$ creates coequalizers of $U$-split pairs, the top diagram is a coequalizer diagram.
  $LKD$ is defined to be the coequalizer of the parallel pair in the diagram.
  % https://q.uiver.app/#q=WzAsMyxbMCwwLCJcXGlEKHgseSkiXSxbMiwwLCJcXGlDKEd4LEd5KSJdLFsyLDIsIlxcaUQoRkd4LHkpIl0sWzAsMSwiRyJdLFsxLDIsIlxcaXNvIiwxXSxbMCwyLCIoXFxlcHNpbG9uX3t4fSleeyp9IiwyXV0=
\[\begin{tikzcd}
	{\iD(x,y)} && {\iC(Gx,Gy)} \\
	\\
	&& {\iD(FGx,y)}
	\arrow["G", from=1-1, to=1-3]
	\arrow["\iso"{description}, from=1-3, to=3-3]
	\arrow["{(\epsilon_{x})^{*}}"', from=1-1, to=3-3]
\end{tikzcd}\]
  The isomorphism $LKD \iso D$ for each $D$ assembles into a natural isomorphism $LK \iso \id_{\iD}$.
\end{proof}

\begin{defn}
  A functor is \emph{finitary} if it preserves filtered colimits.
  Any finitary right adjoint induces a \emph{finitary monad} since left adjoints preserve any colimit.
\end{defn}

\begin{thm}[Reflexive Tripleability Theorem]
  If $U : \iD \to \iC$ admits a left adjoint and if
  \begin{enumerate}
  \item $\iD$ has coequalizers of reflexive pairs;
  \item $U$ preserves coequalizers of reflexive pairs;
  \item $U$ reflects isomorphisms,
  \end{enumerate}
  then $U$ is monadic.
\end{thm}
\begin{proof}
  By \cref{thm:monadicity-theorem}, it suffices to prove that $U$ creates coequalizers of $U$-split pairs.
\end{proof}

\bibliographystyle{alpha}
\bibliography{all}

\end{document}
