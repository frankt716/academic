\documentclass{amsart}
\input{decls}
\title{Limits and Colimits in Categories of Algebras}
\author{Frank Tsai}
\date{\today}
%\thanks{}
\begin{document}
\maketitle
\tableofcontents

\section{Introduction}
\label{sec:introduction}

\begin{lem}
  If $U : \iA \to \iC$ is monadic, then $U$ is conservative.
\end{lem}
\begin{proof}
  Since $U$ is monadic, the canonical comparison functor $K : \iA \to \iC^{T}$ is fully faithful, and thus conservative.
  And since $U = U^{T} \circ K$, it suffices to show that $U^{T}$ is conservative.

  Any $T$-homomorphism $f : (A,\alpha) \to (B,\beta)$ is a morphism $f : A \to B$.
  If $f$ is an isomorphism, then there is an inverse $f\inv : B \to A$, and we wish to show that $f\inv$ is a $T$-homomorphism.
  That is,
  \[
    f\inv \circ \beta = \alpha \circ Tf\inv
  \]
  Since $f$ is an isomorphism (so a monomorphism), it suffices to show that
  \[
    f \circ f\inv \beta = f \circ \alpha \circ Tf\inv
  \]
  This follows immediately from the fact that $f$ is a $T$-homomorphism.
\end{proof}

\begin{cor}
  Any bijective homomorphism in a category of models for an algebraic theory is an isomorphism.
\end{cor}

\begin{thm}\label{thm:monadic-create-limits}
  A monadic functor $U : \iA \to \iC$ creates
  \begin{enumerate}
  \item all limits that $\iC$ has, and
  \item all colimits that $\iC$ has and $T$ and $T^{2}$ preserve.
  \end{enumerate}
\end{thm}
\begin{proof}
  Since $\iA \eqv \iC^{T}$, the comparison functor creates any limit and colimit of $\iC^{T}$.
  Thus, it suffices to prove that $U^{T}$ has these properties.
  
  For (i), consider a diagram $D : \iJ \to \iC^{T}$ spanning $(Dj_{i},\gamma_{i}) \in \iC^{T}$, whose image under $U^{T}$ is a limit cone $\lambda : \ell \to U^{T}D$.
  We need to lift $\ell$ to a $T$-algebra.
  Note that we can define a cone over $U^{T}D$ as follows:
  % https://q.uiver.app/#q=WzAsMyxbMCwyLCJSIl0sWzIsMCwiUyJdLFswLDAsIlJeeyp9Il0sWzAsMiwiXFxldGEiXSxbMCwxLCJmIiwyXSxbMiwxLCIiLDAseyJzdHlsZSI6eyJib2R5Ijp7Im5hbWUiOiJkYXNoZWQifX19XV0=
\[\begin{tikzcd}
	{R^{*}} && S \\
	\\
	R
	\arrow["\eta", from=3-1, to=1-1]
	\arrow["f"', from=3-1, to=1-3]
	\arrow[dashed, from=1-1, to=1-3]
\end{tikzcd}\]
  This suggests that the algebra map should be the map $\alpha : T\ell \to \ell$ given by the universal property of limits.
  It is completely obvious that $(\ell,\alpha)$ is a limit since for any cone $\nu : (D,\kappa) \to D$, the unique map $\nu' : D \to \ell$ is a $T$-homomorphism.
  It remains to check that $(\ell,\alpha)$ is an algebra.
  
  Consider the following cone:
  % https://q.uiver.app/#q=WzAsMyxbMCwyLCJzIl0sWzAsMCwiVUZzIl0sWzIsMCwiVWEiXSxbMCwxLCJcXGV0YV97c30iXSxbMSwyLCIiLDIseyJzdHlsZSI6eyJib2R5Ijp7Im5hbWUiOiJkYXNoZWQifX19XSxbMCwyLCJmIiwyXV0=
\[\begin{tikzcd}
	UFs && Ua \\
	\\
	s
	\arrow["{\eta_{s}}", from=3-1, to=1-1]
	\arrow[dashed, from=1-1, to=1-3]
	\arrow["f"', from=3-1, to=1-3]
\end{tikzcd}\]
  Clearly, $\id_{\ell}$ factors through the legs of the cone.
  Thus, $\alpha \circ \eta_{\ell} = \id_{\ell}$.
  Now, consider the following:
  % https://q.uiver.app/#q=WzAsNCxbMiwwLCJUXnsyfVxcZWxsIl0sWzIsMiwiXFxlbGwiXSxbMCw0LCJEal97MX0iXSxbNCw0LCJEal97Mn0iXSxbMCwxLCJcXGFscGhhIFxcY2lyYyBUXFxhbHBoYSIsMV0sWzIsM10sWzEsMiwiXFxsYW1iZGFfe2pfezF9fSIsMl0sWzEsMywiXFxsYW1iZGFfe2pfezJ9fSJdLFswLDIsIlxcbGFtYmRhX3tqX3sxfX0gXFxjaXJjIFxcYWxwaGEgXFxjaXJjIFxcbXVfe1xcZWxsfSIsMix7ImN1cnZlIjoyfV0sWzAsMywiXFxsYW1iZGFfe2pfezJ9fSBcXGNpcmMgXFxhbHBoYSBcXGNpcmMgXFxtdV97XFxlbGx9IiwwLHsiY3VydmUiOi0yfV1d
\[\begin{tikzcd}
	&& {T^{2}\ell} \\
	\\
	&& \ell \\
	\\
	{Dj_{1}} &&&& {Dj_{2}}
	\arrow["{\alpha \circ T\alpha}"{description}, from=1-3, to=3-3]
	\arrow[from=5-1, to=5-5]
	\arrow["{\lambda_{j_{1}}}"', from=3-3, to=5-1]
	\arrow["{\lambda_{j_{2}}}", from=3-3, to=5-5]
	\arrow["{\lambda_{j_{1}} \circ \alpha \circ \mu_{\ell}}"', curve={height=12pt}, from=1-3, to=5-1]
	\arrow["{\lambda_{j_{2}} \circ \alpha \circ \mu_{\ell}}", curve={height=-12pt}, from=1-3, to=5-5]
\end{tikzcd}\]
  Since $\alpha \circ \mu_{\ell}$ and $\alpha \circ T\alpha$ factor through the legs of the same cone, uniqueness demands that they are equal.

  For (ii), consider a diagram $D : \iJ \to \iC^{T}$ spanning $(Dj_{i},\gamma_{i}) \in \iC^{T}$, whose image under $U^{T}$ is a colimit $\lambda : U^{T}D \to \ell$ preserved by $T$ and $T^{2}$.
  Our goal is to lift $\ell$ to a $T$-algebra defining a colimit of $D$.
  Consider the following tower of colimits:
  % https://q.uiver.app/#q=WzAsMTIsWzAsMCwiRGpfezF9Il0sWzQsMCwiRGpfezJ9Il0sWzIsMSwiXFxlbGwiXSxbMiwzLCJUXFxlbGwiXSxbMCwyLCJURGpfezF9Il0sWzQsMiwiVERqX3syfSJdLFsyLDUsIlReezJ9XFxlbGwiXSxbMCw0LCJUXnsyfURqX3sxfSJdLFs0LDQsIlReezJ9RGpfezJ9Il0sWzIsNywiXFxlbGwiXSxbMCw2LCJEal97MX0iXSxbNCw2LCJEal97Mn0iXSxbMCwxXSxbMCwyLCJcXGxhbWJkYV97al97MX19IiwyXSxbMSwyLCJcXGxhbWJkYV97al97Mn19Il0sWzQsMywiVFxcbGFtYmRhX3tqX3sxfX0iLDJdLFs1LDMsIlRcXGxhbWJkYV97al97Mn19Il0sWzQsNV0sWzcsOF0sWzcsNiwiVF57Mn1cXGxhbWJkYV97al97MX19IiwyXSxbOCw2LCJUXnsyfVxcbGFtYmRhX3tqX3syfX0iXSxbMCw0LCJcXGV0YV97RGpfezF9fSIsMl0sWzEsNSwiXFxldGFfe0RqX3syfX0iXSxbNCw3LCJUXFxldGFfe0RqX3sxfX0iLDJdLFs1LDgsIlRcXGV0YV97RGpfezJ9fSJdLFs3LDEwLCJcXGdhbW1hX3tEal97MX19IFxcY2lyYyBcXG11X3tEal97MX19IiwyXSxbOCwxMSwiXFxnYW1tYV97RGpfezJ9fSBcXGNpcmMgXFxtdV97RGpfezJ9fSJdLFsxMCw5LCJcXGxhbWJkYV97al97MX19IiwyXSxbMTEsOSwiXFxsYW1iZGFfe2pfezJ9fSJdLFsyLDMsIiIsMSx7InN0eWxlIjp7ImJvZHkiOnsibmFtZSI6ImRhc2hlZCJ9fX1dLFszLDYsIlxceGkiLDAseyJsYWJlbF9wb3NpdGlvbiI6MzAsInN0eWxlIjp7ImJvZHkiOnsibmFtZSI6ImRhc2hlZCJ9fX1dLFs2LDksIlxcemV0YSIsMCx7InN0eWxlIjp7ImJvZHkiOnsibmFtZSI6ImRhc2hlZCJ9fX1dXQ==
\[\begin{tikzcd}
	{Dj_{1}} &&&& {Dj_{2}} \\
	&& \ell \\
	{TDj_{1}} &&&& {TDj_{2}} \\
	&& T\ell \\
	{T^{2}Dj_{1}} &&&& {T^{2}Dj_{2}} \\
	&& {T^{2}\ell} \\
	{Dj_{1}} &&&& {Dj_{2}} \\
	&& \ell
	\arrow[from=1-1, to=1-5]
	\arrow["{\lambda_{j_{1}}}"', from=1-1, to=2-3]
	\arrow["{\lambda_{j_{2}}}", from=1-5, to=2-3]
	\arrow["{T\lambda_{j_{1}}}"', from=3-1, to=4-3]
	\arrow["{T\lambda_{j_{2}}}", from=3-5, to=4-3]
	\arrow[from=3-1, to=3-5]
	\arrow[from=5-1, to=5-5]
	\arrow["{T^{2}\lambda_{j_{1}}}"', from=5-1, to=6-3]
	\arrow["{T^{2}\lambda_{j_{2}}}", from=5-5, to=6-3]
	\arrow["{\eta_{Dj_{1}}}"', from=1-1, to=3-1]
	\arrow["{\eta_{Dj_{2}}}", from=1-5, to=3-5]
	\arrow["{T\eta_{Dj_{1}}}"', from=3-1, to=5-1]
	\arrow["{T\eta_{Dj_{2}}}", from=3-5, to=5-5]
	\arrow["{\gamma_{Dj_{1}} \circ \mu_{Dj_{1}}}"', from=5-1, to=7-1]
	\arrow["{\gamma_{Dj_{2}} \circ \mu_{Dj_{2}}}", from=5-5, to=7-5]
	\arrow["{\lambda_{j_{1}}}"', from=7-1, to=8-3]
	\arrow["{\lambda_{j_{2}}}", from=7-5, to=8-3]
	\arrow[dashed, from=2-3, to=4-3]
	\arrow["\xi"{pos=0.3}, dashed, from=4-3, to=6-3]
	\arrow["\zeta", dashed, from=6-3, to=8-3]
\end{tikzcd}\]
  Define $\alpha := \zeta \circ \xi$.
  Again, it is clear that $(\ell,\alpha)$ is a colimit if $\alpha$ is an algebra map.
  Thus, it remains to check that $(\ell,\alpha)$ is an algebra.
  Consider
  % https://q.uiver.app/#q=WzAsNCxbMCwzLCJcXGVsbCJdLFsyLDMsImMiXSxbMSwyLCJcXGVsbCJdLFsxLDAsIngiXSxbMCwxLCJmIiwyXSxbMiwwLCJcXGlkIiwyXSxbMiwxXSxbMywwLCJcXG11X3tcXGVsbH0iLDIseyJjdXJ2ZSI6Mn1dLFszLDEsIiIsMCx7ImN1cnZlIjotMn1dLFszLDIsIlxcbXVfe1xcZWxsfSIsMCx7InN0eWxlIjp7ImJvZHkiOnsibmFtZSI6ImRhc2hlZCJ9fX1dXQ==
\[\begin{tikzcd}
	& x \\
	\\
	& \ell \\
	\ell && c
	\arrow["f"', from=4-1, to=4-3]
	\arrow["\id"', from=3-2, to=4-1]
	\arrow[from=3-2, to=4-3]
	\arrow["{\mu_{\ell}}"', curve={height=12pt}, from=1-2, to=4-1]
	\arrow[curve={height=-12pt}, from=1-2, to=4-3]
	\arrow["{\mu_{\ell}}", dashed, from=1-2, to=3-2]
\end{tikzcd}\]
  Clearly, $\id_{\ell}$ factors through this cocone.
  Thus, uniqueness demands that $\alpha \circ \eta_{\ell} = \id_{\ell}$.
  Also consider
  % https://q.uiver.app/#q=WzAsMyxbMSwwLCJcXGVsbCJdLFswLDIsIlxcZWxsIl0sWzIsMiwiayJdLFsxLDIsIlxca2FwcGFfe2t9IiwyXSxbMCwyLCJcXGxhbWJkYV97a30gPSBcXGthcHBhX3trfSJdLFswLDEsIlxcbGFtYmRhX3tcXGVsbH0iLDJdXQ==
\[\begin{tikzcd}
	& \ell \\
	\\
	\ell && k
	\arrow["{\kappa_{k}}"', from=3-1, to=3-3]
	\arrow["{\lambda_{k} = \kappa_{k}}", from=1-2, to=3-3]
	\arrow["{\lambda_{\ell}}"', from=1-2, to=3-1]
\end{tikzcd}\]
  Uniqueness demands that $\alpha \circ \mu_{\ell} = \alpha \circ T\alpha$.
\end{proof}

\cref{thm:monadic-create-limits} has a number of corollaries.

\begin{cor}
  The inclusion functor of a reflexive subcategory creates all limits.
\end{cor}
\begin{proof}
  The inclusion functor of a reflexive subcategory is monadic.
  Thus, the result follows immediately from \cref{thm:monadic-create-limits}
\end{proof}

\begin{cor}\label{cor:monadic-set-complete}
  Any category that is monadic over $\mathsf{Set}$ is complete.
\end{cor}
\begin{proof}
  This is immediate since $\mathsf{Set}$ is complete.
\end{proof}

\begin{cor}
  $\mathsf{Set}$ is cocomplete.
\end{cor}
\begin{proof}
  We have already proved this.
  An alternate proof of this fact that relies on \cref{thm:monadic-create-limits} is based on the observation that the contravariant powerset functor $\cP : \mathsf{Set}\op \to \mathsf{Set}$ is monadic.
  Thus, by \cref{cor:monadic-set-complete}, $\mathsf{Set}\op$ is complete, proving that $\mathsf{Set}$ is cocomplete.
\end{proof}

\cref{thm:monadic-create-limits} shows that any category of models for an algebraic theory is complete.
In fact, any such a category is also cocomplete.

\begin{thm}
  Suppose $\iC$ is cocomplete and $U : \iA \to \iC$ is monadic.
  Then the following are equivalent:
  \begin{enumerate}
  \item $\iA$ is cocomplete.
  \item $\iA$ has coequalizers.
  \end{enumerate}
\end{thm}

\bibliographystyle{alpha}
\bibliography{all}

\end{document}
