\documentclass{amsart}
\input{decls}
\title{Kan Extensions}
\author{Frank Tsai}
\date{\today}
%\thanks{}
\begin{document}
\maketitle
\tableofcontents

\section{Introduction}
\label{sec:introduction}
% ...
\begin{defn}
  Given functors $F : \iC \to \iE$ and $K : \iC \to \iD$, a \emph{left Kan extension} of $F$ along $K$ is a functor $\mathrm{Kan}_{K}F : \iD \to \iE$ together with a natural transformation $\eta : F \to \mathrm{Kan}_{K} F \circ K$ such that for any other such pair $(G : \iD \to \iE, \gamma : F \to GK)$, $\gamma$ factors uniquely through $\eta$.
  \[% https://q.uiver.app/#q=WzAsMyxbMCwxLCJSIl0sWzIsMSwiUyJdLFsxLDAsIkMiXSxbMCwyLCJmIiwwLHsic3R5bGUiOnsidGFpbCI6eyJuYW1lIjoibW9ubyJ9fX1dLFsxLDIsImciLDIseyJzdHlsZSI6eyJ0YWlsIjp7Im5hbWUiOiJtb25vIn19fV0sWzAsMSwiXFx0YXUiLDJdXQ==
\begin{tikzcd}
	& C \\
	R && S
	\arrow["f", tail, from=2-1, to=1-2]
	\arrow["g"', tail, from=2-3, to=1-2]
	\arrow["\tau"', from=2-1, to=2-3]
\end{tikzcd}\]
  Dually, a \emph{right Kan extension} reverses the direction of $\eta$ and $\gamma$.
\end{defn}

\begin{prop}
  If, for fixed $K : \iC \to \iD$ and $\iE$, the left and right Kan extensions of any functor $F : \iC \to \iE$ along $K$ exist, then these define left and right adjoints to the pre-composition functor $K^{*} : \iE^{\iD} \to \iE^{\iC}$.
  \[% https://q.uiver.app/#q=WzAsNCxbMCwwLCJUXnszfSJdLFsyLDAsIlReezJ9Il0sWzAsMiwiVF57Mn0iXSxbMiwyLCJUIl0sWzAsMSwiVFxcbXUiXSxbMCwyLCJcXG11X3tUfSIsMl0sWzIsMywiXFxtdSIsMl0sWzEsMywiXFxtdSJdXQ==
\begin{tikzcd}
	{T^{3}} && {T^{2}} \\
	\\
	{T^{2}} && T
	\arrow["T\mu", from=1-1, to=1-3]
	\arrow["{\mu_{T}}"', from=1-1, to=3-1]
	\arrow["\mu"', from=3-1, to=3-3]
	\arrow["\mu", from=1-3, to=3-3]
\end{tikzcd}\]
\end{prop}
\begin{proof}
  By definition, we have the following isomorphisms for any $F : \iC \to \iE$.
  \begin{mathpar}
    \iE^{\iD}(\lan_{K}F, G) \iso \iE^{\iC}(F, GK) \and \iE^{\iC}(GK, F) \iso \iE^{\iD}(G, \ran_{K}F)
  \end{mathpar}
  These isomorphisms assemble into adjunctions $\lan_{K} \adj K^{*} \adj \ran_{K}$.
\end{proof}

Observe that the right Kan extension $\ran_{K}F$ of $F$ along $K$ is the terminal object in the category of elements $\int\iE^{\iC}(\blank \circ K, F)$ assuming $\iE^{\iC}$ is locally small.

\bibliographystyle{alpha}
\bibliography{all}

\end{document}
