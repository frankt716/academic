\documentclass{amsart}
\input{decls}
\title{A Formula for Kan Extentions}
\author{Frank Tsai}
\date{\today}
%\thanks{}
\begin{document}
\maketitle
\tableofcontents

\section{Introduction}
\label{sec:introduction}
Recall that a functor between posets is an order-preserving function, and a natural transformation $f \to g$ between two order-preserving functions is a witness for $f \leq g$.
The left Kan extension of $F$ along $K$, if it exists, is the ``smallest'' order-preserving function so that $F \leq \lan_{K}F \circ K$.
\[% https://q.uiver.app/#q=WzAsMyxbMCwwLCJcXGlDIl0sWzEsMSwiXFxpRCJdLFsyLDAsIlxcaUUiXSxbMCwxLCJLIiwyXSxbMCwyLCJGIl0sWzEsMiwiXFxsYW5fe0t9RiIsMix7InN0eWxlIjp7ImJvZHkiOnsibmFtZSI6ImRhc2hlZCJ9fX1dLFs0LDEsIiIsMCx7InNob3J0ZW4iOnsic291cmNlIjoyMH19XV0=
\begin{tikzcd}
	\iC && \iE \\
	& \iD
	\arrow["K"', from=1-1, to=2-2]
	\arrow[""{name=0, anchor=center, inner sep=0}, "F", from=1-1, to=1-3]
	\arrow["{\lan_{K}F}"', dashed, from=2-2, to=1-3]
	\arrow[shorten <=3pt, Rightarrow, from=0, to=2-2]
\end{tikzcd}\]
The universal property demands that for any order-preserving function $G$ such that $F \leq G \circ K$, $\lan_{K}F \leq G$.
This suggests that we take
\[
  \lan_{K}F(d) := \sup\{Fc \mid Kc \leq d\}
\]
Dually, the same argument suggests that we take
\[
  \ran_{K}F(d) := \inf\{Fc \mid d \leq Kc\}
\]
That is, the left (resp., right) Kan extension at $d$ is the supremum (resp., infimum) of all left (resp., right) approximations of $d$ under the image of $F$.
Admittedly, these are the most obvious choices, but they are equivalent to the obvious ones.
Observe that the comma category $K \dn d$ contains pairs $(c, w)$, where $w$ is a witness for $Kc \leq d$.
Thus, the definitions above are equivalent to the followings:
\begin{align}
  \lan_{K}F(d) &:= \colim(\xMor{K \dn d}{\Pi^{d}}{\iC} \xMor{}{F}{\iE})\\
  \ran_{K}F(d) &:= \lim(\xMor{d \dn K}{\Pi_{d}}{\iC} \xMor{}{F}{\iE})
\end{align}

These generalize to any category in which certain limits or colimits exist.

\begin{prop}
  Given functors $F : \iC \to \iE$ and $K : \iC \to \iD$, if for every $d \in \iD$ the colimit
  \[
    \lan_{K}F(d) = \colim(\xMor{K \dn d}{\Pi^{d}}{\iC} \xMor{}{F}{\iE})
  \]
  exists, then they define the left Kan extension.
  Dually, if for every $d \in \iD$ the limit
  \[
    \ran_{K}F(d) = \lim(\xMor{d \dn K}{\Pi_{d}}{\iC} \xMor{}{F}{\iE})
  \]
  exists, then they define the right Kan extension.
\end{prop}
\begin{proof}
  First, we need to extend $\lan_{K}F$ to a functor.
  Consider any morphism $g : d \to d'$.
  This morphism induces a canonical functor $g_{*} : K \dn d \to K \dn d'$ defined by post-composition.
  In particular, the following diagram commutes.
  \[% https://q.uiver.app/#q=WzAsNCxbMCwwLCJcXG1hdGhybXtBZmZ9X3trfShcXG1hdGhybXtBZmZ9X3trfShBKSkiXSxbMiwwLCJcXG1hdGhybXtBZmZ9X3trfShBKSJdLFswLDIsIlxcbWF0aHJte0FmZn1fe2t9KEEpIl0sWzIsMiwiQSJdLFsxLDMsIlxcbWF0aHJte2V2fV97QX0iXSxbMiwzLCJcXG1hdGhybXtldn1fe0F9IiwyXSxbMCwyLCJcXG1hdGhybXtBZmZ9X3trfShcXG1hdGhybXtldn1fe0F9KSIsMl0sWzAsMSwiXFxtdV97QX0iXV0=
\begin{tikzcd}
	{\mathrm{Aff}_{k}(\mathrm{Aff}_{k}(A))} && {\mathrm{Aff}_{k}(A)} \\
	\\
	{\mathrm{Aff}_{k}(A)} && A
	\arrow["{\mathrm{ev}_{A}}", from=1-3, to=3-3]
	\arrow["{\mathrm{ev}_{A}}"', from=3-1, to=3-3]
	\arrow["{\mathrm{Aff}_{k}(\mathrm{ev}_{A})}"', from=1-1, to=3-1]
	\arrow["{\mu_{A}}", from=1-1, to=1-3]
\end{tikzcd}\]
  For each $d \in \iD$, $\lan_{K}F(d)$ provides a specified colimit cocone $\lambda^{d} : F\Pi^{d} \to \lan_{K}F(d)$.
  For $\lambda^{d'} : F\Pi^{d'} \to \lan_{K}F(d')$, we can define a cocone $\mu^{d'} : F\Pi^{d} \to \lan_{K}F(d')$ by reindexing with $g$.
  The universal property then yields a unique morphism $\lan_{K}F(d) \to \lan_{K}F(d')$.
  \[% https://q.uiver.app/#q=WzAsMyxbMSwwLCJGYyJdLFswLDIsIlxcbGFuX3tLfUYoZCkiXSxbMiwyLCJcXGxhbl97S31GKGQnKSJdLFswLDEsIlxcbGFtYmRhXntkfV97Zn0iLDJdLFswLDIsIlxcbGFtYmRhXntkJ31fe2dmfSJdLFsxLDIsIiIsMix7InN0eWxlIjp7ImJvZHkiOnsibmFtZSI6ImRhc2hlZCJ9fX1dXQ==
\begin{tikzcd}
	& Fc \\
	\\
	{\lan_{K}F(d)} && {\lan_{K}F(d')}
	\arrow["{\lambda^{d}_{f}}"', from=1-2, to=3-1]
	\arrow["{\lambda^{d'}_{gf}}", from=1-2, to=3-3]
	\arrow[dashed, from=3-1, to=3-3]
\end{tikzcd}\]
  Uniqueness implies functoriality.
  The unit natural transformation is defined by
  \[
    \eta_{c} := \lambda^{Kc}_{\id_{Kc}} : Fc \to \lan_{K}F(Kc)
  \]
  Consider
  \[% https://q.uiver.app/#q=WzAsNCxbMCwwLCJGYyJdLFsyLDAsIlxcbGFuX3tLfUYoS2MpIl0sWzAsMiwiRmMnIl0sWzIsMiwiXFxsYW5fe0t9RihLYycpIl0sWzAsMiwiRmYiLDJdLFsxLDMsIlxcbGFuX3tLfUYoS2YpIl0sWzAsMSwiXFxldGFfe2N9Il0sWzIsMywiXFxldGFfe2MnfSIsMl0sWzAsMywiXFxsYW1iZGFee0tjJ31fe0tmfSIsMV1d
\begin{tikzcd}
	Fc && {\lan_{K}F(Kc)} \\
	\\
	{Fc'} && {\lan_{K}F(Kc')}
	\arrow["Ff"', from=1-1, to=3-1]
	\arrow["{\lan_{K}F(Kf)}", from=1-3, to=3-3]
	\arrow["{\eta_{c}}", from=1-1, to=1-3]
	\arrow["{\eta_{c'}}"', from=3-1, to=3-3]
	\arrow["{\lambda^{Kc'}_{Kf}}"{description}, from=1-1, to=3-3]
\end{tikzcd}\]
  The bottom composite gives $\lambda^{Kc'}_{Kf}$, while by definition $\lan_{K}F(Kf)$ that makes the top triangle commute.
  Thus, $\eta$ is natural.
  It remains to show that $(\lan_{K}F, \eta)$ has the required universal property.
  For any pair $(G, \gamma)$ defining a left extension, we need to factorize $\gamma$.
  Naturality of $\gamma$ allows us to extend it to a cocone under $F\Pi^{d}$.
  \[% https://q.uiver.app/#q=WzAsNSxbMCwwLCJGXFxQaV57ZH0oYyxmKSJdLFsyLDAsIkZcXFBpXntkfShjJyxmJykiXSxbMCwyLCJHS2MiXSxbMiwyLCJHS2MnIl0sWzEsMywiR2QiXSxbMCwyLCJcXGdhbW1hX3tjfSIsMl0sWzEsMywiXFxnYW1tYV97Yyd9Il0sWzIsNCwiR2YiLDJdLFszLDQsIkdmJyJdLFswLDEsIkZcXFBpXntkfWciXSxbMiwzLCJHS2ciXV0=
\begin{tikzcd}
	{F\Pi^{d}(c,f)} && {F\Pi^{d}(c',f')} \\
	\\
	GKc && {GKc'} \\
	& Gd
	\arrow["{\gamma_{c}}"', from=1-1, to=3-1]
	\arrow["{\gamma_{c'}}", from=1-3, to=3-3]
	\arrow["Gf"', from=3-1, to=4-2]
	\arrow["{Gf'}", from=3-3, to=4-2]
	\arrow["{F\Pi^{d}g}", from=1-1, to=1-3]
	\arrow["GKg", from=3-1, to=3-3]
\end{tikzcd}\]
  Thus, the universal property of colimits yields a unique morphism $\lan_{K}F(d) \to Gd$ so that $\alpha_{d} \circ \lambda^{d}_{f} = Gf \circ \gamma_{c}$.
  We take that unique morphism to be $\alpha_{d}$.
  Clearly,
  \[
    (\alpha \circ \eta)_{c} = \alpha_{Kc} \circ \lambda^{Kc}_{\id_{Kc}} = \gamma_{c}
  \]
\end{proof}

\begin{cor}
  If $K : \iC \to \iD$ is a functor so that $\iC$ is small and $\iD$ is locally small then
  \begin{enumerate}
  \item If $\iE$ is cocomplete, then left Kan extensions $\lan_{K} \adj K^{*}$ exist and are given by the colimit formula.
  \item If $\iE$ is complete, the right Kan extensions $K^{*} \adj \ran_{K}$ exist and are given by the limit formula.
  \end{enumerate}
\end{cor}
\begin{proof}
  Smallness implies that the respective comma categories $K \dn d$ and $d \dn K$ are small.
  Cocompleteness (resp., completeness) of $\iE$ then implies that the necessary colimits (resp., limits) exist.
\end{proof}

\begin{eg}
  Directed graphs are functors from the category with two objects $E, V$ and a parallel pair of morphisms $s, t : E \toto V$ to $\mathsf{Set}$.
  A natural transformations between two such functors is a graph homomorphism.
  The forgetful functor $\mathsf{DirGraph} \to \mathsf{Set}$ that maps a graph to its set of vertices is given by restricting along the functor from the terminal category that picks out the object $V$.
  The forgetful functor admits both left and right adjoints (the free and cofree graphs, respectively).
  These adjoints can be computed as Kan extensions:
  \[% https://q.uiver.app/#q=WzAsMyxbMCwwLCIxIl0sWzIsMCwiXFxtYXRoc2Z7U2V0fSJdLFswLDIsIkUgXFx0b3RvIFYiXSxbMCwyLCJcXG1hdGhybXtyZXN9IiwyXSxbMCwxLCJHIl0sWzIsMSwiIiwyLHsic3R5bGUiOnsiYm9keSI6eyJuYW1lIjoiZGFzaGVkIn19fV1d
\begin{tikzcd}
	1 && {\mathsf{Set}} \\
	\\
	{E \toto V}
	\arrow["{\mathrm{res}}"', from=1-1, to=3-1]
	\arrow["G", from=1-1, to=1-3]
	\arrow[dashed, from=3-1, to=1-3]
\end{tikzcd}\]
  The action on $V$ is trivial since
  \[\mathrm{res} \dn V \iso 1 \iso V \dn \mathrm{res}\]
  Thus, $\lan G(V) = G = \ran G(V)$.
  For edges, $\mathrm{res} \dn E \iso 0$ since there is no morphism $f : V \to E$ and $E \dn \mathrm{res} \iso 2$ since there are exactly two morphisms $s, t : E \toto V$.
  The colimit under an empty diagram is the initial object.
  Thus, $\lan G(E) = \varnothing$.
  This means that $\lan G$ is the discrete graph with vertices $G$.
  
  The colimit under a diagram indexed by the discrete category $2$ is a product.
  Thus, $\ran G(E) = G \times G$.
  This means that $\ran G$ is the complete graph with vertices $G$.
\end{eg}

\bibliographystyle{alpha}
\bibliography{all}

\end{document}
