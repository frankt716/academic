\documentclass{amsart}
\input{decls}
\title{Complete and Cocomplete Categories}
\author{Frank Tsai}
\date{\today}
%\thanks{}
\begin{document}
\maketitle
\tableofcontents

\section{Introduction}
\label{sec:introduction}
\begin{defn}
  A category $\iC$ is \emph{complete} if it has all small limits and \emph{cocomplete} if it has all small colimits.
\end{defn}
\begin{defn}
  A functor $F : \iC \to \iD$ is \emph{continuous} if it preserves all small limits and \emph{cocontinuous} if it preserves all small colimits.
\end{defn}

\begin{thm}
  The category of sets $\mathsf{Set}$ is complete and cocomplete.
\end{thm}
\begin{proof}
  We've already shown that $\mathsf{Set}$ is complete.
  To show that $\mathsf{Set}$ is cocomplete, it suffices to show that it has coproducts and coequalizers.
  The former are given by disjoint unions.
  Given any two functions $f,g : A \toto B$, the latter is given by the quotient $B/\sim$ where $\sim$ an equivalence generated by $f(a) \sim g(a)$ for all $a \in A$.
  The idea is that the coequalizer of $f$ and $g$ consists of a set $C$ and a function $\epsilon : B \epi C$ such that $\epsilon \circ f(a) = \epsilon \circ g(a)$ for all $a \in A$.
  Thus, by identifying $f(a)$ and $g(a)$, the inclusion function $\iota : B \epi B/\sim$ satisfies the universal property of coequalizer.
\end{proof}

\begin{thm}
  The category of topological spaces $\mathsf{Top}$ is complete and cocomplete.
\end{thm}
\begin{proof}
  The forgetful functor $U : \mathsf{Top} \to \mathsf{Set}$ is represented by the singleton topology; and thus, it preserves limits.
  The underlying set of any limit in $\mathsf{Top}$ has to be a limit in $\mathsf{Set}$.
  Thus, any limit in $\mathsf{Top}$ is constructed by topologizing its underlying set.

  The product of a small family of spaces is given by equipping the Cartesian product of their underlying sets with the coarsest topology such that the projection maps are continuous.
  The equalizer of two continuous functions $f,g : A \toto B$ is given by equipping the equalizer of the underlying functions with the subspace topology, i.e., the coarsest topology making the function $\iota : E \mono A$ continuous.

  Dually, olimits in $\mathsf{Top}$ are also given by topologizing colimits in $\mathsf{Set}$.
  This must be the case by a theorem that we will show in Chapter 4.

  The coproduct of a small family of spaces is given by equipping the disjoint union of their underlying sets with the finest topology such that the injection maps are continuous.
  The coequalizers are also given the finest topology such that the quotient map $B \epi B/\sim$ is continuous.
\end{proof}

Complete and cocompleteness of $\mathsf{Set}$ and of $\mathsf{Top}$ implies the same for the category of pointed sets $\mathsf{Set}_{*}$ and the category of pointed spaces $\mathsf{Top}_{*}$ as a consequence of a general result:
\begin{thm}
  If $\iC$ is complete and cocomplete, then so are the slice categories $c/\iC$ and $\iC/c$. 
\end{thm}
\begin{proof}
  It suffices to prove the result for $c/\iC$.
  The other result follows by duality.
  The forgetful functor $\Pi : c/\iC \to \iC$ creates limits and connected colimits.
  Thus, $\iC$ has all limits created by $\Pi$, proving completeness.
  
  For cocompleteness, it suffices to show that $c/\iC$ has coproducts since coequalizers are connected colimits.
  Moreover, $c/\iC$ has all generalized pushouts since they are connected colimits as well.
  Thus, it suffices to show that $c/\iC$ has an initial object, and it does: $\id_{c}$.
\end{proof}

The categories $\mathsf{Cat}$ and $\mathsf{CAT}$ are complete and cocomplete.
The proof is deferred to a later section.

\begin{rmk}
  For any set-valued functor $F : \iC \to \mathsf{Set}$, the pullback along $U : \mathsf{Set}_{*} \to \mathsf{Set}$ in $\mathsf{CAT}$ consists of $c \in \iC$ and $(X, x) \in \mathsf{Set}_{*}$
  as objects so that $Fc = X$; morphisms are pairs $f : c \to d$ and $g : (X, x) \to (X', x')$ so that $Ff = Ug$.
  The objects of this category encode the objects in $\int F$, i.e., an object $c \in C$ and an element $x \in Fc$.
  The morphisms encode the morphisms of $\int F$, i.e., a morphism $f : c \to d$ such that $Ff(x) = Ug(x) = x'$.
  % https://q.uiver.app/#q=WzAsNCxbMCwyLCJcXGlDIl0sWzIsMiwiXFxtYXRoc2Z7U2V0fSJdLFsyLDAsIlxcbWF0aHNme1NldH1feyp9Il0sWzAsMCwiXFxpbnQgRiJdLFswLDEsIkYiLDJdLFsyLDEsIlUiXSxbMywwXSxbMywyXSxbMywxLCIiLDEseyJzdHlsZSI6eyJuYW1lIjoiY29ybmVyIn19XV0=
\[\begin{tikzcd}
	{\int F} && {\mathsf{Set}_{*}} \\
	\\
	\iC && {\mathsf{Set}}
	\arrow["F"', from=3-1, to=3-3]
	\arrow["U", from=1-3, to=3-3]
	\arrow[from=1-1, to=3-1]
	\arrow[from=1-1, to=1-3]
	\arrow["\lrcorner"{anchor=center, pos=0.125}, draw=none, from=1-1, to=3-3]
\end{tikzcd}\]
\end{rmk}

An application of limits and colimits is to define a notion of \emph{equivalence relation} internal to a general category.
In any category with finite limits, the \emph{kernel pair} of a morphism $f : X \to Y$ is the pullback of $f$ along itself: % https://q.uiver.app/#q=WzAsNCxbMCwyLCJYIl0sWzIsMiwiWSJdLFsyLDAsIlgiXSxbMCwwLCJSIl0sWzAsMSwiZiIsMl0sWzIsMSwiZiJdLFszLDAsInMiLDJdLFszLDIsInQiXSxbMywxLCIiLDEseyJzdHlsZSI6eyJuYW1lIjoiY29ybmVyIn19XV0=
\[\begin{tikzcd}
	R && X \\
	\\
	X && Y
	\arrow["f"', from=3-1, to=3-3]
	\arrow["f", from=1-3, to=3-3]
	\arrow["s"', from=1-1, to=3-1]
	\arrow["t", from=1-1, to=1-3]
	\arrow["\lrcorner"{anchor=center, pos=0.125}, draw=none, from=1-1, to=3-3]
\end{tikzcd}\]
These two maps define a morphism $(s,t) : R \to X \times X$, which can be easily shown to be a monomorphism, meaning that $R$ is always a subobject of $X \times X$.
Indeed, subobjects defined by kernel pairs are always equivalence relations.
\begin{enumerate}
\item There is a reflexivity map $\rho : X \to R$ defined by % https://q.uiver.app/#q=WzAsNSxbMiwyLCJSIl0sWzIsNCwiWCJdLFs0LDQsIlkiXSxbNCwyLCJYIl0sWzAsMCwiWCJdLFswLDEsInMiLDJdLFswLDMsInQiXSxbMywyLCJmIl0sWzEsMiwiZiIsMl0sWzAsMiwiIiwxLHsic3R5bGUiOnsibmFtZSI6ImNvcm5lciJ9fV0sWzQsMSwiXFxpZCIsMix7ImN1cnZlIjoyfV0sWzQsMywiXFxpZCIsMCx7ImN1cnZlIjotMn1dLFs0LDAsIlxccmhvIiwxLHsic3R5bGUiOnsiYm9keSI6eyJuYW1lIjoiZGFzaGVkIn19fV1d
\[\begin{tikzcd}
	X \\
	\\
	&& R && X \\
	\\
	&& X && Y
	\arrow["s"', from=3-3, to=5-3]
	\arrow["t", from=3-3, to=3-5]
	\arrow["f", from=3-5, to=5-5]
	\arrow["f"', from=5-3, to=5-5]
	\arrow["\lrcorner"{anchor=center, pos=0.125}, draw=none, from=3-3, to=5-5]
	\arrow["\id"', curve={height=12pt}, from=1-1, to=5-3]
	\arrow["\id", curve={height=-12pt}, from=1-1, to=3-5]
	\arrow["\rho"{description}, dashed, from=1-1, to=3-3]
\end{tikzcd}\]
\item There is a symmetry map $\sigma : R \to R$ defined by % https://q.uiver.app/#q=WzAsNSxbMiwyLCJSIl0sWzIsNCwiWCJdLFs0LDQsIlkiXSxbNCwyLCJYIl0sWzAsMCwiUiJdLFswLDEsInMiLDJdLFswLDMsInQiXSxbMywyLCJmIl0sWzEsMiwiZiIsMl0sWzAsMiwiIiwxLHsic3R5bGUiOnsibmFtZSI6ImNvcm5lciJ9fV0sWzQsMSwidCIsMix7ImN1cnZlIjoyfV0sWzQsMywicyIsMCx7ImN1cnZlIjotMn1dLFs0LDAsIlxcc2lnbWEiLDEseyJzdHlsZSI6eyJib2R5Ijp7Im5hbWUiOiJkYXNoZWQifX19XV0=
\[\begin{tikzcd}
	R \\
	\\
	&& R && X \\
	\\
	&& X && Y
	\arrow["s"', from=3-3, to=5-3]
	\arrow["t", from=3-3, to=3-5]
	\arrow["f", from=3-5, to=5-5]
	\arrow["f"', from=5-3, to=5-5]
	\arrow["\lrcorner"{anchor=center, pos=0.125}, draw=none, from=3-3, to=5-5]
	\arrow["t"', curve={height=12pt}, from=1-1, to=5-3]
	\arrow["s", curve={height=-12pt}, from=1-1, to=3-5]
	\arrow["\sigma"{description}, dashed, from=1-1, to=3-3]
\end{tikzcd}\]
\item There is a transitivity map $\tau : R \times_{X} R \to R$.
  The domain of this map is defined by the pullback of $s$ along $t$ % https://q.uiver.app/#q=WzAsNixbMCwyLCJSIl0sWzIsMCwiUiJdLFsyLDIsIlgiXSxbMCwwLCJSIFxcdGltZXNfe1h9IFIiXSxbMCw0LCJYIl0sWzQsMCwiWCJdLFswLDIsInQiLDJdLFsxLDIsInMiXSxbMywwLCJcXHN0aWwiLDJdLFszLDEsIlxcdHRpbCJdLFszLDIsIiIsMSx7InN0eWxlIjp7Im5hbWUiOiJjb3JuZXIifX1dLFswLDQsInMiLDJdLFsxLDUsInQiXV0=
\[\begin{tikzcd}
	{R \times_{X} R} && R && X \\
	\\
	R && X \\
	\\
	X
	\arrow["t"', from=3-1, to=3-3]
	\arrow["s", from=1-3, to=3-3]
	\arrow["\stil"', from=1-1, to=3-1]
	\arrow["\ttil", from=1-1, to=1-3]
	\arrow["\lrcorner"{anchor=center, pos=0.125}, draw=none, from=1-1, to=3-3]
	\arrow["s"', from=3-1, to=5-1]
	\arrow["t", from=1-3, to=1-5]
\end{tikzcd}\]
  And the transitivity map is given by % https://q.uiver.app/#q=WzAsNSxbMiw0LCJYIl0sWzQsMiwiWCJdLFs0LDQsIlkiXSxbMCwwLCJSIFxcdGltZXNfe1h9IFIiXSxbMiwyLCJSIl0sWzAsMiwiZiIsMl0sWzEsMiwiZiJdLFszLDAsInMgXFxjaXJjIFxcc3RpbCIsMix7ImN1cnZlIjoyfV0sWzMsMSwidCBcXGNpcmMgXFx0dGlsIiwwLHsiY3VydmUiOi0yfV0sWzQsMCwicyIsMl0sWzQsMSwidCJdLFs0LDIsIiIsMSx7InN0eWxlIjp7Im5hbWUiOiJjb3JuZXIifX1dLFszLDQsIlxcdGF1IiwxLHsic3R5bGUiOnsiYm9keSI6eyJuYW1lIjoiZGFzaGVkIn19fV1d
\[\begin{tikzcd}
	{R \times_{X} R} \\
	\\
	&& R && X \\
	\\
	&& X && Y
	\arrow["f"', from=5-3, to=5-5]
	\arrow["f", from=3-5, to=5-5]
	\arrow["{s \circ \stil}"', curve={height=12pt}, from=1-1, to=5-3]
	\arrow["{t \circ \ttil}", curve={height=-12pt}, from=1-1, to=3-5]
	\arrow["s"', from=3-3, to=5-3]
	\arrow["t", from=3-3, to=3-5]
	\arrow["\lrcorner"{anchor=center, pos=0.125}, draw=none, from=3-3, to=5-5]
	\arrow["\tau"{description}, dashed, from=1-1, to=3-3]
\end{tikzcd}\]
\end{enumerate}

In the category $\mathsf{Set}$, the pullback of a function $f : X \to Y$ along itself is the set of pairs $\{(a, b) \in X \times X \mid f(a) = f(b)\} \subseteq X \times X$.
For any $x \in X$, the reflexivity map picks out the element $(x, x) \in R$.
And for any $(x, y) \in R$, the symmetry map picks out the element $(y, x) \in R$.
Finally, transitivity calls for a pair $((a, b), (c, d)) \in R \times R$ such that $b = c$.
This is the pullback of $t$ along $s$.
Thus, the domain of the transitivity map is the pullback $R \times_{X} R$.
The transitivity map then picks out the element $(a, d) \in R$.

\begin{ex}
  Let $G$ be a group regarded as a 1-object category $\iB G$.
  Describe the colimit of a diagram $F : \iB G \to \mathsf{Set}$ in group-theoretic terms.
\end{ex}
\begin{proof}
  The colimit of such a diagram is the coequalizer of two maps % https://q.uiver.app/#q=WzAsMyxbMCwwLCJcXGNvcHJvZF97ZyBcXGluIEd9RihcXHN0YXIpIl0sWzIsMCwiRihcXHN0YXIpIl0sWzQsMCwiXFxjb2xpbSBGIl0sWzAsMSwiYyIsMix7Im9mZnNldCI6MX1dLFswLDEsImQiLDAseyJvZmZzZXQiOi0xfV0sWzEsMiwiIiwyLHsic3R5bGUiOnsiaGVhZCI6eyJuYW1lIjoiZXBpIn19fV1d
\[\begin{tikzcd}
	{\coprod_{g \in G}F(\star)} && {F(\star)} && {\colim F}
	\arrow["c"', shift right=1, from=1-1, to=1-3]
	\arrow["d", shift left=1, from=1-1, to=1-3]
	\arrow[two heads, from=1-3, to=1-5]
\end{tikzcd}\]
  In particular, it is constructed as the quotient of $F(\star)$, where the quotient is generated by $c$ and $d$.
  Filling in the detail, we see that the equivalence relation demands $g \cdot x = x$ for $x \in F(\star)$.
  Thus, $\colim F$ is the orbit space of $F$.
\end{proof}

\begin{ex}
  Prove that the category $\mathsf{DirGraph}$ of directed graphs is complete and cocomplete and explain how to construct its limits and colimits.
\end{ex}
\begin{proof}
  $\mathsf{DirGraph}$ is isomorphic to the functor category $\mathsf{Set}^{\bullet \toto \bullet}$.
  The forgetful functor $\mathsf{Set}^{\bullet \toto \bullet} \to \mathsf{Set}^{2}$ creates limits and colimits.
  Thus, $\mathsf{Set}^{\bullet \toto \bullet}$ is complete and cocomplete.
  The limits and colimits are created componentwise.
  The limit (colimit) of $F : \iJ \to \mathsf{Set}^{A \toto B}$ is a functor whose action on objects is defined to be the limits (colimits) of $F(A)$ and $F(B)$, respectively.
  Taking product as an example, every edge $e$ in the product consists of a pair of edges $e_{1}$ and $e_{2}$.
  Thus, the action on morphisms is given by the product (coproduct) maps.
\end{proof}

\bibliographystyle{alpha}
\bibliography{all}

\end{document}
