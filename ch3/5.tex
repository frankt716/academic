\documentclass{amsart}
\input{decls}
\title{Complete and Cocomplete Categories}
\author{Frank Tsai}
\date{\today}
%\thanks{}
\begin{document}
\maketitle
\tableofcontents

\section{Introduction}
\label{sec:introduction}
\begin{defn}
  A category $\iC$ is \emph{complete} if it has all small limits and \emph{cocomplete} if it has all small colimits.
\end{defn}
\begin{defn}
  A functor $F : \iC \to \iD$ is \emph{continuous} if it preserves all small limits and \emph{cocontinuous} if it preserves all small colimits.
\end{defn}

\begin{thm}
  The category of sets $\mathsf{Set}$ is complete and cocomplete.
\end{thm}
\begin{proof}
  We've already shown that $\mathsf{Set}$ is complete.
  To show that $\mathsf{Set}$ is cocomplete, it suffices to show that it has coproducts and coequalizers.
  The former are given by disjoint unions.
  Given any two functions $f,g : A \toto B$, the latter is given by the quotient $B/\sim$ where $\sim$ an equivalence generated by $f(a) \sim g(a)$ for all $a \in A$.
  The idea is that the coequalizer of $f$ and $g$ consists of a set $C$ and a function $\epsilon : B \epi C$ such that $\epsilon \circ f(a) = \epsilon \circ g(a)$ for all $a \in A$.
  Thus, by identifying $f(a)$ and $g(a)$, the inclusion function $\iota : B \epi B/\sim$ satisfies the universal property of coequalizer.
\end{proof}

\begin{thm}
  The category of topological spaces $\mathsf{Top}$ is complete and cocomplete.
\end{thm}
\begin{proof}
  The forgetful functor $U : \mathsf{Top} \to \mathsf{Set}$ is represented by the singleton topology; and thus, it preserves limits.
  The underlying set of any limit in $\mathsf{Top}$ has to be a limit in $\mathsf{Set}$.
  Thus, any limit in $\mathsf{Top}$ is constructed by topologizing its underlying set.

  The product of a small family of spaces is given by equipping the Cartesian product of their underlying sets with the coarsest topology such that the projection maps are continuous.
  The equalizer of two continuous functions $f,g : A \toto B$ is given by equipping the equalizer of the underlying functions with the subspace topology, i.e., the coarsest topology making the function $\iota : E \mono A$ continuous.

  Dually, olimits in $\mathsf{Top}$ are also given by topologizing colimits in $\mathsf{Set}$.
  This must be the case by a theorem that we will show in Chapter 4.

  The coproduct of a small family of spaces is given by equipping the disjoint union of their underlying sets with the finest topology such that the injection maps are continuous.
  The coequalizers are also given the finest topology such that the quotient map $B \epi B/\sim$ is continuous.
\end{proof}

Complete and cocompleteness of $\mathsf{Set}$ and of $\mathsf{Top}$ implies the same for the category of pointed sets $\mathsf{Set}_{*}$ and the category of pointed spaces $\mathsf{Top}_{*}$ as a consequence of a general result:
\begin{thm}
  If $\iC$ is complete and cocomplete, then so are the slice categories $c/\iC$ and $\iC/c$. 
\end{thm}
\begin{proof}
  It suffices to prove the result for $c/\iC$.
  The other result follows by duality.
  The forgetful functor $\Pi : c/\iC \to \iC$ creates limits and connected colimits.
  Thus, $\iC$ has all limits created by $\Pi$, proving completeness.
  
  For cocompleteness, it suffices to show that $c/\iC$ has coproducts since coequalizers are connected colimits.
  Moreover, $c/\iC$ has all generalized pushouts since they are connected colimits as well.
  Thus, it suffices to show that $c/\iC$ has an initial object, and it does: $\id_{c}$.
\end{proof}

\bibliographystyle{alpha}
\bibliography{all}

\end{document}
