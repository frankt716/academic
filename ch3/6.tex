\documentclass{amsart}
\input{decls}
\title{Functoriality of Limits and Colimits}
\author{Frank Tsai}
\date{\today}
%\thanks{}
\begin{document}
\maketitle
\tableofcontents

\section{Introduction}
\label{sec:introduction}
The goal of this section is to show that natural isomorphisms between diagrams induce naturally defined isomorphisms between their limits and colimits, whenever they exist.
To this end, we establish the functoriality of limits and colimits.

Note that functoriality is not the same as canonicity: given a diagram $D : \iJ \to \iC$, there may be many objects in $\iC$ that satisfy the universal property of limit.
There is seldom a reason to choose one over the other.

\begin{thm}
  If $\iC$ has all $\iJ$-shaped limits, then a choice of a limit for each diagram determines the action on objects of a functor $\lim_{\iJ} : \iC^{\iJ} \to \iC$.
\end{thm}
\begin{proof}
  The action on objects has already been defined.
  It remains to define the action on morphisms and verify functoriality.
  Given a natural transformation $\alpha : D \to E$, the composite
  \[
    \lim_{J}D \overset{\lambda}{\to} D \overset{\alpha}{\to} E
  \]
  is a cone over $E$.
  Thus, by the universal property of limit, there is a unique map $\lim_{J}D \to \lim_{J}E$ that factors through each leg of the limit cone.
  We define the action on morphisms by such unique maps.

  For $\id_{D} : D \to D$, the composite $\id_{D} \circ \lambda = \lambda$ determines a unique map $\lim_{J}D \to \lim_{J}D$ factoring through each leg of the limit cone.
  Clearly, $\id_{\lim_{J}D}: \lim_{J}D \to \lim_{J}D$ satisfies this property.
  Thus, $\lim_{J}(\id_{D}) = \id_{\lim_{J}D}$.
  
\end{proof}
\bibliographystyle{alpha}
\bibliography{all}

\end{document}
