\documentclass{amsart}
\input{decls}
\title{Interactions between Limits and Colimits}
\author{Frank Tsai}
\date{\today}
%\thanks{}
\begin{document}
\maketitle
\tableofcontents

\section{Introduction}
\label{sec:introduction}

Let $\iI$ and $\iJ$ be small categories.
A functor $F : \iI \times \iJ \to \iC$ may be regarded as either $F : \iI \to \iC^{\iJ}$ or $F : \iJ \to \iC^{\iI}$.
Suppose that $\lim_{\iJ}F(i,j)$ exists for all $i \in \iI$, then these values define a functor $\ob F : \ob I \to \iC$, which determines a diagram $\lim_{\iJ} F(\blank, j) : \iI \to \iC$.
Exchanging $\iI$ and $\iJ$ yields a diagram $\lim_{\iI} F(i, \blank) : \iJ \to \iC$.
The following theorem shows that the limit of these two functors are isomorphic.

\begin{thm}
  If $\lim_{\iI}\lim_{\iJ}F(i,j)$ and $\lim_{\iJ}\lim_{\iI}F(i,j)$ exist, then they are isomorphic and define the limit $\lim_{\iI \times \iJ}F(i,j)$.
\end{thm}
\begin{proof}
  By the Yoneda Lemma, it suffices to show that
  \[
    \iC(\blank, \lim_{\iI}\lim_{\iJ}F(i,j)) \iso \iC(\blank, \lim_{\iI \times \iJ}F(i,j)) \iso \iC(\blank, \lim_{\iJ}\lim_{\iI}F(i,j))
  \]
  Since the representable functor preserves limits,
  \[
    \iC(\blank, \lim_{\iI}\lim_{\iJ}F(i,j)) \iso \lim_{\iI}\lim_{\iJ}\iC(\blank, F(i,j))
  \]
  and
  \[
    \iC(\blank, \lim_{\iI \times \iJ}F(i,j)) \iso \lim_{\iI \times \iJ}\iC(\blank, F(i,j))
  \]
  On account of the isomorphism $\iI \times \iJ \iso \iJ \times \iI$, the problem reduces to showing that
  \[
    \lim_{\iI}\lim_{\iJ}\iC(\blank, F(i,j)) \iso \lim_{\iI \times \iJ}\iC(\blank, F(i,j))
  \]
  or in the general case of proving that
  \[
    \lim_{\iI}\lim_{\iJ}H(i,j) \iso \lim_{\iI \times \iJ}H(i,j)
  \]
  for any set valued diagram $H : \iI \times \iJ \to \mathsf{Set}$.
  The limit $\lim_{\iI \times \iJ}H(i,j)$ is the set of all cones over the $\iI \times \iJ$-indexed diagram with summit $1$, while $\lim_{\iI}\lim_{\iJ}H(i,j)$ is the set of cones with submit $1$ over the diagram $\lim_{\iJ}H(\blank,j)$.
  Such a cone consists of legs
  \[
    \left( 1 \overset{\lambda_{i}}{\to} \lim_{\iJ}H(i,j) \right)_{i \in \iI}
  \]
  By the universal property, each leg is determined by a collection
  \[
    \left( 1 \overset{\lambda_{j}}{\to} H(i,j) \right)_{j \in \iJ}
  \]
  The totality of this data gives a construction for a cone over $H$ with submit $1$.
  The cone legs are given by $\lambda_{i,j} = \pi_{i,j} \circ \lambda_{i} = \lambda_{j}$.
  Conversely, given $\lambda_{i,j}$, $\lambda_{j} : 1 \to H(i,j)$ is given by $\lambda_{j} = \lambda_{i,j}$ and $\lambda_{i}$ is given by the collection $\left( 1 \overset{\lambda_{i,j}}{\to} H(i,j) \right)_{j\in\iJ}$.
\end{proof}

Limits seldom commute with colimits.
However, there is a canonical comparison morphism.

\begin{thm}
  For any bifunctor $F : \iI \times \iJ \to \iC$ so that the displayed limits and colimits exist, there is a canonical map
  \[
    \kappa : \colim_{\iI}\lim_{\iJ}F(i,j) \to \lim_{\iJ}\colim_{\iI}F(i,j)
  \]
\end{thm}
\begin{proof}
  By the universal property of colimits, to define a map out of a colimit, it suffices to define a collection of maps
  \[
    \left( \lim_{\iJ}F(i,j) \overset{\kappa_{i}}{\to} \lim_{\iJ}\colim_{\iI}F(i,j) \right)_{i \in \iI}
  \]
  satisfying naturality.
  Similarly, each $\kappa_{i}$ is determined by a collection
  \[
    \left( \lim_{\iJ}F(i,j') \overset{\kappa_{i,j}}{\to} \colim_{\iI}F(i,j) \right)_{j \in \iJ}
  \]
  Define $\kappa_{i,j} = \iota_{i,j} \circ \pi_{i,j}$.
  These assemble into a cone, whose legs collectively define $\kappa$.
\end{proof}

\begin{eg}
  In the poset category $(\dR, \leq)$, we have the following immediate corollary:
  \begin{cor}
    For any pair of sets $X$, $Y$ and any function $f : X \times Y \to \dR$
    \[
      \sup_{X}\inf_{Y}f(x,y) \leq \inf_{Y}\sup_{X}f(x,y)
    \]
  \end{cor}
\end{eg}

\begin{eg}
  A function $x : \dN \to \dR$ defines a sequence $(x_{n})_{n \in \dN}$ of real numbers.
  Define
  \[
    \begin{array}{l}
      \liminf x_{n \to \oo} = \sup_{n}\inf_{m \geq n}x_{m} = \sup_{n}\inf_{m}x_{n+m} = \colim_{n}\lim_{m}x_{n+m}\\
      \limsup x_{n \to \oo} = \inf_{n}\sup_{m \geq n}x_{m} = \inf_{n}\sup_{m}x_{n+m} = \lim_{n}\colim_{m}x_{n+m}\\
    \end{array}
  \]
  Thus, $\liminf_{n \to \oo} x_{n} \leq \limsup_{n \to \oo} x_{n}$
\end{eg}

\bibliographystyle{alpha}
\bibliography{all}

\end{document}
