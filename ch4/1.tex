\documentclass{amsart}
\input{decls}
\title{}
\author{Frank Tsai}
\date{\today}
%\thanks{}
\begin{document}
\maketitle
\tableofcontents

\section{Introduction}
\label{sec:introduction}
\begin{defn}
  An \emph{adjunction} consists of a pair of opposing functors $F, G : \iC \toot \iD$ together with an isomorphism
  \begin{equation}\label{eq:adjunction}
    \iD(Fc, d) \iso \iC(c, Gd)
  \end{equation}
  for each $c \in \iC$ and $d \in \iD$ that is natural in both variables.
\end{defn}

A morphism $f^{\#} : Fc \to d$ corresponds to a morphism $f^{\flat} : c \to Gd$ by the isomorphism \ref{eq:adjunction}.
These morphisms are \emph{adjunct} or are \emph{transpose} of each other.

When $\iC$ and $\iD$ are locally small, the isomorphisms in \ref{eq:adjunction} assemble into a natural isomorphism between two set valued functors:
% https://q.uiver.app/#q=WzAsMixbMCwwLCJcXGlDXFxvcCBcXHRpbWVzIFxcaUQiXSxbMiwwLCJcXG1hdGhzZntTZXR9Il0sWzAsMSwiXFxpRChGXFxibGFuayxcXGJsYW5rKSIsMCx7ImN1cnZlIjotM31dLFswLDEsIlxcaUMoXFxibGFuayxHXFxibGFuaykiLDIseyJjdXJ2ZSI6M31dLFsyLDMsIlxcaXNvIiwwLHsic2hvcnRlbiI6eyJzb3VyY2UiOjIwLCJ0YXJnZXQiOjIwfX1dXQ==
\[\begin{tikzcd}
	{\iC\op \times \iD} && {\mathsf{Set}}
	\arrow[""{name=0, anchor=center, inner sep=0}, "{\iD(F\blank,\blank)}", curve={height=-18pt}, from=1-1, to=1-3]
	\arrow[""{name=1, anchor=center, inner sep=0}, "{\iC(\blank,G\blank)}"', curve={height=18pt}, from=1-1, to=1-3]
	\arrow["\iso", shorten <=5pt, shorten >=5pt, Rightarrow, from=0, to=1]
\end{tikzcd}\]

Naturality in both variables then yields
\begin{mathpar}
  % https://q.uiver.app/#q=WzAsNCxbMCwwLCJcXGlEKEZjLGQpIl0sWzIsMCwiXFxpQyhjLEdkKSJdLFswLDIsIlxcaUQoRmMsZCcpIl0sWzIsMiwiXFxpQyhjLEdkJykiXSxbMCwyLCJrX3sqfSIsMl0sWzEsMywiR2tfeyp9Il0sWzAsMSwiXFxpc28iXSxbMiwzLCJcXGlzbyIsMl1d
\begin{tikzcd}
	{\iD(Fc,d)} && {\iC(c,Gd)} \\
	\\
	{\iD(Fc,d')} && {\iC(c,Gd')}
	\arrow["{k_{*}}"', from=1-1, to=3-1]
	\arrow["{Gk_{*}}", from=1-3, to=3-3]
	\arrow["\iso", from=1-1, to=1-3]
	\arrow["\iso"', from=3-1, to=3-3]
\end{tikzcd} \and % https://q.uiver.app/#q=WzAsNCxbMCwwLCJcXGlEKEZjLGQpIl0sWzIsMCwiXFxpQyhjLEdkKSJdLFswLDIsIlxcaUQoRmMnLGQpIl0sWzIsMiwiXFxpQyhjJyxHZCkiXSxbMCwyLCJGaF57Kn0iLDJdLFsxLDMsImheeyp9Il0sWzAsMSwiXFxpc28iXSxbMiwzLCJcXGlzbyIsMl1d
\begin{tikzcd}
	{\iD(Fc,d)} && {\iC(c,Gd)} \\
	\\
	{\iD(Fc',d)} && {\iC(c',Gd)}
	\arrow["{Fh^{*}}"', from=1-1, to=3-1]
	\arrow["{h^{*}}", from=1-3, to=3-3]
	\arrow["\iso", from=1-1, to=1-3]
	\arrow["\iso"', from=3-1, to=3-3]
\end{tikzcd}
\end{mathpar}
for arbitrary $k : d \to d'$ and $h : c' \to c$.
Thus, for any $f^{\#} : Fc \to d$, one has the following:
\begin{mathpar}
  % https://q.uiver.app/#q=WzAsMyxbMCwwLCJjIl0sWzIsMCwiR2QiXSxbMiwyLCJHZCciXSxbMCwxLCJmXntcXGZsYXR9Il0sWzEsMiwiR2siXSxbMCwyLCIoa1xcY2lyYyBmXntcXCN9KV57XFxmbGF0fSIsMl1d
\begin{tikzcd}
	c && Gd \\
	\\
	&& {Gd'}
	\arrow["{f^{\flat}}", from=1-1, to=1-3]
	\arrow["Gk", from=1-3, to=3-3]
	\arrow["{(k\circ f^{\#})^{\flat}}"', from=1-1, to=3-3]
\end{tikzcd} \and % https://q.uiver.app/#q=WzAsMyxbMCwyLCJjIl0sWzIsMiwiR2QiXSxbMCwwLCJjJyJdLFswLDEsImZee1xcZmxhdH0iLDJdLFsyLDAsImgiLDJdLFsyLDEsIihmXntcXCN9IFxcY2lyYyBGaClee1xcZmxhdH0iXV0=
\begin{tikzcd}
	{c'} \\
	\\
	c && Gd
	\arrow["{f^{\flat}}"', from=3-1, to=3-3]
	\arrow["h"', from=1-1, to=3-1]
	\arrow["{(f^{\#} \circ Fh)^{\flat}}", from=1-1, to=3-3]
\end{tikzcd}
\end{mathpar}
In other words, the transpose of $k \circ f^{\#}$ is the composite $Gk \circ f^{\flat}$, and the transpose of $f^{\#} \circ Fh$ is the composite $f^{\flat} \circ h$.

This result can be assembled into the following convenient lemma.
\begin{lem}
  For a pair of adjoint functors $F, G : \iC \toot \iD$, the left-hand square commutes in $\iD$ if and only if the right-hand square commutes in $\iC$.
  \begin{mathpar}
    % https://q.uiver.app/#q=WzAsNCxbMCwwLCJGYyJdLFsyLDAsImQiXSxbMCwyLCJGYyciXSxbMiwyLCJkJyJdLFswLDIsIkZoIiwyXSxbMSwzLCJrIl0sWzAsMSwiZl57XFwjfSJdLFsyLDMsImdee1xcI30iLDJdXQ==
\begin{tikzcd}
	Fc && d \\
	\\
	{Fc'} && {d'}
	\arrow["Fh"', from=1-1, to=3-1]
	\arrow["k", from=1-3, to=3-3]
	\arrow["{f^{\#}}", from=1-1, to=1-3]
	\arrow["{g^{\#}}"', from=3-1, to=3-3]
\end{tikzcd} \and % https://q.uiver.app/#q=WzAsNCxbMCwwLCJjIl0sWzAsMiwiYyciXSxbMiwwLCJHZCJdLFsyLDIsIkdkJyJdLFswLDEsImgiLDJdLFsyLDMsIkdrIl0sWzAsMiwiZl57XFxmbGF0fSJdLFsxLDMsImdee1xcZmxhdH0iLDJdXQ==
\begin{tikzcd}
	c && Gd \\
	\\
	{c'} && {Gd'}
	\arrow["h"', from=1-1, to=3-1]
	\arrow["Gk", from=1-3, to=3-3]
	\arrow["{f^{\flat}}", from=1-1, to=1-3]
	\arrow["{g^{\flat}}"', from=3-1, to=3-3]
\end{tikzcd}
  \end{mathpar}
\end{lem}
\begin{proof}
  ($1 \implies 2$): Suppose that the left-hand square commutes, then $k \circ f^{\#} = g^{\#} \circ Fh$.
  The transpose of these morphisms are $Gk \circ f^{\flat}$ and $g^{\flat} \circ h$, respectively.
  Thus, $Gk \circ f^{\flat} = g^{\flat} \circ h$.

  The other direction is essentially the same.
  The transpose of $Gk \circ f^{\flat}$ is $k \circ f^{\#}$, and the transpose of $g^{\flat} \circ h$ is $g^{\#} \circ Fh$.
\end{proof}

\begin{lem}
  Any triple of adjoint functors
  % https://q.uiver.app/#q=WzAsMixbMCwwLCJcXGlDIl0sWzIsMCwiXFxpRCJdLFswLDEsIlUiLDFdLFsxLDAsIkwiLDIseyJjdXJ2ZSI6M31dLFsxLDAsIlIiLDAseyJjdXJ2ZSI6LTN9XSxbMiw0LCIiLDIseyJsZXZlbCI6MSwic3R5bGUiOnsibmFtZSI6ImFkanVuY3Rpb24ifX1dLFszLDIsIiIsMix7ImxldmVsIjoxLCJzdHlsZSI6eyJuYW1lIjoiYWRqdW5jdGlvbiJ9fV1d
\[\begin{tikzcd}
	\iC && \iD
	\arrow[""{name=0, anchor=center, inner sep=0}, "U"{description}, from=1-1, to=1-3]
	\arrow[""{name=1, anchor=center, inner sep=0}, "L"', curve={height=18pt}, from=1-3, to=1-1]
	\arrow[""{name=2, anchor=center, inner sep=0}, "R", curve={height=-18pt}, from=1-3, to=1-1]
	\arrow["\dashv"{anchor=center, rotate=-90}, draw=none, from=0, to=2]
	\arrow["\dashv"{anchor=center, rotate=-90}, draw=none, from=1, to=0]
\end{tikzcd}\]
  gives rise to a canonical adjunction $LU \dashv RU$.
\end{lem}
\begin{proof}
  This is given by composing the isomorphisms in the adjunctions $L \dashv U$ and $U \dashv R$.
  % https://q.uiver.app/#q=WzAsMyxbMCwwLCJMKFVhKSBcXHRvIGIiXSxbMiwwLCJVYSBcXHRvIFViIl0sWzIsMiwiYSBcXHRvIFJVYiJdLFswLDEsIkwgXFxkYXNodiBVIiwwLHsic3R5bGUiOnsidGFpbCI6eyJuYW1lIjoiYXJyb3doZWFkIn0sImJvZHkiOnsibmFtZSI6InNxdWlnZ2x5In19fV0sWzEsMiwiVSBcXGRhc2h2IFIiLDAseyJzdHlsZSI6eyJ0YWlsIjp7Im5hbWUiOiJhcnJvd2hlYWQifSwiYm9keSI6eyJuYW1lIjoic3F1aWdnbHkifX19XSxbMCwyLCIiLDIseyJzdHlsZSI6eyJ0YWlsIjp7Im5hbWUiOiJhcnJvd2hlYWQifSwiYm9keSI6eyJuYW1lIjoic3F1aWdnbHkifX19XV0=
\[\begin{tikzcd}
	{L(Ua) \to b} && {Ua \to Ub} \\
	\\
	&& {a \to RUb}
	\arrow["{L \dashv U}", squiggly, tail reversed, from=1-1, to=1-3]
	\arrow["{U \dashv R}", squiggly, tail reversed, from=1-3, to=3-3]
	\arrow[squiggly, tail reversed, from=1-1, to=3-3]
\end{tikzcd}\]
\end{proof}

\bibliographystyle{alpha}
\bibliography{all}

\end{document}
