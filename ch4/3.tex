\documentclass{amsart}
\input{decls}
\title{}
\author{Frank Tsai}
\date{\today}
%\thanks{}
\begin{document}
\maketitle
\tableofcontents

\section{Introduction}
\label{sec:introduction}

\begin{defn}
  By dualizing either $\iC$ or $\iD$ in an adjunction, one obtains
  % https://q.uiver.app/#q=WzAsNCxbMCwwLCJcXGlEKEZjLGQpIl0sWzIsMiwiXFxpQyhjLEdkKSJdLFsyLDAsIlxcaUNcXG9wKEdkLGMpIl0sWzAsMiwiXFxpRFxcb3AoZCxGYykiXSxbMCwyLCJcXGlzbyIsMSx7InN0eWxlIjp7ImhlYWQiOnsibmFtZSI6Im5vbmUifX19XSxbMywxLCJcXGlzbyIsMSx7InN0eWxlIjp7ImhlYWQiOnsibmFtZSI6Im5vbmUifX19XSxbMCwzLCJcXGlzbyIsMSx7InN0eWxlIjp7ImhlYWQiOnsibmFtZSI6Im5vbmUifX19XSxbMiwxLCJcXGlzbyIsMSx7InN0eWxlIjp7ImhlYWQiOnsibmFtZSI6Im5vbmUifX19XSxbMCwxLCJcXGlzbyIsMSx7InN0eWxlIjp7ImhlYWQiOnsibmFtZSI6Im5vbmUifX19XV0=
\[\begin{tikzcd}
	{\iD(Fc,d)} && {\iC\op(Gd,c)} \\
	\\
	{\iD\op(d,Fc)} && {\iC(c,Gd)}
	\arrow["\iso"{description}, no head, from=1-1, to=1-3]
	\arrow["\iso"{description}, no head, from=3-1, to=3-3]
	\arrow["\iso"{description}, no head, from=1-1, to=3-1]
	\arrow["\iso"{description}, no head, from=1-3, to=3-3]
	\arrow["\iso"{description}, no head, from=1-1, to=3-3]
\end{tikzcd}\]
  \begin{itemize}
  \item \emph{mutually left adjoint} when $\iD(F\op c, d) \iso \iC(G\op d, c)$ natural in both variables;
  \item \emph{mutually right adjoint} when $\iD(d, F\op c) \iso \iC(c, G\op d)$ natural in both variables. 
  \end{itemize}
\end{defn}

Mutual right adjoints between preorders are called \emph{antitone Galois connection}.

By dualizing unit or counit, we have the following result:

\begin{lem}
  A mutual left adjunction consists of a pair of contravariant functors $F : C\op \to D$ and $G : D\op \to C$ and a pair of natural transformations $\eta : GF \to \id$ and $\epsilon : FG \to \id$ such that $F\eta = \epsilon F$ and $\eta G = G\epsilon$.
  Dually, a mutual right adjunction consists of $\eta : \id \to GF$ and $\epsilon : \id \to FG$ such that $F\eta = \epsilon F$ and $\eta G = G\epsilon$.
\end{lem}

\begin{eg}
  Let $\mathrm{Axiom}_{\sigma}$ be a set of axioms, i.e., sentences in a fixed first-order language whose signature $\sigma$ specifies a list of function, constant, and relation symbols to be used with the standard logical symbols.

  Let $\mathrm{Struct}_{\sigma}$ be a set of $\sigma$-structures, i.e., sets with interpretations of the given function, constant, and relation symbols.
  For example, the language of natural numbers consists of a constant symbol ``0'', a binary function symbol ``+'', and a binary relation symbol ``$\leq$'' with some axioms.
  A structure for this language consists of a specified constant to interpret ``0'', a binary function to interpret ``+'', and a binary relation to interpret ``$\leq$''.

  Given a set of $\sigma$-structures $\cM$ and a set of axioms $A$, we write $\cM \vDash A$ if each of the axioms in $A$ is satisfied by each of the $\sigma$-structures in $\cM$.

  There are two contravariant functors $\mathrm{truein} : P(\mathrm{Axiom}_{\sigma})\op \to P(\mathrm{Struct}_{\sigma})$ and $\mathrm{satisfying} : P(\mathrm{Struct}_{\sigma})\op \to P(\mathrm{Axiom}_{\sigma})$ sending a set of axioms to a set of $\sigma$-structures, called \emph{models}, satisfying those axioms and sending a set of $\sigma$-structures to a set of axioms that they satisfy.

  These two functors are mutual right adjoints and are called the \emph{Galois connection between syntax and semantics}, i.e, $\cM \subseteq \mathrm{truein}(A)$ if and only if $\cM \vDash A$.
\end{eg}

Now, we explore adjoint functors is multiple variables.
First, we need to prove a useful theorem.

\begin{thm}
  Consider a functor $F : A \to B$ so that for each $b \in \iB$ there is an object $Gb \in \iA$ together with an isomorphism
  \[
    \iB(Fa,b) \iso \iA(a, Gb)
  \]
  natural in $a \in \iA$.
  Then there is a unique way to extend the assignment $G : \ob\iB \to \ob\iA$ to a functor $G : \iB \to \iA$ so that the family of isomorphisms is also natural in $b \in \iB$.

  In other words, if the functor
  \[
    \iB(F\blank, b) : \iA\op \to \mathsf{Set}
  \]
  admits a representation for all $b \in \iB$ then the data assembles into a right adjoint $F \dashv G$.
\end{thm}

\begin{proof}
  For $f : b \to b'$, the composite
  \[
    \iA(a, Gb) \iso \iB(Fa, b) \overset{f_{*}}{\to} \iB(Fa, b') \iso \iA(a, Gb')
  \]
  assembles into a natural transformation $\iA(\blank, Gb) \to \iA(\blank, Gb')$ since the isomorphism is natural in $a$.
  This uniquely determines the action of $G$ on morphisms by the Yoneda Lemma, yielding
  \[
    \iB(Fa, b) \iso \iA(a, Gb)
  \]
  natural in both variables.
\end{proof}

This result can be extended to functors in multiple variables.

\begin{lem}
  Suppose that $F : \iA \times \iB \to \iC$ is a bifunctor so that for each $a \in \iA$, the induced functor $F(a, \blank) : \iB \to \iC$ admits a right adjoint $G_{a} : \iC \to \iB$.
  Then, these right adjoints assemble into a unique bifunctor $G : \iA\op \times \iC \to \iB$, defined so that $G(a,c) = G_{a}c$ and so that the isomorphisms $\iC(F(a,b),c) \iso \iB(b, G(a,c))$ is natural in all three variables.
\end{lem}
\begin{proof}
  It remains to define the action on morphisms.
  To this end, let $f : (a, c) \to (a', c')$ be a morphism in $\iA\op \times \iC$, which consists of $f_{1} : a \to a'$ in $\iA\op$ and $f_{2} : c \to c'$ in $\iB$.

  The composite
  \[
    \iB(b, G(a,c)) \iso \iC(F(a,b),c) \overset{F(f_{1},b)^{*}}{\to} \iC(F(a',b),c) \iso \iB(b, G(a',c))
  \]
  assembles into $\iB(\blank, G(a,c)) \to \iB(\blank, G(a', c))$, defining $G(\blank, c) : \iA\op \to \iB$.
  Similarly, the composite
  \[
    \iB(b, G(a,c)) \iso \iC(F(a,b), c) \overset{f_{2_{*}}}{\to} \iC(F(a,b), c') \iso \iB(b, G(a,c'))
  \]
  defines $G(a, \blank) : \iC \to \iB$.

  These data defines a unique $G : \iA\op \times \iC \to \iB$.
  Naturality in $b$ and $c$ follows from the fact that $F(a, \blank) \dashv G_{a}$, while naturality in $a$ follows from the construction.
\end{proof}

\bibliographystyle{alpha}
\bibliography{all}

\end{document}
