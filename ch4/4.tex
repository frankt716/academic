\documentclass{amsart}
\input{decls}
\title{}
\author{Frank Tsai}
\date{\today}
%\thanks{}
\begin{document}
\maketitle
\tableofcontents

\section{Introduction}
\label{sec:introduction}

This section introduces a syntactic proof technique for adjoint functors, and compares this method to Yoneda-style proofs.

\begin{lem}
  If $F \dashv G$ and $F' \dashv G$, then $F \iso F'$ and there is a unique $\theta : F \iso F'$ commuting with the units and counits of the adjunction.
  \begin{mathpar}
    % https://q.uiver.app/#q=WzAsMyxbMCwwLCJcXGlkX3tcXGlDfSJdLFsyLDAsIkdGIl0sWzIsMiwiR0YnIl0sWzEsMiwiR1xcdGhldGEiXSxbMCwxLCJcXGV0YSJdLFswLDIsIlxcZXRhJyIsMl1d
\begin{tikzcd}
	{\id_{\iC}} && GF \\
	\\
	&& {GF'}
	\arrow["G\theta", from=1-3, to=3-3]
	\arrow["\eta", from=1-1, to=1-3]
	\arrow["{\eta'}"', from=1-1, to=3-3]
\end{tikzcd} \and % https://q.uiver.app/#q=WzAsNCxbMCwwLCJ4Il0sWzIsMCwieSJdLFs0LDAsInoiXSxbNCwyLCJhIl0sWzAsMSwiZiIsMCx7Im9mZnNldCI6LTF9XSxbMCwxLCJnIiwyLHsib2Zmc2V0IjoxfV0sWzEsMiwiaCJdLFsxLDAsInQiLDAseyJjdXJ2ZSI6LTN9XSxbMiwxLCJzIiwwLHsiY3VydmUiOi0zfV0sWzEsMywidSIsMix7ImN1cnZlIjozfV0sWzIsMywidXMiLDAseyJzdHlsZSI6eyJib2R5Ijp7Im5hbWUiOiJkYXNoZWQifX19XV0=
\[\begin{tikzcd}
	x && y && z \\
	\\
	&&&& a
	\arrow["f", shift left, from=1-1, to=1-3]
	\arrow["g"', shift right, from=1-1, to=1-3]
	\arrow["h", from=1-3, to=1-5]
	\arrow["t", curve={height=-18pt}, from=1-3, to=1-1]
	\arrow["s", curve={height=-18pt}, from=1-5, to=1-3]
	\arrow["u"', curve={height=18pt}, from=1-3, to=3-5]
	\arrow["us", dashed, from=1-5, to=3-5]
\end{tikzcd}\]
  \end{mathpar}
\end{lem}
\begin{proof}[Proof by formal category theory]
  To define a natural transformation $\theta : F \to F'$, we use the following fact
  % https://q.uiver.app/#q=WzAsOCxbMCwwLCJGYyJdLFswLDIsIkZjJyJdLFsyLDAsIkYnYyJdLFsyLDIsIkYnYyciXSxbNCwwLCJjIl0sWzQsMiwiYyciXSxbNiwwLCJHRidjIl0sWzYsMiwiR0YnYyciXSxbMCwyLCJcXHRoZXRhX3tjfSJdLFsxLDMsIlxcdGhldGFfe2MnfSIsMl0sWzAsMSwiRmYiLDJdLFsyLDMsIkYnZiJdLFs0LDUsImYiLDJdLFs2LDcsIkdGJ2YiXSxbNCw2LCJcXGV0YV97Y30nIl0sWzUsNywiXFxldGFfe2MnfSciLDJdLFsxMSwxMiwiXFxsZWZ0cmlnaHRzcXVpZ2Fycm93IiwxLHsic2hvcnRlbiI6eyJzb3VyY2UiOjIwLCJ0YXJnZXQiOjIwfSwic3R5bGUiOnsiYm9keSI6eyJuYW1lIjoibm9uZSJ9LCJoZWFkIjp7Im5hbWUiOiJub25lIn19fV1d
\[\begin{tikzcd}
	Fc && {F'c} && c && {GF'c} \\
	\\
	{Fc'} && {F'c'} && {c'} && {GF'c'}
	\arrow["{\theta_{c}}", from=1-1, to=1-3]
	\arrow["{\theta_{c'}}"', from=3-1, to=3-3]
	\arrow["Ff"', from=1-1, to=3-1]
	\arrow[""{name=0, anchor=center, inner sep=0}, "{F'f}", from=1-3, to=3-3]
	\arrow[""{name=1, anchor=center, inner sep=0}, "f"', from=1-5, to=3-5]
	\arrow["{GF'f}", from=1-7, to=3-7]
	\arrow["{\eta_{c}'}", from=1-5, to=1-7]
	\arrow["{\eta_{c'}'}"', from=3-5, to=3-7]
	\arrow["\leftrightsquigarrow"{description}, draw=none, from=0, to=1]
\end{tikzcd}\]
  defining $\theta$ to be the transpose of $\eta'$.
  That is,
  % https://q.uiver.app/#q=WzAsNCxbMSwwLCJGIl0sWzMsMCwiRkdGJyJdLFs1LDAsIkYnIl0sWzAsMCwiXFx0aGV0YSJdLFswLDEsIkZcXGV0YSciXSxbMSwyLCJcXGVwc2lsb24gRiciXSxbMywwLCI6PSIsMSx7InN0eWxlIjp7ImJvZHkiOnsibmFtZSI6Im5vbmUifSwiaGVhZCI6eyJuYW1lIjoibm9uZSJ9fX1dXQ==
\[\begin{tikzcd}
	\theta & F && {FGF'} && {F'}
	\arrow["{F\eta'}", from=1-2, to=1-4]
	\arrow["{\epsilon F'}", from=1-4, to=1-6]
	\arrow["{:=}"{description}, draw=none, from=1-1, to=1-2]
\end{tikzcd}\]
  Exchanging the role of $F$ and $F'$, we define $\theta' : F' \to F$ as follows
  % https://q.uiver.app/#q=WzAsNixbMCwwLCJUQSJdLFsyLDAsIlRCIl0sWzQsMCwiVEMiXSxbMCwyLCJBIl0sWzIsMiwiQiJdLFs0LDIsIkMiXSxbMyw0LCJmIiwwLHsib2Zmc2V0IjotMX1dLFszLDQsImciLDIseyJvZmZzZXQiOjF9XSxbNCw1LCJoIiwyXSxbMCwxLCJUZiIsMCx7Im9mZnNldCI6LTF9XSxbMCwxLCJUZyIsMix7Im9mZnNldCI6MX1dLFsxLDIsIlRoIl0sWzAsMywiXFxhbHBoYSIsMl0sWzEsNCwiXFxiZXRhIiwxXSxbMiw1LCJcXGdhbW1hIiwwLHsic3R5bGUiOnsiYm9keSI6eyJuYW1lIjoiZGFzaGVkIn19fV1d
\[\begin{tikzcd}
	TA && TB && TC \\
	\\
	A && B && C
	\arrow["f", shift left, from=3-1, to=3-3]
	\arrow["g"', shift right, from=3-1, to=3-3]
	\arrow["h"', from=3-3, to=3-5]
	\arrow["Tf", shift left, from=1-1, to=1-3]
	\arrow["Tg"', shift right, from=1-1, to=1-3]
	\arrow["Th", from=1-3, to=1-5]
	\arrow["\alpha"', from=1-1, to=3-1]
	\arrow["\beta"{description}, from=1-3, to=3-3]
	\arrow["\gamma", dashed, from=1-5, to=3-5]
\end{tikzcd}\]

  To show that $\theta'\theta = \id_{F}$, it suffices to show that $(\theta'\theta)^{\dag} = \eta$.
  % https://q.uiver.app/#q=WzAsNixbMCwwLCJcXGlkX3tcXGlDfSIsWzMwLDYwLDYwLDFdXSxbMCwyLCJHRiIsWzMwLDYwLDYwLDFdXSxbMiwyLCJHRkdGJyIsWzMwLDYwLDYwLDFdXSxbNCwyLCJHRiciXSxbNCw0LCJHRidHRiJdLFs0LDYsIkdGIl0sWzAsMSwiXFxldGEiLDIseyJjb2xvdXIiOlszMCw2MCw2MF19LFszMCw2MCw2MCwxXV0sWzEsMiwiR0ZcXGV0YSciLDAseyJjb2xvdXIiOlszMCw2MCw2MF19LFszMCw2MCw2MCwxXV0sWzIsMywiR1xcZXBzaWxvbiBGJyJdLFszLDQsIkdGJ1xcZXRhIl0sWzQsNSwiR1xcZXBzaWxvbicgRiJdXQ==
\[\begin{tikzcd}
	\textcolor{rgb,255:red,214;green,153;blue,92}{\id_{\iC}} \\
	\\
	\textcolor{rgb,255:red,214;green,153;blue,92}{GF} && \textcolor{rgb,255:red,214;green,153;blue,92}{GFGF'} && {GF'} \\
	\\
	&&&& {GF'GF} \\
	\\
	&&&& GF
	\arrow["\eta"', color={rgb,255:red,214;green,153;blue,92}, from=1-1, to=3-1]
	\arrow["{GF\eta'}", color={rgb,255:red,214;green,153;blue,92}, from=3-1, to=3-3]
	\arrow["{G\epsilon F'}", from=3-3, to=3-5]
	\arrow["{GF'\eta}", from=3-5, to=5-5]
	\arrow["{G\epsilon' F}", from=5-5, to=7-5]
\end{tikzcd}\]
  By naturality in $\eta$,
  % https://q.uiver.app/#q=WzAsNCxbMCwwLCJUXnsyfUMiXSxbMiwwLCJUQyJdLFswLDIsIlRDIl0sWzIsMiwiQyJdLFswLDIsIlRcXGdhbW1hIiwyXSxbMCwxLCJcXG11X3tDfSJdLFsxLDMsIlxcZ2FtbWEiXSxbMiwzLCJcXGdhbW1hIiwyXV0=
\begin{tikzcd}
	{T^{2}C} && TC \\
	\\
	TC && C
	\arrow["T\gamma"', from=1-1, to=3-1]
	\arrow["{\mu_{C}}", from=1-1, to=1-3]
	\arrow["\gamma", from=1-3, to=3-3]
	\arrow["\gamma"', from=3-1, to=3-3]
\end{tikzcd}
  By the triangle equality,
  % https://q.uiver.app/#q=WzAsNCxbMCwwLCJBIl0sWzIsMCwiQiJdLFs0LDAsIkMiXSxbNCwyLCJEIl0sWzAsMSwiZiIsMCx7Im9mZnNldCI6LTF9XSxbMCwxLCJnIiwyLHsib2Zmc2V0IjoxfV0sWzEsMiwiaCJdLFsyLDMsImoiLDAseyJzdHlsZSI6eyJib2R5Ijp7Im5hbWUiOiJkYXNoZWQifX19XSxbMSwzLCJrIiwyXV0=
\[\begin{tikzcd}
	A && B && C \\
	\\
	&&&& D
	\arrow["f", shift left, from=1-1, to=1-3]
	\arrow["g"', shift right, from=1-1, to=1-3]
	\arrow["h", from=1-3, to=1-5]
	\arrow["j", dashed, from=1-5, to=3-5]
	\arrow["k"', from=1-3, to=3-5]
\end{tikzcd}\]
  By naturality in $\eta'$,
  % https://q.uiver.app/#q=WzAsNCxbMCwwLCJcXGlkX3tcXGlDfSJdLFs0LDAsIkdGJ0dGIixbMzAsNjAsNjAsMV1dLFs2LDAsIkdGIixbMzAsNjAsNjAsMV1dLFsyLDAsIkdGIl0sWzEsMiwiR1xcZXBzaWxvbicgRiIsMCx7ImNvbG91ciI6WzMwLDYwLDYwXX0sWzMwLDYwLDYwLDFdXSxbMywxLCJcXGV0YScgR0YiLDAseyJjb2xvdXIiOlszMCw2MCw2MF19LFszMCw2MCw2MCwxXV0sWzAsMywiXFxldGEiXV0=
\[\begin{tikzcd}
	{\id_{\iC}} && GF && \textcolor{rgb,255:red,214;green,153;blue,92}{GF'GF} && \textcolor{rgb,255:red,214;green,153;blue,92}{GF}
	\arrow["{G\epsilon' F}", color={rgb,255:red,214;green,153;blue,92}, from=1-5, to=1-7]
	\arrow["{\eta' GF}", color={rgb,255:red,214;green,153;blue,92}, from=1-3, to=1-5]
	\arrow["\eta", from=1-1, to=1-3]
\end{tikzcd}\]
  Finally, the triangle equality yields
  % https://q.uiver.app/#q=WzAsMixbMCwwLCJcXGlkX3tcXGlDfSJdLFsyLDAsIkdGIl0sWzAsMSwiXFxldGEiXV0=
\[\begin{tikzcd}
	{\id_{\iC}} && GF
	\arrow["\eta", from=1-1, to=1-3]
\end{tikzcd}\]

  For the converse direction, we need to show that $\theta\theta' = \id_{F'}$.
  It is sufficient to show that $(\theta\theta')^{\dag} = \eta'$.
  % https://q.uiver.app/#q=WzAsNixbMCwwLCJcXGlkX3tcXGlDfSIsWzMwLDYwLDYwLDFdXSxbMCwyLCJHRiciLFszMCw2MCw2MCwxXV0sWzIsMiwiR0YnR0YiLFszMCw2MCw2MCwxXV0sWzQsMiwiR0YiXSxbNCw0LCJHRkdGJyJdLFs0LDYsIkdGJyJdLFswLDEsIlxcZXRhJyIsMix7ImNvbG91ciI6WzMwLDYwLDYwXX0sWzMwLDYwLDYwLDFdXSxbMSwyLCJHRidcXGV0YSIsMCx7ImNvbG91ciI6WzMwLDYwLDYwXX0sWzMwLDYwLDYwLDFdXSxbMiwzLCJHXFxlcHNpbG9uJyBGIl0sWzMsNCwiR0ZcXGV0YSciXSxbNCw1LCJHXFxlcHNpbG9uIEYnIl1d
\[\begin{tikzcd}
	\textcolor{rgb,255:red,214;green,153;blue,92}{\id_{\iC}} \\
	\\
	\textcolor{rgb,255:red,214;green,153;blue,92}{GF'} && \textcolor{rgb,255:red,214;green,153;blue,92}{GF'GF} && GF \\
	\\
	&&&& {GFGF'} \\
	\\
	&&&& {GF'}
	\arrow["{\eta'}"', color={rgb,255:red,214;green,153;blue,92}, from=1-1, to=3-1]
	\arrow["{GF'\eta}", color={rgb,255:red,214;green,153;blue,92}, from=3-1, to=3-3]
	\arrow["{G\epsilon' F}", from=3-3, to=3-5]
	\arrow["{GF\eta'}", from=3-5, to=5-5]
	\arrow["{G\epsilon F'}", from=5-5, to=7-5]
\end{tikzcd}\]
  By naturality in $\eta'$,
  % https://q.uiver.app/#q=WzAsMyxbMCwwLCJGVUZVRCJdLFsyLDAsIkZVRCJdLFs0LDAsIkQiXSxbMCwxLCJGVVxcZXBzaWxvbl97RH0iLDAseyJvZmZzZXQiOi0xfV0sWzAsMSwiXFxlcHNpbG9uX3tGVUR9IiwyLHsib2Zmc2V0IjoxfV0sWzEsMiwiXFxlcHNpbG9uX3tEfSJdXQ==
\[\begin{tikzcd}
	FUFUD && FUD && D
	\arrow["{FU\epsilon_{D}}", shift left, from=1-1, to=1-3]
	\arrow["{\epsilon_{FUD}}"', shift right, from=1-1, to=1-3]
	\arrow["{\epsilon_{D}}", from=1-3, to=1-5]
\end{tikzcd}\]
  By the triangle equality,
  % https://q.uiver.app/#q=WzAsNCxbMCwwLCJcXGlkX3tcXGlDfSIsWzMwLDYwLDYwLDFdXSxbMiwwLCJHRiIsWzMwLDYwLDYwLDFdXSxbNCwwLCJHRkdGJyIsWzMwLDYwLDYwLDFdXSxbNiwwLCJHRiciXSxbMSwyLCJHRlxcZXRhJyIsMCx7ImNvbG91ciI6WzMwLDYwLDYwXX0sWzMwLDYwLDYwLDFdXSxbMiwzLCJHXFxlcHNpbG9uIEYnIl0sWzAsMSwiXFxldGEiLDAseyJjb2xvdXIiOlszMCw2MCw2MF19LFszMCw2MCw2MCwxXV1d

\begin{equation}\label{eqn:eta-to-eta'}
  \begin{tikzcd}
    \textcolor{rgb,255:red,214;green,153;blue,92}{\id_{\iC}} && \textcolor{rgb,255:red,214;green,153;blue,92}{GF} && \textcolor{rgb,255:red,214;green,153;blue,92}{GFGF'} && {GF'}
    \arrow["{GF\eta'}", color={rgb,255:red,214;green,153;blue,92}, from=1-3, to=1-5]
    \arrow["{G\epsilon F'}", from=1-5, to=1-7]
    \arrow["\eta", color={rgb,255:red,214;green,153;blue,92}, from=1-1, to=1-3]
  \end{tikzcd}
\end{equation}

  By naturality in $\eta$,
  % https://q.uiver.app/#q=WzAsMyxbMCwwLCJVRlVGVUQiXSxbMiwwLCJVRlVEIl0sWzQsMCwiVUQiXSxbMCwxLCJVRlVcXGVwc2lsb25fe0R9IiwwLHsib2Zmc2V0IjotMX1dLFswLDEsIlVcXGVwc2lsb25fe0ZVRH0iLDIseyJvZmZzZXQiOjF9XSxbMSwyLCJVXFxlcHNpbG9uX3tEfSJdLFsyLDEsIlxcZXRhX3tVRH0iLDAseyJjdXJ2ZSI6LTN9XSxbMSwwLCJcXGV0YV97VUZVRH0iLDAseyJjdXJ2ZSI6LTN9XV0=
\[\begin{tikzcd}
	UFUFUD && UFUD && UD
	\arrow["{UFU\epsilon_{D}}", shift left, from=1-1, to=1-3]
	\arrow["{U\epsilon_{FUD}}"', shift right, from=1-1, to=1-3]
	\arrow["{U\epsilon_{D}}", from=1-3, to=1-5]
	\arrow["{\eta_{UD}}", curve={height=-18pt}, from=1-5, to=1-3]
	\arrow["{\eta_{UFUD}}", curve={height=-18pt}, from=1-3, to=1-1]
\end{tikzcd}\]
  Finally, the triangle equality yields
  \input{img/4-15}
  Note that the fact that the diagram
  % https://q.uiver.app/#q=WzAsNCxbMCwwLCJcXGlkX3tcXGlDfSIsWzMwLDYwLDYwLDFdXSxbMiwwLCJHRiIsWzMwLDYwLDYwLDFdXSxbNCwwLCJHRkdGJyIsWzMwLDYwLDYwLDFdXSxbNiwwLCJHRiciXSxbMSwyLCJHRlxcZXRhJyIsMCx7ImNvbG91ciI6WzMwLDYwLDYwXX0sWzMwLDYwLDYwLDFdXSxbMiwzLCJHXFxlcHNpbG9uIEYnIl0sWzAsMSwiXFxldGEiLDAseyJjb2xvdXIiOlszMCw2MCw2MF19LFszMCw2MCw2MCwxXV1d

\begin{equation}\label{eqn:eta-to-eta'}
  \begin{tikzcd}
    \textcolor{rgb,255:red,214;green,153;blue,92}{\id_{\iC}} && \textcolor{rgb,255:red,214;green,153;blue,92}{GF} && \textcolor{rgb,255:red,214;green,153;blue,92}{GFGF'} && {GF'}
    \arrow["{GF\eta'}", color={rgb,255:red,214;green,153;blue,92}, from=1-3, to=1-5]
    \arrow["{G\epsilon F'}", from=1-5, to=1-7]
    \arrow["\eta", color={rgb,255:red,214;green,153;blue,92}, from=1-1, to=1-3]
  \end{tikzcd}
\end{equation}

  reduces to $\eta$ means that the left-hand diagram in the statement commute.
  Furthermore, the left-hand diagram says that the transpose of $\theta$ along $F \dashv G$ is $\eta'$, which is precisely the definition of $\theta$, proving uniqueness.
  It remains to verify the right-hand diagram commutes.
  % https://q.uiver.app/#q=WzAsNCxbMCwwLCJGRyJdLFsyLDAsIkZHRidHIixbMzAsNjAsNjAsMV1dLFs0LDAsIkYnRyIsWzMwLDYwLDYwLDFdXSxbNiwwLCJcXGlkX3tcXGlEfSIsWzMwLDYwLDYwLDFdXSxbMCwxLCJGXFxldGEnIEciXSxbMSwyLCJcXGVwc2lsb24gRidHIiwwLHsiY29sb3VyIjpbMzAsNjAsNjBdfSxbMzAsNjAsNjAsMV1dLFsyLDMsIlxcZXBzaWxvbiciLDAseyJjb2xvdXIiOlszMCw2MCw2MF19LFszMCw2MCw2MCwxXV1d
\[\begin{tikzcd}
	FG && \textcolor{rgb,255:red,214;green,153;blue,92}{FGF'G} && \textcolor{rgb,255:red,214;green,153;blue,92}{F'G} && \textcolor{rgb,255:red,214;green,153;blue,92}{\id_{\iD}}
	\arrow["{F\eta' G}", from=1-1, to=1-3]
	\arrow["{\epsilon F'G}", color={rgb,255:red,214;green,153;blue,92}, from=1-3, to=1-5]
	\arrow["{\epsilon'}", color={rgb,255:red,214;green,153;blue,92}, from=1-5, to=1-7]
\end{tikzcd}\]
  By naturality in $\epsilon'$,
  % https://q.uiver.app/#q=WzAsNCxbMCwwLCJGRyIsWzMwLDYwLDYwLDFdXSxbMiwwLCJGR0YnRyIsWzMwLDYwLDYwLDFdXSxbNCwwLCJGJ0ciLFszMCw2MCw2MCwxXV0sWzYsMCwiXFxpZF97XFxpRH0iXSxbMCwxLCJGXFxldGEnIEciLDAseyJjb2xvdXIiOlszMCw2MCw2MF19LFszMCw2MCw2MCwxXV0sWzEsMiwiRkdcXGVwc2lsb24nIiwwLHsiY29sb3VyIjpbMzAsNjAsNjBdfSxbMzAsNjAsNjAsMV1dLFsyLDMsIlxcZXBzaWxvbiJdXQ==
\[\begin{tikzcd}
	\textcolor{rgb,255:red,214;green,153;blue,92}{FG} && \textcolor{rgb,255:red,214;green,153;blue,92}{FGF'G} && \textcolor{rgb,255:red,214;green,153;blue,92}{F'G} && {\id_{\iD}}
	\arrow["{F\eta' G}", color={rgb,255:red,214;green,153;blue,92}, from=1-1, to=1-3]
	\arrow["{FG\epsilon'}", color={rgb,255:red,214;green,153;blue,92}, from=1-3, to=1-5]
	\arrow["\epsilon", from=1-5, to=1-7]
\end{tikzcd}\]
  And by the triangle equality,
  % https://q.uiver.app/#q=WzAsMixbMCwwLCJGJ0ciXSxbMiwwLCJcXGlkX3tcXGlEfSJdLFswLDEsIlxcZXBzaWxvbiJdXQ==
\[\begin{tikzcd}
	{F'G} && {\id_{\iD}}
	\arrow["\epsilon", from=1-1, to=1-3]
\end{tikzcd}\]
\end{proof}

\begin{proof}[Proof by Yoneda Lemma]
  By hypothesis,
  \[
    \iD(F'c,d) \iso \iC(c,Gd) \iso \iD(Fc,d)
  \]
  Thus, $\iD(F'\blank,\blank) \iso \iD(F\blank,\blank)$.
  By the Yoneda Lemma, this isomorphism corresponds uniquely to an isomorphism $\theta : F \iso F'$ whose component $\theta_{c}$ is defined by the image of $1_{F'c}$ under the isomorphism.
  The isomorphism carries $1_{F'c}$ to $\eta'_{c}$, then to the transpose of $\eta'_{c}$ along $F \dashv G$, yielding $\theta_{c} = \epsilon F'c \circ F\eta'_{c}$.
\end{proof}

\begin{lem}
  Given $F \dashv G$ and $F' \dashv G'$, $F'F \dashv GG'$.
  % https://q.uiver.app/#q=WzAsNSxbMCwwLCJcXGlDIl0sWzIsMCwiXFxpRCJdLFs0LDAsIlxcaUUiXSxbNiwwLCJcXGlDIl0sWzgsMCwiXFxpRSJdLFswLDEsIkYiLDAseyJvZmZzZXQiOi0yfV0sWzEsMCwiRyIsMCx7Im9mZnNldCI6LTJ9XSxbMSwyLCJGJyIsMCx7Im9mZnNldCI6LTJ9XSxbMiwxLCJHJyIsMCx7Im9mZnNldCI6LTJ9XSxbMyw0LCJGJ0YiLDAseyJvZmZzZXQiOi0yfV0sWzQsMywiR0cnIiwwLHsib2Zmc2V0IjotMn1dLFsyLDMsIlxccmlnaHRzcXVpZ2Fycm93IiwxLHsic3R5bGUiOnsiYm9keSI6eyJuYW1lIjoibm9uZSJ9LCJoZWFkIjp7Im5hbWUiOiJub25lIn19fV0sWzUsNiwiIiwwLHsibGV2ZWwiOjEsInN0eWxlIjp7Im5hbWUiOiJhZGp1bmN0aW9uIn19XSxbNyw4LCIiLDAseyJsZXZlbCI6MSwic3R5bGUiOnsibmFtZSI6ImFkanVuY3Rpb24ifX1dLFs5LDEwLCIiLDAseyJsZXZlbCI6MSwic3R5bGUiOnsibmFtZSI6ImFkanVuY3Rpb24ifX1dXQ==
\[\begin{tikzcd}
	\iC && \iD && \iE && \iC && \iE
	\arrow[""{name=0, anchor=center, inner sep=0}, "F", shift left=2, from=1-1, to=1-3]
	\arrow[""{name=1, anchor=center, inner sep=0}, "G", shift left=2, from=1-3, to=1-1]
	\arrow[""{name=2, anchor=center, inner sep=0}, "{F'}", shift left=2, from=1-3, to=1-5]
	\arrow[""{name=3, anchor=center, inner sep=0}, "{G'}", shift left=2, from=1-5, to=1-3]
	\arrow[""{name=4, anchor=center, inner sep=0}, "{F'F}", shift left=2, from=1-7, to=1-9]
	\arrow[""{name=5, anchor=center, inner sep=0}, "{GG'}", shift left=2, from=1-9, to=1-7]
	\arrow["\rightsquigarrow"{description}, draw=none, from=1-5, to=1-7]
	\arrow["\dashv"{anchor=center, rotate=-90}, draw=none, from=0, to=1]
	\arrow["\dashv"{anchor=center, rotate=-90}, draw=none, from=2, to=3]
	\arrow["\dashv"{anchor=center, rotate=-90}, draw=none, from=4, to=5]
\end{tikzcd}\]
\end{lem}
\begin{proof}
  \[
    \iE(F'Fc,e) \iso \iD(Fc,G'e) \iso \iC(c,GG'e)
  \]
  The first isomorphism is given by $F' \dashv G'$ and the second isomorphism is given by $F \dashv G$.
\end{proof}

\begin{lem}
  Any equivalence
  \begin{mathpar}
    % https://q.uiver.app/#q=WzAsMixbMCwwLCJcXGlDIl0sWzIsMCwiXFxpRCJdLFswLDEsIkYiLDAseyJvZmZzZXQiOi0yfV0sWzEsMCwiRyIsMCx7Im9mZnNldCI6LTJ9XV0=
\begin{tikzcd}
	\iC && \iD
	\arrow["F", shift left=2, from=1-1, to=1-3]
	\arrow["G", shift left=2, from=1-3, to=1-1]
\end{tikzcd} \and \eta : \id_{\iC} \iso GF \and \epsilon : FG \iso \id_{\iD}
  \end{mathpar}
  can be promoted to an adjoint equivalence.
\end{lem}
\begin{proof}
  The composite
  % https://q.uiver.app/#q=WzAsNCxbMSwwLCJHIl0sWzMsMCwiR0ZHIl0sWzMsMiwiRyJdLFswLDAsIlxcZ2FtbWEiXSxbMCwxLCJcXGV0YSBHIl0sWzEsMiwiR1xcZXBzaWxvbiJdLFsyLDAsIlxcZ2FtbWFcXGludiJdLFszLDAsIjo9IiwxLHsic3R5bGUiOnsiYm9keSI6eyJuYW1lIjoibm9uZSJ9LCJoZWFkIjp7Im5hbWUiOiJub25lIn19fV1d
\[\begin{tikzcd}
	\gamma & G && GFG \\
	\\
	&&& G
	\arrow["{\eta G}", from=1-2, to=1-4]
	\arrow["G\epsilon", from=1-4, to=3-4]
	\arrow["\gamma\inv", from=3-4, to=1-2]
	\arrow["{:=}"{description}, draw=none, from=1-1, to=1-2]
\end{tikzcd}\]
  is an isomorphism but $\gamma\inv$ is not necessarily the identity.
  This can be remedied by adjusting $\eta$ or $\epsilon$.
  Naturality in $\eta$ suggests that the missing arrow should be $GF\gamma\inv$.
  % https://q.uiver.app/#q=WzAsNSxbMCwwLCJHIl0sWzIsMCwiR0ZHIl0sWzIsMiwiRyJdLFsyLDEsIkdGRyJdLFsxLDEsIkciXSxbMCwxLCJcXGV0YSBHIl0sWzMsMiwiR1xcZXBzaWxvbiJdLFsxLDMsIkdGXFxnYW1tYVxcaW52IiwwLHsic3R5bGUiOnsiYm9keSI6eyJuYW1lIjoiZGFzaGVkIn19fV0sWzAsNCwiXFxnYW1tYVxcaW52IiwyXSxbNCwyLCJcXGdhbW1hIiwyXSxbNCwzLCJcXGV0YSBHIl1d
\[\begin{tikzcd}
	G && GFG \\
	& G & GFG \\
	&& G
	\arrow["{\eta G}", from=1-1, to=1-3]
	\arrow["G\epsilon", from=2-3, to=3-3]
	\arrow["GF\gamma\inv", dashed, from=1-3, to=2-3]
	\arrow["\gamma\inv"', from=1-1, to=2-2]
	\arrow["\gamma"', from=2-2, to=3-3]
	\arrow["{\eta G}", from=2-2, to=2-3]
\end{tikzcd}\]
  Define $\epsilon' = \epsilon \circ F\gamma\inv$.
  For the other triangle identity, we need to show that $\epsilon'F \circ F\eta = \id_{F}$.
  This is tricky to show directly.
  Instead, we can show that it's an idempotent.
  Then we can use the fact that every idempotent isomorphism is the identity to conclude what we want.

  Note that the following diagram commutes
  % https://q.uiver.app/#q=WzAsOCxbMCwwLCJGIl0sWzIsMCwiRkdGIl0sWzQsMCwiRiJdLFs0LDIsIkZHRiJdLFs0LDQsIkYiXSxbMiwyLCJGR0ZHRiJdLFsyLDQsIkZHRiJdLFswLDIsIkZHRiJdLFswLDEsIkZcXGV0YSJdLFsxLDIsIlxcZXBzaWxvbidGIl0sWzIsMywiRlxcZXRhIl0sWzMsNCwiXFxlcHNpbG9uJ0YiXSxbMSw1LCJGR0ZcXGV0YSIsMl0sWzUsMywiXFxlcHNpbG9uJ0ZHRiJdLFs2LDQsIlxcZXBzaWxvbidGIiwyXSxbNSw2LCJcXGVwc2lsb24nRkdGIl0sWzAsNywiRlxcZXRhIiwyXSxbNyw1LCJGXFxldGEgR0YiXSxbNyw2LCIiLDEseyJsZXZlbCI6Miwic3R5bGUiOnsiaGVhZCI6eyJuYW1lIjoibm9uZSJ9fX1dXQ==
\[\begin{tikzcd}
	F && FGF && F \\
	\\
	FGF && FGFGF && FGF \\
	\\
	&& FGF && F
	\arrow["F\eta", from=1-1, to=1-3]
	\arrow["{\epsilon'F}", from=1-3, to=1-5]
	\arrow["F\eta", from=1-5, to=3-5]
	\arrow["{\epsilon'F}", from=3-5, to=5-5]
	\arrow["FGF\eta"', from=1-3, to=3-3]
	\arrow["{\epsilon'FGF}", from=3-3, to=3-5]
	\arrow["{\epsilon'F}"', from=5-3, to=5-5]
	\arrow["{\epsilon'FGF}", from=3-3, to=5-3]
	\arrow["F\eta"', from=1-1, to=3-1]
	\arrow["{F\eta GF}", from=3-1, to=3-3]
	\arrow[Rightarrow, no head, from=3-1, to=5-3]
\end{tikzcd}\]
  proving that $\epsilon'F \circ F\eta$ is an idempotent.
\end{proof}

\bibliographystyle{alpha}
\bibliography{all}

\end{document}
