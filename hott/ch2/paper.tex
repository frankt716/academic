\documentclass{amsart}
\input{decls}
\title{Homotopy Type Theory}
\author{Frank Tsai}
\date{\today}
%\thanks{}
\begin{document}
\maketitle
\tableofcontents

\section{Homotopies and Equivalences}
\label{sec:homotopies-and-equivalences}
\begin{defn}\label{def:quasi-inverse}
  For a function $f : A \to B$, a \emph{quasi-inverse} of $f$ is a triple $(g, \alpha, \beta)$ consisting of a function $g : B \to A$ and homotopies $\alpha : f \circ g \sim \id_{B}$ and $\beta : g \circ f \sim \id_{A}$.
  The type of quasi-inverse of $f$ is defined as
  \[
    \qinv(f) := \sum_{g : B \to A}(f \circ g \sim \id_{B}) \times (g \circ f \sim \id_{A})
  \]
\end{defn}

\begin{lem}
  For any $p : x =_{A} y$ and any family $P : A \to \cU$, the function
  \[
    \transport^{P}(p,\blank) : P(x) \to P(y)
  \]
  is a quasi-inverse.
\end{lem}
\begin{proof}
  Take $\transport^{P}(p\inv,\blank) : P(y) \to P(x)$.
  It remains to provide the homotopies.
  Both homotopies follow immediately by path induction on $p$.
\end{proof}

\cref{def:quasi-inverse} is not the right notion of equivalence between types because it has nontrivial high dimensional structures.
The right notion of inverse should have the following properties:
\begin{enumerate}
\item for any $f : A \to B$, there is a function $\qinv(f) \to \isequiv(f)$;
\item and there is also a function $\isequiv(f) \to \qinv(f)$;
\item moreover, $\isequiv(f)$ is a proposition.
\end{enumerate}

\begin{defn}
  Let $f : A \to B$ be a function.
  The type of equivalence $\isequiv(f)$ is defined as
  \[
    \isequiv(f) := \left(\sum_{g : B \to A}f \circ g \sim \id_{B}\right) \times \left(\sum_{h : B \to A}h \circ f \sim \id_{A}\right)
  \]
\end{defn}

\begin{lem}
  $\isequiv(f)$ satisfies all three conditions.
\end{lem}
\begin{proof}
  Given $(g,\alpha,\beta) : \qinv(f)$, we have $((g,\alpha),(g,\beta)) : \isequiv(f)$.
  This defines the sought after function.
  Similarly, given $((g,\alpha),(h,\beta)) : \isequiv(f)$, define $\gamma : g \sim h$ by
  \[
    g \sim g \circ f \circ h \sim h
  \]
  and $\beta' : g \circ f \sim \id_{A}$ by $\beta \circ \gamma(\beta) : g \circ f \sim h \circ f \sim \id_{A}$.
  Then $(g,\alpha,\beta') : \qinv(f)$.

  Condition (iii) is deferred to a later section.
\end{proof}

\begin{defn}
  Given two types $A, B : \cU$, the type of equivalences from $A$ to $B$ is defined as
  \[
    (A \eqv B) := \sum_{f : A \to B}\isequiv(f)
  \]
\end{defn}
Unsurprisingly, $\eqv$ is an equivalence relation.

\section{Cartesian Products}
\label{sec:cartesian-products}

Given $p : x =_{A \times B} y$, functoriality of $\pi_{1}$ and $\pi_{2}$ yields two paths of type $\pi_{1}(x) =_{A} \pi_{1}(y)$ and $\pi_{2}(x) =_{B} \pi_{2}(y)$.
Thus, there is a function
\begin{equation}\label{eqn:cartesian-product-paths}
  (x =_{A \times B} y) \to (\pi_{1}(x) =_{A} \pi_{1}(y)) \times (\pi_{2}(x) =_{B} \pi_{2}(y))
\end{equation}

\begin{lem}
  The function in \eqref{eqn:cartesian-product-paths} is an equivalence.
\end{lem}
\begin{proof}
  It suffices to show that \eqref{eqn:cartesian-product-paths} is a quasi-inverse.
  We need to find a function
  \[
    (\pi_{1}(x) =_{A} \pi_{1}(y)) \times (\pi_{2}(x) =_{B} \pi_{2}(y)) \to (x =_{A \times B} y)
  \]
  Given $p : \pi_{1}(x) =_{A} \pi_{1}(y)$ and $q : \pi_{2}(x) =_{B} \pi_{2}(y)$, by path induction, we may assume $p$ is $\refl_{\pi_{1}(x)}$ and $q$ is $\refl_{\pi_{2}(x)}$.
  Then the result follows by $\refl$.

  The two homotopies are also easy:
  \begin{enumerate}
  \item $\alpha : \refl_{x} \mapsto (\refl_{\pi_{1}(x)},\refl_{\pi_{2}(x)}) \mapsto \refl_{x}$;
  \item $\beta : (\refl_{\pi_{1}(x)},\refl_{\pi_{2}(x)}) \mapsto \refl_{x} \mapsto (\refl_{\pi_{1}(x)},\refl_{\pi_{2}(x)})$.
  \end{enumerate}
\end{proof}

\bibliographystyle{alpha}
\bibliography{all}

\end{document}
