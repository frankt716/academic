\documentclass[t]{beamer}
\newif\ifbeamer\beamertrue
\input{decls}
\tikzcdset{diagrams={ampersand replacement=\&}}

% \usetheme{Warsaw}
% \setbeamertemplate{headline}{}

\usetheme[height=0cm]{Rochester}
\useinnertheme{circles}
% \useoutertheme[compress,subsection=false]{miniframes}
\usecolortheme{rose}
\setbeamertemplate{navigation symbols}{}
\setbeamertemplate{frametitle continuation}{}

\title{Abstract nonsense}
\author[Frank T]{Frank Tsai\inst{1}}
\institute{\inst{1} (SUNY at Buffalo)}
\date{\today}

\AtBeginSection[]
{
  \begin{frame}<beamer>{Outline}
    \tableofcontents[currentsection]
  \end{frame}
}

\setbeamercolor{emph}{fg=blue}
\renewcommand<>{\emph}[1]{%
  {\usebeamercolor[fg]{emph}\only#2{\itshape}#1}%
}

\renewcommand<>{\bf}[1]{%
  {\usebeamercolor[fg]{emph}\only#2{\bfseries}#1}%
}

\begin{document}

\begin{frame}
  \maketitle
\end{frame}

\begin{frame}{Motivation}
  Lots of mathematical theories capturing various things
  \begin{columns}[t]
    \begin{column}<1->{5cm}
      \begin{itemize}
      \item[] Topology
      \item Spaces
      \item Continuous functions
      \item<2-> Product topology, quotient topology, etc
      \end{itemize}
    \end{column}
    \begin{column}<3->{5cm}
      \begin{itemize}
      \item[] Group theory
      \item Symmetries of objects
      \item Symmetry preserving functions
      \item<4-> Product groups, quotient groups, etc
      \end{itemize}
    \end{column}
  \end{columns}
  \begin{enumerate}
  \item<5-> A topological space equipped with a group structure?
  \item<6-> Translate a topological problem into a problem about groups (e.g., the fundamental group of a space)?
  \item<7-> A unified framework for various constructions (e.g., products, quotients, etc)?
  \end{enumerate}
\end{frame}

\section{Categories}
\label{sec:categories}

\begin{frame}{Categories}
  A \emph{category} consists of...\\\vspace{1em}
  \begin{onlyenv}<1>
    A collection of \emph{objects}.
    % https://q.uiver.app/#q=WzAsNCxbMCwwLCJcXGJ1bGxldCJdLFswLDIsIlxcYnVsbGV0Il0sWzIsMiwiXFxidWxsZXQiXSxbMiwwLCJcXGJ1bGxldCJdXQ==
    \[\begin{tikzcd}
	\bullet \&\& \bullet \\
	\\
	\bullet \&\& \bullet
      \end{tikzcd}\]
  \end{onlyenv}
  \begin{onlyenv}<2>
    A collection of \emph{morphisms}.
    % https://q.uiver.app/#q=WzAsNyxbMiwwLCJcXGJ1bGxldCJdLFswLDIsIlxcYnVsbGV0Il0sWzMsMywiXFxidWxsZXQiXSxbNCwxLCJcXGJ1bGxldCJdLFs2LDIsIlxcYnVsbGV0Il0sWzUsMCwiXFxidWxsZXQiXSxbMiwyLCJcXGJ1bGxldCJdLFsxLDBdLFswLDNdLFsxLDNdLFs2LDIsIiIsMCx7ImN1cnZlIjotMn1dLFs2LDIsIiIsMix7ImN1cnZlIjoyfV1d
    % https://q.uiver.app/#q=WzAsNCxbMCwwLCJcXGJ1bGxldCJdLFswLDIsIlxcYnVsbGV0Il0sWzIsMiwiXFxidWxsZXQiXSxbMiwwLCJcXGJ1bGxldCJdLFswLDMsIiIsMCx7ImN1cnZlIjotMX1dLFswLDMsIiIsMix7ImN1cnZlIjoxfV0sWzMsMl0sWzAsMl1d
    \[\begin{tikzcd}
	\bullet \&\& \bullet \\
	\\
	\bullet \&\& \bullet
	\arrow[curve={height=-6pt}, from=1-1, to=1-3]
	\arrow[curve={height=6pt}, from=1-1, to=1-3]
	\arrow[from=1-3, to=3-3]
	\arrow[from=1-1, to=3-3]
        \arrow[loop above, from=1-1, to=1-1]
        \arrow[loop left, from=1-1, to=1-1]
        \arrow[loop above, from=1-3, to=1-3]
        \arrow[loop below, from=3-1, to=3-1]
        \arrow[loop below, from=3-3, to=3-3]
      \end{tikzcd}\]
  \end{onlyenv}
  \begin{onlyenv}<3>
    A specified \emph{identity} morphism for each object.
    \[\begin{tikzcd}
	\bullet \&\& \bullet \\
	\\
	\bullet \&\& \bullet
	\arrow[curve={height=-6pt}, from=1-1, to=1-3]
	\arrow[curve={height=6pt}, from=1-1, to=1-3]
	\arrow[from=1-3, to=3-3]
	\arrow[from=1-1, to=3-3]
        \arrow[loop above, from=1-1, to=1-1]{}{\id}
        \arrow[loop left, from=1-1, to=1-1]
        \arrow[loop above, from=1-3, to=1-3]{}{\id}
        \arrow[loop below, from=3-1, to=3-1]{}{\id}
        \arrow[loop below, from=3-3, to=3-3]{}{\id}
      \end{tikzcd}\]
  \end{onlyenv}
  \begin{onlyenv}<4>
    A specified \emph{composite} morphism for any two composable morphisms.
    \[\begin{tikzcd}
	\bullet \&\& \bullet \\
	\\
	\bullet \&\& \bullet
	\arrow[curve={height=-6pt}, from=1-1, to=1-3]{}{f}
	\arrow[curve={height=6pt}, from=1-1, to=1-3]
	\arrow[from=1-3, to=3-3]{}{g}
	\arrow[from=1-1, to=3-3]{}[swap]{g \circ f}
        \arrow[loop above, from=1-1, to=1-1]
        \arrow[loop left, from=1-1, to=1-1]
        \arrow[loop above, from=1-3, to=1-3]
        \arrow[loop below, from=3-1, to=3-1]
        \arrow[loop below, from=3-3, to=3-3]
      \end{tikzcd}\]
  \end{onlyenv}
\end{frame}

\begin{frame}{Categories}
  These data are subject to the following requirements:
  \begin{itemize}
  \item \bf{Associativity}: $f \circ (g \circ h) = (f \circ g) \circ h$.
  \item \bf{Unitality}: $\id \circ f = f = f \circ \id$.
  \end{itemize}
\end{frame}

\begin{frame}{Examples}
  \begin{columns}
    \begin{column}{0.5\linewidth}
      \begin{itemize}
      \item<1-> $\mathsf{Set}$
        \begin{itemize}
        \item sets
        \item functions
        \item identity functions
        \item function composition
        \end{itemize}
      \item<2-> $\mathsf{Grp}$
        \begin{itemize}
        \item groups
        \item group homomorphisms
        \item identity functions
        \item function composition
        \end{itemize}
      \end{itemize}
    \end{column}
    \begin{column}{0.5\linewidth}
      \begin{itemize}
      \item<3-> $(P,\leq)$
        \begin{itemize}
        \item elements of $P$
        \item $a \to b$ iff $a \leq b$
        \item reflexivity of $\leq$
        \item transitivity of $\leq$
        \end{itemize}
      \end{itemize}
      \begin{itemize}
      \item<4-> $\cT$
        \begin{itemize}
        \item sorts
        \item terms
        \item variables
        \item substitution
        \end{itemize}
      \end{itemize}
    \end{column}
  \end{columns}
\end{frame}

\begin{frame}[allowframebreaks]{Example}{Internal groups}
  Recall the usual presentation of the theory of groups.
  To specify a group structure on an object $G$ (an \emph{internal group}) is to specify the following data.
  \begin{itemize}
  \item The group identity: $e : 1 \to G$
  \item The group inverse: $()\inv : G \to G$
  \item The group multiplication: $m : G \times G \to G$
  \end{itemize}
  These data are required to satisfy the following conditions.
  \begin{itemize}
  \item $m(x,e) = x = m(e,x)$
  \item $m(x,x\inv) = e = m(x\inv,x)$
  \item $m(m(x,y),z) = m(x,m(y,z))$
  \end{itemize}
  \begin{mathpar}
    % https://q.uiver.app/#q=WzAsNCxbNCwwLCJHIl0sWzIsMCwiRyBcXHRpbWVzIEciXSxbMiwxLCJHIl0sWzAsMCwiRyJdLFswLDEsImUgXFx0aW1lcyBcXGlkX3tHfSIsMl0sWzEsMiwibSJdLFswLDIsIiIsMCx7ImxldmVsIjoyLCJzdHlsZSI6eyJoZWFkIjp7Im5hbWUiOiJub25lIn19fV0sWzMsMSwiXFxpZF97R30gXFx0aW1lcyBlIl0sWzMsMiwiIiwwLHsibGV2ZWwiOjIsInN0eWxlIjp7ImhlYWQiOnsibmFtZSI6Im5vbmUifX19XV0=
    \begin{tikzcd}
      G \&\& {G \times G} \&\& G \\
      \&\& G
      \arrow[from=1-5, to=1-3]{}[swap]{(e,\id_{G})}
      \arrow[from=1-3, to=2-3]{}{m}
      \arrow[Rightarrow, no head, from=1-5, to=2-3]
      \arrow[from=1-1, to=1-3]{}{{(\id_{G}, e)}}
      \arrow[Rightarrow, no head, from=1-1, to=2-3]
    \end{tikzcd} \and%
    % https://q.uiver.app/#q=WzAsNCxbMCwwLCJHIl0sWzIsMCwiRyBcXHRpbWVzIEciXSxbNCwwLCJHIl0sWzIsMSwiRyJdLFswLDEsIihcXGlkX3tHfSwoKVxcaW52KSJdLFsyLDEsIigoKVxcaW52LCBcXGlkX3tHfSkiLDJdLFsxLDMsIm0iXSxbMCwzLCIiLDIseyJsZXZlbCI6Miwic3R5bGUiOnsiaGVhZCI6eyJuYW1lIjoibm9uZSJ9fX1dLFsyLDMsIiIsMix7ImxldmVsIjoyLCJzdHlsZSI6eyJoZWFkIjp7Im5hbWUiOiJub25lIn19fV1d
    \begin{tikzcd}
      G \&\& {G \times G} \&\& G \\
      \&\& G
      \arrow[from=1-1, to=1-3]{}{(\id_{G},()\inv)}
      \arrow[from=1-5, to=1-3]{}[swap]{(()\inv, \id_{G})}
      \arrow[from=1-3, to=2-3]{}{m}
      \arrow[Rightarrow, no head, from=1-1, to=2-3]
      \arrow[Rightarrow, no head, from=1-5, to=2-3]
    \end{tikzcd} \and%
    % https://q.uiver.app/#q=WzAsNCxbMCwwLCJHIFxcdGltZXMgRyBcXHRpbWVzIEciXSxbMCwyLCJHIFxcdGltZXMgRyJdLFsyLDAsIkcgXFx0aW1lcyBHIl0sWzIsMiwiRyJdLFsxLDMsIm0iLDJdLFsyLDMsIm0iXSxbMCwxLCJcXGlkX3tHfSBcXHRpbWVzIG0iLDJdLFswLDIsIm0gXFx0aW1lcyBcXGlkX3tHfSJdXQ==
    \begin{tikzcd}
      {G \times G \times G} \&\& {G \times G} \\
      \\
      {G \times G} \&\& G
      \arrow[from=3-1, to=3-3]{}[swap]{m}
      \arrow[from=1-3, to=3-3]{}{m}
      \arrow[from=1-1, to=3-1]{}[swap]{\id_{G} \times m}
      \arrow[from=1-1, to=1-3]{}{m \times \id_{G}}
    \end{tikzcd}
  \end{mathpar}
\end{frame}

\begin{frame}{Observations}
  \begin{itemize}
  \item Internal groups can be defined in any category $\iC$ (with sufficient structure)
  \item An internal group in $\mathsf{Set}$ is just a group in the usual sense
  \item Many things can be internalized in a category with sufficient structure
    \begin{itemize}
    \item Internal monoid
    \item Internal Heyting algebra
    \item Internal categories
    \end{itemize}
  \end{itemize}
\end{frame}

\section{Functors}
\label{sec:functors}

\begin{frame}{Functors}
  Let $\iC$ and $\iD$ be categories.
  A \emph{functor} $F : \iC \to \iD$ consists of the following data.
  \begin{itemize}
  \item An \emph{action on objects}: each object of $\iC$ is mapped to an object of $\iD$.
  \item An \emph{action on morphisms}: each morphism $c \to c'$ is mapped to a morphism $Fc \to Fc'$.
  \end{itemize}
  \pause
  These data are required to satisfy the following conditions.
  \begin{itemize}
  \item $F(\id_{a}) = \id_{Fa}$.
  \item $F(f \circ g) = Ff \circ Fg$.
  \end{itemize}
\end{frame}

\begin{frame}{Examples}
  \begin{itemize}
  \item<1-> The forgetful functor $U : \mathsf{Grp} \to \mathsf{Set}$ mapping each group to its underlying set and each group homomorphism to its underlying function
  \item<2-> The free functor $F : \mathsf{Set} \to \mathsf{Grp}$ mapping each set to the free group on that set
  \item<3-> The discrete topology functor $D : \mathsf{Set} \to \mathsf{Top}$ equipping each set with the discrete topology
  \item<4-> The indiscrete topology functor $I : \mathsf{Set} \to \mathsf{Top}$ equipping each set with the indiscrete topology
  \item<5-> The fundamental group functor $\pi_{1} : \mathsf{Top}_{*} \to \mathsf{Grp}$ mapping each pointed space to the group of closed paths in that space
  \item<6-> The functor $\mathsf{Maybe} : \mathsf{Set} \to \mathsf{Set}$ mapping each set $S$ to the underlying set of $S$ freely adjoined with a point
  \end{itemize}
\end{frame}

\section{Universality}
\label{sec:universality}

\begin{frame}{Universality}
  \begin{itemize}
  \item<1-> Product group is the ``most efficient'' group so that the projections are group homomorphisms
  \item<2-> Quotient topology is the ``most efficient'' topology so that the quotient map is continuous
  \item<3-> The Sierpi\'{n}ski space is the ``most efficient'' topological space equipped with an open set
  \item<4-> The complete graph $K_{n}$ is the ``most efficient'' $n$-colorable graph
  \item<5-> Boolean values form the ``most efficient'' set that distinguishes subsets
  \end{itemize}
\end{frame}

\begin{frame}{Identity functor on Set}
  \begin{itemize}
  \item[] $\id_{\mathsf{Set}} : \mathsf{Set} \to \mathsf{Set}$
    \begin{itemize}
    \item[] action on objects: $S \mapsto S$
    \item[] action on morphisms: $f \mapsto f$
    \end{itemize}
  \end{itemize}
  \begin{block}{Fact}
    $\mathsf{Set}(1, S) \iso \id_{\mathsf{Set}}(S) = S$.
  \end{block}
  \begin{itemize}
  \item<2-> The identity functor on $\mathsf{Set}$ is \emph{represented} by the singleton set
  \item<3-> To understand $S$, it suffices to understand all functions from the singleton set to $S$
  \item<4-> Every element of a set $S$ corresponds to a unique function from the singleton set to $S$
  \end{itemize}
\end{frame}

\begin{frame}{Graph coloring}
  Graphs and their homomorphisms form a category $\mathsf{Graph}$.
  \begin{itemize}
  \item[] $n\text{-Color} : \mathsf{Graph} \to \mathsf{Set}$
    \begin{itemize}
    \item[] action on objects: $G \mapsto$ set of all $n$-colorings
    \item[] action on morphisms : $f \mapsto$ a consistent recoloring
    \end{itemize}
  \end{itemize}
  \begin{block}{Fact}
    $\mathsf{Graph}(G,K_{n}) \iso n\text{-Color}(G)$.
  \end{block}
  \begin{itemize}
  \item<2-> $n\text{-Color}$ is \emph{represented} by the complete graph $K_{n}$
  \item<3-> To understand the set of all $n$-colorings of $G$, it suffices to understand the set of all graph homomorphisms from $G$ to $K_{n}$
  \item<4-> Every $n$-coloring on a graph $G$ determines a graph homomorphism $G \to K_{n}$
  \end{itemize}
\end{frame}

\begin{frame}{Powerset}
  \begin{itemize}
  \item[] $\mathcal{P} : \mathsf{Set} \to \mathsf{Set}$
    \begin{itemize}
    \item[] action on objects: $S \mapsto \mathcal{P}(S)$
    \item[] action on morphisms: magic
    \end{itemize}
  \end{itemize}
  \begin{block}{Fact}
    $\mathsf{Set}(S,\mathbb{B}) \iso \mathcal{P}(S)$, where $\mathbb{B}$ is the two point set (Boolean values).
  \end{block}
  \begin{itemize}
  \item<2-> The powerset functor is \emph{represented} by the two point set
  \item<3-> To understand the powerset, it suffices to understand all \emph{characteristic functions} (Boolean functions)
  \item<4-> Every subset is uniquely determined by a characteristic function.
  \end{itemize}
\end{frame}

\end{document}
