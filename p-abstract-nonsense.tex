\documentclass{beamer}
\newif\ifbeamer\beamertrue
\input{decls}
\tikzcdset{diagrams={ampersand replacement=\&}}

% \usetheme{Warsaw}
% \setbeamertemplate{headline}{}

\usetheme[height=0cm]{Rochester}
\useinnertheme{circles}
% \useoutertheme[compress,subsection=false]{miniframes}
\usecolortheme{rose}
\setbeamertemplate{navigation symbols}{}

\title{Abstract nonsense}
\author[Frank T]{Frank Tsai\inst{1}}
\institute{\inst{1} (SUNY at Buffalo)}
\date{\today}

\AtBeginSection[]
{
  \begin{frame}<beamer>{Outline}
    \tableofcontents[currentsection]
  \end{frame}
}

\setbeamercolor{emph}{fg=blue}
\renewcommand<>{\emph}[1]{%
  {\usebeamercolor[fg]{emph}\only#2{\itshape}#1}%
}

\renewcommand<>{\bf}[1]{%
  {\usebeamercolor[fg]{emph}\only#2{\bfseries}#1}%
}

\begin{document}

\begin{frame}
  \maketitle
\end{frame}

\section{Categories}
\label{sec:categories}

\begin{frame}{Categories}
  A \emph{category} consists of...\\\vspace{1em}
  \begin{onlyenv}<1>
    A collection of \emph{objects}.
    % https://q.uiver.app/#q=WzAsNCxbMCwwLCJcXGJ1bGxldCJdLFswLDIsIlxcYnVsbGV0Il0sWzIsMiwiXFxidWxsZXQiXSxbMiwwLCJcXGJ1bGxldCJdXQ==
    \[\begin{tikzcd}
	\bullet \&\& \bullet \\
	\\
	\bullet \&\& \bullet
      \end{tikzcd}\]
  \end{onlyenv}
  \begin{onlyenv}<2>
    A collection of \emph{morphisms}.
    % https://q.uiver.app/#q=WzAsNyxbMiwwLCJcXGJ1bGxldCJdLFswLDIsIlxcYnVsbGV0Il0sWzMsMywiXFxidWxsZXQiXSxbNCwxLCJcXGJ1bGxldCJdLFs2LDIsIlxcYnVsbGV0Il0sWzUsMCwiXFxidWxsZXQiXSxbMiwyLCJcXGJ1bGxldCJdLFsxLDBdLFswLDNdLFsxLDNdLFs2LDIsIiIsMCx7ImN1cnZlIjotMn1dLFs2LDIsIiIsMix7ImN1cnZlIjoyfV1d
    % https://q.uiver.app/#q=WzAsNCxbMCwwLCJcXGJ1bGxldCJdLFswLDIsIlxcYnVsbGV0Il0sWzIsMiwiXFxidWxsZXQiXSxbMiwwLCJcXGJ1bGxldCJdLFswLDMsIiIsMCx7ImN1cnZlIjotMX1dLFswLDMsIiIsMix7ImN1cnZlIjoxfV0sWzMsMl0sWzAsMl1d
    \[\begin{tikzcd}
	\bullet \&\& \bullet \\
	\\
	\bullet \&\& \bullet
	\arrow[curve={height=-6pt}, from=1-1, to=1-3]
	\arrow[curve={height=6pt}, from=1-1, to=1-3]
	\arrow[from=1-3, to=3-3]
	\arrow[from=1-1, to=3-3]
        \arrow[loop above, from=1-1, to=1-1]
        \arrow[loop left, from=1-1, to=1-1]
        \arrow[loop above, from=1-3, to=1-3]
        \arrow[loop below, from=3-1, to=3-1]
        \arrow[loop below, from=3-3, to=3-3]
      \end{tikzcd}\]
  \end{onlyenv}
  \begin{onlyenv}<3>
    A specified \emph{identity} morphism for each object.
    \[\begin{tikzcd}
	\bullet \&\& \bullet \\
	\\
	\bullet \&\& \bullet
	\arrow[curve={height=-6pt}, from=1-1, to=1-3]
	\arrow[curve={height=6pt}, from=1-1, to=1-3]
	\arrow[from=1-3, to=3-3]
	\arrow[from=1-1, to=3-3]
        \arrow[loop above, from=1-1, to=1-1]{}{\id}
        \arrow[loop left, from=1-1, to=1-1]
        \arrow[loop above, from=1-3, to=1-3]{}{\id}
        \arrow[loop below, from=3-1, to=3-1]{}{\id}
        \arrow[loop below, from=3-3, to=3-3]{}{\id}
      \end{tikzcd}\]
  \end{onlyenv}
  \begin{onlyenv}<4>
    A specified \emph{composite} morphism for any two composable morphisms.
    \[\begin{tikzcd}
	\bullet \&\& \bullet \\
	\\
	\bullet \&\& \bullet
	\arrow[curve={height=-6pt}, from=1-1, to=1-3]{}{f}
	\arrow[curve={height=6pt}, from=1-1, to=1-3]
	\arrow[from=1-3, to=3-3]{}{g}
	\arrow[from=1-1, to=3-3]{}[swap]{g \circ f}
        \arrow[loop above, from=1-1, to=1-1]
        \arrow[loop left, from=1-1, to=1-1]
        \arrow[loop above, from=1-3, to=1-3]
        \arrow[loop below, from=3-1, to=3-1]
        \arrow[loop below, from=3-3, to=3-3]
      \end{tikzcd}\]
  \end{onlyenv}
\end{frame}

\begin{frame}{Categories}
  These data are subject to the following requirements:
  \begin{itemize}
  \item \bf{Associativity}: $f \circ (g \circ h) = (f \circ g) \circ h$.
  \item \bf{Unitality}: $\id \circ f = f = f \circ \id$.
  \end{itemize}
\end{frame}

\begin{frame}{Examples}
  \begin{columns}
    \begin{column}{0.5\linewidth}
      \begin{itemize}
      \item $\mathsf{Set}$
        \begin{itemize}
        \item sets
        \item functions
        \item identity functions
        \item function composition
        \end{itemize}
      \item $\mathsf{Grp}$
        \begin{itemize}
        \item groups
        \item group homomorphisms
        \item identity functions
        \item function composition
        \end{itemize}
      \end{itemize}
    \end{column}
    \begin{column}{0.5\linewidth}
      \begin{itemize}
      \item $\mathcal{M}$
        \begin{itemize}
        \item a single abstract object
        \item elements of the monoid $\mathcal{M}$
        \item the identity element
        \item monoid multiplication
        \end{itemize}
      \item $(P,\leq)$
        \begin{itemize}
        \item elements of $P$
        \item $a \to b$ iff $a \leq b$
        \item reflexivity of $\leq$
        \item transitivity of $\leq$
        \end{itemize}
      \end{itemize}
    \end{column}
  \end{columns}
\end{frame}

\section{Functors}
\label{sec:functors}

\begin{frame}{Functors}
  Let $\iC$ and $\iD$ be categories.
  A \emph{functor} $F : \iC \to \iD$ consists of
  \begin{itemize}
  \item an \emph{action on objects}: each object of $\iC$ is mapped to an object of $\iD$.
  \item an \emph{action on morphisms}: each morphism $c \to c'$ is mapped to a morphism $Fc \to Fc'$.
  \end{itemize}
  subject to the following requirements:
  \begin{itemize}
  \item $F(\id_{a}) = \id_{Fa}$.
  \item $F(f \circ g) = Ff \circ Fg$.
  \end{itemize}
\end{frame}

\begin{frame}{Examples}
  \begin{itemize}
  \item A functor between posets $(P, \leq_{P})$ and $(Q, \leq_{Q})$ encodes an order-preserving function
  \item The forgetful functor $U : \mathsf{Grp} \to \mathsf{Set}$ mapping each group to its underlying set and each group homomorphism to its underlying function.
  \item The free functor $F : \mathsf{Set} \to \mathsf{Grp}$ mapping each set $S$ to the free group generated by $S$ and each function $f$ to a group homomorphism induced by $f$.
  \item The cofree functor $I : \mathsf{Set} \to \mathsf{Top}$ that equips each set with the indiscrete topology.
  \item The forgetful functor $\iota : \mathsf{Ab} \to \mathsf{Grp}$ that forgets the ``abelianness'' of an abelian group.
  \end{itemize}
\end{frame}

\section{Natural transformations}
\label{sec:natural-transformations}

\section{Representability}
\label{sec:representability}

\section{Limits and colimits}
\label{sec:limits-and-colimits}



\end{document}
