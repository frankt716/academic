\documentclass{amsart}
\usepackage{amssymb,amsmath,stmaryrd,mathrsfs}
\usepackage{quiver}

% This is a generally good thing to do (https://tex.stackexchange.com/q/664/535), but make sure that the package cm-super is installed (https://tex.stackexchange.com/a/1293/535) or it will make bad-looking PDFs with bitmap (type 3) fonts.
\usepackage[T1]{fontenc}

%% Set this to true before loading if we're using the TAC style file.
%% Note that eventually, TAC requires everything to be in one source file.
\def\definetac{\newif\iftac}    % Can't define a \newif inside another \if!
\ifx\tactrue\undefined
  \definetac
  %% Guess whether we're using TAC by whether \state is defined.
  \ifx\state\undefined\tacfalse\else\tactrue\fi
\fi

% Similarly detect beamer
\def\definebeamer{\newif\ifbeamer}
\ifx\beamertrue\undefined
  \definebeamer
  %% Guess whether we're using Beamer by whether \uncover is defined.
  \ifx\uncover\undefined\beamerfalse\else\beamertrue\fi
\fi

% And cleveref
\def\definecref{\newif\ifcref}
\ifx\creftrue\undefined
  \definecref
  % Default to false
  \creftrue
\fi

\iftac\else\usepackage{amsthm}\fi
%\usepackage[all,2cell]{xy}
%\UseAllTwocells
\usepackage{tikz,tikz-cd}
\usetikzlibrary{arrows}
\ifbeamer\else
  \usepackage{enumitem}
  \usepackage{xcolor}
  \definecolor{darkgreen}{rgb}{0,0.45,0} 
  \iftac\else\usepackage[pagebackref,colorlinks,citecolor=darkgreen,linkcolor=darkgreen]{hyperref}
  \renewcommand*{\backref}[1]{}
  \renewcommand*{\backrefalt}[4]{({%
      \ifcase #1 Not cited.%
            \or Cited on p.~#2%
            \else Cited on pp.~#2%
      \fi%
    })}\fi
\fi
\usepackage{mathtools}          % for all sorts of things
%\usepackage{graphics}
\usepackage{ifmtarg}            % used in \jd
%\usepackage{microtype}
%\usepackage{color,epsfig}
%\usepackage{fullpage}
%\usepackage{eucal}
%\usepackage{wasysym}
%\usepackage{txfonts}            % for \invamp, or for the nice fonts
\usepackage{braket}             % for \Set, etc.
\let\setof\Set
\usepackage{url}                % for citations to web sites
\usepackage{xspace}             % put spaces after a \command in text
%\usepackage{cite}               % compress and sort grouped citations (only use with numbered citations)
\ifcref\usepackage{cleveref,aliascnt}\fi
\usepackage[status=draft,author=]{fixme}
\fxusetheme{color}
\usepackage{mathpartir}

%% If you want to use biblatex, e.g. if a journal requires (Author name YEAR) citations.
%% You also need to remove the pagebackref option and \backref commands above.
% \usepackage[style=authoryear,
%  backend=biber, bibencoding=utf8,
%  backref=true,
%  maxnames=4,
%  maxbibnames=99,
%  uniquename=false,
%  firstinits=true
% ]{biblatex}
% \addbibresource{all.bib}

% \let\cite\parencite
% \DeclareNameAlias{sortname}{last-first}

\makeatletter
\let\ea\expandafter

%% Defining commands that are always in math mode.
\def\mdef#1#2{\ea\ea\ea\gdef\ea\ea\noexpand#1\ea{\ea\ensuremath\ea{#2}\xspace}}
\def\alwaysmath#1{\ea\ea\ea\global\ea\ea\ea\let\ea\ea\csname your@#1\endcsname\csname #1\endcsname
  \ea\def\csname #1\endcsname{\ensuremath{\csname your@#1\endcsname}\xspace}}

%% SIMPLE COMMANDS FOR FONTS AND DECORATIONS

\newcount\foreachcount

\def\foreachletter#1#2#3{\foreachcount=#1
  \ea\loop\ea\ea\ea#3\@alph\foreachcount
  \advance\foreachcount by 1
  \ifnum\foreachcount<#2\repeat}

\def\foreachLetter#1#2#3{\foreachcount=#1
  \ea\loop\ea\ea\ea#3\@Alph\foreachcount
  \advance\foreachcount by 1
  \ifnum\foreachcount<#2\repeat}

% Script: \sA is \mathscr{A}
\def\definescr#1{\ea\gdef\csname s#1\endcsname{\ensuremath{\mathscr{#1}}\xspace}}
\foreachLetter{1}{27}{\definescr}
% Calligraphic: \cA is \mathcal{A}
\def\definecal#1{\ea\gdef\csname c#1\endcsname{\ensuremath{\mathcal{#1}}\xspace}}
\foreachLetter{1}{27}{\definecal}
% Bold: \bA is \mathbf{A}
\def\definebold#1{\ea\gdef\csname b#1\endcsname{\ensuremath{\mathbf{#1}}\xspace}}
\foreachLetter{1}{27}{\definebold}
% Blackboard Bold: \dA is \mathbb{A}
\def\definebb#1{\ea\gdef\csname d#1\endcsname{\ensuremath{\mathbb{#1}}\xspace}}
\foreachLetter{1}{27}{\definebb}
% Fraktur: \fa is \mathfrak{a}, except for \fi; \fA is \mathfrak{A}
\def\definefrak#1{\ea\gdef\csname f#1\endcsname{\ensuremath{\mathfrak{#1}}\xspace}}
\foreachletter{1}{9}{\definefrak}
\foreachletter{10}{27}{\definefrak}
\foreachLetter{1}{27}{\definefrak}
% Sans serif: \ia is \mathsf{a}, except for \if and \in and \it
\def\definesf#1{\ea\gdef\csname i#1\endcsname{\ensuremath{\mathsf{#1}}\xspace}}
\foreachletter{1}{6}{\definesf}
\foreachletter{7}{14}{\definesf}
\foreachletter{15}{20}{\definesf}
\foreachletter{21}{27}{\definesf}
\foreachLetter{1}{27}{\definesf}
% Bar: \Abar is \overline{A}, \abar is \overline{a}
\def\definebar#1{\ea\gdef\csname #1bar\endcsname{\ensuremath{\overline{#1}}\xspace}}
\foreachLetter{1}{27}{\definebar}
\foreachletter{1}{8}{\definebar} % \hbar is something else!
\foreachletter{9}{15}{\definebar} % \obar is something else!
\foreachletter{16}{27}{\definebar}
% Tilde: \Atil is \widetilde{A}, \atil is \widetilde{a}
\def\definetil#1{\ea\gdef\csname #1til\endcsname{\ensuremath{\widetilde{#1}}\xspace}}
\foreachLetter{1}{27}{\definetil}
\foreachletter{1}{27}{\definetil}
% Hats: \Ahat is \widehat{A}, \ahat is \widehat{a}
\def\definehat#1{\ea\gdef\csname #1hat\endcsname{\ensuremath{\widehat{#1}}\xspace}}
\foreachLetter{1}{27}{\definehat}
\foreachletter{1}{27}{\definehat}
% Checks: \Achk is \widecheck{A}, \achk is \widecheck{a}
\def\definechk#1{\ea\gdef\csname #1chk\endcsname{\ensuremath{\widecheck{#1}}\xspace}}
\foreachLetter{1}{27}{\definechk}
\foreachletter{1}{27}{\definechk}
% Underline: \uA is \underline{A}, \ua is \underline{a}
\def\defineul#1{\ea\gdef\csname u#1\endcsname{\ensuremath{\underline{#1}}\xspace}}
\foreachLetter{1}{27}{\defineul}
\foreachletter{1}{27}{\defineul}

% Particular commands for typefaces, sometimes with the first letter
% different.
\def\autofmt@n#1\autofmt@end{\mathrm{#1}}
\def\autofmt@b#1\autofmt@end{\mathbf{#1}}
\def\autofmt@d#1#2\autofmt@end{\mathbb{#1}\mathsf{#2}}
\def\autofmt@c#1#2\autofmt@end{\mathcal{#1}\mathit{#2}}
\def\autofmt@s#1#2\autofmt@end{\mathscr{#1}\mathit{#2}}
\def\autofmt@i#1\autofmt@end{\mathsf{#1}}
\def\autofmt@f#1\autofmt@end{\mathfrak{#1}}
% Particular commands for decorations.
\def\autofmt@u#1\autofmt@end{\underline{\smash{\mathsf{#1}}}}
\def\autofmt@U#1\autofmt@end{\underline{\underline{\smash{\mathsf{#1}}}}}
\def\autofmt@h#1\autofmt@end{\widehat{#1}}
\def\autofmt@r#1\autofmt@end{\overline{#1}}
\def\autofmt@t#1\autofmt@end{\widetilde{#1}}
\def\autofmt@k#1\autofmt@end{\check{#1}}

% Defining multi-letter commands.  Use this like so:
% \autodefs{\bSet\cCat\cCAT\kBicat\lProf}
\def\auto@drop#1{}
\def\autodef#1{\ea\ea\ea\@autodef\ea\ea\ea#1\ea\auto@drop\string#1\autodef@end}
\def\@autodef#1#2#3\autodef@end{%
  \ea\def\ea#1\ea{\ea\ensuremath\ea{\csname autofmt@#2\endcsname#3\autofmt@end}\xspace}}
\def\autodefs@end{blarg!}
\def\autodefs#1{\@autodefs#1\autodefs@end}
\def\@autodefs#1{\ifx#1\autodefs@end%
  \def\autodefs@next{}%
  \else%
  \def\autodefs@next{\autodef#1\@autodefs}%
  \fi\autodefs@next}

%% FONTS AND DECORATION FOR GREEK LETTERS

%% the package `mathbbol' gives us blackboard bold greek and numbers,
%% but it does it by redefining \mathbb to use a different font, so that
%% all the other \mathbb letters look different too.  Here we import the
%% font with bb greek and numbers, but assign it a different name,
%% \mathbbb, so as not to replace the usual one.
\DeclareSymbolFont{bbold}{U}{bbold}{m}{n}
\DeclareSymbolFontAlphabet{\mathbbb}{bbold}
\newcommand{\dDelta}{\ensuremath{\mathbbb{\Delta}}\xspace}
\newcommand{\done}{\ensuremath{\mathbbb{1}}\xspace}
\newcommand{\dtwo}{\ensuremath{\mathbbb{2}}\xspace}
\newcommand{\dthree}{\ensuremath{\mathbbb{3}}\xspace}

% greek with bars
\newcommand{\albar}{\ensuremath{\overline{\alpha}}\xspace}
\newcommand{\bebar}{\ensuremath{\overline{\beta}}\xspace}
\newcommand{\gmbar}{\ensuremath{\overline{\gamma}}\xspace}
\newcommand{\debar}{\ensuremath{\overline{\delta}}\xspace}
\newcommand{\phibar}{\ensuremath{\overline{\varphi}}\xspace}
\newcommand{\psibar}{\ensuremath{\overline{\psi}}\xspace}
\newcommand{\xibar}{\ensuremath{\overline{\xi}}\xspace}
\newcommand{\ombar}{\ensuremath{\overline{\omega}}\xspace}

% greek with tildes
\newcommand{\altil}{\ensuremath{\widetilde{\alpha}}\xspace}
\newcommand{\betil}{\ensuremath{\widetilde{\beta}}\xspace}
\newcommand{\gmtil}{\ensuremath{\widetilde{\gamma}}\xspace}
\newcommand{\phitil}{\ensuremath{\widetilde{\varphi}}\xspace}
\newcommand{\psitil}{\ensuremath{\widetilde{\psi}}\xspace}
\newcommand{\xitil}{\ensuremath{\widetilde{\xi}}\xspace}
\newcommand{\omtil}{\ensuremath{\widetilde{\omega}}\xspace}

% MISCELLANEOUS SYMBOLS
\let\del\partial
\mdef\delbar{\overline{\partial}}
\let\sm\wedge
\newcommand{\dd}[1]{\ensuremath{\frac{\partial}{\partial {#1}}}}
\newcommand{\inv}{^{-1}}
\newcommand{\dual}{^{\vee}}
\mdef\hf{\textstyle\frac12 }
\mdef\thrd{\textstyle\frac13 }
\mdef\qtr{\textstyle\frac14 }
\let\meet\wedge
\let\join\vee
\let\dn\downarrow
\newcommand{\op}{^{\mathrm{op}}}
\newcommand{\co}{^{\mathrm{co}}}
\newcommand{\coop}{^{\mathrm{coop}}}
\let\adj\dashv
\let\iso\cong
\let\eqv\simeq
\let\cng\equiv
\mdef\Id{\mathrm{Id}}
\mdef\id{\mathrm{id}}
\alwaysmath{ell}
\alwaysmath{infty}
\let\oo\infty
\def\io{\ensuremath{(\infty,1)}}
\alwaysmath{odot}
\def\frc#1/#2.{\frac{#1}{#2}}   % \frc x^2+1 / x^2-1 .
\mdef\ten{\mathrel{\otimes}}
\let\bigten\bigotimes
\mdef\sqten{\mathrel{\boxtimes}}
\def\lt{<}                      % For iTex compatibility
\def\gt{>}

%%% Blanks (shorthand for lambda abstractions)
\newcommand{\blank}{\mathord{\hspace{1pt}\text{--}\hspace{1pt}}}
%%% Nameless objects
\newcommand{\nameless}{\mathord{\hspace{1pt}\underline{\hspace{1ex}}\hspace{1pt}}}

% Hiragana "yo" for the Yoneda embedding, from https://arxiv.org/abs/1506.08870
\DeclareFontFamily{U}{min}{}
\DeclareFontShape{U}{min}{m}{n}{<-> udmj30}{}
\newcommand{\yon}{\!\text{\usefont{U}{min}{m}{n}\symbol{'210}}\!}

%% Get some new symbols from mathabx, without changing the old ones by
%% importing the package.  Font declarations copied from mathabx.sty:
\DeclareFontFamily{U}{mathb}{\hyphenchar\font45}
\DeclareFontShape{U}{mathb}{m}{n}{
      <5> <6> <7> <8> <9> <10> gen * mathb
      <10.95> mathb10 <12> <14.4> <17.28> <20.74> <24.88> mathb12
      }{}
\DeclareSymbolFont{mathb}{U}{mathb}{m}{n}
\DeclareFontSubstitution{U}{mathb}{m}{n}
\DeclareFontFamily{U}{mathx}{\hyphenchar\font45}
\DeclareFontShape{U}{mathx}{m}{n}{
      <5> <6> <7> <8> <9> <10>
      <10.95> <12> <14.4> <17.28> <20.74> <24.88>
      mathx10
      }{}
\DeclareSymbolFont{mathx}{U}{mathx}{m}{n}
\DeclareFontSubstitution{U}{mathx}{m}{n}
%% And now the symbols we want, copied from mathabx.dcl
\DeclareMathSymbol{\dotplus}       {2}{mathb}{"00}% name to be checked
\DeclareMathSymbol{\dotdiv}        {2}{mathb}{"01}% name to be checked
\DeclareMathSymbol{\dottimes}      {2}{mathb}{"02}% name to be checked
\DeclareMathSymbol{\divdot}        {2}{mathb}{"03}% name to be checked
\DeclareMathSymbol{\udot}          {2}{mathb}{"04}% name to be checked
\DeclareMathSymbol{\square}        {2}{mathb}{"05}% name to be checked
\DeclareMathSymbol{\Asterisk}      {2}{mathb}{"06}
\DeclareMathSymbol{\bigast}        {1}{mathb}{"06}
\DeclareMathSymbol{\coAsterisk}    {2}{mathb}{"07}
\DeclareMathSymbol{\bigcoast}      {1}{mathb}{"07}
\DeclareMathSymbol{\circplus}      {2}{mathb}{"08}% name to be checked
\DeclareMathSymbol{\pluscirc}      {2}{mathb}{"09}% name to be checked
\DeclareMathSymbol{\convolution}   {2}{mathb}{"0A}% name to be checked
\DeclareMathSymbol{\divideontimes} {2}{mathb}{"0B}% name to be checked
\DeclareMathSymbol{\blackdiamond}  {2}{mathb}{"0C}% name to be checked
\DeclareMathSymbol{\sqbullet}      {2}{mathb}{"0D}% name to be checked
\DeclareMathSymbol{\bigstar}       {2}{mathb}{"0E}
\DeclareMathSymbol{\bigvarstar}    {2}{mathb}{"0F}
\DeclareMathSymbol{\corresponds}   {3}{mathb}{"1D}% name to be checked
\DeclareMathSymbol{\boxleft}      {2}{mathb}{"68}
\DeclareMathSymbol{\boxright}     {2}{mathb}{"69}
\DeclareMathSymbol{\boxtop}       {2}{mathb}{"6A}
\DeclareMathSymbol{\boxbot}       {2}{mathb}{"6B}
\DeclareMathSymbol{\updownarrows}          {3}{mathb}{"D6}
\DeclareMathSymbol{\downuparrows}          {3}{mathb}{"D7}
\DeclareMathSymbol{\Lsh}                   {3}{mathb}{"E8}
\DeclareMathSymbol{\Rsh}                   {3}{mathb}{"E9}
\DeclareMathSymbol{\dlsh}                  {3}{mathb}{"EA}
\DeclareMathSymbol{\drsh}                  {3}{mathb}{"EB}
\DeclareMathSymbol{\looparrowdownleft}     {3}{mathb}{"EE}
\DeclareMathSymbol{\looparrowdownright}    {3}{mathb}{"EF}
% \DeclareMathSymbol{\curvearrowleft}        {3}{mathb}{"F0}
% \DeclareMathSymbol{\curvearrowright}       {3}{mathb}{"F1}
\DeclareMathSymbol{\curvearrowleftright}   {3}{mathb}{"F2}
\DeclareMathSymbol{\curvearrowbotleft}     {3}{mathb}{"F3}
\DeclareMathSymbol{\curvearrowbotright}    {3}{mathb}{"F4}
\DeclareMathSymbol{\curvearrowbotleftright}{3}{mathb}{"F5}
% \DeclareMathSymbol{\circlearrowleft}       {3}{mathb}{"F6}
% \DeclareMathSymbol{\circlearrowright}      {3}{mathb}{"F7}
\DeclareMathSymbol{\leftsquigarrow}        {3}{mathb}{"F8}
\DeclareMathSymbol{\rightsquigarrow}       {3}{mathb}{"F9}
\DeclareMathSymbol{\leftrightsquigarrow}   {3}{mathb}{"FA}
\DeclareMathSymbol{\lefttorightarrow}      {3}{mathb}{"FC}
\DeclareMathSymbol{\righttoleftarrow}      {3}{mathb}{"FD}
\DeclareMathSymbol{\uptodownarrow}         {3}{mathb}{"FE}
\DeclareMathSymbol{\downtouparrow}         {3}{mathb}{"FF}
\DeclareMathSymbol{\varhash}       {0}{mathb}{"23}
\DeclareMathSymbol{\sqSubset}       {3}{mathb}{"94}
\DeclareMathSymbol{\sqSupset}       {3}{mathb}{"95}
\DeclareMathSymbol{\nsqSubset}      {3}{mathb}{"96}
\DeclareMathSymbol{\nsqSupset}      {3}{mathb}{"97}
% WIDECHECK
\DeclareMathAccent{\widecheck}    {0}{mathx}{"71}


%% OPERATORS
\DeclareMathOperator\lan{Lan}
\DeclareMathOperator\ran{Ran}
\DeclareMathOperator\colim{colim}
\DeclareMathOperator\coeq{coeq}
\DeclareMathOperator\ob{ob}
\DeclareMathOperator\cod{cod}
\DeclareMathOperator\dom{dom}
\DeclareMathOperator\ev{ev}
\DeclareMathOperator\eq{eq}
\DeclareMathOperator\Tot{Tot}
\DeclareMathOperator\cosk{cosk}
\DeclareMathOperator\sk{sk}
\DeclareMathOperator\img{im}
\DeclareMathOperator\Spec{Spec}
\DeclareMathOperator\Ho{Ho}
\DeclareMathOperator\Aut{Aut}
\DeclareMathOperator\End{End}
\DeclareMathOperator\Hom{Hom}
\DeclareMathOperator\Map{Map}

%% ARROWS
% \to already exists
\newcommand{\too}[1][]{\ensuremath{\overset{#1}{\longrightarrow}}}
\newcommand{\ot}{\ensuremath{\leftarrow}}
\newcommand{\oot}[1][]{\ensuremath{\overset{#1}{\longleftarrow}}}
\let\toot\rightleftarrows
\let\otto\leftrightarrows
\let\Impl\Rightarrow
\let\imp\Rightarrow
\let\toto\rightrightarrows
\let\into\hookrightarrow
\let\xinto\xhookrightarrow
\mdef\we{\overset{\sim}{\longrightarrow}}
\mdef\leftwe{\overset{\sim}{\longleftarrow}}
\let\mono\rightarrowtail
\let\leftmono\leftarrowtail
\let\cof\rightarrowtail
\let\leftcof\leftarrowtail
\let\epi\twoheadrightarrow
\let\leftepi\twoheadleftarrow
\let\fib\twoheadrightarrow
\let\leftfib\twoheadleftarrow
\let\cohto\rightsquigarrow
\let\maps\colon
\newcommand{\spam}{\,:\!}       % \maps for left arrows
\def\acof{\mathrel{\mathrlap{\hspace{3pt}\raisebox{4pt}{$\scriptscriptstyle\sim$}}\mathord{\rightarrowtail}}}

% diagxy redefines \to, along with \toleft, \two, \epi, and \mon.

%% EXTENSIBLE ARROWS
\let\xto\xrightarrow
\let\xot\xleftarrow
% See Voss' Mathmode.tex for instructions on how to create new
% extensible arrows.
\def\rightarrowtailfill@{\arrowfill@{\Yright\joinrel\relbar}\relbar\rightarrow}
\newcommand\xrightarrowtail[2][]{\ext@arrow 0055{\rightarrowtailfill@}{#1}{#2}}
\let\xmono\xrightarrowtail
\let\xcof\xrightarrowtail
\def\twoheadrightarrowfill@{\arrowfill@{\relbar\joinrel\relbar}\relbar\twoheadrightarrow}
\newcommand\xtwoheadrightarrow[2][]{\ext@arrow 0055{\twoheadrightarrowfill@}{#1}{#2}}
\let\xepi\xtwoheadrightarrow
\let\xfib\xtwoheadrightarrow
% Let's leave the left-going ones until I need them.

%% EXTENSIBLE SLASHED ARROWS
% Making extensible slashed arrows, by modifying the underlying AMS code.
% Arguments are:
% 1 = arrowhead on the left (\relbar or \Relbar if none)
% 2 = fill character (usually \relbar or \Relbar)
% 3 = slash character (such as \mapstochar or \Mapstochar)
% 4 = arrowhead on the left (\relbar or \Relbar if none)
% 5 = display mode (\displaystyle etc)
\def\slashedarrowfill@#1#2#3#4#5{%
  $\m@th\thickmuskip0mu\medmuskip\thickmuskip\thinmuskip\thickmuskip
   \relax#5#1\mkern-7mu%
   \cleaders\hbox{$#5\mkern-2mu#2\mkern-2mu$}\hfill
   \mathclap{#3}\mathclap{#2}%
   \cleaders\hbox{$#5\mkern-2mu#2\mkern-2mu$}\hfill
   \mkern-7mu#4$%
}
% Here's the idea: \<slashed>arrowfill@ should be a box containing
% some stretchable space that is the "middle of the arrow".  This
% space is created as a "leader" using \cleader<thing>\hfill, which
% fills an \hfill of space with copies of <thing>.  Here instead of
% just one \cleader, we use two, with the slash in between (and an
% extra copy of the filler, to avoid extra space around the slash).
\def\rightslashedarrowfill@{%
  \slashedarrowfill@\relbar\relbar\mapstochar\rightarrow}
\newcommand\xslashedrightarrow[2][]{%
  \ext@arrow 0055{\rightslashedarrowfill@}{#1}{#2}}
\mdef\hto{\xslashedrightarrow{}}
\mdef\htoo{\xslashedrightarrow{\quad}}
\let\xhto\xslashedrightarrow

%% To get a slashed arrow in XYmatrix, do
% \[\xymatrix{A \ar[r]|-@{|} & B}\]
%% To get it in diagxy, do
% \morphism/{@{>}|-*@{|}}/[A`B;p]

%% Here is an \hto for diagxy:
% \def\htopppp/#1/<#2>^#3_#4{\:%
% \ifnum#2=0%
%    \setwdth{#3}{#4}\deltax=\wdth \divide \deltax by \ul%
%    \advance \deltax by \defaultmargin  \ratchet{\deltax}{100}%
% \else \deltax #2%
% \fi%
% \xy\ar@{#1}|-@{|}^{#3}_{#4}(\deltax,0) \endxy%
% \:}%
% \def\htoppp/#1/<#2>^#3{\ifnextchar_{\htopppp/#1/<#2>^{#3}}{\htopppp/#1/<#2>^{#3}_{}}}%
% \def\htopp/#1/<#2>{\ifnextchar^{\htoppp/#1/<#2>}{\htoppp/#1/<#2>^{}}}%
% \def\htoop/#1/{\ifnextchar<{\htopp/#1/}{\htopp/#1/<0>}}%
% \def\hto{\ifnextchar/{\htoop}{\htoop/>/}}%

% LABELED ISOMORPHISMS
\def\xiso#1{\mathrel{\mathrlap{\smash{\xto[\smash{\raisebox{1.3mm}{$\scriptstyle\sim$}}]{#1}}}\hphantom{\xto{#1}}}}
\def\toiso{\xto{\smash{\raisebox{-.5mm}{$\scriptstyle\sim$}}}}

% SHADOWS
\def\shvar#1#2{{\ensuremath{%
  \hspace{1mm}\makebox[-1mm]{$#1\langle$}\makebox[0mm]{$#1\langle$}\hspace{1mm}%
  {#2}%
  \makebox[1mm]{$#1\rangle$}\makebox[0mm]{$#1\rangle$}%
}}}
\def\sh{\shvar{}}
\def\scriptsh{\shvar{\scriptstyle}}
\def\bigsh{\shvar{\big}}
\def\Bigsh{\shvar{\Big}}
\def\biggsh{\shvar{\bigg}}
\def\Biggsh{\shvar{\Bigg}}

%% Paul Taylor: noncommutative diagrams
\def\puncture{
  \begingroup
  \setbox0\hbox{A}%
  \vrule height.53\ht0 depth-.47\ht0 width.35\ht0
  \kern .12\ht0
  \vrule height\ht0 depth-.65\ht0 width.06\ht0
  \kern-.06\ht0
  \vrule height.35\ht0 depth0pt width.06\ht0
  \kern .12\ht0
  \vrule height.53\ht0 depth-.47\ht0 width.35\ht0
  \endgroup
}

% TYPING JUDGMENTS
% Call this macro as \jd{x:A, y:B |- c:C}.  It adds (what I think is)
% appropriate spacing, plus auto-sized parentheses around each typing judgment.
\def\jd#1{\@jd#1\ej}
\def\@jd#1|-#2\ej{\@@jd#1,,\;\vdash\;\left(#2\right)}
\def\@@jd#1,{\@ifmtarg{#1}{\let\next=\relax}{\left(#1\right)\let\next=\@@@jd}\next}
\def\@@@jd#1,{\@ifmtarg{#1}{\let\next=\relax}{,\,\left(#1\right)\let\next=\@@@jd}\next}
% Here's a version which puts a line break before the turnstyle.
\def\jdm#1{\@jdm#1\ej}
\def\@jdm#1|-#2\ej{\@@jd#1,,\\\vdash\;\left(#2\right)}
% Make an actual comma that doesn't separate typing judgments (e.g. A,B,C : Type).
\def\cm{,}

% 2-(CO)MONAD STUFF
\def\alg{\text{-}\mathcal{A}\mathit{lg}}
\def\algs{\text{-}\mathcal{A}\mathit{lg}_s}
\def\algl{\text{-}\mathcal{A}\mathit{lg}_l}
\def\algc{\text{-}\mathcal{A}\mathit{lg}_c}
\def\algw{\text{-}\mathcal{A}\mathit{lg}_w}
\def\psalg{\text{-}\mathcal{P}\mathit{s}\mathcal{A}\mathit{lg}}
\def\dalg{\text{-}\mathbb{A}\mathsf{lg}}
\def\coalg{\text{-}\mathcal{C}\mathit{oalg}}
\def\coalgs{\text{-}\mathcal{C}\mathit{oalg}_s}
\def\coalgl{\text{-}\mathcal{C}\mathit{oalg}_l}
\def\coalgc{\text{-}\mathcal{C}\mathit{oalg}_c}
\def\coalgw{\text{-}\mathcal{C}\mathit{oalg}_w}
\def\pscoalg{\text{-}\mathcal{P}\mathit{s}\mathcal{C}\mathit{oalg}}
\def\dcoalg{\text{-}\mathbb{C}\mathsf{oalg}}

%% SKIPIT in TikZ
% See http://tex.stackexchange.com/questions/3513/draw-only-some-segments-of-a-path-in-tikz
\long\def\my@drawfill#1#2;{%
\@skipfalse
\fill[#1,draw=none] #2;
\@skiptrue
\draw[#1,fill=none] #2;
}
\newif\if@skip
\newcommand{\skipit}[1]{\if@skip\else#1\fi}
\newcommand{\drawfill}[1][]{\my@drawfill{#1}}

%% TWOCELLS AND PULLBACKS in TIKZ-CD
\newcounter{nodemaker}
\setcounter{nodemaker}{0}
\newcommand{\twocell}[2][]{%
  \global\edef\mynodeone{twocell\arabic{nodemaker}}%
  \stepcounter{nodemaker}%
  \global\edef\mynodetwo{twocell\arabic{nodemaker}}%
  \stepcounter{nodemaker}%
  \ar[#2,phantom,shift left=3,""{name=\mynodeone}]%
  \ar[#2,phantom,shift right=3,""'{name=\mynodetwo}]%
  \ar[Rightarrow,from=\mynodeone,to=\mynodetwo,"{#1}"]%
}
\newcommand{\twocellop}[2][]{%
  \global\edef\mynodeone{twocell\arabic{nodemaker}}%
  \stepcounter{nodemaker}%
  \global\edef\mynodetwo{twocell\arabic{nodemaker}}%
  \stepcounter{nodemaker}%
  \ar[#2,phantom,shift left=3,""{name=\mynodeone}]%
  \ar[#2,phantom,shift right=3,""'{name=\mynodetwo}]%
  \ar[Rightarrow,from=\mynodetwo,to=\mynodeone,"{#1}"]%
}
\newcommand{\drpullback}[1][dr]{\ar[#1,phantom,near start,"\lrcorner"]}
\newcommand{\dlpullback}[1][dl]{\ar[#1,phantom,near start,"\llcorner"]}
\newcommand{\urpullback}[1][ur]{\ar[#1,phantom,near start,"\urcorner"]}
\newcommand{\ulpullback}[1][ul]{\ar[#1,phantom,near start,"\ulcorner"]}


%%%% THEOREM-TYPE ENVIRONMENTS, hacked to
%%% (a) number all with the same numbers, and
%%% (b) have the right names.
%% The following code should work in TAC or out of it, and with
%% hyperref or without it.  In all cases, we use \label to label
%% things and \autoref to refer to them (ordinary \ref declines to
%% include names).  The non-hyperref label and reference hack is from
%% Mike Mandell, I believe.
\newif\ifhyperref
\@ifpackageloaded{hyperref}{\hyperreftrue}{\hyperreffalse}
\iftac
% TODO: In TAC style, \currthmtype isn't getting set by the theorem environments.  They need to be a \your@thing *followed* by a \gdef\currthmtype{#2}.

  %% In the TAC style, all theorems are actually subsections.  So
  %% the hyperref \autoref doesn't work and we have to use our own code
  %% in any case.  We also have to hook into the \state macros instead
  %% of \@thm since those are what know about the current theorem type.
  \let\your@state\state
  \def\state#1{\my@state#1}
  \def\my@state#1.{\gdef\currthmtype{#1}\your@state{#1.}}
  \let\your@staterm\staterm
  \def\staterm#1{\my@staterm#1}
  \def\my@staterm#1.{\gdef\currthmtype{#1}\your@staterm{#1.}}
  \let\@defthm\newtheorem
  \def\switchtotheoremrm{\let\@defthm\newtheoremrm}
  \def\defthm#1#2#3{\@defthm{#1}{#2}} % Ignore the third argument (for cleveref only)
  % The following allows us to use \cref for sections, subsections,
  % and figures too, as if it were cleveref.  (But not for multiple
  % references at once.)
  \let\your@section\section
  \def\section{\gdef\currthmtype{section}\your@section}
  \let\your@subsection\subsection
  \def\subsection{\gdef\currthmtype{subsection}\your@subsection}
  \let\your@figure\figure
  \def\figure{\gdef\currthmtype{Figure}\your@figure}
  % Start out \currthmtype as empty
  \def\currthmtype{}
  % In a bit, we're going to redefine \label so that \label{athm} will
  % also make a label "label@name@athm" which is the current value of
  % \currthmtype.  Now \autoref{athm} just adds a reference to this in
  % front of the reference.
  \ifhyperref
    \def\autoref#1{\ref*{label@name@#1}~\ref{#1}}
  \else
    \def\autoref#1{\ref{label@name@#1}~\ref{#1}}
  \fi
  % This has to go AFTER the \begin{document} because apparently
  % hyperref resets the definition of \label at that point.
  \AtBeginDocument{%
    % Save the old definition of \label
    \let\old@label\label%
    % Redefine \label so that \label{athm} will also make a label
    % "label@name@athm" which is the current value of \currthmtype.
    \def\label#1{%
      {\let\your@currentlabel\@currentlabel%
        \edef\@currentlabel{\currthmtype}%
        \old@label{label@name@#1}}%
      \old@label{#1}}
  }
  \let\cref\autoref
\else\ifcref
  % Cleveref does most of it for us.
  \def\defthm#1#2#3{%
    %% Ensure all theorem types are numbered with the same counter
    \newaliascnt{#1}{thm}
    \newtheorem{#1}[#1]{#2}
    \aliascntresetthe{#1}
    %% This command tells cleveref's \cref what to call things
    \crefname{#1}{#2}{#3}% following brace must be on separate line to support poorman cleveref sed file
  }
  % \let\autoref\cref  % May want to use \autoref for xr-ed links
\else
  % In non-TAC styles without cleveref, theorems have their own counters and so the
  % hyperref \autoref works, if hyperref is loaded.
  \ifhyperref
    %% If we have hyperref, then we have to make sure all the theorem
    %% types appear to use different counters so that hyperref can tell
    %% them apart.  However, we want them actually to use the same
    %% counter, so we don't have both Theorem 9.1 and Definition 9.1.
    \def\defthm#1#2#3{% Ignore the third argument (for cleveref only)
      %% All types of theorems are number inside sections
      \newtheorem{#1}{#2}[section]%
      %% This command tells hyperref's \autoref what to call things
      \expandafter\def\csname #1autorefname\endcsname{#2}%
      %% This makes all the theorem counters actually the same counter
      \expandafter\let\csname c@#1\endcsname\c@thm}
  \else
    %% Without hyperref, we have to roll our own.  This code is due to
    %% Mike Mandell.  First, all theorems can now "officially" use the
    %% same counter.
    \def\defthm#1#2#3{\newtheorem{#1}[thm]{#2}} % Ignore the third argument (for cleveref only)
    %% Save the label- and theorem-making commands
    \ifx\SK@label\undefined\let\SK@label\label\fi
    \let\old@label\label
    \let\your@thm\@thm
    %% Save the current type of theorem whenever we start one
    \def\@thm#1#2#3{\gdef\currthmtype{#3}\your@thm{#1}{#2}{#3}}
    %% Start that out as empty
    \def\currthmtype{}
    %% Redefine \label so that \label{athm} defines, in addition to a
    %% label "athm" pointing to "9.1," a label "athm@" pointing to
    %% "Theorem 9.1."
    \def\label#1{{\let\your@currentlabel\@currentlabel\def\@currentlabel%
        {\currthmtype~\your@currentlabel}%
        \SK@label{#1@}}\old@label{#1}}
    %% Now \autoref just looks at "athm@" instead of "athm."
    \def\autoref#1{\ref{#1@}}
  \fi
  \let\cref\autoref
\fi\fi

%% Now the code that works in all cases.  Note that TAC allows the
%% optional arguments, but ignores them.  It also defines environments
%% called "theorem," etc.
\newtheorem{thm}{Theorem}[section]
\ifcref
  \crefname{thm}{Theorem}{Theorems}
\else
  \newcommand{\thmautorefname}{Theorem}
\fi
\defthm{cor}{Corollary}{Corollaries}
\defthm{prop}{Proposition}{Propositions}
\defthm{lem}{Lemma}{Lemmas}
\defthm{sch}{Scholium}{Scholia}
\defthm{assume}{Assumption}{Assumptions}
\defthm{claim}{Claim}{Claims}
\defthm{conj}{Conjecture}{Conjectures}
\defthm{hyp}{Hypothesis}{Hypotheses}
\iftac\switchtotheoremrm\else\theoremstyle{definition}\fi
\defthm{defn}{Definition}{Definitions}
\defthm{notn}{Notation}{Notations}
\iftac\switchtotheoremrm\else\theoremstyle{remark}\fi
\defthm{rmk}{Remark}{Remarks}
\defthm{eg}{Example}{Examples}
\defthm{egs}{Examples}{Examples}
\defthm{ex}{Exercise}{Exercises}
\defthm{ceg}{Counterexample}{Counterexamples}

\ifcref
  % Display format for sections
  \crefformat{section}{\S#2#1#3}
  \Crefformat{section}{Section~#2#1#3}
  \crefrangeformat{section}{\S\S#3#1#4--#5#2#6}
  \Crefrangeformat{section}{Sections~#3#1#4--#5#2#6}
  \crefmultiformat{section}{\S\S#2#1#3}{ and~#2#1#3}{, #2#1#3}{ and~#2#1#3}
  \Crefmultiformat{section}{Sections~#2#1#3}{ and~#2#1#3}{, #2#1#3}{ and~#2#1#3}
  \crefrangemultiformat{section}{\S\S#3#1#4--#5#2#6}{ and~#3#1#4--#5#2#6}{, #3#1#4--#5#2#6}{ and~#3#1#4--#5#2#6}
  \Crefrangemultiformat{section}{Sections~#3#1#4--#5#2#6}{ and~#3#1#4--#5#2#6}{, #3#1#4--#5#2#6}{ and~#3#1#4--#5#2#6}
  % Display format for appendices
  \crefformat{appendix}{Appendix~#2#1#3}
  \Crefformat{appendix}{Appendix~#2#1#3}
  \crefrangeformat{appendix}{Appendices~#3#1#4--#5#2#6}
  \Crefrangeformat{appendix}{Appendices~#3#1#4--#5#2#6}
  \crefmultiformat{appendix}{Appendices~#2#1#3}{ and~#2#1#3}{, #2#1#3}{ and~#2#1#3}
  \Crefmultiformat{appendix}{Appendices~#2#1#3}{ and~#2#1#3}{, #2#1#3}{ and~#2#1#3}
  \crefrangemultiformat{appendix}{Appendices~#3#1#4--#5#2#6}{ and~#3#1#4--#5#2#6}{, #3#1#4--#5#2#6}{ and~#3#1#4--#5#2#6}
  \Crefrangemultiformat{appendix}{Appendices~#3#1#4--#5#2#6}{ and~#3#1#4--#5#2#6}{, #3#1#4--#5#2#6}{ and~#3#1#4--#5#2#6}
  \crefformat{subappendix}{\S#2#1#3}
  \Crefformat{subappendix}{Section~#2#1#3}
  \crefrangeformat{subappendix}{\S\S#3#1#4--#5#2#6}
  \Crefrangeformat{subappendix}{Sections~#3#1#4--#5#2#6}
  \crefmultiformat{subappendix}{\S\S#2#1#3}{ and~#2#1#3}{, #2#1#3}{ and~#2#1#3}
  \Crefmultiformat{subappendix}{Sections~#2#1#3}{ and~#2#1#3}{, #2#1#3}{ and~#2#1#3}
  \crefrangemultiformat{subappendix}{\S\S#3#1#4--#5#2#6}{ and~#3#1#4--#5#2#6}{, #3#1#4--#5#2#6}{ and~#3#1#4--#5#2#6}
  \Crefrangemultiformat{subappendix}{Sections~#3#1#4--#5#2#6}{ and~#3#1#4--#5#2#6}{, #3#1#4--#5#2#6}{ and~#3#1#4--#5#2#6}
  % Display format for parts
  \crefname{part}{Part}{Parts}
  % Display format for figures
  \crefname{figure}{Figure}{Figures}
\fi

% Knowledge package
\usepackage[quotation,protect quotation=tikzcd,notion]{knowledge}
% Hack the knowledge "notion" command to introduce things in boldface
\ifKnowledgePaperMode
  \knowledgestyle*{intro notion}{boldface}
\fi
\ifKnowledgeCompositionMode
  \ifXcolor
    \knowledgestyle*{intro notion}{boldface,color={blue}}
  \else
    \knowledgestyle*{intro notion}{boldface, underline}
  \fi
\fi
\ifKnowledgeElectronicMode
  \ifXcolor
    \knowledgestyle*{intro notion}{boldface,color={blue}}
  \else
    \knowledgestyle*{intro notion}{boldface, underline}
 \fi
\fi

% \qedhere for TAC
\iftac
  \let\qed\endproof
  \let\your@endproof\endproof
  \def\my@endproof{\your@endproof}
  \def\endproof{\my@endproof\gdef\my@endproof{\your@endproof}}
  \def\qedhere{\tag*{\endproofbox}\gdef\my@endproof{\relax}}
\fi

% Make the optional arguments to TAC's \proof behave like everyone else's
\iftac
  \def\pr@@f[#1]{\subsubsection*{\sc #1.}}
\fi

% How to get QED symbols inside equations at the end of the statements
% of theorems.  AMS LaTeX knows how to do this inside equations at the
% end of *proofs* with \qedhere, and at the end of the statement of a
% theorem with \qed (meaning no proof will be given), but it can't
% seem to combine the two.  Use this instead.
\def\thmqedhere{\expandafter\csname\csname @currenvir\endcsname @qed\endcsname}

% Number numbered lists as (i), (ii), ...
\ifbeamer\else
  \renewcommand{\theenumi}{(\roman{enumi})}
  \renewcommand{\labelenumi}{\theenumi}
\fi

% Left margins for enumitem
\ifbeamer\else
  \setitemize[1]{leftmargin=2em}
  \setenumerate[1]{leftmargin=*}
\fi

% Also number formulas with the theorem counter
\iftac
  \let\c@equation\c@subsection
\else
  \let\c@equation\c@thm
\fi
\numberwithin{equation}{section}

% Only show numbers for equations that are actually referenced (or
% whose tags are specified manually) - uses mathtools.  All equations
% need to be referenced with \eqref, not \ref, for this to work!
\ifcref\else
  \@ifpackageloaded{mathtools}{\mathtoolsset{showonlyrefs,showmanualtags}}{}
\fi

% GREEK LETTERS, ETC.
\alwaysmath{alpha}
\alwaysmath{beta}
\alwaysmath{gamma}
\alwaysmath{Gamma}
\alwaysmath{delta}
\alwaysmath{Delta}
\alwaysmath{epsilon}
\mdef\ep{\varepsilon}
\alwaysmath{zeta}
\alwaysmath{eta}
\alwaysmath{theta}
\alwaysmath{Theta}
\alwaysmath{iota}
\alwaysmath{kappa}
\alwaysmath{lambda}
\alwaysmath{Lambda}
\alwaysmath{mu}
\alwaysmath{nu}
\alwaysmath{xi}
\alwaysmath{pi}
\alwaysmath{rho}
\alwaysmath{sigma}
\alwaysmath{Sigma}
\alwaysmath{tau}
\alwaysmath{upsilon}
\alwaysmath{Upsilon}
\alwaysmath{phi}
\alwaysmath{Pi}
\alwaysmath{Phi}
\mdef\ph{\varphi}
\alwaysmath{chi}
\alwaysmath{psi}
\alwaysmath{Psi}
\alwaysmath{omega}
\alwaysmath{Omega}
\let\al\alpha
\let\be\beta
\let\gm\gamma
\let\Gm\Gamma
\let\de\delta
\let\De\Delta
\let\si\sigma
\let\Si\Sigma
\let\om\omega
\let\ka\kappa
\let\la\lambda
\let\La\Lambda
\let\ze\zeta
\let\th\theta
\let\Th\Theta
\let\vth\vartheta
\let\Om\Omega

%% Include or exclude solutions
% This code is basically copied from version.sty, except that when the
% solutions are included, we put them in a `proof' environment as
% well.  To include solutions, say \includesolutions; to exclude them
% say \excludesolutions.
% \begingroup
% 
% \catcode`{=12\relax\catcode`}=12\relax%
% \catcode`(=1\relax \catcode`)=2\relax%
% \gdef\includesolutions(\newenvironment(soln)(\begin(proof)[Solution])(\end(proof)))%
% \gdef\excludesolutions(%
%   \gdef\soln(\@bsphack\catcode`{=12\relax\catcode`}=12\relax\soln@NOTE)%
%   \long\gdef\soln@NOTE##1\end{soln}(\solnEND@NOTE)%
%   \gdef\solnEND@NOTE(\@esphack\end(soln))%
% )%
% \endgroup

\makeatother

% Local Variables:
% mode: latex
% TeX-master: ""
% End:

\title{Categorical Semantics of Dependent Types}
\author{Frank Tsai}
\date{\today}
% \thanks{}
\begin{document}
\maketitle
\tableofcontents

\newcommand{\Ty}{\ensuremath{\mathcal{U}}}
\newcommand{\Tm}{\ensuremath{\widetilde{\mathcal{U}}}}
\newcommand{\El}{\ensuremath{\mathrm{El}}}
\newcommand{\app}{\ensuremath{\mathrm{App}}}
\newcommand{\Block}[1]{\paragraph{\fbox{\textbf{#1}}}}
\newcommand{\ctxt}{\ensuremath{~\mathsf{ctxt}}}
\newcommand{\type}{\ensuremath{~\mathsf{type}}}
\newcommand{\todo}{{\color{red}{TODO}}}

\newcommand{\Fam}{\ensuremath{\mathsf{Fam}}}
\newcommand{\Cop}{\ensuremath{\mathsf{op}}}

\section{Introduction}
\label{sec:introduction}
This is my personal note for \cite{hofmann:ssdts}, \cite{awodey:natmodels}, and \cite{coraglia:context-judgment-deduction}.
The goal of this note is to survey some existing approaches to categorical semantics in the literature, and hopefully explore the differences between and unity in them.

Categorical semantics provides a framework for semantic investigation of dependent types.
A concrete mathematical model can be identified from an abstract one constructed in the categorical framework.

To develop categorical semantics, certain structures are postulated to avoid distracting ourselves from the task at hand.
As a result, certain complex operations, such as substitution, that one encounters when working with a particular presentation of type theory become primitive ones.

\section{Dependent Type Theory}
\label{sec:dependent-type-theory}

The notion of type theory itself is informal.
However, most would probably agree that a dependent type theory is (informally) a formal system dealing with \emph{types} and \emph{terms} in a \emph{context}, and consisting of the following data:
\begin{itemize}
\item Syntax: a collection of glyphs that make up the atoms of the language.
  These are things such as $\Gamma, A$ or $a : A$.
\item Judgments: sentences that bind together atomic data from the syntax.
  These sentences are intended to describe the status of the given atomic data or how they are related.
  For example,
  \begin{mathpar}
    \vdash \Gamma \ctxt \and \Gamma \vdash A \type \and \Gamma \vdash a : A \and \Gamma \vdash A = B \type \and \Gamma \vdash a = b : A
  \end{mathpar}
  can be informally read as ``$\Gamma$ is a context'', ``$A$ is a type in context $\Gamma$'', ``$a$ has type $A$ in context $\Gamma$'', ``$A$ and $B$ are definitionally equal types in context $\Gamma$'', and ``$a$ and $b$ are definitionally equal terms of type $A$ in context $\Gamma$'', respectively.
\item Rules: a mapping from judgments to judgments.
  The intention of rules is to specify when a new judgment can be made given that some other judgments have already been made.
  For example,
  \begin{mathpar}
    \inferrule*[left=(dTy)]
    { \Gamma \vdash a : A \\ \Gamma.A \vdash B \type }
    { \Gamma \vdash B[a] \type }
  \end{mathpar}
\end{itemize}

Contexts play an important role in dependent type theory.
They can be thought of as the content of a \emph{where} clause in English.
For example, ``$\mathrm{Vec}(x)$ is a type where $x$ is a natural number.''
Then the rule (\textsc{dTy}) yields a new type, say $\mathrm{Vec}(7)$, by substituting $7$ for $x$.

Most type theories that we will discuss feature the following rules regarding contexts and variables.
\begin{mathpar}
  \inferrule*[left=(Empty)]
  { }
  { \diamond \ctxt }
  \and
  \inferrule*[left=(Ctx-Extend)]
  { \Gamma \ctxt \\ \Gamma \vdash A \type }
  { \Gamma, x : A \ctxt }
  \and
  \inferrule*[left=(Var)]
  { \Gamma, x : A \ctxt }
  { \Gamma, x : A \vdash x : A }
\end{mathpar}
These rules should be natural to computer scientists especially the last rule, which states that the variable $x$ has type $A$ when it has been declared to have type $A$.

\subsection{Type Formers}
\label{sec:type-formers}

Type theories can be extended with new types and new terms.
The general scheme for adding new types and new terms is to specify:
\begin{itemize}
\item Formation rules: how to form a new type of this kind.
  For example, given a type $B$ in context $\Gamma.A$, we can form a function type $A \to B$.
\item Introduction rules: how to form a new term of that type.
  For example, given a term $b$ of type $B$ in context $\Gamma.A$, we can form an element of type $A \to B$ via $\lambda$-abstraction: $\lambda_{A}B$.
\item Elimination rules: how to use an element of that type.
  For example, given a dependent function $f$ of type $A \to B$ and a term $a$ of type $A$, we can form an element of type $B$ via function application: $f(a)$.
\item Computation rules ($\beta$-rules): how elimination rules act on introduction rules.
  For example, $(\lambda_{A}b)(a)$ is definitionally equal to $b[a]$.
\item Extensionality principle ($\eta$-rules): how introduction rules act on elimination rules.
  For example, a function $f$ is definitionally equal to $\lambda_{A}(f~x)$.
\end{itemize}

Some quintessential examples of type formers are $\Pi$-types, $\Sigma$-types, $\Id$-types, and universes.
\begin{mathpar}
  \inferrule*
  { \Gamma \vdash A \type \\ \Gamma.A \vdash B \type }
  { \Gamma \vdash \prod_{A}B \type }
  \and
  \inferrule*
  { \Gamma \vdash A \type \\ \Gamma.A \vdash B \type }
  { \Gamma \vdash \sum_{A}B \type }
  \and
  \inferrule*
  { \Gamma \vdash a : A \\ \Gamma \vdash b : A }
  { \Gamma \vdash \Id(a, b) \type }
  \and
  \inferrule*
  { }
  { \Gamma \vdash \cU \type }
\end{mathpar}

\subsection{Examples of Type Theories}
\label{sec:examples-of-type-theories}

In this section, we record a number of type theories.

\subsubsection{Martin-L\"{o}f Type Theory}
\label{sec:martin-lof-type-theory}

Martin-L\"{o}f Type Theory is a collective name for dependent type theories that include several type formers mentioned in \cref{sec:type-formers}, but not a universe closed under impredicative quantification.

\subsubsection{The Logical Framework}
\label{sec:logical-framework}

Martin-L\"{o}f's logical framework is a type theory with $\Pi$-types and a universe.
Its intended use is to define theories as extensions of the LF by constants and equations.
The LF can also be used to encode the syntax of other logical systems, e.g., predicate logic and modal logic.

\subsubsection{The Calculus of Construction}
\label{sec:calculus-of-construction}

The calculus of construction (CoC) is a type theory with $\Pi$-types and a universe $\mathsf{Prop}$ closed under impredicative quantification.
Datatypes defined within $\mathsf{Prop}$ lack induction principles.

\section{Substitution}
\label{sec:substitution}
Write $\Gamma \vdash A \type$ for an element $A$ of the set of semantic types under the semantic context $\Gamma$, and $\Gamma \vdash a : A$ for an element $a$ of the set of semantic terms of type $A$.
For any substitution $\sigma : \Delta \to \Gamma$, write $\blank[\sigma]$ for the action of substitution with $\sigma$ on both types and terms.
We impose the following axioms.
\begin{mathpar}
  A[\sigma \circ \xi] = (A[\sigma])[\xi]
  \and
  a[\sigma \circ \xi] = (a[\sigma])[\xi]
  \and
  A[\id] = A
  \and
  a[\id] = a
\end{mathpar}
These equations correspond to the familiar properties of the inductively defined substitution.
\begin{mathpar}
  A[t[u/x]/x] = (A[t/x])[u/x]
  \and
  a[t[u/x]/x] = (a[t/x])[u/x]
  \and
  A[x/x] = A
  \and
  a[x/x] = a
\end{mathpar}

\section{Category with Families}
\label{sec:category-with-families}

\begin{defn}[Category of families of sets]
  The category $\iF$ of families of sets consists of
  \begin{itemize}
  \item Objects: an indexed family of sets $(I, A_{i \in I})$ where for each $i \in I$, $A_{i}$ is a set.
  \item Morphisms: a morphism $f : (I, A_{i \in I}) \to (J, B_{j \in J})$ consists of a function $f : I \to J$, and an $I$-indexed family of functions $f_{i} : A_{i} \to B_{f(i)}$.
  \end{itemize}
  Identities and compositions are defined in an obvious way.
\end{defn}

\begin{defn}[Category with families]
  A category with families consists of
  \begin{itemize}
  \item A base category $\iC$, whose objects are called contexts and whose morphisms are called substitutions.
  \item A terminal object $\diamond \in \iC$, called the empty context.
  \item A functor $T : \iC\op \to \iF$, mapping each context $\Gamma$ to a family of sets $(\Ty(\Gamma), \Tm_{A \in \Ty(\Gamma)})$, where the set $\Ty(\Gamma)$ is the set of types in context $\Gamma$, and for each $A \in \Ty(\Gamma)$ the set $\Tm_{A}$ is the set of terms of type $A$ in context $\Gamma$.
    For each substitution $\sigma : \Delta \to \Gamma$, $T$ induces a morphism $T(\sigma) : (\Ty(\Gamma), \Tm_{A \in \Ty(\Gamma)}) \to (\Ty(\Delta), \Tm_{A \in \Ty(\Delta)})$, which consists of a function $\blank[\sigma] : \Ty(\Gamma) \to \Ty(\Delta)$, and a family of functions $\blank[\sigma]_{A} : \Tm_{A} \to \Tm_{A[\sigma]}$.
    These functions are the substitution with $\sigma$ on types and on terms, respectively.
    We omit the subscript in the substitution on terms whenever possible.
  \item For each $\Gamma \in \iC$ and each $A \in \Ty(\Gamma)$, there is a context extension $\Gamma.A \in \iC$, a morphism $p_{A} : \Gamma.A \to \Gamma$, and a term $q_{A} \in \Tm(\Gamma.A)_{A[p_{A}]}$ with the universal property that for any $\Delta \in \iC$, any $\sigma : \Delta \to \Gamma$, and any $a \in \Tm(\Delta)_{A[\sigma]}$, there is a unique morphism $(\sigma, a) : \Delta \to \Gamma.A$ such that $p_{A} \circ (\sigma, a) = \sigma$ and that $q_{A}[(\sigma, a)] = a$.
    % https://q.uiver.app/#q=WzAsMyxbMCwyLCJcXEdhbW1hIl0sWzAsMCwiXFxHYW1tYS5BIl0sWzIsMiwiXFxEZWx0YSJdLFsxLDAsInBfe0F9IiwyXSxbMiwwLCJcXHNpZ21hIl0sWzIsMSwiKFxcc2lnbWEsYSkiLDIseyJzdHlsZSI6eyJib2R5Ijp7Im5hbWUiOiJkYXNoZWQifX19XV0=
\[\begin{tikzcd}
	{\Gamma.A} \\
	\\
	\Gamma && \Delta
	\arrow["{p_{A}}"', from=1-1, to=3-1]
	\arrow["\sigma", from=3-3, to=3-1]
	\arrow["{(\sigma,a)}"', dashed, from=3-3, to=1-1]
\end{tikzcd}\]
  \end{itemize}
\end{defn}

Any CwF $\iC$ has the structure of dependent type theory described in \S \ref{sec:dependent-type-theory}.
The terminal object $\diamond \in \iC$ models the empty context.
Context extension and its universal property model the context extension rule and the variable rule.
Additionally, substitutions satisfy the required equations.
This follows directly by functoriality.
For $\sigma : \Delta \to \Gamma$ and $\xi : \Xi \to \Delta$, we have $T(\sigma \circ \xi) = T(\xi) \circ T(\sigma)$.
The left-hand side consists of a function $\blank[\sigma \circ \xi] : \Ty(\Gamma) \to \Ty(\Xi)$, which equals $(\blank[\sigma])[\xi]$ by the right-hand side, and a family of functions $\blank[\sigma \circ \xi]_{A} : \Tm(\Gamma) \to \Tm(\Xi)$, which equals $(\blank[\sigma]_{A[\xi]})[\xi]_{A}$ by the right-hand side.
The other two equations, $A[\id] = A$ and $a[\id] = a$, also follow by functoriality.


\subsection{Terms and Sections}
\label{sec:cwf-terms-and-sections}
Let $\Gamma \in \iC$ and $\Gamma \vdash A \type$.
There is a one-to-one correspondence between the set of sections of $p_{A}$ and the set of terms $\Tm(\Gamma)_{A}$.
Let $\Gamma \vdash a : A$, then evidently $\Gamma \vdash a : A[\id]$.
The universal property for context extension then states that there is a unique section $(\id, a) : \Gamma \to \Gamma.A$.
\[
  \begin{array}{r@{~:~}ccc}
    \varphi & \Tm(\Gamma)_{A} & \to & \mathrm{Section}(p_{A})\\
    \varphi & a & \mapsto & (\id, a)
  \end{array}
\]
Conversely, for any section $\sigma$ of $p_{A}$, there is a term
\[
  q_{A}[\sigma] \in \Tm(\Gamma)_{(A[p_{A}])[\sigma]} = \Tm(\Gamma)_{A[p_{A} \circ \sigma]} = \Tm(\Gamma)_{A}
\]
Thus, we define
\[
  \begin{array}{r@{~:~}ccc}
    \psi & \mathrm{Section}(p_{A}) & \to & \Tm(\Gamma)_{A}\\
    \psi & \sigma & \mapsto & q_{A}[\sigma]
  \end{array}
\]
The two functions are mutual inverses by the universal property of context extension.
Thus, the set of terms of type $A$ in context $\Gamma$ can be viewed as the set of sections of the morphism $p_{A}$.
% https://q.uiver.app/#q=WzAsMixbMCwzLCJcXEdhbW1hIl0sWzAsMCwiXFxHYW1tYS5BIl0sWzEsMCwicF97QX0iLDFdLFswLDEsImFfezJ9IiwxLHsiY3VydmUiOi0zfV0sWzAsMSwiYV97MX0iLDAseyJjdXJ2ZSI6LTV9XSxbMCwxLCJhX3szfSIsMSx7ImN1cnZlIjozfV0sWzAsMSwiYV97NH0iLDIseyJjdXJ2ZSI6NX1dXQ==
\[\begin{tikzcd}
	{\Gamma.A} \\
	\\
	\\
	\Gamma
	\arrow["{p_{A}}"{description}, from=1-1, to=4-1]
	\arrow["{a_{2}}"{description}, curve={height=-18pt}, from=4-1, to=1-1]
	\arrow["{a_{1}}", curve={height=-30pt}, from=4-1, to=1-1]
	\arrow["{a_{3}}"{description}, curve={height=18pt}, from=4-1, to=1-1]
	\arrow["{a_{4}}"', curve={height=30pt}, from=4-1, to=1-1]
\end{tikzcd}\]

\subsection{Weakening}
\label{sec:cwf-weakening}
Suppose that $\sigma : \Delta \to \Gamma$ and that $\Gamma \vdash A \type$.
Then $\Delta \vdash A[\sigma] \type$.
Thus, there is a term $\Delta.A[\sigma] \vdash q_{A[\sigma]} : A[\sigma \circ p_{A[\sigma]}]$.
The universal property of context extension gives a substitution $\Delta.A[\sigma] \to \Gamma.A$ as indicated in
% https://q.uiver.app/#q=WzAsNCxbNCwzLCJcXERlbHRhLkFbXFxzaWdtYV0iXSxbMiwzLCJcXERlbHRhIl0sWzAsMywiXFxHYW1tYSJdLFswLDAsIlxcR2FtbWEuQSJdLFsxLDIsIlxcc2lnbWEiXSxbMCwxLCJwX3tBW1xcc2lnbWFdfSJdLFszLDIsInBfe0F9IiwyXSxbMCwzLCIoXFxzaWdtYSBcXGNpcmMgcF97QVtcXHNpZ21hXX0sIHFfe0FbXFxzaWdtYV19KSIsMix7InN0eWxlIjp7ImJvZHkiOnsibmFtZSI6ImRhc2hlZCJ9fX1dXQ==
\[\begin{tikzcd}
	{\Gamma.A} \\
	\\
	\\
	\Gamma && \Delta && {\Delta.A[\sigma]}
	\arrow["\sigma", from=4-3, to=4-1]
	\arrow["{p_{A[\sigma]}}", from=4-5, to=4-3]
	\arrow["{p_{A}}"', from=1-1, to=4-1]
	\arrow["{(\sigma \circ p_{A[\sigma]}, q_{A[\sigma]})}"', dashed, from=4-5, to=1-1]
\end{tikzcd}\]
We call such a substitution a weakening substitution, and denote it as $\sigma_{A}$.
In the special case where $\sigma$ is $\id_{\Gamma}$, the weakening substitution is the identity.
% https://q.uiver.app/#q=WzAsNCxbNCwzLCJcXEdhbW1hLkEiXSxbMiwzLCJcXEdhbW1hIl0sWzAsMywiXFxHYW1tYSJdLFswLDAsIlxcR2FtbWEuQSJdLFsxLDIsIlxcaWRfe1xcR2FtbWF9Il0sWzAsMSwicF97QX0iXSxbMywyLCJwX3tBfSIsMl0sWzAsMywiKHBfe0F9LCBxX3tBfSkgPSBcXGlkX3tcXEdhbW1hLkF9IiwyLHsic3R5bGUiOnsiYm9keSI6eyJuYW1lIjoiZGFzaGVkIn19fV1d
\[\begin{tikzcd}
	{\Gamma.A} \\
	\\
	\\
	\Gamma && \Gamma && {\Gamma.A}
	\arrow["{\id_{\Gamma}}", from=4-3, to=4-1]
	\arrow["{p_{A}}", from=4-5, to=4-3]
	\arrow["{p_{A}}"', from=1-1, to=4-1]
	\arrow["{(p_{A}, q_{A}) = \id_{\Gamma.A}}"', dashed, from=4-5, to=1-1]
\end{tikzcd}\]
And in the special case where $\sigma$ is $p_{A}$, the weakening substitution adds a new variable of type $A$ to the context.
In dependent type theory, types in a context can mention variables declared prior to it.
The operation $\blank[p_{A}]$ ``up-shifts'' the variables so that the newly added type $A$ mentions the same variables as the already existing type $A$.
% https://q.uiver.app/#q=WzAsNCxbNCwzLCJcXEdhbW1hLkEuQVtwX3tBfV0iXSxbMiwzLCJcXEdhbW1hLkEiXSxbMCwzLCJcXEdhbW1hIl0sWzAsMCwiXFxHYW1tYS5BIl0sWzEsMiwicF97QX0iXSxbMCwxLCJwX3tBW3Bfe0F9XX0iXSxbMywyLCJwX3tBfSIsMl0sWzAsMywiKHBfe0F9IFxcY2lyYyBwX3tBW3Bfe0F9XX0sIHFfe0FbcF97QX1dfSkiLDIseyJzdHlsZSI6eyJib2R5Ijp7Im5hbWUiOiJkYXNoZWQifX19XV0=
\[\begin{tikzcd}
	{\Gamma.A} \\
	\\
	\\
	\Gamma && {\Gamma.A} && {\Gamma.A.A[p_{A}]}
	\arrow["{p_{A}}", from=4-3, to=4-1]
	\arrow["{p_{A[p_{A}]}}", from=4-5, to=4-3]
	\arrow["{p_{A}}"', from=1-1, to=4-1]
	\arrow["{(p_{A} \circ p_{A[p_{A}]}, q_{A[p_{A}]})}"', dashed, from=4-5, to=1-1]
\end{tikzcd}\]

\subsection{Supporting Type Formers}
\label{sec:cwf-supporting-type-formers}

\subsubsection{$\Pi$-Types}
\label{sec:cwf-pi-types}

\begin{defn}
  A CwF supports intensional $\Pi$-types if the following data are given
  \begin{itemize}
  \item An operator $\Pi : \Ty(\Gamma) \times \Ty(\Gamma.A) \to \Ty(\Gamma)$ such that for any $\sigma : \Delta \to \Gamma$
    \[
      \left(\prod_{A}B\right)[\sigma] = \prod_{A[\sigma]}B[\sigma_{A}]
    \]
  \item An operator $\lambda_{A,B} : \Tm(\Gamma.A)_{B} \to \Tm(\Gamma)_{\prod_{A}B}$ such that for any $\sigma : \Delta \to \Gamma$
    \[
      (\lambda_{A,B}b)[\sigma] = \lambda_{A[\sigma],B[\sigma_{A}]}b[\sigma_{A}]
    \]
  \item A morphism $\app_{A,B} : \Gamma.A.(\Pi_{A}B) \to \Gamma.A.B$ such that % https://q.uiver.app/#q=WzAsMyxbMCwwLCJcXEdhbW1hLkEuQiJdLFswLDIsIlxcR2FtbWEuQSJdLFsyLDAsIlxcR2FtbWEuQS4oXFxQaV97QX1CKSJdLFswLDEsInBfe0J9IiwyXSxbMiwxLCJwX3soXFxQaV97QX1CKX0iXSxbMiwwLCJcXGFwcF97QSxCfSIsMl1d
\[\begin{tikzcd}
	{\Gamma.A.B} && {\Gamma.A.(\Pi_{A}B)} \\
	\\
	{\Gamma.A}
	\arrow["{p_{B}}"', from=1-1, to=3-1]
	\arrow["{p_{\Pi_{A}B}}", from=1-3, to=3-1]
	\arrow["{\app_{A,B}}"', from=1-3, to=1-1]
\end{tikzcd}\] and % https://q.uiver.app/#q=WzAsMyxbMiwwLCJcXEdhbW1hIl0sWzIsMiwiXFxHYW1tYS5BLihcXFBpX3tBfUIpIl0sWzAsMiwiXFxHYW1tYS5BLkIiXSxbMCwxLCIoYSxcXGxhbWJkYV97QSxCfWIpIl0sWzEsMiwiXFxhcHBfe0EsQn0iXSxbMCwyLCIoYSxiW2FdKSIsMl1d
\[\begin{tikzcd}
	&& \Gamma \\
	\\
	{\Gamma.A.B} && {\Gamma.A.(\Pi_{A}B)}
	\arrow["{(a,\lambda_{A,B}b)}", from=1-3, to=3-3]
	\arrow["{\app_{A,B}}", from=3-3, to=3-1]
	\arrow["{(a,b[a])}"', from=1-3, to=3-1]
\end{tikzcd}\] commute.
    And for any morphism $\sigma : \Delta \to \Gamma$,
    \[
      \app_{A,B} \circ \sigma_{A_{(\Pi_{A}B)[\sigma_{A}]}} = \sigma_{A_{B}} \circ \app_{A[\sigma],B[\sigma_{A}]}
    \]
  \end{itemize}
\end{defn}
The morphism $\app_{A,B}$ is rather mysterious.
To unravel what it means, observe that a term $\Gamma \vdash t : B[a]$ can be obtained from the variable $\Gamma.A.B \vdash q_{B} : B[p_{B}]$ with a suitable substitution.

In the elimination rule, two terms $\Gamma \vdash a : A$ and $\Gamma \vdash f : \prod_{A}B$ are given.
The idea is to take $t$ as the application of $f$ to $a$.
The universal property of context extension gives a morphism $(a, f) : \Gamma \to \Gamma.A.(\Pi_{A}B)$.
The first equation yields % https://q.uiver.app/#q=WzAsNCxbMCwxLCJcXEdhbW1hIl0sWzEsMiwiXFxHYW1tYS5BLihcXFBpX3tBfUIpIl0sWzEsMSwiXFxHYW1tYS5BLkIiXSxbMSwwLCJcXEdhbW1hLkEiXSxbMCwxLCIoYSxmKSIsMl0sWzEsMiwiXFxhcHBfe0EsQn0iLDJdLFsyLDMsInBfe0J9IiwyXSxbMCwzLCJhIl1d
\[\begin{tikzcd}
	& {\Gamma.A} \\
	\Gamma & {\Gamma.A.B} \\
	& {\Gamma.A.(\Pi_{A}B)}
	\arrow["{(a,f)}"', from=2-1, to=3-2]
	\arrow["{\app_{A,B}}"', from=3-2, to=2-2]
	\arrow["{p_{B}}"', from=2-2, to=1-2]
	\arrow["a", from=2-1, to=1-2]
\end{tikzcd}\]
Thus, there is a term $\Gamma \vdash q_{B}[\app_{A,B} \circ (a,f)] : B[a]$.

Next to validate the $\beta$-rule, the equation $\Gamma \vdash q_{B}[\app_{A,B} \circ (a,\lambda_{A,B}b)] = b[a] : B[a]$ must hold for any $\Gamma.A \vdash b : B$.
The second equation states that $\app_{A,B} \circ (a,\lambda_{A,B}b)$ is precisely $(a,b[a])$, and $q_{B}[a,b[a]] = b[a]$ follows from the universal property of context extension.

Finally, the last equation states that function application is stable under substitution.

\begin{defn}
  A CwF supports extensional $\Pi$-types if it supports intensional $\Pi$-types, and whenever $b \in \Tm(\Gamma.A)_{B}$ and $f \in \Tm(\Gamma)_{\Pi_{A}B}$ and $\app_{A,B} \circ f = b$, then $f = \lambda_{A,B}b$.
\end{defn}

\section{Natural Model}
\label{sec:natural-model}
\begin{defn}[Natural models]
  Let $\iC$ be a small category.
  A natural transformation $\tau : \Tm \to \Ty$ between presheaves on $\iC$ is \emph{representable} when all of its fibers are representable objects in the sense that for any $\Gamma \in C$ and $A \in \Ty(\Gamma)$, we have a pullback.
  % https://q.uiver.app/#q=WzAsMyxbMCwwLCJcXGJ1bGxldCJdLFswLDIsIlxcYnVsbGV0Il0sWzIsMiwiXFxidWxsZXQiXSxbMCwxXSxbMSwyXSxbMCwyXV0=
\[\begin{tikzcd}
	\bullet \\
	\\
	\bullet && \bullet
	\arrow[from=1-1, to=3-1]
	\arrow[from=3-1, to=3-3]
	\arrow[from=1-1, to=3-3]
\end{tikzcd}\]
  A \emph{natural model} of type theory is a representable natural transformation between presheaves.
  $\Tm$ is the presheaf of terms and $\Ty$ is the presheaf of types.
\end{defn}
The Yoneda Lemma identifies $A : \yon \Gamma \to \Ty$ with a type $A \in \Ty(\Gamma)$ and $b : \yon \Delta \to \Tm$ with a term $b \in \Tm(\Delta)$.
The component $\tau_{\Gamma} : \Tm(\Gamma) \to \Ty(\Gamma)$ can be thought of as the typing of terms in context $\Gamma$.
We write $\Gamma \vdash A \type$ for $A \in \Ty(\Gamma)$ and $\Gamma \vdash b : B$ for $a \in \Tm(\Gamma)$, where $B = \tau \circ b$ as indicated in
% https://q.uiver.app/#q=WzAsMyxbMSwwLCJcXG1hdGhzZntUbX0iXSxbMCwxLCJcXG1hdGhzZnt5fVxcR2FtbWEiXSxbMSwxLCJcXG1hdGhzZntUeX0iXSxbMCwyLCJwIl0sWzEsMiwiQiIsMl0sWzEsMCwiYiJdXQ==
\[\begin{tikzcd}
	& {\Tm} \\
	{\yon \Gamma} & {\Ty}
	\arrow["\tau", from=1-2, to=2-2]
	\arrow["B"', from=2-1, to=2-2]
	\arrow["b", from=2-1, to=1-2]
\end{tikzcd}\]
Henceforth, we omit $\yon$ whenever possible.

For an arbitrary morphism $\sigma$ in $C$, we use the following notations for the action on morphisms of the respective presheaf.
\begin{mathpar}
  \Ty(\sigma) \iff \blank[\sigma]
  \and
  \Tm(\sigma) \iff \blank[\sigma]
\end{mathpar}
As a consequence of the Yoneda Lemma, $A[\sigma] = A \circ \yon\sigma$ and $a[\sigma] = a \circ \yon\sigma$ for any $\Gamma \vdash A \type$ and $\Gamma \vdash a : A$.

\begin{thm}
  Let $\tau : \Tm \to \Ty$ be a natural transformation of presheaves on a small category $\iC$ with a terminal object.
  Then $\tau$ is representable (a natural model of type theory) just when $(\iC, \tau)$ is a category with families.
\end{thm}

\begin{proof}
  Naturality of $\tau$ means that for any substitution $\sigma : \Delta \to \Gamma$, the diagram
  % https://q.uiver.app/#q=WzAsNCxbMCwwLCJcXG1hdGhzZntUbX0oXFxHYW1tYSkiXSxbMiwwLCJcXG1hdGhzZntUeX0oXFxHYW1tYSkiXSxbMCwyLCJcXG1hdGhzZntUbX0oXFxEZWx0YSkiXSxbMiwyLCJcXG1hdGhzZntUeX0oXFxEZWx0YSkiXSxbMCwxLCJwX3tcXEdhbW1hfSJdLFsyLDMsInBfe1xcRGVsdGF9IiwyXSxbMCwyLCJcXG1hdGhzZntUbX0oXFxzaWdtYSkiLDJdLFsxLDMsIlxcbWF0aHNme1R5fShcXHNpZ21hKSJdXQ==
\[\begin{tikzcd}
	{\Tm(\Gamma)} && {\Ty(\Gamma)} \\
	\\
	{\Tm(\Delta)} && {\Ty(\Delta)}
	\arrow["{\tau_{\Gamma}}", from=1-1, to=1-3]
	\arrow["{\tau_{\Delta}}"', from=3-1, to=3-3]
	\arrow["{\Tm(\sigma)}"', from=1-1, to=3-1]
	\arrow["{\Ty(\sigma)}", from=1-3, to=3-3]
\end{tikzcd}\]
  commutes.
  Thus, we have that
  \begin{mathpar}
    \Gamma \vdash A \type \imp \Delta \vdash A[\sigma] \type
    \and
    \Gamma \vdash a : A \imp \Delta \vdash a[\sigma] : A[\sigma]
  \end{mathpar}
  The former is given by $\Ty(\sigma)$, and the latter is given by $\Tm(\sigma)$.
  Moreover, the latter ``type-checks'' because the diagram commutes, i.e., $A[\sigma] = (\tau_{\Gamma}(a))[\sigma] = \tau_{\Delta}(a[\sigma])$.

  By functoriality, the following equations are satisfied
  \begin{mathpar}
    A[\sigma \circ \xi] = (A[\sigma])[\xi]
    \and
    A[\id] = A
    \and
    a[\sigma \circ \xi] = (a[\sigma])[\xi]
    \and
    a[\id] = a
  \end{mathpar}
  where $\xi$ is an arbitrary substitution $\xi : \Xi \to \Delta$.
  Note that the contravariance of the presheves $\Tm$ and $\Ty$ plays a role in the order of the compositions, which works out nicely with the chosen notations.
  These data provide a canonical functor $T : \iC\op \to \iF$.
  % https://q.uiver.app/#q=WzAsMTIsWzAsMCwiXFxHYW1tYSJdLFsyLDAsIlxcRGVsdGEiXSxbMCwzLCIoXFxUeShcXEdhbW1hKSwoXFx0YXVfe1xcR2FtbWF9XFxpbnYoQSkpX3tBIFxcaW4ge1xcVHkoXFxHYW1tYSl9fSkiXSxbMiwzLCIoXFxUeShcXERlbHRhKSwoXFx0YXVfe1xcRGVsdGF9XFxpbnYoQikpX3tCIFxcaW4ge1xcVHkoXFxEZWx0YSl9fSkiXSxbMCwxXSxbMCwyXSxbMSwxXSxbMSwyXSxbMiwxXSxbMiwyXSxbMywwLCJcXGluXFxpQyJdLFszLDMsIlxcaW5cXGlGIl0sWzAsMSwiXFxzaWdtYSJdLFszLDJdLFs0LDUsIiIsMCx7InN0eWxlIjp7InRhaWwiOnsibmFtZSI6Im1hcHMgdG8ifX19XSxbNiw3LCIiLDAseyJzdHlsZSI6eyJ0YWlsIjp7Im5hbWUiOiJtYXBzIHRvIn19fV0sWzgsOSwiIiwwLHsic3R5bGUiOnsidGFpbCI6eyJuYW1lIjoibWFwcyB0byJ9fX1dXQ==
\[\begin{tikzcd}
	\Gamma && \Delta & \in\iC \\
	{} & {} & {} \\
	{} & {} & {} \\
	{(\Ty(\Gamma),(\tau_{\Gamma}\inv(A))_{B \in {\Ty(\Gamma)}})} && {(\Ty(\Delta),(\tau_{\Delta}\inv(B))_{A \in {\Ty(\Delta)}})} & \in\iF
	\arrow["\sigma", from=1-1, to=1-3]
	\arrow[from=4-3, to=4-1]
	\arrow[maps to, from=2-1, to=3-1]
	\arrow[maps to, from=2-2, to=3-2]
	\arrow[maps to, from=2-3, to=3-3]
\end{tikzcd}\]
  As usual, $\tau_{\Delta}\inv(A) \subseteq \Tm(\Delta)$ is the fiber over $A$.
  $T(\sigma)$ consists of the function $\blank[\sigma] : \Ty(\Delta) \to \Ty(\Gamma)$ and a family of functions $\blank[\sigma]_{A} : \tau_{\Delta}\inv(A) \to \tau_{\Gamma}\inv(A[\sigma])$ defined by restricting $\blank[\sigma] : \Tm(\Delta) \to \Tm(\Gamma)$ to the fiber $\tau_{\Delta}\inv(A)$.

  The final piece of data associated with a CwF is context extension, which is encoded precisely by the representability of $\tau$.
  Given any $\Gamma \vdash A \type$, by Yoneda there is a corresponding map $A : \Gamma \to \Ty$.
  Representability of $\tau$ then means that for any $\Delta \vdash a : A[\sigma]$, we have
  % https://q.uiver.app/#q=WzAsNSxbMiw0LCJcXG1hdGhzZnt5fVxcR2FtbWEiXSxbNCw0LCJcXG1hdGhzZntUeX0iXSxbNCwyLCJcXG1hdGhzZntUbX0iXSxbMiwyLCJcXG1hdGhzZnt5fVxcR2FtbWEuQSJdLFswLDAsIlxcbWF0aHNme3l9XFxEZWx0YSJdLFswLDEsIkEiLDJdLFsyLDEsInQiXSxbMywwLCJcXG1hdGhzZnt5fXAiLDJdLFszLDIsInEiXSxbMywxLCIiLDEseyJzdHlsZSI6eyJuYW1lIjoiY29ybmVyIn19XSxbNCwwLCJcXG1hdGhzZnt5fVxcc2lnbWEiLDIseyJjdXJ2ZSI6Mn1dLFs0LDIsImEiLDAseyJjdXJ2ZSI6LTJ9XSxbNCwzLCJcXG1hdGhzZnt5fShcXHNpZ21hLCBhKSIsMSx7InN0eWxlIjp7ImJvZHkiOnsibmFtZSI6ImRhc2hlZCJ9fX1dXQ==
\[\begin{tikzcd}
	{\Delta} \\
	\\
	&& {\Gamma.A} && {\Tm} \\
	\\
	&& {\Gamma} && {\Ty}
	\arrow["A"', from=5-3, to=5-5]
	\arrow["\tau", from=3-5, to=5-5]
	\arrow["p"', from=3-3, to=5-3]
	\arrow["q", from=3-3, to=3-5]
	\arrow["\lrcorner"{anchor=center, pos=0.125}, draw=none, from=3-3, to=5-5]
	\arrow["{\sigma}"', curve={height=12pt}, from=1-1, to=5-3]
	\arrow["a", curve={height=-12pt}, from=1-1, to=3-5]
	\arrow["{(\sigma, a)}"{description}, dashed, from=1-1, to=3-3]
\end{tikzcd}\]
  The notations used in the diagram above is justified because $\yon$ is fully faithful.
  In particular, we have that $q \circ \yon(\sigma, a) = a$, which implies that $q_{\Delta} \circ (\sigma, a)_{*} = a_{\Delta}$.
  The component $a_{\Delta} : C(\Delta, \Delta) \to \Tm(\Delta)$ is a function.
  Thus, we have that $q_{\Delta} \circ (\sigma, a)_{*} (\id_{\Delta}) = a_{\Delta}(\id_{\Delta})$.
  By Yoneda, the left-hand side determines the element $q[(\sigma, a)] \in \Tm(\Delta)$, while the right-hand side determines the element $a \in \Tm(\Delta)$.
  Thus, we can extract the required equations from the pullback diagram.
  \begin{mathpar}
    p \circ (\sigma, a) = \sigma
    \and
    q[(\sigma, a)] = a
  \end{mathpar}
  Additionally, by uniqueness, we also have the following required equations.
  \begin{mathpar}
    (\sigma, a) \circ \xi = (\sigma \circ \xi, a[\xi])
    \and
    (p, q) = \id_{\Gamma.A}
  \end{mathpar}
  where $\xi : \Xi \to \Delta$ is an arbitrary substitution.
\end{proof}

By waving a magic wand, every natural model $f : \Tm \to \Ty$ has an associated polynomial functor $F(X) = \sum_{A : \Ty}X^{A}$ such that
\begin{mathpar}
  \mprset{
    fraction={===},
    fractionaboveskip=2.3 ex
  }
  \inferrule
  { (A, B) : \Gamma \to \sum_{A : \Ty}\Ty^{A} }
  { \Gamma \vdash A \type \\ \Gamma.A \vdash B \type }
  \and
  \inferrule
  { (A, b) : \Gamma \to \sum_{A : \Ty}\Tm^{A} }
  { \Gamma \vdash A \type \\ \Gamma.A \vdash b : B }
\end{mathpar}

\subsection{Supporting Type Formers}
\label{sec:nm-supporting-type-formers}

\subsubsection{$\Pi$-Types}
\label{sec:nm-pi-types}

\begin{thm}
  Let $F(X) = \sum_{A : \Ty}X^{A}$ be the polynomial functor associated to a natural model $f : \Tm \to \Ty$.
  Then the type-theoretic rules for (extensional) $\Pi$-types are modeled by maps of the form
  \begin{mathpar}
    \lambda : F(\Tm) \to \Tm
    \and
    \Pi : F(\Ty) \to \Ty
  \end{mathpar}
  making the following diagram a pullback.
  % https://q.uiver.app/#q=WzAsNCxbMCwwLCJQKFRtKSJdLFsyLDAsIlRtIl0sWzAsMiwiUChUeSkiXSxbMiwyLCJUeSJdLFswLDEsIlxcbGFtYmRhIl0sWzIsMywiXFxQaSIsMl0sWzAsMiwiUChmKSIsMl0sWzEsMywiZiJdLFswLDMsIiIsMSx7InN0eWxlIjp7Im5hbWUiOiJjb3JuZXIifX1dXQ==
\[\begin{tikzcd}
	{F(\Tm)} && \Tm \\
	\\
	{F(\Ty)} && \Ty
	\arrow["\lambda", from=1-1, to=1-3]
	\arrow["\Pi"', from=3-1, to=3-3]
	\arrow["{F(f)}"', from=1-1, to=3-1]
	\arrow["f", from=1-3, to=3-3]
	\arrow["\lrcorner"{anchor=center, pos=0.125}, draw=none, from=1-1, to=3-3]
\end{tikzcd}\]
\end{thm}
\begin{proof}
  The type-theoretic $\Pi$-formation rule is
  \begin{mathpar}
    \inferrule
    { \Gamma \vdash A \type \\ \Gamma.A \vdash B \type }
    { \Gamma \vdash \prod_{A}B \type }
  \end{mathpar}
  The premises correspond to a map $(A, B) : \Gamma \to F(\Ty)$.
  Composing $\Pi$ with $(A, B)$ yields $\Gamma \vdash \prod_{A}B \type$.
  % https://q.uiver.app/#q=WzAsMyxbMiwwLCJGKFR5KSJdLFs0LDAsIlR5Il0sWzAsMCwiXFxHYW1tYSJdLFswLDEsIlxcUGkiXSxbMiwwLCIoQSwgQikiXV0=
\[\begin{tikzcd}
	\Gamma && {F(\Ty)} && \Ty
	\arrow["\Pi", from=1-3, to=1-5]
	\arrow["{(A, B)}", from=1-1, to=1-3]
\end{tikzcd}\]
  Thus, $\Pi : F(\Ty) \to \Ty$ models the type-theoretic formation rule.

  Similarly, the type-theoretic $\Pi$-introduction rule is
  \begin{mathpar}
    \inferrule
    { \Gamma \vdash A \type \\ \Gamma.A \vdash b : B }
    { \Gamma \vdash \lambda_{A}b : \prod_{A}B }
  \end{mathpar}
  The premises correspond to a map $(A, b) : \Gamma \to F(\Tm)$, whose composition with $\lambda : F(\Tm) \to \Tm$ yields $\Gamma \vdash \lambda_{A}b : \prod_{A}B$.
  Thus, $\lambda : F(\Tm) \to \Tm$ models the type-theoretic introduction rule.
  % https://q.uiver.app/#q=WzAsMyxbMiwwLCJGKFR5KSJdLFs0LDAsIlR5Il0sWzAsMCwiXFxHYW1tYSJdLFswLDEsIlxcUGkiXSxbMiwwLCIoQSwgQikiXV0=
\[\begin{tikzcd}
	\Gamma && {F(\Tm)} && \Tm
	\arrow["\lambda", from=1-3, to=1-5]
	\arrow["{(A, b)}", from=1-1, to=1-3]
\end{tikzcd}\]

  Now consider the elimination rule and its associated $\beta$-rule and $\eta$-rule.
  \begin{mathpar}
    \inferrule
    { \Gamma \vdash g : \prod_{A}B \\ \Gamma \vdash a : A }
    { \Gamma \vdash g(a) : B[a] }
    \and
    \inferrule
    { \Gamma.A \vdash b : B \\ \Gamma \vdash a : A }
    { \Gamma \vdash (\lambda_{A}b)(a) = b[a] : B[a] }
    \and
    \inferrule
    { \Gamma \vdash g : \prod_{A}B }
    { \Gamma \vdash \lambda_{A}(g[p_{A}](q_{A})) = g : \prod_{A}B }
  \end{mathpar}
  where $b[a]$ and $B[a]$ are defined respectively as
  \begin{mathpar}
    b \circ (1, a) \and B \circ (1, a)
  \end{mathpar}
  as indicated in
  % https://q.uiver.app/#q=WzAsNCxbNCwzLCJUbSJdLFswLDMsIlR5Il0sWzIsMiwiXFxHYW1tYS5BIl0sWzIsMCwiXFxHYW1tYSJdLFswLDEsImYiXSxbMiwwLCJiIl0sWzMsMCwiYlthXSIsMCx7ImN1cnZlIjotMn1dLFszLDIsIigxLGEpIiwxLHsic3R5bGUiOnsiYm9keSI6eyJuYW1lIjoiZGFzaGVkIn19fV0sWzIsMSwiQiIsMl0sWzMsMSwiQlthXSIsMix7ImN1cnZlIjoyfV1d
\[\begin{tikzcd}
	&& \Gamma \\
	\\
	&& {\Gamma.A} \\
	\Ty &&&& \Tm
	\arrow["f", from=4-5, to=4-1]
	\arrow["b", from=3-3, to=4-5]
	\arrow["{b[a]}", curve={height=-12pt}, from=1-3, to=4-5]
	\arrow["{(1,a)}"{description}, dashed, from=1-3, to=3-3]
	\arrow["B"', from=3-3, to=4-1]
	\arrow["{B[a]}"', curve={height=12pt}, from=1-3, to=4-1]
\end{tikzcd}\]
  Intuitively, $B[a]$ corresponds to the type obtained by filling in $a$ for the last variable in the context, and $b[a]$ corresponds to the term obtained by filling in $a$ for the last variable in the context.
  
  The first premise of the elimination rule states that we have a commutative diagram.
  % https://q.uiver.app/#q=WzAsNCxbMCwwLCJcXEdhbW1hIl0sWzQsMiwiVG0iXSxbNCw0LCJUeSJdLFsyLDQsIkYoVHkpIl0sWzAsMSwiZyIsMCx7ImN1cnZlIjotMn1dLFsxLDIsImYiXSxbMCwzLCIoQSxCKSIsMix7ImN1cnZlIjoyfV0sWzMsMiwiXFxQaSIsMl1d
\[\begin{tikzcd}
	\Gamma \\
	\\
	&&&& \Tm \\
	\\
	&& {F(\Ty)} && \Ty
	\arrow["g", curve={height=-12pt}, from=1-1, to=3-5]
	\arrow["f", from=3-5, to=5-5]
	\arrow["{(A,B)}"', curve={height=12pt}, from=1-1, to=5-3]
	\arrow["\Pi"', from=5-3, to=5-5]
\end{tikzcd}\]
  By the universal property of the pullback, we then have a unique map $(A, \widetilde{g}) : \Gamma \to F(\Tm)$, corresponding to $\Gamma \vdash A \type$ and $\Gamma.A \vdash \widetilde{g} : B$.
  % https://q.uiver.app/#q=WzAsNSxbMCwwLCJcXEdhbW1hIl0sWzQsMiwiVG0iXSxbNCw0LCJUeSJdLFsyLDQsIkYoVHkpIl0sWzIsMiwiRihUbSkiXSxbMCwxLCJnIiwwLHsiY3VydmUiOi0yfV0sWzEsMiwiZiJdLFswLDMsIihBLEIpIiwyLHsiY3VydmUiOjJ9XSxbMywyLCJcXFBpIiwyXSxbNCwxLCJcXGxhbWJkYSJdLFs0LDMsIkYoZikiLDJdLFswLDQsIihBLFxcd2lkZXRpbGRle2d9KSIsMSx7InN0eWxlIjp7ImJvZHkiOnsibmFtZSI6ImRhc2hlZCJ9fX1dXQ==
\[\begin{tikzcd}
	\Gamma \\
	\\
	&& {F(\Tm)} && \Tm \\
	\\
	&& {F(\Ty)} && \Ty
	\arrow["g", curve={height=-12pt}, from=1-1, to=3-5]
	\arrow["f", from=3-5, to=5-5]
	\arrow["{(A,B)}"', curve={height=12pt}, from=1-1, to=5-3]
	\arrow["\Pi"', from=5-3, to=5-5]
	\arrow["\lambda", from=3-3, to=3-5]
	\arrow["{F(f)}"', from=3-3, to=5-3]
	\arrow["{(A,\widetilde{g})}"{description}, dashed, from=1-1, to=3-3]
\end{tikzcd}\]
  Thus, we have a term $\Gamma \vdash \widetilde{g}[a] : B[a]$, and we take $g(a) := \gtil[a]$.
  % https://q.uiver.app/#q=WzAsNCxbMiwyLCJcXEdhbW1hLkEiXSxbNCwzLCJUbSJdLFswLDMsIlR5Il0sWzIsMCwiXFxHYW1tYSJdLFsxLDIsImYiXSxbMywxLCJcXHdpZGV0aWxkZXtnfVthXSIsMCx7ImN1cnZlIjotMn1dLFszLDAsIigxLGEpIiwxLHsic3R5bGUiOnsiYm9keSI6eyJuYW1lIjoiZGFzaGVkIn19fV0sWzAsMSwiXFx3aWRldGlsZGV7Z30iXSxbMCwyLCJCIiwyXSxbMywyLCJCW2FdIiwyLHsiY3VydmUiOjJ9XV0=
\[\begin{tikzcd}
	&& \Gamma \\
	\\
	&& {\Gamma.A} \\
	\Ty &&&& \Tm
	\arrow["f", from=4-5, to=4-1]
	\arrow["{\widetilde{g}[a]}", curve={height=-12pt}, from=1-3, to=4-5]
	\arrow["{(1,a)}"{description}, dashed, from=1-3, to=3-3]
	\arrow["{\widetilde{g}}", from=3-3, to=4-5]
	\arrow["B"', from=3-3, to=4-1]
	\arrow["{B[a]}"', curve={height=12pt}, from=1-3, to=4-1]
\end{tikzcd}\]

  From the first premise of the $\beta$-rule, we can extract that
  \begin{mathpar}
    \Gamma \vdash A \type
    \and
    \Gamma.A \vdash B \type
    \and
    \Gamma.A \vdash b : B
  \end{mathpar}
  Thus, there are maps $(A, B) : \Gamma \to F(\Ty)$ and $(A, b) : \Gamma \to F(\Tm)$.
  The introduction rule gives a term $\Gamma \vdash \lambda_{A}b : B$ so that the diagram commutes.
  % https://q.uiver.app/#q=WzAsNCxbNCwyLCJUbSJdLFs0LDQsIlR5Il0sWzIsNCwiRihUeSkiXSxbMCwwLCJcXEdhbW1hIl0sWzMsMCwiXFxsYW1iZGFfe0F9YiIsMCx7ImN1cnZlIjotMn1dLFszLDIsIihBLEIpIiwyLHsiY3VydmUiOjJ9XSxbMCwxLCJmIl0sWzIsMSwiXFxQaSIsMl1d
\[\begin{tikzcd}
	\Gamma \\
	\\
	&&&& \Tm \\
	\\
	&& {F(\Ty)} && \Ty
	\arrow["{\lambda_{A}b}", curve={height=-12pt}, from=1-1, to=3-5]
	\arrow["{(A,B)}"', curve={height=12pt}, from=1-1, to=5-3]
	\arrow["f", from=3-5, to=5-5]
	\arrow["\Pi"', from=5-3, to=5-5]
\end{tikzcd}\]
  Thus, there is a unique map $(A, \widetilde{\lambda_{A}b}) : \Gamma \to F(\Tm)$.
  % https://q.uiver.app/#q=WzAsNSxbNCwyLCJUbSJdLFs0LDQsIlR5Il0sWzIsNCwiRihUeSkiXSxbMCwwLCJcXEdhbW1hIl0sWzIsMiwiRihUbSkiXSxbMywwLCJcXGxhbWJkYV97QX1iIiwwLHsiY3VydmUiOi0yfV0sWzMsMiwiKEEsQikiLDIseyJjdXJ2ZSI6Mn1dLFswLDEsImYiXSxbMiwxLCJcXFBpIiwyXSxbNCwwLCJcXGxhbWJkYSJdLFs0LDIsIkYoZikiLDJdLFszLDQsIihBLFxcd2lkZXRpbGRle1xcbGFtYmRhX3tBfWJ9KSIsMSx7InN0eWxlIjp7ImJvZHkiOnsibmFtZSI6ImRhc2hlZCJ9fX1dXQ==
\[\begin{tikzcd}
	\Gamma \\
	\\
	&& {F(\Tm)} && \Tm \\
	\\
	&& {F(\Ty)} && \Ty
	\arrow["{\lambda_{A}b}", curve={height=-12pt}, from=1-1, to=3-5]
	\arrow["{(A,B)}"', curve={height=12pt}, from=1-1, to=5-3]
	\arrow["f", from=3-5, to=5-5]
	\arrow["\Pi"', from=5-3, to=5-5]
	\arrow["\lambda", from=3-3, to=3-5]
	\arrow["{F(f)}"', from=3-3, to=5-3]
	\arrow["{(A,\widetilde{\lambda_{A}b})}"{description}, dashed, from=1-1, to=3-3]
\end{tikzcd}\]
  By definition, $\Gamma \vdash (\lambda_{A}b)(a) = \widetilde{\lambda_{A}b}[a] : B[a]$, where $a$ is given by the second premise.
  Additionally, $\lambda \circ (A, b) = \lambda_{A}b$.
  By uniqueness, $\Gamma.A \vdash b = \widetilde{\lambda_{A}b} : B$, yielding the desired equation: $\Gamma \vdash (\lambda_{A}b)(a) = b[a] : B[a]$.

  Finally, the $\eta$-law is also satisfied.
  The premise says that there is a unique map $(A, \gtil) \maps \Gamma \to F(\Tm)$.
  % https://q.uiver.app/#q=WzAsNSxbMiwyLCJGKFxcVG0pIl0sWzIsNCwiRihcXFR5KSJdLFs0LDQsIlxcVHkiXSxbNCwyLCJcXFRtIl0sWzAsMCwiXFxHYW1tYSJdLFswLDEsIkYoZikiLDJdLFsxLDIsIlxcUGkiLDJdLFszLDIsImYiXSxbNCwxLCIoQSxCKSIsMix7ImN1cnZlIjoyfV0sWzQsMywiZyIsMCx7ImN1cnZlIjotMn1dLFswLDMsIlxcbGFtYmRhIl0sWzQsMCwiKEEsXFx3aWRldGlsZGV7Z30pIiwxLHsic3R5bGUiOnsiYm9keSI6eyJuYW1lIjoiZGFzaGVkIn19fV1d
\[\begin{tikzcd}
	\Gamma \\
	\\
	&& {F(\Tm)} && \Tm \\
	\\
	&& {F(\Ty)} && \Ty
	\arrow["{F(f)}"', from=3-3, to=5-3]
	\arrow["\Pi"', from=5-3, to=5-5]
	\arrow["f", from=3-5, to=5-5]
	\arrow["{(A,B)}"', curve={height=12pt}, from=1-1, to=5-3]
	\arrow["g", curve={height=-12pt}, from=1-1, to=3-5]
	\arrow["\lambda", from=3-3, to=3-5]
	\arrow["{(A,\widetilde{g})}"{description}, dashed, from=1-1, to=3-3]
\end{tikzcd}\]
  By definition, $(g[p_{A}])(q_{A}) = \widetilde{g[p_{A}]}[q_{A}] = \widetilde{g[p_{A}]} \circ (1, q_{A})$.

  The type and term constructors $\Pi$, $\lambda$, and $\blank(\blank)$ also respect substitutions $\sigma \maps \Delta \to \Gamma$.
  Consider the formation rule,
  \begin{mathpar}
    \inferrule
    { \Gamma \vdash A \type \\ \Gamma.A \vdash B \type }
    { \Gamma \vdash \prod_{A}B \type }
  \end{mathpar}
  Applying $\sigma$ to each premise yields a new instance of the formation rule.

  $\Gamma.A \vdash B \type$,
  $\Delta.A[\sigma] \vdash B[\sigma] \type$
  
  \begin{mathpar}
    \inferrule
    { \Delta \vdash A[\sigma] \type \\ \Delta.A[\sigma] \vdash B[\sigma] \type }
    { \Delta \vdash \prod_{A[\sigma]}B[\sigma] \type }
  \end{mathpar}
\end{proof}

\section{Judgmental Theory}
\label{sec:judgmental-theory}
\begin{defn}
  A \emph{prejudgmental theory} consists of the following data:
  \begin{itemize}
  \item a category $\iC$ with a terminal object $\diamond$;
  \item a set $\cJ$ of functors $f : \dF \to \iC$;
  \item a set $\cR$ of functors $\lambda : \dF \to \dG$;
  \item a set $\cP$ of 2-cells;
  \end{itemize}
\end{defn}

\paragraph{\textbf{Terminology}}
\begin{itemize}
\item The objects of the category $\iC$ are called contexts and the morphisms are substitutions.
  The terminal object is called the empty context.
\item An element $f : \dF \to \iC$ of $\cJ$ is a \emph{classifier for the judgment} $\dF$.
  The boundary between a judgment and a judgment classifier is often blurred.
\item A \emph{rule} is an element $\lambda$ of $\cR$.
\item A \emph{policy} is an element $\lambda^{\#}$ of $\cP$.
\end{itemize}

\begin{defn}
  A \emph{judgmental theory} is a prejudgmental theory such that the following coherence conditions are satisfied:
  \begin{itemize}
  \item $\cR$ and $\cP$ are closed under composition;
  \item the judgments are precisely those rules whose codomain is $\iC$;
  \item $\cR$ and $\cP$ are closed under finite limits, $\#$-liftings, and whiskering.
  \end{itemize}
  A classifier is said to be \emph{(op)substitutional} if it is a Grothendieck (op)fibration.
\end{defn}

\begin{rmk}
  Both the slice and the coslice over an object $\Gamma \in \iC$ can be obtained by pulling back the universal discrete opfibration $\textsf{Set}_{*} \to \textsf{Set}$ along the (co)representable functor.
  Thus, the coslice is a discrete Grothendieck opfibration, while the slice is a discrete Grothendieck fibration.
\end{rmk}

\begin{defn}
  A (pre)judgmental theory has axioms if the initial fibration over $\iC$ is a judgment.
\end{defn}

\begin{defn}
  A (pre)judgmental theory has substitution if
  \begin{itemize}
  \item all slices are judgments;
  \item for all fibrations $\dF \in \cR$ and for all $F \in f\inv(\Gamma)$, the corresponding arrow $F : \iC/\Gamma \to \dF$ is a rule.
  \end{itemize}
\end{defn}

\subsection{Judgmental calculi}
\label{sec:judgmental-calculi}

Recall that a dependent type theory consists of syntax, judgments, and rules.
Given a judgmental theory, the corresponding judgmental calculus provides these data.
\begin{itemize}
\item Syntax: a letter $\Gamma$ for each context in $\iC$, and a letter $F$ for each object in a judgment classifier $\dF$.
\item Judgments: there are two main kinds of judgment for each classifier $\dG \in \cJ$.
  \begin{itemize}
  \item The first kind of judgment acknowledges a piece of $\dG$-empirical evidence and clarifies the status of an object.
    For $g : \dG \to \iC$, we write
    \[
      \Gamma \vdash G~\dG(g)
    \]
    for an object in the fiber category $\dG_{\Gamma}$.
    This can be understood as, for example, ``$G$ is green given $\Gamma$''.
  \item The second kind of judgment is an identity checker.
    For two equal objects $G, G' \in \dG_{\Gamma}$, we write
    \[
      \Gamma \vdash G =_{\dG(g)} G'
    \]
    This can be understood as ``$G$ and $G'$ are indistinguishable by green given $\Gamma$''.
  \end{itemize}
\end{itemize}

\subsection{Nested Judgments}
\label{sec:nested-judgments}

Often, we find ourselves in a situation where we need to classify a pair of judgments that are related in some way.
For example, in the identity formation rule the premise is a pair of judgments, $\Gamma \vdash a : A$ and $\Gamma \vdash b : B$, under the same context.
A discussion of identity types is deferred to a later section. 

Say, we wish to classify the pair of judgments: $\Gamma \vdash H~\dH(g\lambda)$ and $g\lambda(H) \vdash F~\dF(f)$.
Note that $H$ and $F$ form a cone over the cospan $g\lambda$ and $f$.
% https://q.uiver.app/#q=WzAsNixbMiw0LCJcXGRIIl0sWzMsNCwiXFxkRyJdLFs0LDQsIlxcaUMiXSxbNCwyLCJcXGRGIl0sWzIsMiwiXFxkSC5cXGxhbWJkYVxcZEYiXSxbMCwwLCJcXGJ1bGxldCJdLFswLDEsIlxcbGFtYmRhIiwyXSxbMSwyLCJnIiwyXSxbMywyLCJmIl0sWzQsMCwiIiwyLHsic3R5bGUiOnsiYm9keSI6eyJuYW1lIjoiZGFzaGVkIn19fV0sWzQsMywiIiwwLHsic3R5bGUiOnsiYm9keSI6eyJuYW1lIjoiZGFzaGVkIn19fV0sWzUsMywiRiIsMCx7ImN1cnZlIjotMn1dLFs1LDAsIkgiLDIseyJjdXJ2ZSI6Mn1dLFs1LDQsIkguXFxsYW1iZGEgRiIsMSx7InN0eWxlIjp7ImJvZHkiOnsibmFtZSI6ImRhc2hlZCJ9fX1dLFs0LDIsIiIsMSx7InN0eWxlIjp7Im5hbWUiOiJjb3JuZXIifX1dXQ==
\[\begin{tikzcd}
	\bullet \\
	\\
	&& {\dH.\lambda\dF} && \dF \\
	\\
	&& \dH & \dG & \iC
	\arrow["\lambda"', from=5-3, to=5-4]
	\arrow["g"', from=5-4, to=5-5]
	\arrow["f", from=3-5, to=5-5]
	\arrow[dashed, from=3-3, to=5-3]
	\arrow[dashed, from=3-3, to=3-5]
	\arrow["F", curve={height=-12pt}, from=1-1, to=3-5]
	\arrow["H"', curve={height=12pt}, from=1-1, to=5-3]
	\arrow["{H.\lambda F}"{description}, dashed, from=1-1, to=3-3]
	\arrow["\lrcorner"{anchor=center, pos=0.125}, draw=none, from=3-3, to=5-5]
\end{tikzcd}\]
Thus, by the universal property of pullbacks there is a one-to-one correspondence
\begin{mathpar}
  \mprset{
    fraction={===}
  }
  \inferrule
  { \Gamma \vdash H.\lambda F~\dH.\lambda\dF }
  { \Gamma \vdash H~\dH \\ g\lambda(H) \vdash F~\dF }
\end{mathpar}
That is, the nested judgment classifier $\dH.\lambda\dF$ classifies a pair of judgments of the form $\Gamma \vdash H~\dH$ and $g\lambda(H) \vdash F~\dF$.

\subsection{Rules}
\label{sec:jt-rules}

Given a rule $\lambda : \dF \to \dG$ and $\Gamma \vdash F~\dF$, then one has $g\lambda(F) \vdash \lambda(F)~\dG$.
\begin{mathpar}
  % https://q.uiver.app/#q=WzAsNCxbMCwyLCJcXGlDIl0sWzIsMiwiXFxpQyJdLFswLDAsIlxcZEYiXSxbMiwwLCJcXGRHIl0sWzIsMCwiZiIsMl0sWzMsMSwiZyJdLFsyLDMsIlxcbGFtYmRhIl1d
\begin{tikzcd}
	\dF && \dG \\
	\\
	\iC && \iC
	\arrow["f"', from=1-1, to=3-1]
	\arrow["g", from=1-3, to=3-3]
	\arrow["\lambda", from=1-1, to=1-3]
\end{tikzcd} \and (\lambda)\,\inferrule{ \Gamma \vdash F~\dF }{ g\lambda(F) \vdash \lambda(F)~\dG }
\end{mathpar}

\subsection{Substitutionality}
\label{sec:jt-substitutionality}

In a substitutional judgmental theory, the corresponding functor of an object $\Gamma \vdash F~\dF$ is a rule indicated as follows.
Thus, for any substitution $\sigma : \Delta \to \Gamma$ in $\iC$, one has $\Delta \vdash F(\sigma)~\dF$.
\begin{mathpar}
  % https://q.uiver.app/#q=WzAsMyxbMCwwLCJcXGlDL1xcR2FtbWEiXSxbMiwwLCJcXGRGIl0sWzIsMiwiXFxpQyJdLFswLDEsIkYiXSxbMSwyLCJmIl0sWzAsMl1d
\begin{tikzcd}
	{\iC/\Gamma} && \dF \\
	\\
	&& \iC
	\arrow["F", from=1-1, to=1-3]
	\arrow["f", from=1-3, to=3-3]
	\arrow[from=1-1, to=3-3]
\end{tikzcd} \and (F)\,\inferrule{ \Delta \vdash \sigma~\iC/\Gamma }{ \Delta \vdash F(\sigma)~\dF }
\end{mathpar}

\subsection{Policies}
\label{sec:jt-policies}

Suppose that we are given a substitutional judgmental theory with
% https://q.uiver.app/#q=WzAsMyxbMCwwLCJcXGRGIl0sWzIsMCwiXFxkRyJdLFsxLDEsIlxcaUMiXSxbMCwxLCJcXGxhbWJkYSJdLFswLDIsImYiLDJdLFsxLDIsImciXSxbMSw0LCJcXGxhbWJkYV57XFwjfSIsMSx7InNob3J0ZW4iOnsidGFyZ2V0IjoyMH19XV0=
\[\begin{tikzcd}
	\dF && \dG \\
	& \iC
	\arrow["\lambda", from=1-1, to=1-3]
	\arrow[""{name=0, anchor=center, inner sep=0}, "f"', from=1-1, to=2-2]
	\arrow["g", from=1-3, to=2-2]
	\arrow["{\lambda^{\#}}"{description}, shorten >=8pt, Rightarrow, from=1-3, to=0]
\end{tikzcd}\]
Then for any $\Gamma \vdash F~\dF$, there is a substitution $\lambda^{\#}_{F} : g\lambda(F) \to \Gamma$.
Thus,
\begin{mathpar}
  \inferrule
  { \Gamma \vdash F~\dF }
  { g\lambda(F) \vdash F(\lambda^{\#}_{F})~\dF }
\end{mathpar}

\subsection{$\#$-Lifting}
\label{sec:sharp-lifting}

Let $\lambda^{\#}$ be a policy and $\cR^{*}$ be a rule which happens to be a fibration depicted as follows.
% https://q.uiver.app/#q=WzAsNCxbMiwzLCJcXGRHIl0sWzQsMiwiXFxkWCJdLFs0LDAsIlxcZEgiXSxbMCwyLCJcXGRGIl0sWzAsMSwiXFxjUiIsMl0sWzIsMSwiXFxjUl57Kn0iXSxbMywwLCJcXGxhbWJkYSIsMl0sWzMsMSwiXFxjUiIsMCx7ImxhYmVsX3Bvc2l0aW9uIjo2MH1dLFswLDcsIlxcbGFtYmRhXntcXCN9IiwwLHsiY3VydmUiOi0xLCJzaG9ydGVuIjp7InRhcmdldCI6MjB9fV1d
\[\begin{tikzcd}
	&&&& \dH \\
	\\
	\dF &&&& \dX \\
	&& \dG
	\arrow["\cR"', from=4-3, to=3-5]
	\arrow["{\cR^{*}}", from=1-5, to=3-5]
	\arrow["\lambda"', from=3-1, to=4-3]
	\arrow[""{name=0, anchor=center, inner sep=0}, "\cR"{pos=0.6}, from=3-1, to=3-5]
	\arrow["{\lambda^{\#}}", curve={height=-6pt}, shorten >=3pt, Rightarrow, from=4-3, to=0]
\end{tikzcd}\]

We can pull back two maps along $\cR^{*}$.
% https://q.uiver.app/#q=WzAsNSxbMiwzLCJcXGRHIl0sWzQsMiwiXFxkWCJdLFs0LDAsIlxcZEgiXSxbMCwyLCJcXGRGIl0sWzIsMSwiXFxkR1xcY1Jeeyp9LlxcY1JcXGRIIl0sWzAsMSwiXFxjUiIsMl0sWzIsMSwiXFxjUl57Kn0iXSxbMywwLCJcXGxhbWJkYSIsMl0sWzMsMSwiXFxjUiIsMCx7ImxhYmVsX3Bvc2l0aW9uIjo2MH1dLFs0LDJdLFs0LDBdLFs0LDEsIiIsMCx7InN0eWxlIjp7Im5hbWUiOiJjb3JuZXIifX1dLFswLDgsIlxcbGFtYmRhXntcXCN9IiwwLHsiY3VydmUiOi0xLCJzaG9ydGVuIjp7InRhcmdldCI6MjB9fV1d
\[\begin{tikzcd}
	&&&& \dH \\
	&& {\dG\cR^{*}.\cR\dH} \\
	\dF &&&& \dX \\
	&& \dG
	\arrow["\cR"', from=4-3, to=3-5]
	\arrow["{\cR^{*}}", from=1-5, to=3-5]
	\arrow["\lambda"', from=3-1, to=4-3]
	\arrow[""{name=0, anchor=center, inner sep=0}, "\cR"{pos=0.6}, from=3-1, to=3-5]
	\arrow[from=2-3, to=1-5]
	\arrow[from=2-3, to=4-3]
	\arrow["\lrcorner"{anchor=center, pos=0.125, rotate=45}, draw=none, from=2-3, to=3-5]
	\arrow["{\lambda^{\#}}", curve={height=-6pt}, shorten >=3pt, Rightarrow, from=4-3, to=0]
\end{tikzcd}\]

Then $\lambda$ and $\lambda^{\#}$ can be lifted to complete the triangle above.
% https://q.uiver.app/#q=WzAsNixbMiwzLCJcXGRHIl0sWzQsMiwiXFxkWCJdLFs0LDAsIlxcZEgiXSxbMCwyLCJcXGRGIl0sWzIsMSwiXFxkR1xcY1Jeeyp9LlxcY1JcXGRIIl0sWzAsMCwiXFxkRlxcY1Jeeyp9LlxcY1JcXGRIIl0sWzAsMSwiXFxjUiIsMl0sWzIsMSwiXFxjUl57Kn0iXSxbMywwLCJcXGxhbWJkYSIsMl0sWzMsMSwiXFxjUiIsMCx7ImxhYmVsX3Bvc2l0aW9uIjo2MH1dLFs0LDJdLFs0LDBdLFs0LDEsIiIsMCx7InN0eWxlIjp7Im5hbWUiOiJjb3JuZXIifX1dLFs1LDNdLFs1LDJdLFs1LDEsIiIsMCx7InN0eWxlIjp7Im5hbWUiOiJjb3JuZXIifX1dLFs1LDQsIlxcY1Jeeyp9XFxsYW1iZGEiLDIseyJjb2xvdXIiOlswLDYwLDYwXSwic3R5bGUiOnsiYm9keSI6eyJuYW1lIjoiZGFzaGVkIn19fSxbMCw2MCw2MCwxXV0sWzAsOSwiXFxsYW1iZGFee1xcI30iLDAseyJjdXJ2ZSI6LTEsInNob3J0ZW4iOnsidGFyZ2V0IjoyMH19XSxbNCwxNCwiXFxjUl57Kn1cXGxhbWJkYV57XFwjfSIsMCx7ImN1cnZlIjotMSwic2hvcnRlbiI6eyJ0YXJnZXQiOjIwfSwiY29sb3VyIjpbMCw2MCw2MF19LFswLDYwLDYwLDFdXV0=
\[\begin{tikzcd}
	{\dF\cR^{*}.\cR\dH} &&&& \dH \\
	&& {\dG\cR^{*}.\cR\dH} \\
	\dF &&&& \dX \\
	&& \dG
	\arrow["\cR"', from=4-3, to=3-5]
	\arrow["{\cR^{*}}", from=1-5, to=3-5]
	\arrow["\lambda"', from=3-1, to=4-3]
	\arrow[""{name=0, anchor=center, inner sep=0}, "\cR"{pos=0.6}, from=3-1, to=3-5]
	\arrow[from=2-3, to=1-5]
	\arrow[from=2-3, to=4-3]
	\arrow["\lrcorner"{anchor=center, pos=0.125, rotate=45}, draw=none, from=2-3, to=3-5]
	\arrow[from=1-1, to=3-1]
	\arrow[""{name=1, anchor=center, inner sep=0}, from=1-1, to=1-5]
	\arrow["\lrcorner"{anchor=center, pos=0.125, rotate=45}, draw=none, from=1-1, to=3-5]
	\arrow["{\cR^{*}\lambda}"', color={rgb,255:red,214;green,92;blue,92}, dashed, from=1-1, to=2-3]
	\arrow["{\lambda^{\#}}", curve={height=-6pt}, shorten >=4pt, Rightarrow, from=4-3, to=0]
	\arrow["{\cR^{*}\lambda^{\#}}", color={rgb,255:red,214;green,92;blue,92}, curve={height=-6pt}, shorten >=3pt, Rightarrow, from=2-3, to=1]
\end{tikzcd}\]

\section{Dependent Type Theory in Judgmental Theory}
\label{sec:dependent-type-theory-in-judgmental-theory}

\begin{defn}
  A dependent type theory is a substitutional judgmental theory generated by
  % https://q.uiver.app/#q=WzAsMyxbMCwwLCJcXFRtIl0sWzIsMCwiXFxUeSJdLFsxLDEsIlxcaUMiXSxbMCwxLCJcXFNpZ21hIiwyXSxbMCwyLCJcXHV0aWwiLDJdLFsxLDIsInUiXSxbMSwwLCJcXERlbHRhIiwyLHsiY3VydmUiOjN9XSxbMyw2LCIiLDAseyJsZXZlbCI6MSwic3R5bGUiOnsibmFtZSI6ImFkanVuY3Rpb24ifX1dXQ==
\[\begin{tikzcd}
	\Tm && \Ty \\
	& \iC
	\arrow[""{name=0, anchor=center, inner sep=0}, "\Sigma"', from=1-1, to=1-3]
	\arrow["\util"', from=1-1, to=2-2]
	\arrow["u", from=1-3, to=2-2]
	\arrow[""{name=1, anchor=center, inner sep=0}, "\Delta"', curve={height=18pt}, from=1-3, to=1-1]
	\arrow["\dashv"{anchor=center, rotate=91}, draw=none, from=0, to=1]
\end{tikzcd}\]
\end{defn}

We can think of $\Tm$ and $\Ty$ as the categories classifying terms and types, respectively.
The functor $\Sigma$ performs typing, while the functor $\Delta$ performs context extension.
Indeed, given $\Gamma \vdash a~\Tm$, one has $\Gamma \vdash \Sigma(a)~\Ty$, which is precisely the right-hand rule in a less familiar notation. 
\begin{mathpar}
  \inferrule
  { \Gamma \vdash a~\Tm }
  { \Gamma \vdash \Sigma(a)~\Ty }
  \and
  \inferrule
  { \Gamma \vdash a : A }
  { \Gamma \vdash A \type }
\end{mathpar}

By whiskering the following diagram
% https://q.uiver.app/#q=WzAsNCxbMCwwLCJcXFR5Il0sWzIsMCwiXFxUbSJdLFs0LDAsIlxcVHkiXSxbNiwwLCJcXGlDIl0sWzAsMSwiXFxEZWx0YSJdLFsxLDIsIlxcU2lnbWEiXSxbMiwzLCJ1Il0sWzEsMywiXFx1dGlsIiwwLHsiY3VydmUiOi01fV0sWzAsMiwiXFxpZCIsMix7ImN1cnZlIjo1fV0sWzEsOCwiXFxlcHNpbG9uIiwyLHsibGFiZWxfcG9zaXRpb24iOjQwLCJzaG9ydGVuIjp7InRhcmdldCI6MjB9fV0sWzcsMiwiIiwwLHsic2hvcnRlbiI6eyJzb3VyY2UiOjIwfSwic3R5bGUiOnsiaGVhZCI6eyJuYW1lIjoibm9uZSJ9fX1dXQ==
\[\begin{tikzcd}
	\Ty && \Tm && \Ty && \iC
	\arrow["\Delta", from=1-1, to=1-3]
	\arrow["\Sigma", from=1-3, to=1-5]
	\arrow["u", from=1-5, to=1-7]
	\arrow[""{name=0, anchor=center, inner sep=0}, "\util", curve={height=-30pt}, from=1-3, to=1-7]
	\arrow[""{name=1, anchor=center, inner sep=0}, "\id"', curve={height=30pt}, from=1-1, to=1-5]
	\arrow["\epsilon"'{pos=0.4}, shorten >=3pt, Rightarrow, from=1-3, to=1]
	\arrow[shorten <=3pt, Rightarrow, no head, from=0, to=1-5]
\end{tikzcd}\]
one obtains the following policy
% https://q.uiver.app/#q=WzAsMyxbMCwwLCJcXFR5Il0sWzIsMCwiXFxUbSJdLFsxLDEsIlxcaUMiXSxbMCwxLCJcXERlbHRhIl0sWzAsMiwidSIsMl0sWzEsMiwiXFx1dGlsIl0sWzEsNCwiXFxkZWx0YSIsMSx7InNob3J0ZW4iOnsidGFyZ2V0IjoyMH19XV0=
\[\begin{tikzcd}
	\Ty && \Tm \\
	& \iC
	\arrow["\Delta", from=1-1, to=1-3]
	\arrow[""{name=0, anchor=center, inner sep=0}, "u"', from=1-1, to=2-2]
	\arrow["\util", from=1-3, to=2-2]
	\arrow["\delta"{description}, shorten >=8pt, Rightarrow, from=1-3, to=0]
\end{tikzcd}\]
whose components $\delta_{A}$ are given by $u(\epsilon_{A})$.
Thus, given $\Gamma \vdash A~\Ty$, one obtains $\util\Delta(A) \vdash A(\delta_{A})~\Ty$ and $\util\Delta(A) \vdash \Delta(A)~\Tm$.
\begin{mathpar}
  \inferrule
  { \Gamma \vdash A~\Ty }
  { \util\Delta(A) \vdash \Delta(A)~\Tm }
  \and
  \inferrule
  { \Gamma \vdash A \type }
  { \Gamma.A \vdash q_{A} : A(\delta_{A}) }
\end{mathpar}
The following lemma shows that $\Delta(A)$ has the expected type.

\begin{lem}[Emergence of Substitution]
  Let $A \in \Ty$ be an object.
  Then $\Sigma\Delta(A) = A(\delta_{A})$.
  Here, the $A$ on the right-hand side is given by the fibrational Yoneda Lemma.
\end{lem}
\begin{proof}
  The Cartesian lift of $\delta_{A} : u(\Sigma\Delta(A)) \to u(A)$ is a morphism $\widetilde{\delta_{A}} : A(\delta_{A}) \to A$.
  It is then sufficient to show that the Cartesian lift of $\delta_{A}$ is a map $f : \Sigma\Delta(A) \to A$.
  This follows immediately from the construction of $\delta_{A}$ as $\epsilon_{A} : \Sigma\Delta(A) \to A$ is precisely the sought-after morphism.
\end{proof}

The classifier $\Ty.\Delta\Ty$ given by the pullback
% https://q.uiver.app/#q=WzAsNyxbMiwyLCJcXFR5Il0sWzMsMiwiXFxUbSJdLFs0LDIsIlxcaUMiXSxbNCwwLCJcXFR5Il0sWzIsMCwiXFxUeS5cXERlbHRhXFxUeSJdLFswLDIsIlxcVG0iXSxbMCwwLCJcXFRtLlxcU2lnbWFcXERlbHRhXFxUeSJdLFswLDEsIlxcRGVsdGEiLDJdLFsxLDIsIlxcdXRpbCIsMl0sWzMsMiwidSJdLFs0LDBdLFs0LDNdLFs0LDIsIiIsMSx7InN0eWxlIjp7Im5hbWUiOiJjb3JuZXIifX1dLFs1LDAsIlxcU2lnbWEiLDJdLFs2LDVdLFs2LDRdLFs2LDAsIiIsMCx7InN0eWxlIjp7Im5hbWUiOiJjb3JuZXIifX1dXQ==
\[\begin{tikzcd}
	{\Tm.\Sigma\Delta\Ty} && {\Ty.\Delta\Ty} && \Ty \\
	\\
	\Tm && \Ty & \Tm & \iC
	\arrow["\Delta"', from=3-3, to=3-4]
	\arrow["\util"', from=3-4, to=3-5]
	\arrow["u", from=1-5, to=3-5]
	\arrow[from=1-3, to=3-3]
	\arrow[from=1-3, to=1-5]
	\arrow["\lrcorner"{anchor=center, pos=0.125}, draw=none, from=1-3, to=3-5]
	\arrow["\Sigma"', from=3-1, to=3-3]
	\arrow[from=1-1, to=3-1]
	\arrow[from=1-1, to=1-3]
	\arrow["\lrcorner"{anchor=center, pos=0.125}, draw=none, from=1-1, to=3-3]
\end{tikzcd}\]
classifies a pair of judgments of the form $\Gamma \vdash A~\Ty$ and $\util\Delta(A) \vdash B~\Ty$.
The map $u.\util\Delta$ projects out $\Gamma \vdash A~\Ty$, which ends up landing in $\Gamma.A$ after traveling along $\util\Delta$.
Alternatively, one can travel along $u$ directly without visiting the middleman $\Tm$, ending up in $\Gamma$.
These two paths are related by a policy $\delta'$ obtained by whiskering the following diagram.
% https://q.uiver.app/#q=WzAsMyxbMCwwLCJcXFR5LlxcRGVsdGFcXFR5Il0sWzIsMCwiXFxUeSJdLFs0LDAsIlxcaUMiXSxbMCwxLCJ1LlxcdXRpbFxcRGVsdGEiXSxbMSwyLCJcXHV0aWxcXERlbHRhIiwwLHsiY3VydmUiOi0zfV0sWzEsMiwidSIsMix7ImN1cnZlIjozfV0sWzQsNSwiXFxkZWx0YSIsMCx7InNob3J0ZW4iOnsic291cmNlIjoyMCwidGFyZ2V0IjoyMH19XV0=
\[\begin{tikzcd}
	{\Ty.\Delta\Ty} && \Ty && \iC
	\arrow["{u.\util\Delta}", from=1-1, to=1-3]
	\arrow[""{name=0, anchor=center, inner sep=0}, "\util\Delta", curve={height=-18pt}, from=1-3, to=1-5]
	\arrow[""{name=1, anchor=center, inner sep=0}, "u"', curve={height=18pt}, from=1-3, to=1-5]
	\arrow["\delta", shorten <=5pt, shorten >=5pt, Rightarrow, from=0, to=1]
\end{tikzcd}\]
Henceforth, we let $v = u \circ u.\util\Delta$.

\subsection{Type Dependency}
\label{sec:type-dependency}

In dependent type theory, if one has a type $B$ that depends on $A$ and an element of $A$, then one can obtain a new type $B[a]$ by substituting $a$ into $B$.
\begin{mathpar}
  \inferrule
  { \Gamma \vdash a : A \\ \Gamma.A \vdash B \type }
  { \Gamma \vdash B[a] \type }
  \and
  \inferrule
  { \Gamma \vdash a : A \\ \Gamma.A \vdash b : B }
  { \Gamma \vdash b[a] : B[a] }
\end{mathpar}
First, we need to classify the premises.
% https://q.uiver.app/#q=WzAsMTAsWzIsNCwiXFxUeSJdLFszLDQsIlxcVG0iXSxbNCw0LCJcXGlDIl0sWzQsMiwiXFxUeSJdLFsyLDIsIlxcVHkuXFxEZWx0YVxcVHkiXSxbMCw0LCJcXFRtIl0sWzAsMiwiXFxUbS5cXFNpZ21hXFxEZWx0YVxcVHkiXSxbNCwwLCJcXFRtIl0sWzIsMCwiXFxUeS5cXERlbHRhXFxUeSJdLFswLDAsIlxcVG0uXFxTaWdtYVxcRGVsdGFcXFRtIl0sWzAsMSwiXFxEZWx0YSIsMl0sWzEsMiwiXFx1dGlsIiwyXSxbMywyLCJ1Il0sWzQsMF0sWzQsM10sWzUsMCwiXFxTaWdtYSIsMl0sWzYsNV0sWzYsNF0sWzcsMywiXFxTaWdtYSJdLFs4LDRdLFs4LDddLFs5LDhdLFs5LDZdXQ==
\[\begin{tikzcd}
	{\Tm.\Sigma\Delta\Tm} && {\Ty.\Delta\Ty} && \Tm \\
	\\
	{\Tm.\Sigma\Delta\Ty} && {\Ty.\Delta\Ty} && \Ty \\
	\\
	\Tm && \Ty & \Tm & \iC
	\arrow["\Delta"', from=5-3, to=5-4]
	\arrow["\util"', from=5-4, to=5-5]
	\arrow["u", from=3-5, to=5-5]
	\arrow[from=3-3, to=5-3]
	\arrow[from=3-3, to=3-5]
	\arrow["\Sigma"', from=5-1, to=5-3]
	\arrow[from=3-1, to=5-1]
	\arrow[from=3-1, to=3-3]
	\arrow["\Sigma", from=1-5, to=3-5]
	\arrow[from=1-3, to=3-3]
	\arrow[from=1-3, to=1-5]
	\arrow[from=1-1, to=1-3]
	\arrow[from=1-1, to=3-1]
\end{tikzcd}\]
Then for $\Gamma \vdash a : A$ and $\Gamma.A \vdash B \type$, we have $\Gamma.A \vdash (a, B)~\Tm.\Sigma\Delta\Ty$.

Let $\eta' = \util\eta$.
Then each component $\eta'_{a} : \Gamma \to \Gamma.A$ is a substitution, which corresponds to the substitution that substitutes $a$.
% https://q.uiver.app/#q=WzAsMyxbMCwxLCJcXFRtIl0sWzIsMSwiXFxpQyJdLFsxLDAsIlxcVG0iXSxbMCwxLCJcXHV0aWxcXERlbHRhXFxTaWdtYSIsMl0sWzAsMiwiXFxpZCJdLFsyLDEsIlxcdXRpbCJdLFsyLDMsIlxcZXRhJyIsMCx7InNob3J0ZW4iOnsidGFyZ2V0IjoyMH19XV0=
\[\begin{tikzcd}
	& \Tm \\
	\Tm && \iC
	\arrow[""{name=0, anchor=center, inner sep=0}, "\util\Delta\Sigma"', from=2-1, to=2-3]
	\arrow["\id", from=2-1, to=1-2]
	\arrow["\util", from=1-2, to=2-3]
	\arrow["{\eta'}", shorten >=3pt, Rightarrow, from=1-2, to=0]
\end{tikzcd}\]

This triangle can be lifted to a triangle involving the classifier $\Tm.\Sigma\Delta\Ty$.
% https://q.uiver.app/#q=WzAsNixbMCwzLCJcXFRtIl0sWzQsMywiXFxpQyJdLFsyLDIsIlxcVG0iXSxbNCwxLCJcXFR5Il0sWzAsMSwiXFxUbS5cXFNpZ21hXFxEZWx0YVxcVHkiXSxbMiwwLCJcXFRtIFxcdGltZXMgXFxUeSJdLFswLDIsIlxcaWQiXSxbMiwxLCJcXHV0aWwiXSxbMCwxLCJcXHV0aWxcXERlbHRhXFxTaWdtYSIsMl0sWzMsMSwidSJdLFs0LDNdLFs0LDBdLFs1LDJdLFs1LDNdLFs0LDUsInVeeyp9XFxpZCIsMCx7ImNvbG91ciI6WzAsNjAsNjBdLCJzdHlsZSI6eyJib2R5Ijp7Im5hbWUiOiJkYXNoZWQifX19LFswLDYwLDYwLDFdXSxbMiw4LCJcXGV0YSciLDAseyJjdXJ2ZSI6LTIsInNob3J0ZW4iOnsidGFyZ2V0IjoyMH19XSxbNSwxMCwidV57Kn1cXGV0YSciLDAseyJjdXJ2ZSI6LTIsInNob3J0ZW4iOnsidGFyZ2V0IjoyMH0sImNvbG91ciI6WzAsNjAsNjBdLCJzdHlsZSI6eyJib2R5Ijp7Im5hbWUiOiJkYXNoZWQifX19LFswLDYwLDYwLDFdXV0=
\[\begin{tikzcd}
	&& {\Tm \times \Ty} \\
	{\Tm.\Sigma\Delta\Ty} &&&& \Ty \\
	&& \Tm \\
	\Tm &&&& \iC
	\arrow["\id", from=4-1, to=3-3]
	\arrow["\util", from=3-3, to=4-5]
	\arrow[""{name=0, anchor=center, inner sep=0}, "\util\Delta\Sigma"', from=4-1, to=4-5]
	\arrow["u", from=2-5, to=4-5]
	\arrow[""{name=1, anchor=center, inner sep=0}, from=2-1, to=2-5]
	\arrow[from=2-1, to=4-1]
	\arrow[from=1-3, to=3-3]
	\arrow[from=1-3, to=2-5]
	\arrow["{u^{*}\id}", color={rgb,255:red,214;green,92;blue,92}, dashed, from=2-1, to=1-3]
	\arrow["{\eta'}", curve={height=-12pt}, shorten >=4pt, Rightarrow, from=3-3, to=0]
	\arrow["{u^{*}\eta'}", color={rgb,255:red,214;green,92;blue,92}, curve={height=-12pt}, shorten >=4pt, Rightarrow, dashed, from=1-3, to=1]
\end{tikzcd}\]

Traveling along $u^{*}\id$ yields $\Gamma.A \vdash u^{*}\id(a,B)~\Tm \times \Ty$.
We want the object to live in the fiber over $\Gamma$ instead of $\Gamma.A$.
This can be done by adjusting the policy as follows.
% https://q.uiver.app/#q=WzAsNSxbNCwxLCJcXFR5Il0sWzAsMSwiXFxUbS5cXFNpZ21hXFxEZWx0YVxcVHkiXSxbMiwwLCJcXFRtIFxcdGltZXMgXFxUeSJdLFsyLDEsIlxcVHkuXFxEZWx0YVxcVHkiXSxbNCwyLCJcXGlDIl0sWzIsMF0sWzEsMiwidV57Kn1cXGlkIiwwLHsiY29sb3VyIjpbMCw2MCw2MF0sInN0eWxlIjp7ImJvZHkiOnsibmFtZSI6ImRhc2hlZCJ9fX0sWzAsNjAsNjAsMV1dLFsxLDNdLFszLDBdLFswLDQsInUiXSxbMyw0LCJ2IiwyXSxbMiwzLCJ1XnsqfVxcZXRhJyIsMCx7ImxldmVsIjoyfV0sWzAsMTAsIlxcZGVsdGEnIiwwLHsic2hvcnRlbiI6eyJ0YXJnZXQiOjIwfX1dXQ==
\[\begin{tikzcd}
	&& {\Tm \times \Ty} \\
	{\Tm.\Sigma\Delta\Ty} && {\Ty.\Delta\Ty} && \Ty \\
	&&&& \iC
	\arrow[from=1-3, to=2-5]
	\arrow["{u^{*}\id}", from=2-1, to=1-3]
	\arrow[from=2-1, to=2-3]
	\arrow[from=2-3, to=2-5]
	\arrow["u", from=2-5, to=3-5]
	\arrow[""{name=0, anchor=center, inner sep=0}, "v"', from=2-3, to=3-5]
	\arrow["{u^{*}\eta'}", Rightarrow, from=1-3, to=2-3]
	\arrow["{\delta'}", shorten >=5pt, Rightarrow, from=2-5, to=0]
\end{tikzcd}\]

\subsection{JDDT $\Pi$-Types}
\label{sec:jddt-pi-types}

Having $\Pi$-types means to implement the following rules:
\begin{mathpar}
  \inferrule
  { \Gamma \vdash A \type \\ \Gamma.A \vdash B \type }
  { \Gamma \vdash \Pi_{A}B \type }
  \and
  \inferrule
  { \Gamma \vdash A \type \\ \Gamma.A \vdash b : B }
  { \Gamma \vdash \lambda_{A}b : \Pi_{A}B }
  \and
  \inferrule
  { \Gamma \vdash f : \Pi_{A}B \\ \Gamma \vdash a : A }
  { \Gamma \vdash f(a) : B[a] }
  \and
  \inferrule
  { \Gamma.A \vdash b : B \\ \Gamma \vdash a : A }
  { \Gamma \vdash (\lambda_{A}b)(a) = b[a] : B[a] }
\end{mathpar}

The classifiers for the premises of the formation and the introduction rules are constructed as indicated below.
% https://q.uiver.app/#q=WzAsMTAsWzAsNCwiXFxUbSJdLFsyLDQsIlxcVHkiXSxbMyw0LCJcXFRtIl0sWzQsNCwiXFxpQyJdLFs0LDIsIlxcVHkiXSxbMiwyLCJcXFR5LlxcRGVsdGFcXFR5Il0sWzAsMiwiXFxUbS5cXFNpZ21hXFxEZWx0YVxcVHkiXSxbNCwwLCJcXFRtIl0sWzIsMCwiXFxUeS5cXERlbHRhXFxUbSJdLFswLDAsIlxcVG0uXFxTaWdtYVxcRGVsdGFcXFRtIl0sWzAsMSwiXFxTaWdtYSIsMl0sWzEsMiwiXFxEZWx0YSIsMl0sWzIsMywiXFx1dGlsIiwyXSxbNCwzLCJ1Il0sWzUsMV0sWzUsNF0sWzYsMF0sWzYsNV0sWzcsNCwiXFxTaWdtYSJdLFs4LDVdLFs4LDddLFs5LDZdLFs5LDhdXQ==
\[\begin{tikzcd}
	{\Tm.\Sigma\Delta\Tm} && {\Ty.\Delta\Tm} && \Tm \\
	\\
	{\Tm.\Sigma\Delta\Ty} && {\Ty.\Delta\Ty} && \Ty \\
	\\
	\Tm && \Ty & \Tm & \iC
	\arrow["\Sigma"', from=5-1, to=5-3]
	\arrow["\Delta"', from=5-3, to=5-4]
	\arrow["\util"', from=5-4, to=5-5]
	\arrow["u", from=3-5, to=5-5]
	\arrow[from=3-3, to=5-3]
	\arrow[from=3-3, to=3-5]
	\arrow[from=3-1, to=5-1]
	\arrow[from=3-1, to=3-3]
	\arrow["\Sigma", from=1-5, to=3-5]
	\arrow[from=1-3, to=3-3]
	\arrow[from=1-3, to=1-5]
	\arrow[from=1-1, to=3-1]
	\arrow[from=1-1, to=1-3]
\end{tikzcd}\]

A dependent type theory has $\Pi$-types if the following is a commutative diagram.
% https://q.uiver.app/#q=WzAsNSxbMSwzLCJcXGlDIl0sWzIsMiwiXFxUeSJdLFsyLDAsIlxcVG0iXSxbMCwyLCJcXFR5LlxcRGVsdGFcXFR5Il0sWzAsMCwiXFxUeS5cXERlbHRhXFxUbSJdLFsxLDAsInUiXSxbMiwxLCJcXFNpZ21hIl0sWzMsMSwiXFxQaSIsMV0sWzQsMywiXFxTaWdtYS4oXFx1dGlsXFxEZWx0YS51KSIsMl0sWzQsMiwiXFxsYW1iZGEiXSxbNCwxLCIiLDIseyJzdHlsZSI6eyJuYW1lIjoiY29ybmVyIn19XSxbMywwLCJ2IiwyXV0=
\[\begin{tikzcd}
	{\Ty.\Delta\Tm} && \Tm \\
	\\
	{\Ty.\Delta\Ty} && \Ty \\
	& \iC
	\arrow["u", from=3-3, to=4-2]
	\arrow["\Sigma", from=1-3, to=3-3]
	\arrow["\Pi"{description}, from=3-1, to=3-3]
	\arrow["{\Sigma.(\util\Delta.u)}"', from=1-1, to=3-1]
	\arrow["\lambda", from=1-1, to=1-3]
	\arrow["\lrcorner"{anchor=center, pos=0.125}, draw=none, from=1-1, to=3-3]
	\arrow["v"', from=3-1, to=4-2]
\end{tikzcd}\]

The formation rule is precisely $\Pi$ and the introduction rule is precisely $\lambda$.
For the other rules, we need to classify
\begin{mathpar}
  \Gamma \vdash A \type \and \Gamma.A \vdash B \type \and \Gamma \vdash a : A \and \Gamma \vdash f : \Pi_{A}B
\end{mathpar}
It is clear that $\Tm.\Sigma\Delta\Ty$ classifies the first three.
The last judgment is classified by pulling back $\Sigma$ along the composite $\Pi' = \Pi \circ \pi$.

% https://q.uiver.app/#q=WzAsNSxbMCwyLCJcXFRtLlxcU2lnbWFcXERlbHRhXFxUeSJdLFsxLDIsIlxcVHkuXFxEZWx0YVxcVHkiXSxbMiwyLCJcXFR5Il0sWzIsMCwiXFxUbSJdLFswLDAsIihcXFRtLlxcU2lnbWFcXERlbHRhXFxUeSlcXFNpZ21hLlxcUGknXFxUbSJdLFszLDIsIlxcU2lnbWEiXSxbMCwxLCJcXHBpIiwyXSxbMSwyLCJcXFBpIiwyXSxbNCwwXSxbNCwzXSxbNCwyLCIiLDEseyJzdHlsZSI6eyJuYW1lIjoiY29ybmVyIn19XV0=
\[\begin{tikzcd}
	{(\Tm.\Sigma\Delta\Ty)\Sigma.\Pi'\Tm} && \Tm \\
	\\
	{\Tm.\Sigma\Delta\Ty} & {\Ty.\Delta\Ty} & \Ty
	\arrow["\Sigma", from=1-3, to=3-3]
	\arrow["\pi"', from=3-1, to=3-2]
	\arrow["\Pi"', from=3-2, to=3-3]
	\arrow[from=1-1, to=3-1]
	\arrow[from=1-1, to=1-3]
	\arrow["\lrcorner"{anchor=center, pos=0.125}, draw=none, from=1-1, to=3-3]
\end{tikzcd}\]

\bibliographystyle{alpha}
\bibliography{all}

\end{document}

% LocalWords:  prejudgmental
