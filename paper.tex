\documentclass{amsart}
\input{decls}
\title{}
\author{Frank Tsai}
\date{\today}
%\thanks{}
\begin{document}
\maketitle
\tableofcontents

\section{Introduction}
\label{sec:introduction}
Basically, a fibration is a categorical structure of \emph{indexing} and \emph{substitution}.

\begin{defn}
  Let $p : \iE \to \iB$ be a functor and $I$ be an object in the base category $\iB$.
  The \emph{fiber category} $\iE_{I} = p\inv(I)$ over $I$ consists of
  \begin{center}
    \begin{tabularx}{\linewidth}{lX}
      \textbf{objects} & an object $X \in \iE_{I}$ is an object in $X \in \iE$ so that $pX = I$.\\
      \textbf{morphisms} & a morphism $f : X \to Y$ in $\iE_{I}$ is a \emph{vertical} morphism $f : X \to Y$ in $\iE_{I}$.
                           That is, $pf = \id_{I}$.
    \end{tabularx}
  \end{center}
\end{defn}

\begin{defn}
  Let $p : \iE \to \iB$ be a functor.
  A morphism $f : X \to Y$ in $\iE$ is \emph{Cartesian} if for any $g : Z \to Y$ in $\iE$, any filling of the cospan $pf : pX \to pY$ and $pg : pZ \to pY$ has a unique lift that fills the cospan in $\iE$.
  % https://q.uiver.app/#q=WzAsOCxbMSwyLCJYIl0sWzMsMiwiWSJdLFsxLDAsIloiXSxbMSw2LCJwWCJdLFszLDYsInBZIl0sWzEsNCwicFoiXSxbMCwxLCJcXGlFIl0sWzAsNSwiXFxpQiJdLFswLDEsImYiLDJdLFsyLDEsImciLDAseyJjdXJ2ZSI6LTJ9XSxbMyw0LCJwZiIsMl0sWzUsNCwicGciLDAseyJjdXJ2ZSI6LTJ9XSxbNSwzLCJoIiwyXSxbMiwwLCJcXGh0aWwiLDIseyJzdHlsZSI6eyJib2R5Ijp7Im5hbWUiOiJkYXNoZWQifX19XSxbNiw3LCJwIiwyXV0=
\[\begin{tikzcd}
	& Z \\
	\iE \\
	& X && Y \\
	\\
	& pZ \\
	\iB \\
	& pX && pY
	\arrow["f"', from=3-2, to=3-4]
	\arrow["g", curve={height=-12pt}, from=1-2, to=3-4]
	\arrow["pf"', from=7-2, to=7-4]
	\arrow["pg", curve={height=-12pt}, from=5-2, to=7-4]
	\arrow["h"', from=5-2, to=7-2]
	\arrow["\htil"', dashed, from=1-2, to=3-2]
	\arrow["p"', from=2-1, to=6-1]
\end{tikzcd}\]
  
  A morphism $f : X \to Y$ is said to be \emph{Cartesian over} a morphism $u : I \to J$ in $\iB$ if $f$ is Cartesian and $pf = u$.
\end{defn}

\begin{defn}
  A fibration $p : \iE \to \iB$ is a functor so that for any $Y \in \iE$ and any morphism $u : I \to pY$ in $\iB$, there is a Cartesian morphism $f : X \to Y$ over $u$.
  % https://q.uiver.app/#q=WzAsOCxbMCwxLCJcXGlFIl0sWzAsMywiXFxpQiJdLFsxLDEsIlgiXSxbMiwxLCJZIl0sWzEsMCwiWiJdLFsxLDQsInBYIl0sWzIsNCwicFkiXSxbMSwzLCJwWiJdLFswLDEsInAiLDJdLFsyLDMsImYiLDJdLFs0LDMsImciXSxbNSw2LCJwZiIsMl0sWzcsNiwicGciXSxbNyw1LCJoIiwyXSxbNCwyLCJcXGh0aWwiLDIseyJzdHlsZSI6eyJib2R5Ijp7Im5hbWUiOiJkYXNoZWQifX19XV0=
\[\begin{tikzcd}
	& Z \\
	\iE & X & Y \\
	\\
	\iB & pZ \\
	& pX & pY
	\arrow["p"', from=2-1, to=4-1]
	\arrow["f"', from=2-2, to=2-3]
	\arrow["g", from=1-2, to=2-3]
	\arrow["pf"', from=5-2, to=5-3]
	\arrow["pg", from=4-2, to=5-3]
	\arrow["h"', from=4-2, to=5-2]
	\arrow["\htil"', dashed, from=1-2, to=2-2]
\end{tikzcd}\]

  $f$ is often called the \emph{terminal} or \emph{Cartesian} lifting of $u$.
  Indeed, the use of the definite article ``the'' is justified since Cartesian liftings are unique up to isomorphism (in the slice category).
\end{defn}

\begin{defn}
  A \emph{cloven} fibration $p : \iE \to \iB$ is a fibration equipped with a \emph{cleavage}, i.e., a choice of Cartesian liftings.
\end{defn}

Given a cloven fibration $p : \iE \to \iB$, every morphism $u : I \to J$ in $\iB$ induces a functor $u^{*} : \iE_{J} \to \iE_{I}$, which is often called a \emph{substitution functor}.

For $u : I \to pY$ in the base category $\iB$, the cleavage determines a Cartesian lifting $\ubar(Y) : X \to Y$.
The domain $X$ of the Cartesian lift $\ubar(X)$ is an object in $\iE_{J}$.
This determines the action on objects.

For $f : A \to B$ in $\iE_{J}$, the composite $f \circ \ubar(A)$ is also a Cartesian morphism over $u$.
The projected cospan in $\iB$ can be filled with the identity, which lifts to a unique morphism $u^{*}(A) \to u^{*}(B)$ in $\iE$.
This determines the action on morphisms; uniqueness guarantees that this assignment is functorial.
% https://q.uiver.app/#q=WzAsNyxbMiwwLCJBIl0sWzIsMiwiQiJdLFswLDAsInVeeyp9KEEpIl0sWzAsMiwidV57Kn0oQikiXSxbMCw2LCJJIl0sWzIsNiwiSiJdLFswLDQsIkkiXSxbMCwxLCJmIl0sWzIsMCwiXFx1YmFyKEEpIl0sWzMsMSwiXFx1YmFyKEIpIiwyXSxbNCw1LCJ1IiwyXSxbMiwxXSxbNiw1LCJ1Il0sWzYsNCwiXFxpZCIsMl0sWzIsMywiIiwwLHsic3R5bGUiOnsiYm9keSI6eyJuYW1lIjoiZGFzaGVkIn19fV1d
\[\begin{tikzcd}
	{u^{*}(A)} && A \\
	\\
	{u^{*}(B)} && B \\
	\\
	I \\
	\\
	I && J
	\arrow["f", from=1-3, to=3-3]
	\arrow["{\ubar(A)}", from=1-1, to=1-3]
	\arrow["{\ubar(B)}"', from=3-1, to=3-3]
	\arrow["u"', from=7-1, to=7-3]
	\arrow[from=1-1, to=3-3]
	\arrow["u", from=5-1, to=7-3]
	\arrow["\id"', from=5-1, to=7-1]
	\arrow[dashed, from=1-1, to=3-1]
\end{tikzcd}\]

There is an assignment $* : \iB\op \to \mathsf{Cat}$, mapping objects in the base category to its fiber category and morphisms in the base category to its substitution functor.
\begin{mathpar}
  I \mapsto \iE_{I} \and u \mapsto u^{*}
\end{mathpar}
However, this assignment is not functorial.
For any pair of composable morphisms $u, v$ in the base category $\iB$, $u^{*} \circ v^{*}$ is in general not equal to $(v \circ u)^{*}$ on the nose.
Rather, they are naturally isomorphic: for any $k \in \iE_{K}$, both $\ubar(k) \circ \ubar(u^{*}(k))$ and $\overline{v \circ u}(k)$ are Cartesian over $v \circ u$.
Thus, $v^{*}(u^{*}(k)) \iso (v \circ u)^{*}(k)$, which assembles into a natural isomorphism $v^{*} \circ u^{*} \iso (v \circ u)^{*}$.
\begin{mathpar}
  % https://q.uiver.app/#q=WzAsNixbMCwyLCJJIl0sWzEsMiwiSiJdLFsyLDIsIksiXSxbMiwwLCJrIl0sWzEsMCwidV57Kn0oaykiXSxbMCwwLCJ2XnsqfSh1XnsqfShrKSkiXSxbMCwxLCJ1IiwyXSxbMSwyLCJ2IiwyXSxbNCwzLCJcXHViYXIoaykiXSxbNSw0LCJcXHViYXIodV57Kn0oaykpIl0sWzQsMSwicCIsMSx7InNob3J0ZW4iOnsic291cmNlIjoyMCwidGFyZ2V0IjoyMH19XV0=
\begin{tikzcd}
	{v^{*}(u^{*}(k))} & {u^{*}(k)} & k \\
	\\
	I & J & K
	\arrow["u"', from=3-1, to=3-2]
	\arrow["v"', from=3-2, to=3-3]
	\arrow["{\ubar(k)}", from=1-2, to=1-3]
	\arrow["{\ubar(u^{*}(k))}", from=1-1, to=1-2]
	\arrow["p"{description}, shorten <=6pt, shorten >=6pt, from=1-2, to=3-2]
\end{tikzcd} \and % https://q.uiver.app/#q=WzAsNixbMCwyLCJJIl0sWzEsMiwiSiJdLFsyLDIsIksiXSxbMiwwLCJrIl0sWzAsMCwiKHYgXFxjaXJjIHUpXnsqfShrKSJdLFsxLDBdLFswLDEsInUiLDJdLFsxLDIsInYiLDJdLFs0LDNdLFs1LDEsInAiLDEseyJzaG9ydGVuIjp7InNvdXJjZSI6MjAsInRhcmdldCI6MjB9fV1d
\begin{tikzcd}
	{(v \circ u)^{*}(k)} & {} & k \\
	\\
	I & J & K
	\arrow["u"', from=3-1, to=3-2]
	\arrow["v"', from=3-2, to=3-3]
	\arrow["\overline{v \circ u}(k)", from=1-1, to=1-3]
	\arrow["p"{description}, shorten <=6pt, shorten >=6pt, from=1-2, to=3-2]
\end{tikzcd}
\end{mathpar}

\bibliographystyle{alpha}
\bibliography{all}

\end{document}
