\documentclass{amsart}
\input{decls}
\title{Algebraic Theory}
\author{Frank Tsai}
\date{\today}
%\thanks{}
\begin{document}
\maketitle
\tableofcontents

\newcommand{\gmult}{\hat{+}}
\newcommand{\ginv}{\hat{-}}
\newcommand{\gid}{\hat{0}}
\newcommand{\interp}[1]{\llbracket #1 \rrbracket}

\section{Introduction}
\label{sec:introduction}

The theory of groups, rings, modules, etc, fits into a nice framework.
This is my personal note for \cite{borceaux:handbook-2}.

\section{The Theory of Groups}
\label{sec:the-theory-of-groups}

The theory of groups can be described by giving
\begin{itemize}
\item A denumerable set of variables $x, y, z, \ldots$.
\item Three formal symbols $+, -, 0$.
\end{itemize}

The terms of the theory of groups are defined inductively as follows:
\begin{itemize}
\item all variables are terms.
\item if $t_{1}$ and $t_{2}$ are terms, then $t_{1} + t_{2}$ is a term.
\item if $t$ is a term, then $-t$ is a term.
\item $0$ is a term.
\end{itemize}

The axioms of the theory of groups are the following:
\begin{itemize}
\item $x + (y + z) = (x + y) + z$
\item $x + 0 = x$
\item $0 + x = x$
\item $x + (-x) = 0$
\item $(-x) + x = 0$
\end{itemize}

\begin{defn}
  A \emph{model} of this formal theory of groups consists of
  \begin{itemize}
  \item a set $G$;
  \item a function $\gmult : G \times G \to G$;
  \item a function $\ginv : G \to G$;
  \item an element $\gid \in G$.
  \end{itemize}
  The terms are interpreted recursively:
  \begin{itemize}
  \item the interpretation of a variable is any element of $G$;
  \item $\interp{x + y} = \interp{x} \gmult \interp{y}$;
  \item $\interp{-x} = \ginv \interp{x}$;
  \item $\interp{0} = \gid$.
  \end{itemize}
\end{defn}

The data $(G,\gmult,\ginv,\gid)$ constitute a model of the theory of groups when for any possible interpretation of variables, the interpretations of those group axioms hold.

\section{Universal Algebra}
\label{sec:universal-algebra}

\begin{defn}
  A \emph{presentation} of an algebraic theory is a theory with equality specified by
  \begin{itemize}
  \item a denumerable set of variables $x, y, z, \ldots$;
  \item for each $n \in \dN$, a denumerable set $\cO_{n}$ of $n$-ary function symbols: $f, g, h,\ldots$;
  \item a set of axioms
  \end{itemize}
  The terms of the theory are defined inductively:
  \begin{itemize}
  \item each variable is a term;
  \item if $f$ is an $n$-ary function symbol and $t_{1},\ldots,t_{n}$ are terms, then $f(t_{1},\ldots,t_{n})$ is a term.
  \end{itemize}
  An axiom is an equality between terms.
\end{defn}

\begin{defn}
  Let $\cT$ be a presentation of an algebraic theory.
  A \emph{model} of $\cT$ consists of
  \begin{itemize}
  \item a set $M$;
  \item for each $f \in \cO_{n}$, a function $\hat{f} : M^{n} \to M$.
  \end{itemize}
  The terms of $\cT$ are interpreted recursively:
  \begin{itemize}
  \item the interpretation of a variable is any element of $M$;
  \item $\interp{f(t_{1},\ldots,t_{n})} = \hat{f}(\interp{t_{1}},\ldots,\interp{t_{n}})$.
  \end{itemize}
\end{defn}

\begin{defn}
  Let $L$ and $M$ be models of a presentation $\cT$.
  A $\cT$-homomorphism $\varphi : L \to M$ is a function $\varphi : L \to M$ such that
  \[
    \varphi(\hat{f}(x_{1},\ldots,x_{n})) = \hat{f}(\varphi(x_{1}),\ldots,\varphi(x_{n}))
  \]
  for any function symbol $f$ and elements $x_{1},\ldots,x_{n} \in L$.
  Note that $\hat{f}$ on the left-hand side is a function $\hat{f} : L^{n} \to L$, while $\hat{f}$ on the right-hand side is a function $\hat{f} : M^{n} \to M$.
\end{defn}

\begin{lem}
  Let $\cT$ be a presentation of an algebraic theory.
  The models of $\cT$ and their homomorphisms constitute a category.
\end{lem}

There is a categorical presentation of this notion.

\begin{defn}
  Let $\cT$ be a presentation of an algebraic theory.
  We define a congruence relation $\sim$ on the set of terms as follows:
  \begin{mathpar}
    \inferrule
    { s = t }
    { s \sim t }\and
    \inferrule
    { s \sim t }
    { s[t_{1},\ldots,t_{n}/x_{1},\ldots,x_{n}] \sim t[t_{1},\ldots,t_{n}/x_{1},\ldots,x_{n}] }\and
    \inferrule
    { t_{1} \sim s_{1} \\ \cdots \\ t_{n} \sim s_{n} }
    { f(t_{1},\ldots,t_{n}) \sim f(s_{1},\ldots,s_{n}) }
  \end{mathpar}
\end{defn}

\begin{lem}
  Let $\cT$ be a presentation of an algebraic theory.
  Let $T_{n}$ be the set of terms containing only the variables $x_{1},\ldots,x_{n}$.
  Define $F_{n}$ to be the quotient of $T_{n}$ by the congruence $\sim$.
  Then $F_{n}$ has the structure of a $\cT$-model.
\end{lem}
\begin{proof}
  For each function symbol $f$ in $\cT$, the function $\hat{f} : F_{n}^{m} \to F_{n}$ is defined by
  the last rule defining $\sim$.
  The first two rules ensures that this interpretation satisfies all axioms regardless of how variables are interpreted.
\end{proof}

\begin{lem}
  Let $\cT$ be a presentation of an algebraic theory.
  In the category $\mathsf{Mod}_{\cT}$, $F_{n}$ is the $n$-th copower of $F_{1}$.
\end{lem}
\begin{proof}
  Let $F_{1}^{(i)}$ be the set of equivalence classes of terms involving just the variable $x_{i}$.
  This yields a $\cT$-model isomorphic to $F_{1}$.
  The obvious injections $\iota_{i} : F_{1}^{(i)} \to F_{n}$ form a colimit cone with nadir $F_{n}$.
\end{proof}

\begin{lem}
  Let $\cT$ be a presentation of an algebraic theory.
  The model $F_{n}$ is the free model on $n$ generators.
\end{lem}
\begin{proof}
  
\end{proof}

\begin{lem}
  Let $\cT$ be a presentation of an algebraic theory.
  Write $\sF$ for the full subcategory of $\mathsf{Mod}_{\cT}$ spanned by the free models $F_{n}$ on finitely many generators.
  The dual category $\sF\op$ has finite products and $\mathsf{Mod}_{\cT}$ is equivalent to the category of product-preserving functors from $\sF\op$ to $\mathsf{Set}$, and natural transformations between them.
\end{lem}

\section{A Categorical Approach to Universal Algebra}
\label{sec:a-categorical-approach-to-universal-algebra}

\begin{defn}
  An algebraic theory $\cT$ is a category with a denumerable set of objects $\{T^{0}, T^{1}, T^{2}, \ldots\}$.
  Each object $T^{n}$ is the $n$-th power of the object $T^{1}$.
  A model of $\cT$ is a finite-product-preserving functor $F : \cT \to \mathsf{Set}$.
  A homomorphism of $\cT$-models is a natural transformation.
\end{defn}

\bibliographystyle{alpha}
\bibliography{all}

\end{document}
