\documentclass{amsart}
\input{decls}
\title{Connectedness and Compactness}
\author{Frank Tsai}
\date{\today}
%\thanks{}
\begin{document}
\maketitle
\tableofcontents

\section{Connectedness}
\label{sec:connectedness}

\begin{defn}
  Let $X$ be a topological space.
  A \emph{separation} of $X$ is a pair $U, V$ of disjoint open sets such that $U \cup V = X$.
  The space $X$ is \emph{connected} of no separations exist.

  An equivalent condition for connectedness is that any continuous function
  \[
    f : X \to \{0,1\}
  \]
  is a constant function, where $\{0,1\}$ is equipped with the discrete topology.
\end{defn}

\begin{lem}\label{lem:connected-fiber}
  If $C$ and $D$ form a separation of $X$, and if $Y$ is a connected subspace of $X$, then $Y$ lies entirely in $C$ or $D$.
\end{lem}
\begin{proof}
  The separation is determined by a continuous function $f : X \to \{0,1\}$.
  Since $Y$ is connected, the restriction $f \circ \iota : Y \to \{0,1\}$ is a constant function.
  This is possible if and only if $\iota\inv(C) = Y$ or $\iota\inv(D) = Y$.
\end{proof}

\begin{lem}\label{lem:connected-union-connected}
  The union of a collection of connected subspaces of $X$ that have a point in common is connected.
\end{lem}
\begin{proof}
  Let $\{Y_{\alpha}\}_{\alpha \in J}$ be a family of connected subspaces of $X$.
  Let $f : \cup\{Y_{\alpha}\}_{\alpha \in J} \to \{0,1\}$ be a continuous function.
  Any such function is determined by a family of continuous functions $f_{\alpha} : Y_{\alpha} \to \{0,1\}$.

  Since $Y_{\alpha}$ is connected, $f_{\alpha}$ is constant.
  Furthermore, since those subspaces share a common point, that common point completely determines every continuous function $f_{\alpha}$.
  Thus, $f$ is also constant.
\end{proof}

\section{Connected Subspaces of the Real Line}
\label{sec:connected-subspaces-of-the-real-line}

\begin{thm}[Intermediate Value Theorem]
  Let $f : X \to Y$ be a continuous function, where $X$ is connected and $Y$ is an ordered set in the ordered topology.
  If $a$ and $b$ are two points of $X$ and if $r$ is a point of $Y$ lying between $f(a)$ and $f(b)$, then there is a point $c$ of $X$ such that $f(c) = r$.
\end{thm}
\begin{proof}
  Assume the hypothesis.
  Since $f$ is continuous and $X$ is connected, its image under $f$ must be connected.
  Consider the two open sets in $f(X)$.
  \begin{mathpar}
    A = f(X) \cap (-\oo, r) \and B = f(X) \cap (r, \oo)
  \end{mathpar}
  If $f\inv(r) = \varnothing$, then $A$ and $B$ constitute a separation, contradicting the fact that $f(X)$ is connected.
\end{proof}

\begin{defn}
  Given points $x$ and $y$ of the space $X$, a \emph{path} from $x$ to $y$ is a continuous map $f : [0,1] \to X$ such that $f(0) = x$ and $f(1) = y$.
  A space is \emph{path connected} if any two points is connected by a path.
\end{defn}

\begin{lem}\label{lem:path-connected-connected}
  A path connected space is connected.
\end{lem}
\begin{proof}
  Suppose that $X$ is a path connected space.
  Assume that there is a nonconstant continuous function $f : X \to \{0, 1\}$.
  Then, choose arbitrary points $a \in f\inv(0)$ and $b \in f\inv(1)$.
  Since $X$ is path connected, there is a continuous function $g : [0,1] \to X$ with $g(0) = a$ and $g(1) = b$, but this contradicts \cref{lem:connected-fiber}.
\end{proof}

The converse of \cref{lem:path-connected-connected} does not hold.
The classic example is the \emph{topologist's sine curve}.
\[
  S = \{x\sin(1/x) \mid 0 < x \leq 1\}
\]
Since $S$ is the image of a connected space $(0,1]$ under a continuous function, $S$ is connected, but $S$ is not path connected.

\section{Components and Local Connectedness}
\label{sec:components-and-local-connectedness}

\begin{defn}
  Given an arbitrary topological space $X$, define a relation $x \sim y$ iff there is a (path-)connected subspace of $X$ containing both $x$ and $y$.
  The equivalence class $[x]$ is a \emph{(path-)connected component}.
\end{defn}

Clearly, the equivalence class $[x]$ is a connected subspace of $X$.
This is a consequence of \cref{lem:connected-union-connected}.

\begin{defn}
  A space $X$ is \emph{locally (path-)connected at $x$} if for every neighborhood $U$ of $x$, there is a (path-)connected neighborhood $V$ of $x$ contained in $U$.
  A space is \emph{locally (path-)connected} if it is locally (path-)connected at every point.
\end{defn}

\begin{lem}
  A space $X$ is locally connected if and only if for every open set $U$ of $X$, each component of $U$ is open in $X$.
\end{lem}
\begin{proof}
  Suppose that $X$ is locally connected.
  Let $U$ be an open set of $X$ and $C$ be a component of $U$.
  For any $x \in U$, there is a connected neighborhood $V$ of $x$ contained in $U$.
  Connectedness demands that $V$ lies entirely within $C$, proving that $C$ is open.

  Conversely, let $x \in X$ and $U$ be a neighborhood of $x$.
  The hypothesis yields a connected neighborhood of $x$ contained within $U$.
\end{proof}

\section{Compactness}
\label{sec:compactness}

\begin{defn}
  A collection $\sA$ of subsets of a space $X$ is a \emph{covering} of $X$ if the union of the elements of $\sA$ is equal to $X$.
  $\sA$ is called an \emph{open covering} if each of its elements is open.
\end{defn}

\begin{defn}
  A space $X$ is \emph{compact} if every open covering $\sA$ of $X$ has a finite subcover.
\end{defn}

Compactness can be characterized in a different way.

\begin{defn}
  A collection $\sC$ of subsets of $X$ has the \emph{finite intersection property} if for every finite subcollection
  \[
    \{C_{1},\ldots,C_{n}\}
  \]
  of $\sC$, the intersection is nonempty.
\end{defn}

\begin{thm}
  Let $X$ be a topological space.
  Then $X$ is compact if and only if every collection $\sC$ of closed sets in $X$ having the finite intersection property, the intersection $\bigcap_{C \in \sC}C$ of all the elements of $\sC$ is nonempty.
\end{thm}
\begin{proof}
  If $X$ is compact and $\sA$ is an open covering, then for each $A_{i} \in \sA$, consider the corresponding closed set:
  \[
    C_{i} = X \setminus A_{i}
  \]
  The following hold:
  \begin{enumerate}
  \item $\cup\sA = X$ iff $\cap\sC = \varnothing$.
  \item Any finite subcover $\sA'$ corresponds to a finite subcollection $\sC'$ of $\cC$ so that $\cap\sC' = \varnothing$.
  \end{enumerate}
  $X$ is compact iff for any $\sA$, if no finite subcollection of $\sA$ covers $X$, then $\sA$ does not cover $X$ iff for any $\sC$, if the intersection of any finite subcollection of $\sC$ is nonempty, then $\cap\sC$ is nonempty.
\end{proof}

\begin{thm}[Extreme Value Theorem]
  Let $f : X \to Y$ be continuous, where $Y$ is an ordered set in the order topology.
  If $X$ is compact, then there are points $c$ and $d$ in $X$ so that $f(c) \leq f(x) \leq f(d)$ for all $x \in X$.
\end{thm}
\begin{proof}
  Since $X$ is compact, the image $A = f(X)$ is also compact.
  Note that if $A$ does not have a largest element, then
  \[
    \sA = \{(-\oo,a) \mid a \in A\}
  \]
  is an open cover of $A$.
  Compactness implies that finitely many of these open sets cover $A$, i.e.,
  \[
    \sA' = \{(-\oo,a_{i}) \mid i \in [n]\}
  \]
  is a finite subcover.
  Let $a_{i}$ be the largest element of $a_{1},\ldots,a_{n}$.
  Then $a_{i}$ does not lie in $f(X)$, contradicting the fact that $\sA'$ covers $A$.

  Conversely, if $A$ does not have a smallest element, then
  \[
    \sB = \{(a,\oo) \mid a \in A\}
  \]
  covers $A$.
  A contradiction can then be derived in the same way.
\end{proof}

\end{document}
