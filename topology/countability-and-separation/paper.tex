\documentclass{amsart}
\input{decls}
\title{}
\author{Frank Tsai}
\date{\today}
%\thanks{}
\begin{document}
\maketitle
\tableofcontents

\section{Countability Axioms}
\label{sec:countability-axioms}

\begin{defn}
  A space $X$ has a \emph{countable basis at $x$} if there is a countable collection $\sB$ of neighborhoods of $x$ such that each neighborhood of $x$ contains at least one of the elements of $\sB$.
  A space with a countable basis at each of its points satisfies the \emph{first countability axiom}.
\end{defn}

An immediate consequence of this definition is that elements $B \in \cB$ can be arranged into an ascending sequence ordered by inclusion.
That is, one has
\[
  B_{1} \subseteq B_{2} \subseteq \cdots
\]

\begin{thm}
  Let $X$ be a topological space.
  \begin{enumerate}
  \item Let $A \subseteq X$.
    If there is a sequence of points of $A$ converging to $x$, then $x \in \Abar$; the converse holds if $X$ is first-countable.
  \item Let $f : X \to Y$.
    If $f$ is continuous, then for every convergent sequence $x_{n} \to x$ in $X$, the sequence $f(x_{n})$ converges to $f(x)$.
    The converse holds if $X$ is first-countable.
  \end{enumerate}
\end{thm}

\end{document}
