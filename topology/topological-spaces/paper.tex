\documentclass{amsart}
\input{decls}
\title{}
\author{Frank Tsai}
\date{\today}
%\thanks{}
\begin{document}
\maketitle
\tableofcontents

\section{Introduction}
\label{sec:introduction}
% ...
\begin{defn}
  A \emph{topology} on a set $X$ consists of a collection $\cT$ of subsets of $X$ such that
  \begin{enumerate}
  \item $\varnothing$ and $X$ are in $\cT$;
  \item the union of any subcollection of $\cT$ is in $\cT$;
  \item the intersection of any finite subcollection of $\cT$ is in $\cT$.
  \end{enumerate}
  A set equipped with a topology is called a \emph{topological space}.
  Subsets $U$ of $X$ in $\cT$ are called \emph{open sets} and their complements are called \emph{closed sets}.
\end{defn}

\begin{eg}
  For any set $X$ the collection of all subsets of $X$ is a topology on $X$.
  It is called the \emph{discrete topology}.
  Dually, the collection of just $\varnothing$ and $X$ is also a topology on $X$.
  It is called the \emph{indiscrete topology} or the \emph{trivial topology}.
\end{eg}

\begin{eg}
  Let $X$ be a set and $\cT$ be the collection of all subsets $U$ of $X$ such that $X \setminus U$ is either finite or is all of $X$.
  The topology $\cT$ on $X$ is called the \emph{finite complement topology}.
\end{eg}

\begin{defn}
  Let $X$ be a set.
  Topologies on $X$ form a lattice ordered by set inclusion.
  If $\cT \subseteq \cT'$ we say that $\cT'$ is \emph{finer} than $\cT$, and dually that $\cT$ is \emph{coarser} than $\cT'$.
\end{defn}

\section{Basis for a Topology}
\label{sec:basis-for-a-topology}

Usually it is too difficult to specify all the open sets of a topology on a set $X$.
We can specify a smaller collection of open sets of $X$ and define a topology in terms of that.
\begin{defn}
  Let $X$ be a set.
  A \emph{basis} for a topology on $X$ is a collection $\cB$ of subsets of $X$ (called \emph{basis elements}) such that
  \begin{enumerate}
  \item for each $x \in X$, there is a basis element $B$ containing $x$;
  \item if $x \in B_{1} \cap B_{2}$ for some basis elements $B_{1}$ and $B_{2}$, then there is a basis element $B_{3}$ containing $x$ such that $B_{3} \subseteq B_{1} \cap B_{2}$.
  \end{enumerate}
\end{defn}

\begin{defn}
  Let $\cB$ be a basis.
  The topology $\cT$ \emph{generated by} $\cB$ is defined inductively as follows:
  \begin{enumerate}
  \item all basis elements are open;
  \item if $\{U_{\alpha}\}_{\alpha \in J}$ is a family of open sets, then $\bigcup_{\alpha \in J}U_{\alpha}$ is open.
  \end{enumerate}
\end{defn}

Let's verify that $\cT$ is a topology.
First, $\varnothing$ is vacuously open, and for each $x \in X$, by the first condition of a basis, there is a basis element $B_{x}$ such that $x \in B_{x}$.
Thus, $\{B_{x}\}_{x \in X}$ is a collection of open sets, and $X = \bigcup_{x \in X}B_{x}$ is open.

Second, the union of an arbitrary collection of open sets is open by definition.

Finally, suppose that $U = U_{1} \cap U_{2}$.
Then for any $x \in U_{1} \cap U_{2}$, there are two basis elements such that $x \in A_{x} \cap B_{x}$.
Then by the second condition of a basis, there is a basis such that $C_{x} \subseteq A_{x} \cap B_{x}$.
This defines a family of open sets $\{C_{x}\}_{x \in U_{1} \cap U_{2}}$.
Thus, $U_{1} \cap U_{2} = \bigcup_{x \in U_{1} \cap U_{2}}$ is open.
This argument can be extended to any finite intersection of open sets by induction.

\begin{lem}
  Let $X$ be a topological space.
  Let $\cC$ be a collection of open sets of $X$ such that for each open set $U$ of $X$ and each $x \in U$, there is an element $C \in \cC$ such that $x \in C \subseteq U$.
  Then $\cC$ is a basis for the topology of $X$.
\end{lem}

\begin{lem}
  Let $\cB$ and $\cB'$ be the bases for the topologies $\cT$ and $\cT'$, respectively, on $X$.
  Then the following are equivalent:
  \begin{enumerate}
  \item $\cT'$ is finer than $\cT$;
  \item for each $x \in X$ and each basis element $B \in \cB$ containing $x$, there is a basis element $B' \in \cB'$ such that $x \in B' \subseteq B$.
  \end{enumerate}
\end{lem}
\begin{proof}
  The ``if'' direction is trivial.
  For the ``only if'' direction, suppose that (ii) holds.
  Let $U \in \cT$, then $U$ is the union of a collection of basis elements in $\cT$.
  For each $x \in U$ and $x \in B_{x}$, (ii) yields a family of open sets $(B'_{x})_{x \in U}$, each of which is contained in the corresponding $B_{x}$.
  Thus, $U = \bigcup_{x \in U}B'_{x}$ is open in $\cT'$.
\end{proof}

We now define topologies on the real line $\dR$.

\begin{defn}
  Let $\cB$ be the collection of all open intervals
  \[
    (a,b) = \{x \mid a < x < b\}
  \]
  the topology generated by $\cB$ is called the \emph{standard topology} on the real line.
\end{defn}

\begin{defn}
  Let $\cB$ be the collection of all half-open intervals
  \[
    [a,b) = \{x \mid a \leq x < b\}
  \]
  the topology generated by $\cB$ is called the \emph{lower limit topology} on the real line.
  We write $\dR_{\ell}$ when $\dR$ is given the lower limit topology.
\end{defn}

\begin{defn}
  Let $K$ denote the set of all numbers of the form $1/n$ for positive integer $n$.
  Let $\cB$ be the collection of all open intervals $(a,b)$, along with all sets of the form $(a,b) \setminus K$.
  The topology generated by $\cB$ is called the \emph{$K$-topology}.
  We write $\dR_{K}$ when $\dR$ is given the $K$ topology.
\end{defn}

\begin{defn}
  A \emph{subbasis} $\cS$ for a topology on $X$ is a collection of subsets of $X$ whose union equals $X$.
  The topology generated by $\cS$ is defined to be the collection $\cT$ of all unions of finite intersections of elements of $\cS$.
\end{defn}
To verify that $\cT$ is a topology, it suffices to verify that the collection of all finite intersections of elements of $\cS$ is a basis.
First, for any $x \in X$, there is some $S \in \cS$ that contains $x$ since the union of all $S \in \cS$ is $X$.
Second, for any two finite intersections $S_{1}$ and $S_{2}$ of elements of $\cS$ and any $x \in S_{1} \cap S_{2}$, the intersection itself is a finite intersection of elements of $\cS$ that contains $\cS$.

\section{Continuous Functions}
\label{sec:continuous-functions}

\begin{defn}
  Let $(X,\cT)$ and $(Y,\cT')$ be topological spaces.
  A \emph{continuous} function $f : (X,\cT) \to (Y,\cT')$ is a function $f : X \to Y$ such that for all $U \in \cT'$, $f\inv(U) \in \cT$.
\end{defn}
One can check that topological spaces and continuous functions assemble into a category.

\begin{lem}\label{lem:basis-continuous}
  Let $\cB$ be a basis for a topology on $Y$.
  A function $f : X \to Y$ is continuous if and only if $f\inv(B)$ is open for all basis element $B \in \cB$.
\end{lem}
\begin{proof}
  The ``if'' direction is trivial.
  For the ``only if'' direction, note that every open set of $Y$ can be expressed as the union of some basis elements
  \[
    U = \bigcup_{\alpha \in J}B_{\alpha}
  \]
  Then
  \[
    f\inv(U) = \bigcup_{\alpha \in J} f\inv(B_{\alpha})
  \]
  Topology is closed under arbitrary union.
  Thus, $f\inv(U)$ is an open set of $X$.
\end{proof}

\begin{lem}
  Furthermore, if $\cB$ is given by a subbasis $\cS$, then a function $f : X \to Y$ is continuous if and only if $f\inv(S)$ is open for all element $S \in \cS$.
\end{lem}
\begin{proof}
  Again, the ``if'' direction is trivial.
  For the ``only if'' direction, note that every basis element is a finite intersection of elements of $\cS$, i.e.,
  \[
    B = \bigcap_{\alpha \in [n]} S_{\alpha}
  \]
  Thus,
  \[
    f\inv(B) = \bigcap_{\alpha \in [n]}f\inv(S_{\alpha})
  \]
  Topology is closed under finite intersection.
  Thus, $f\inv(B)$ is an open set of $X$.
  By \cref{lem:basis-continuous}, $f$ is continuous.
\end{proof}

\end{document}
