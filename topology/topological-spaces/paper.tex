\documentclass{amsart}
\input{decls}
\title{}
\author{Frank Tsai}
\date{\today}
%\thanks{}
\begin{document}
\maketitle
\tableofcontents

\section{Introduction}
\label{sec:introduction}
% ...
\begin{defn}
  A \emph{topology} on a set $X$ consists of a collection $\cT$ of subsets of $X$ such that
  \begin{enumerate}
  \item $\varnothing$ and $X$ are in $\cT$;
  \item the union of any subcollection of $\cT$ is in $\cT$;
  \item the intersection of any finite subcollection of $\cT$ is in $\cT$.
  \end{enumerate}
  A set equipped with a topology is called a \emph{topological space}.
  Subsets $U$ of $X$ in $\cT$ are called \emph{open sets} and their complements are called \emph{closed sets}.
\end{defn}

\begin{eg}
  For any set $X$ the collection of all subsets of $X$ is a topology on $X$.
  It is called the \emph{discrete topology}.
  Dually, the collection of just $\varnothing$ and $X$ is also a topology on $X$.
  It is called the \emph{indiscrete topology} or the \emph{trivial topology}.
\end{eg}

\begin{eg}
  Let $X$ be a set and $\cT$ be the collection of all subsets $U$ of $X$ such that $X \setminus U$ is either finite or is all of $X$.
  The topology $\cT$ on $X$ is called the \emph{finite complement topology}.
\end{eg}

\begin{defn}
  Let $X$ be a set.
  Topologies on $X$ form a lattice ordered by set inclusion.
  If $\cT \subseteq \cT'$ we say that $\cT'$ is \emph{finer} than $\cT$, and dually that $\cT$ is \emph{coarser} than $\cT'$.
\end{defn}

\begin{defn}
  Given a topological space $X$, open sets containing a point $x \in X$ are given a more descriptive name.
  A \emph{neighborhood} of $x$ is an open set $U$ containing $x$.
\end{defn}

Note that every point $x \in X$ has at lease one neighborhood because the entire set $X$ itself is open.

\section{Continuous Functions}
\label{sec:continuous-functions}

\begin{defn}
  Let $(X,\cT)$ and $(Y,\cT')$ be topological spaces.
  A \emph{continuous} function $f : (X,\cT) \to (Y,\cT')$ is a function $f : X \to Y$ such that for all $U \in \cT'$, $f\inv(U) \in \cT$.
\end{defn}
One can check that topological spaces and continuous functions assemble into a category.

\begin{lem}
  Topological spaces and continuous functions assemble into a category $\mathsf{Top}$.
\end{lem}

Since the forgetful functor $U$ is faithful, it reflects monomorphisms and epimorphisms.
Thus, every continuous function whose underlying function is bijective is both a monomorphism and an epimorphism.
However, not all monic epimorphisms in $\mathsf{Top}$ are isomorphisms.

\begin{defn}
  A homeomorphism $f : X \to Y$ is a bijection so that both $f$ and $f\inv$ are continuous.
\end{defn}

\begin{lem}[Pasting Lemma]
  Let $X = A \cup B$, where $A$ and $B$ are closed in $X$.
  Let $f : A \to Y$ and $g : B \to Y$ be continuous.
  If $f$ and $g$ agree for every point in $A \cap B$, the $f$ and $g$ determine a continuous function $h : X \to Y$.
\end{lem}

\section{Basis for a Topology}
\label{sec:basis-for-a-topology}

Usually it is too difficult to specify all the open sets of a topology on a set $X$.
We can specify a smaller collection of open sets of $X$ and define a topology in terms of that.
\begin{defn}
  Let $X$ be a set.
  A \emph{basis} for a topology on $X$ is a collection $\cB$ of subsets of $X$ (called \emph{basis elements}) such that
  \begin{enumerate}
  \item for each $x \in X$, there is a basis element $B$ containing $x$;
  \item if $x \in B_{1} \cap B_{2}$ for some basis elements $B_{1}$ and $B_{2}$, then there is a basis element $B_{3}$ containing $x$ such that $B_{3} \subseteq B_{1} \cap B_{2}$.
  \end{enumerate}
\end{defn}

\begin{defn}
  Let $\cB$ be a basis.
  The topology $\cT$ \emph{generated by} $\cB$ is defined inductively as follows:
  \begin{enumerate}
  \item all basis elements are open;
  \item if $\{U_{\alpha}\}_{\alpha \in J}$ is a family of open sets, then $\bigcup_{\alpha \in J}U_{\alpha}$ is open.
  \end{enumerate}
\end{defn}

Let's verify that $\cT$ is a topology.
First, $\varnothing$ is vacuously open, and for each $x \in X$, by the first condition of a basis, there is a basis element $B_{x}$ such that $x \in B_{x}$.
Thus, $\{B_{x}\}_{x \in X}$ is a collection of open sets, and $X = \bigcup_{x \in X}B_{x}$ is open.

Second, the union of an arbitrary collection of open sets is open by definition.

Finally, suppose that $U = U_{1} \cap U_{2}$.
Then for any $x \in U_{1} \cap U_{2}$, there are two basis elements such that $x \in A_{x} \cap B_{x}$.
Then by the second condition of a basis, there is a basis such that $C_{x} \subseteq A_{x} \cap B_{x}$.
This defines a family of open sets $\{C_{x}\}_{x \in U_{1} \cap U_{2}}$.
Thus, $U_{1} \cap U_{2} = \bigcup_{x \in U_{1} \cap U_{2}}$ is open.
This argument can be extended to any finite intersection of open sets by induction.

\begin{lem}
  Let $X$ be a topological space.
  Let $\cC$ be a collection of open sets of $X$ such that for each open set $U$ of $X$ and each $x \in U$, there is an element $C \in \cC$ such that $x \in C \subseteq U$.
  Then $\cC$ is a basis for the topology of $X$.
\end{lem}

\begin{lem}
  Let $\cB$ and $\cB'$ be the bases for the topologies $\cT$ and $\cT'$, respectively, on $X$.
  Then the following are equivalent:
  \begin{enumerate}
  \item $\cT'$ is finer than $\cT$;
  \item for each $x \in X$ and each basis element $B \in \cB$ containing $x$, there is a basis element $B' \in \cB'$ such that $x \in B' \subseteq B$.
  \end{enumerate}
\end{lem}
\begin{proof}
  The ``if'' direction is trivial.
  For the ``only if'' direction, suppose that (ii) holds.
  Let $U \in \cT$, then $U$ is the union of a collection of basis elements in $\cT$.
  For each $x \in U$ and $x \in B_{x}$, (ii) yields a family of open sets $(B'_{x})_{x \in U}$, each of which is contained in the corresponding $B_{x}$.
  Thus, $U = \bigcup_{x \in U}B'_{x}$ is open in $\cT'$.
\end{proof}

We now define topologies on the real line $\dR$.

\begin{defn}
  Let $\cB$ be the collection of all open intervals
  \[
    (a,b) = \{x \mid a < x < b\}
  \]
  the topology generated by $\cB$ is called the \emph{standard topology} on the real line.
\end{defn}

\begin{defn}
  Let $\cB$ be the collection of all half-open intervals
  \[
    [a,b) = \{x \mid a \leq x < b\}
  \]
  the topology generated by $\cB$ is called the \emph{lower limit topology} on the real line.
  We write $\dR_{\ell}$ when $\dR$ is given the lower limit topology.
\end{defn}

\begin{defn}
  Let $K$ denote the set of all numbers of the form $1/n$ for positive integer $n$.
  Let $\cB$ be the collection of all open intervals $(a,b)$, along with all sets of the form $(a,b) \setminus K$.
  The topology generated by $\cB$ is called the \emph{$K$-topology}.
  We write $\dR_{K}$ when $\dR$ is given the $K$ topology.
\end{defn}

\begin{defn}
  A \emph{subbasis} $\cS$ for a topology on $X$ is a collection of subsets of $X$ whose union equals $X$.
  The topology generated by $\cS$ is defined to be the collection $\cT$ of all unions of finite intersections of elements of $\cS$.
\end{defn}
To verify that $\cT$ is a topology, it suffices to verify that the collection of all finite intersections of elements of $\cS$ is a basis.
First, for any $x \in X$, there is some $S \in \cS$ that contains $x$ since the union of all $S \in \cS$ is $X$.
Second, for any two finite intersections $S_{1}$ and $S_{2}$ of elements of $\cS$ and any $x \in S_{1} \cap S_{2}$, the intersection itself is a finite intersection of elements of $\cS$ that contains $\cS$.

\begin{lem}\label{lem:basis-continuous}
  Let $\cB$ be a basis for a topology on $Y$.
  A function $f : X \to Y$ is continuous if and only if $f\inv(B)$ is open for all basis element $B \in \cB$.
\end{lem}
\begin{proof}
  The ``if'' direction is trivial.
  For the ``only if'' direction, note that every open set of $Y$ can be expressed as the union of some basis elements
  \[
    U = \bigcup_{\alpha \in J}B_{\alpha}
  \]
  Then
  \[
    f\inv(U) = \bigcup_{\alpha \in J} f\inv(B_{\alpha})
  \]
  Topology is closed under arbitrary union.
  Thus, $f\inv(U)$ is an open set of $X$.
\end{proof}

\begin{lem}
  Furthermore, if $\cB$ is given by a subbasis $\cS$, then a function $f : X \to Y$ is continuous if and only if $f\inv(S)$ is open for all element $S \in \cS$.
\end{lem}
\begin{proof}
  Again, the ``if'' direction is trivial.
  For the ``only if'' direction, note that every basis element is a finite intersection of elements of $\cS$, i.e.,
  \[
    B = \bigcap_{\alpha \in [n]} S_{\alpha}
  \]
  Thus,
  \[
    f\inv(B) = \bigcap_{\alpha \in [n]}f\inv(S_{\alpha})
  \]
  Topology is closed under finite intersection.
  Thus, $f\inv(B)$ is an open set of $X$.
  By \cref{lem:basis-continuous}, $f$ is continuous.
\end{proof}

\section{Interior and Closure}
\label{sec:interior-and-closure}

Given a topological space $X$, $X$ the poset $PX$ admits a closure operator $C$ and a kernel operator $K$.
The closure operator maps a subset $A \subseteq S$ to the smallest closed set containing $A$, while the kernel operator maps $A$ to the largest open set contained in $A$.

\begin{lem}
  Let $A$ be a subset of a topological space $X$.
  \begin{enumerate}
  \item $x \in C(A)$ if and only if every neighborhood $U_{x}$ intersects $A$.
  \item Suppose that $\cB$ is a basis for the topology of $X$, then $x \in C(A)$ if and only if every basis element $B$ containing $x$ intersects $A$.
  \end{enumerate}
\end{lem}
\begin{proof}
  (i): Let $x \notin C(A)$.
  Then $x$ is in the complement of $C(A)$, which is an open set not intersecting $A$.
  Conversely, if $U_{x}$ is an open set not intersecting $A$, then its complement is a closed set containing $A$.
  Since $C(A) \subseteq X \setminus U_{x}$, $C(A)$ cannot contain $x$.

  (ii): By (i), it suffices to prove that every neighborhood $U_{x}$ intersects $A$ if and only if every basis element $B$ containing $x$ intersects $A$.
  The ``if'' direction is trivial because every basis element is open.
  Conversely, if every basis element $B$ containing $x$ intersects $A$, then any open set $U$ contains a basis that contains $x$.
  Thus, $U$ intersects $A$.
\end{proof}

\section{Hausdorff Space}
\label{sec:hausdorff-space}

\begin{defn}
  A topological space $X$ is a \emph{Hausdorff space} if for any distinct points $x$ and $y$ in $X$, there are some neighborhoods $U_{x}$ and $U_{y}$ that are disjoint.
\end{defn}

\section{Subspace Topology}
\label{sec:subspace-topology}

Given a topological space $X$ and a subset $A \subseteq X$.
We wish to topologize $A$ so that the inclusion function $\iota : A \into X$ is continuous.
This is always possible because we can always pick the discrete topology.
However, we want the ``most efficient'' topology.

\begin{defn}
  Given a topological space $X$ and a subset $A \subseteq X$.
  The \emph{subspace topology} is given by the subbasis
  \[
    S = \{\iota\inv(U) \mid U\,\text{is open in}\,X\}
  \]
\end{defn}

\begin{lem}
  Given a topological space $(X, \cT)$ and a subset $A \subseteq X$.
  Let $\cS$ be the subspace topology.
  For any topology $\cT'$ so that the inclusion $\iota : (A, \cT') \to (X, \cT)$ is continuous, the identity function $\id : A \to A$ is continuous.
  % https://q.uiver.app/#q=WzAsNCxbMCwwLCJHIFxcdGltZXMgRyBcXHRpbWVzIEciXSxbMiwwLCJHIFxcdGltZXMgRyJdLFswLDIsIkcgXFx0aW1lcyBHIl0sWzIsMiwiRyJdLFswLDIsIlxcaWRfe0d9IFxcdGltZXMgXFxjZG90IiwyXSxbMSwzLCJcXGNkb3QiXSxbMCwxLCJcXGNkb3QgXFx0aW1lcyBcXGlkX3tHfSJdLFsyLDMsIlxcY2RvdCIsMl1d
\begin{tikzcd}
	{G \times G \times G} && {G \times G} \\
	\\
	{G \times G} && G
	\arrow["{\id_{G} \times \cdot}"', from=1-1, to=3-1]
	\arrow["\cdot", from=1-3, to=3-3]
	\arrow["{\cdot \times \id_{G}}", from=1-1, to=1-3]
	\arrow["\cdot"', from=3-1, to=3-3]
\end{tikzcd}
\end{lem}
\begin{proof}
  It suffices to verify that $\id\inv(\iota\inv(U)) = \iota\inv(U)$ is open for all $U \in \cT$, but this is immediate since $\cT'$ makes the inclusion map $\iota$ continuous.
\end{proof}
Note that the identity function going the other direction $\id : (A, \cT') \to (A, \cS)$ is not necessarily continuous.

\section{Product Topology}
\label{sec:product-topology}

Any product in $\mathsf{Top}$ is necessarily a product in $\mathsf{Set}$.
Thus, one way to construct products in $\mathsf{Top}$ is to topologize the product of a given family of underlying sets.
Concretely, this construction can be described in terms of subbasis.

\begin{defn}
  Let $\{X_{i}\}_{i \in I}$ be an indexed family of topological spaces.
  The \emph{product topology} is given by the following subbasis
  \[
    S_{i} = \{\pi_{i}\inv(U_{i}) \mid U_{i}\,\text{is open in}\,X_{i}\}
  \]
  Note that this is the coarsest topology for which the projection maps are continuous.
\end{defn}

\section{Quotient Topology}
\label{sec:quotient-topology}

Given a topological space $X$ and an equivalence relation $\sim$ on $X$.
Once can construct the quotient set $X/\sim$ and there is a canonical quotient map $\pi : X \to X/\sim$.
The goal is to equip $X/\sim$ with a topology for which the quotient map $\pi$ is continuous.

Unlike \cref{sec:subspace-topology,sec:product-topology}, the set that we want to topologize is the codomain of $\pi$.
Thus, a trivial solution is to equip $X/\sim$ with the indiscrete topology.
Now, the ``most efficient'' solution is the finest topology for which $\pi$ is continuous.
This leads to the following definition: the quotient space of $X$ by $\sim$ is given by the topology where $U$ is open whenever $\pi\inv(U)$ is open.

This admits a simple description.
A subset $U \subseteq X/\sim$ is a collection of equivalence classes, and $\pi\inv(U)$ is just the union of the equivalence classes in $U$.
Thus, an open set of $X/\sim$ is a collection of equivalence classes whose union is open in $X$.

\section{Metric Topology}
\label{sec:metric-topology}

\begin{defn}
  A \emph{metric} on a set $X$ is a binary function
  \[
    d : X \times X \to \dR
  \]
  satisfying the following properties:
  \begin{enumerate}
  \item $d(x,y) \geq 0$ for all $x,y \in X$; equality holds if and only if $x = y$.
  \item $d(x,y) = d(y,x)$ for all $x,y \in X$.
  \item $d(x,y) + d(y,z) \geq d(x,z)$ for all $x,y,z \in X$.
  \end{enumerate}
\end{defn}

\begin{defn}
  Given a metric $d$ on $X$, the \emph{$\epsilon$-ball centered at $x$} is
  \[
    B_{d}(x,\epsilon) = \{ y \mid d(x,y) < \epsilon \}
  \]
\end{defn}

\begin{defn}
  Given a metric $d$ on $X$, the collection of all $\epsilon$-balls is a basis for a topology on $X$, called the \emph{metric topology} induced by $d$.
\end{defn}

\begin{defn}
  If $X$ is a topological space, $X$ is \emph{metrizable} if there is a metric $d$ that induces the topology of $X$.
  A \emph{metric space} is a metrizable topological space with a specified metric.
\end{defn}

The notion of continuous function coincides with the usual $\epsilon\delta$ definition.
\begin{thm}
  Let $f : X \to Y$; let $X$ and $Y$ be metrizable with metrics $d_{X}$ and $d_{Y}$, respectively.
  Then $f$ is continuous if and only if given $x \in X$ and $\epsilon > 0$, there is $\delta > 0$ such that
  \[
    d_{X}(x,y) < \delta \imp d_{Y}(f(x),f(y)) < \epsilon
  \]
\end{thm}
\begin{proof}
  Suppose that $f$ is continuous, then consider the open set $f\inv(B_{d_{Y}}(f(x),\epsilon))$, contained within it is an open ball $B_{d_{X}}(x,\delta)$.
  Then for any $y$ in this open ball, $f(y)$ is in the open ball $B_{d_{Y}}(f(x),\epsilon)$.

  Conversely, suppose that the $\epsilon\delta$ condition is satisfied.
  Let $V$ be open in $Y$ and $x \in f\inv(V)$.
  Then there is an open ball $B_{d_{Y}}(f(x),\epsilon)$ contained within $V$.
  The $\epsilon\delta$ condition implies that there is some $\delta$ so that $f(B_{d_{X}}(x,\delta)) \subseteq B_{d_{Y}}(f(x),\epsilon)$.
  Thus, $B_{d_{X}}(x,\delta)$ is a neighborhood of $x$ contained in $f\inv(V)$.
  This implies that $f\inv(V)$ is open.
\end{proof}

\end{document}
