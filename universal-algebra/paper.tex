\documentclass{amsart}
\input{decls}
\title{Algebraic Theory}
\author{Frank Tsai}
\date{\today}
%\thanks{}
\begin{document}
\maketitle
\tableofcontents

\newcommand{\interp}[1]{\llbracket #1 \rrbracket}

\section{Universal Algebra}
\label{sec:universal-algebra}

The theory of groups $\cG$ can be described by specifying
\begin{enumerate}
\item a denumerable set of variables $x, y, z, \ldots$;
\item three formal symbols $+, -, 0$.
\end{enumerate}

The terms of the theory of groups are defined inductively:
\begin{enumerate}
\item variables are terms;
\item $0$ is a term;
\item if $t$ is a term, then $-t$ is a term;
\item if $t_{1}$ and $t_{2}$ are terms, then $t_{1} + t_{2}$ is a term.
\end{enumerate}

The axioms of the theory of groups are the following:
\begin{enumerate}
\item $x + (y + z) = (x + y) + z$
\item $x + 0 = x$
\item $0 + x = x$
\item $x + (-x) = 0$
\item $(-x) + x = 0$
\end{enumerate}

\begin{defn}
  A \emph{premodel} of the theory of groups $\cG$ consists of
  \begin{enumerate}
  \item a set $G$;
  \item an element $\hat{0} \in G$;
  \item a unary function $\hat{-} : G \to G$;
  \item a binary function $\hat{+} : G \times G \to G$.
  \end{enumerate}
\end{defn}

Given a premodel $(G, \hat{0}, \hat{-}, \hat{+})$, terms are interpreted recursively in this premodel:
\begin{enumerate}
\item $\interp{x}$ is any element of $G$;
\item $\interp{0}$ is $\hat{0}$;
\item $\interp{-t}$ is $\hat{-}\interp{t}$;
\item $\interp{t_{1} + t_{2}}$ is $\interp{t_{1}} \hat{+} \interp{t_{2}}$.
\end{enumerate}

\begin{defn}
  A premodel $(G, \hat{0}, \hat{-}, \hat{+})$ constitutes a \emph{model} of $\cG$ if for every interpretation of variables, every axiom is satisfied, i.e.,
  \begin{enumerate}
  \item $\interp{x} \hat{+} (\interp{y} \hat{+} \interp{z}) = (\interp{x} \hat{+} \interp{y}) \hat{+} \interp{z}$
  \item $\interp{x} \hat{+} \hat{0} = \interp{x}$
  \item $\hat{0} \hat{+} \interp{x} = \interp{x}$
  \item $\interp{x} \hat{+} (\hat{-}\interp{x}) = \hat{0}$
  \item $(\hat{-}\interp{x}) \hat{+} \interp{x} = \hat{0}$
  \end{enumerate}
\end{defn}

We can repeat this process for rings, $R$-modules, monoids, etc.
In general, algebraic theories fit in the following formal framework:

\begin{defn}
  A \emph{presentation} $\dP$ of an algebraic theory $\cT$ is a theory with equality specified by
  \begin{enumerate}
  \item a denumerable set of variables: $x, y, z, \ldots$;
  \item for each $n \in \dN$, a set $\cO_{n}$ of $n$-ary function symbols;
  \item a set of axioms.
  \end{enumerate}
  Axioms are equalities between terms; terms are defined inductively as follows:
  \begin{enumerate}
  \item variables are terms;
  \item if $f$ is an $n$-ary function symbol and $t_{1},\ldots,t_{n}$ are terms, then $f(t_{1},\ldots,t_{n})$ is a term.
  \end{enumerate}
  A nullary function symbol is also called a \emph{constant}.
\end{defn}

\begin{defn}
  Let $\dP$ be a presentation of an algebraic theory.
  A \emph{premodel} of $\dP$ consists of
  \begin{enumerate}
  \item a set $M$;
  \item a function $\hat{f} : M^{n} \to M$ for each $n$-ary function symbol $f$.
  \end{enumerate}
\end{defn}

Given a premodel, terms are interpreted recursively:
\begin{enumerate}
\item $\interp{x}$ is any element of $M$;
\item $\interp{f(t_{1},\ldots,t_{n})} = \hat{f}(\interp{t_{1}},\ldots,\interp{t_{n}})$.
\end{enumerate}

\begin{defn}
  Let $\sL$ and $\sM$ be models of a presentation $\dP$.
  A $\dP$-homomorphism is a function $\varphi : L \to M$ so that
  \[
    \varphi(\hat{f}_{\sL}(a_{1},\ldots,a_{n})) = \hat{f}_{\sM}(\varphi(a_{1}),\ldots,\varphi(a_{n}))
  \]
  for any function symbol $f$ of $\dP$ and any elements $a_{1},\ldots,a_{n}$ of $L$.
\end{defn}

\begin{lem}
  Let $\dP$ be a presentation of an algebraic theory.
  The models of $\dP$ and their homomorphisms constitute a category.
\end{lem}

\section{Lawvere Theory}
\label{sec:lawvere-theory}

\begin{defn}
  An algebraic theory $\cT$ is a category with a denumerable set of objects $\{T^{0}, T^{1}, T^{2}, \ldots\}$.
  Each object $T^{n}$ is the $n$-th power of $T^{1}$.

  A model of $\cT$ is a finite-product-preserving functor $F : \cT \to \mathsf{Set}$, and a homomorphism of $\cT$-models is a natural transformation.
\end{defn}

$\cT$-models and their homomorphisms form a category, which we call $\mathsf{Mod}_{\cT}$.

\begin{lem}
  Let $\cT$ be an algebraic theory.
  If $\alpha : F \to G$ is a morphism in $\mathsf{Mod}_{\cT}$, then the diagram
  % https://q.uiver.app/#q=WzAsNCxbMCwwLCJHIFxcdGltZXMgRyBcXHRpbWVzIEciXSxbMiwwLCJHIFxcdGltZXMgRyJdLFswLDIsIkcgXFx0aW1lcyBHIl0sWzIsMiwiRyJdLFswLDIsIlxcaWRfe0d9IFxcdGltZXMgXFxjZG90IiwyXSxbMSwzLCJcXGNkb3QiXSxbMCwxLCJcXGNkb3QgXFx0aW1lcyBcXGlkX3tHfSJdLFsyLDMsIlxcY2RvdCIsMl1d
\begin{tikzcd}
	{G \times G \times G} && {G \times G} \\
	\\
	{G \times G} && G
	\arrow["{\id_{G} \times \cdot}"', from=1-1, to=3-1]
	\arrow["\cdot", from=1-3, to=3-3]
	\arrow["{\cdot \times \id_{G}}", from=1-1, to=1-3]
	\arrow["\cdot"', from=3-1, to=3-3]
\end{tikzcd}
  commutes and the isomorphisms are the canonical ones.
\end{lem}
\begin{proof}
  By naturality,
  \[
    G(\pi_{i}) \circ \alpha_{T^{n}} = \alpha_{T} \circ F(\pi_{i})
  \]
  for any projection $\pi_{i} : T^{n} \to T$.
  The isomorphisms are uniquely determined by the projections $F(\pi_{i})$ and $G(\pi_{i})$.
\end{proof}

A presentation $\dT$ of the algebraic theory $\cT$ consists of
\begin{enumerate}
\item for each $n \in \dN$, a variable $x^{n}_{i}$ for the projection $\pi_{i} \in \cT(T^{n},T^{1})$;
\item for each $n \in \dN$, a function symbol $f^{n}$ for each $f^{n} \in \cT(T^{n},T^{1})$.
\end{enumerate}
Terms are defined inductively as
\begin{enumerate}
\item variables are terms;
\item if $f^{n}$ is an $n$-ary function symbol and $t_{1},\ldots,t_{n}$ are terms, then $f^{n}(t_{1},\ldots,t_{n})$ is a term.
  This term corresponds to the composite $f^{n} \circ (\beta_{1},\ldots,\beta_{n})$, where $\beta_{1},\ldots,\beta_{n}$ correspond to $t_{1},\ldots,t_{n}$.
\end{enumerate}
In this presentation, a term $t$ corresponds to a morphism $T^{n} \to T^{1}$.

\section{Limits and Colimits in Algebraic Categories}
\label{sec:limits-and-colimits-in-algebraic-categories}


\section{Algebraic Functors}
\label{sec:algebraic-functors}

\section{Freely Generated Models}
\label{sec:freely-generated-models}

\section{Characterization of Algebraic Categories}
\label{sec:characterization-of-algebraic-categories}

\section{Tensor Products of Theories}
\label{sec:tensor-products-of-theories}



\bibliographystyle{alpha}
\bibliography{all}

\end{document}
