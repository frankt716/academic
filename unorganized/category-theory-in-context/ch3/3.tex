\documentclass{amsart}
\input{decls}
\title{}
\author{Frank Tsai}
\date{\today}
%\thanks{}
\begin{document}
\maketitle
\tableofcontents

\section{Introduction}
\label{sec:introduction}

The following theorem shows that limits and colimits in a functor category are constructed objectwise from the base category.
\begin{thm}\label{thm:objectwise}
  If $\iA$ is small, then the forgetful functor $\iC^{\iA} \to \iC^{\ob \iA}$ strictly creates limits and colimits that exist in $\iC$.
  These limits are defined objectwise, meaning that for each $a \in \iA$, the evaluation functor $\ev_{a} : \iC^{A} \to \iC$ preserves all limits and colimits in $\iC$.
\end{thm}
The forgetful functor $\Pi$ forgets the action on morphisms of a given functor $F \in \iC^{\iA}$.

We show the case for limits.
The case for colimits follows by duality.
The proof idea is that
\begin{enumerate}
\item first, we show that the functor category $\iC^{\ob \iA}$ has all limits and colimits that $\iC$ does, and the evaluation functor preserves them;
\item\label{stp:2} then, we show that the functor category $\iC^{\iA}$ also has all limits and colimits that $\iC$ does, defined componentwise.
\item finally, when there is a limit or colimit in $\iC^{\ob \iA}$, we can project it into limits or colimits in $\iC$, then use \ref{stp:2} to construct the desired limit or colimit.
\end{enumerate}

\begin{lem}\label{lem:step-1}
  Let $\iA$ be a small category.
  The functor category $\iC^{\ob \iA}$ has all limits and colimits that $\iC$ does.
  Moreover, these are preserved by the evaluation functor $\ev_{a} : \iC^{\ob \iA} \to \iC$.
\end{lem}
\begin{proof}
  The functor category $\iC^{\ob \iA}$ is isomorphic to the product category $\prod_{\ob \iA}\iC$.
  Thus, by the universal property of product, a diagram $D$ of shape $\iJ$ in $\iC^{\ob \iA}$ is an $\ob \iA$-indexed family of diagrams $D_{a} : \iJ \to \iC$.
  % https://q.uiver.app/#q=WzAsNSxbMCwzLCJcXGlDIl0sWzIsMywiXFxjZG90cyJdLFs0LDMsIlxcaUMiXSxbMiwwLCJKIl0sWzIsMiwiXFxpQyBcXHRpbWVzIFxcY2RvdHMgXFx0aW1lcyBcXGlDIl0sWzMsNF0sWzMsMCwiIiwyLHsiY3VydmUiOjN9XSxbMywyLCIiLDIseyJjdXJ2ZSI6LTN9XSxbNCwwXSxbNCwyXSxbNCwxXSxbMywxLCIiLDAseyJjdXJ2ZSI6LTJ9XV0=
\[\begin{tikzcd}
	&& \iJ \\
	\\
	&& {\iC \times \cdots \times \iC} \\
	\iC && \cdots && \iC
	\arrow[from=1-3, to=3-3]
	\arrow[curve={height=18pt}, from=1-3, to=4-1]
	\arrow[curve={height=-18pt}, from=1-3, to=4-5]
	\arrow[from=3-3, to=4-1]
	\arrow[from=3-3, to=4-5]
	\arrow[from=3-3, to=4-3]
	\arrow[curve={height=-12pt}, from=1-3, to=4-3]
\end{tikzcd}\]

  A limit $\ell_{a_{i}}$ of each of these component diagrams assembles into a limit $\langle \ell_{a_{1}},\ldots, \ell_{a_{n}} \rangle$ of the diagram $\iJ \to \iC^{\ob \iA}$.
  This construction shows that $\iC^{\ob \iA}$ has all limits or colimits that $\iC$ does.
  Moreover, these are clearly preserved by the evaluation functor $\ev_{a} : \iC^{\ob \iA} \to \iC$.
\end{proof}

\begin{lem}\label{lem:step-2}
  Let $\iA$ be a small category.
  The functor category $\iC^{\iA}$ has all limits and colimits that $\iC$ does, constructed objectwise.
  Namely, whenever the diagram
  \[
    \iJ \overset{F}{\to} \iC^{\iA} \overset{\ev_{a}}{\to} \iC
  \]
  has a limit for all $a \in \iA$, then these limits extend to a functor $\lim F \in \iC^{\iA}$, the limit of $F$.
\end{lem}
\begin{proof}
  To define a functor $\lim F : \iA \to \iC$, we specify its action on objects and its action on morphisms.
  The hypothesis gives an action on objects: each $a \in \iA$ is mapped to the limit of the diagram $\ev_{a} \circ F : \iJ \to \iC$.

  Write $D_{a}$ for the diagram $\ev_{a} \circ F : \iC^{\iA} \to \iC$.
  A morphism, $f : a \to b$ in $\iA$ extends to a natural transformation $D_{a} \to D_{b}$ with components given by $F(j)(f)$ for all $j \in \iJ$.
  This yields a cone over $D_{b}$ with submit $\lim D_{a}$.
  The universal property of limit then yields a map $\lim D_{a} \to \lim D_{b}$, determining the action on morphisms, and guaranteeing functoriality.

  It remains to verify that $\lim F$ is a limit in $\iC^{\iA}$.
  A leg of the ``limit'' cone of $\lim F$ is a natural transformation $\lim F \to F(j)$.
  The component $\lim F(a) \to F(j)(a)$ is precisely the projection map $\lim\ev_{a} \circ F \to F(j)(a)$.

  Let $\Gtil \to F$ be a cone over $F$.
  Each leg consists of a natural transformation $G \to F(j)$.
  For each $a \in \iA$, the cone $\lim F \to F$ restricts to a limit cone $\lim\ev_{a} \circ F \to \ev_{a} \circ F$ in $\iC$, and the cone $\Gtil \to F$ restricts to a cone $G(a) \to \ev_{a} \circ F$.
  % https://q.uiver.app/#q=WzAsOSxbMSwwLCJHIl0sWzEsMSwiXFxsaW0gRiJdLFswLDIsIkYoaSkiXSxbMiwyLCJGKGopIl0sWzMsMSwiXFxyaWdodHNxdWlnYXJyb3ciXSxbNSwwLCJHKGEpIl0sWzUsMSwiXFxsaW0gRihhKSJdLFs0LDIsIkYoaSkoYSkiXSxbNiwyLCJGKGopKGEpIl0sWzEsMiwiXFx2YXJwaV97aX0iLDAseyJsZXZlbCI6Mn1dLFsxLDMsIlxcdmFycGlfe2p9IiwyLHsibGV2ZWwiOjJ9XSxbMCwyLCJcXHpldGFfe2l9IiwyLHsiY3VydmUiOjIsImxldmVsIjoyfV0sWzAsMywiXFx6ZXRhX3tqfSIsMCx7ImN1cnZlIjotMiwibGV2ZWwiOjJ9XSxbNiw3LCJcXHBpX3tpfSJdLFs2LDgsIlxccGlfe2p9IiwyXSxbNSw3LCJ6X3tpfSIsMix7ImN1cnZlIjoyfV0sWzUsOCwiel97an0iLDAseyJjdXJ2ZSI6LTJ9XSxbNSw2LCIiLDEseyJzdHlsZSI6eyJib2R5Ijp7Im5hbWUiOiJkYXNoZWQifX19XV0=
\[\begin{tikzcd}
	& G &&&& {G(a)} \\
	& {\lim F} && \rightsquigarrow && {\lim F(a)} \\
	{F(i)} && {F(j)} && {F(i)(a)} && {F(j)(a)}
	\arrow["{\varpi_{i}}", Rightarrow, from=2-2, to=3-1]
	\arrow["{\varpi_{j}}"', Rightarrow, from=2-2, to=3-3]
	\arrow["{\zeta_{i}}"', curve={height=12pt}, Rightarrow, from=1-2, to=3-1]
	\arrow["{\zeta_{j}}", curve={height=-12pt}, Rightarrow, from=1-2, to=3-3]
	\arrow["{\pi_{i}}", from=2-6, to=3-5]
	\arrow["{\pi_{j}}"', from=2-6, to=3-7]
	\arrow["{z_{i}}"', curve={height=12pt}, from=1-6, to=3-5]
	\arrow["{z_{j}}", curve={height=-12pt}, from=1-6, to=3-7]
	\arrow[dashed, from=1-6, to=2-6]
\end{tikzcd}\]
  
  The universal property then yields a unique morphism $G(a) \to \lim F(a)$, which assembles into a unique natural transformation $G \to \lim F$.
\end{proof}

We are now ready to prove the main theorem.
\begin{proof}[Proof of Theorem \ref{thm:objectwise}]
  Whenever the diagram
  \[
    \iJ \overset{F}{\to} \iC^{\iA} \overset{\Pi}{\to} \iC^{\ob \iA}
  \]
  has a limit, we can project them to limits in $\iC$ by Lemma \ref{lem:step-1}.
  Then by \ref{lem:step-2}, these limits extend to a limit in $\iC^{\iA}$.
  Moreover, $\Pi$ reflects these limits since the construction in Lemma \ref{lem:step-2} computes the action on morphisms, i.e., it supplies the data that $\Pi$ forgets.
\end{proof}

\bibliographystyle{alpha}
\bibliography{all}

\end{document}
