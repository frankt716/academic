\documentclass{amsart}
\input{decls}
\title{Size Matters}
\author{Frank Tsai}
\date{\today}
%\thanks{}
\begin{document}
\maketitle
\tableofcontents

\section{Introduction}
\label{sec:introduction}

This section explains why the size restriction in the notion of completeness is necessary.
First, we record a fact.

\begin{lem}
  For any category $\iC$, the identity functor $\id_{\iC}$ admits a limit if and only if $\iC$ has an initial object.
\end{lem}

\begin{defn}
  The \emph{cardinality} of a small category $\iC$ is the cardinality of the set of its morphisms.
  A category whose cardinality is less than $\kappa$ is called $\kappa$-small.
\end{defn}

Freyd's lemma explains why categories don't admit limits or colimits of unrestricted size.

\begin{lem}
  Any $\kappa$-small category that admits all $\kappa$-small limits or all $\kappa$-small colimits is a preorder.
\end{lem}

\bibliographystyle{alpha}
\bibliography{all}

\end{document}
