\documentclass{amsart}
\input{decls}
\title{Existence of Adjoint Functors}
\author{Frank Tsai}
\date{\today}
%\thanks{}
\begin{document}
\maketitle
\tableofcontents

\section{Introduction}
\label{sec:introduction}
Consider the inclusion functor $\mathsf{Ring} \into \mathsf{Rng}$.
The left adjoint ought to be the ``most efficient way'' to attach a multiplicative unit to a rng $R$.
In other words, for any possible way $f$ to adjoin a multiplicative unit to $R$, we want $\eta$ to uniquely factor into this construction.
% https://q.uiver.app/#q=WzAsMyxbMCwyLCJSIl0sWzIsMCwiUyJdLFswLDAsIlJeeyp9Il0sWzAsMiwiXFxldGEiXSxbMCwxLCJmIiwyXSxbMiwxLCIiLDAseyJzdHlsZSI6eyJib2R5Ijp7Im5hbWUiOiJkYXNoZWQifX19XV0=
\[\begin{tikzcd}
	{R^{*}} && S \\
	\\
	R
	\arrow["\eta", from=3-1, to=1-1]
	\arrow["f"', from=3-1, to=1-3]
	\arrow[dashed, from=1-1, to=1-3]
\end{tikzcd}\]
Namely, we want
\[
  \mathsf{Ring}(R^{*}, S) \iso \mathsf{Rng}(R, S)
\]
Note that the inclusion functor is often omitted.
This amounts to finding a representation $R^{*}$ for the functor $\mathsf{Rng}(R,\blank)$.
A representation for this functor defines an initial object in $\int\mathsf{Rng}(R,\blank)$.
This category is isomorphic to the comma category $R \dn \mathsf{Ring}$.
If we can find a unital ring $R^{*}$ and a ring homomorphism $\eta : R \to R^{*}$ that is initial in $R \dn \mathsf{Ring}$, then the optimization problem is solved.

\begin{lem}\label{lem:left-adjoint-initial}
  A functor $U : \iA \to \iS$ admits a left adjoint if and only if $s \dn U$ has an initial object for all $s \in \iS$.
\end{lem}
\begin{proof}
  The ``if'' direction: suppose that $U : \iA \to \iS$ admits a left adjoint $F : \iS \to \iA$.
  Then for each $s \in \iS$, the component $\eta_{s} : s \to UFs$ is an initial object in $s \dn U$.
  This is immediate by the following defining diagram:
  % https://q.uiver.app/#q=WzAsMyxbMCwyLCJzIl0sWzAsMCwiVUZzIl0sWzIsMCwiVWEiXSxbMCwxLCJcXGV0YV97c30iXSxbMSwyLCIiLDIseyJzdHlsZSI6eyJib2R5Ijp7Im5hbWUiOiJkYXNoZWQifX19XSxbMCwyLCJmIiwyXV0=
\[\begin{tikzcd}
	UFs && Ua \\
	\\
	s
	\arrow["{\eta_{s}}", from=3-1, to=1-1]
	\arrow[dashed, from=1-1, to=1-3]
	\arrow["f"', from=3-1, to=1-3]
\end{tikzcd}\]

  Conversely, suppose that $s \dn U$ has an initial object, which is suggestively denoted as $(Fs \in \iA, \eta_{s} : s \to UFs)$ for all $s \in \iS$.
  This define a function $F : \ob\iS \to \ob\iA$, which can be extended to a functor.
  For $f : s \to s'$, we define $Ff$ to be the unique morphism $Fs \to Fs'$ given by the initiality of $(Fs \in \iA, \eta_{s} : s \to UFs)$.
  This data assembles into a natural transformation $\eta : \id_{\iS} \to UF$.
  We then define a natural transformation $\varphi : \iA(F\blank,\blank) \to \iS(\blank,U\blank)$ by
  % https://q.uiver.app/#q=WzAsNCxbMiwwLCJUXnsyfVxcZWxsIl0sWzIsMiwiXFxlbGwiXSxbMCw0LCJEal97MX0iXSxbNCw0LCJEal97Mn0iXSxbMCwxLCJcXGFscGhhIFxcY2lyYyBUXFxhbHBoYSIsMV0sWzIsM10sWzEsMiwiXFxsYW1iZGFfe2pfezF9fSIsMl0sWzEsMywiXFxsYW1iZGFfe2pfezJ9fSJdLFswLDIsIlxcbGFtYmRhX3tqX3sxfX0gXFxjaXJjIFxcYWxwaGEgXFxjaXJjIFxcbXVfe1xcZWxsfSIsMix7ImN1cnZlIjoyfV0sWzAsMywiXFxsYW1iZGFfe2pfezJ9fSBcXGNpcmMgXFxhbHBoYSBcXGNpcmMgXFxtdV97XFxlbGx9IiwwLHsiY3VydmUiOi0yfV1d
\[\begin{tikzcd}
	&& {T^{2}\ell} \\
	\\
	&& \ell \\
	\\
	{Dj_{1}} &&&& {Dj_{2}}
	\arrow["{\alpha \circ T\alpha}"{description}, from=1-3, to=3-3]
	\arrow[from=5-1, to=5-5]
	\arrow["{\lambda_{j_{1}}}"', from=3-3, to=5-1]
	\arrow["{\lambda_{j_{2}}}", from=3-3, to=5-5]
	\arrow["{\lambda_{j_{1}} \circ \alpha \circ \mu_{\ell}}"', curve={height=12pt}, from=1-3, to=5-1]
	\arrow["{\lambda_{j_{2}} \circ \alpha \circ \mu_{\ell}}", curve={height=-12pt}, from=1-3, to=5-5]
\end{tikzcd}\]
  Initiality provides existence and uniqueness of maps $UFs \to Ua$.
  Existence proves surjectivity, while uniqueness proves injectivity.
  Thus $\varphi_{s,a}(g)$ is a bijection, proving that
  \[
    \iA(Fs,a) \iso \iS(s,Ua)
  \]
\end{proof}

Recall that the canonical forgetful functor $\Pi : s \dn U \to \iA$
\begin{enumerate}
\item strictly creates all limits that $\iA$ admits and that $\iS(s, U\blank)$ preserves;
\item strictly creates all connected colimits that $\iA$ admits.
\end{enumerate}
Thus,
\begin{lem}\label{lem:forgetful-create}
  For any functor $U : \iA \to \iS$ and object $s \in \iS$, the associated forgetful functor $\Pi : s \dn U \to \iA$ strictly creates the limit of any diagram whose limit exists in $\iA$ and is preserved by $U$.
  In particular, if $\iA$ is complete and $U$ is continuous, then $s \dn U$ is complete.
\end{lem}
\begin{proof}
  It suffices to show that those limits that $U$ preserves are precisely those that $\iS(s,U\blank)$ preserves.
  Since $U(\lim G) \iso \lim UG$, by the Yoneda Lemma,
  \[
    \iS(s, U(\lim G)) \iso \iS(s, \lim UG)
  \]
  Since represented functors preserve limits, we have
  \[
    \iS(s, \lim UG) \iso \lim \iS(s, UG)
  \]
  Now suppose that $\iA$ is complete and $U$ is continuous, then $U$ preserves any small limit that exists in $\iA$.
  Then $\Pi : s \dn U \to \iA$ strictly creates any limit in $\iA$, proving that $s \dn U$ is complete.
\end{proof}

Recall that an initial object in $s \dn U$ is the limit of the identity functor $\id_{s \dn U}$.
By \cref{lem:forgetful-create}, $s \dn U$ has an initial object if and only if $\Pi : s \dn U \to \iA$ has a limit.
However, even if $\iA$ is complete and $U$ is continuous, $\Pi : s \dn U \to \iA$ does not necessarily have a limit since $s \dn U$ is in general large.

The general adjoint functor theorem identifies a condition for which this large limit can be reduced to a small one.

\begin{lem}\label{lem:joint-initial}
  If $\iC$ is complete, locally small, and has a jointly weakly initial set of objects $\Phi$, then $\iC$ has an initial object.
\end{lem}
\begin{proof}
  Consider the inclusion functor from the full subcategory spanned by objects of $\Phi$ into $\iC$.
  Since $\Phi$ is a set and $\iC$ is locally small, the subcategory spanned by $\Phi$ has a set worth of morphisms, i.e., it is small.
  Thus, the inclusion functor defines a small diagram.
  And since $\iC$ is complete, this diagram has a limit in $\iC$ with limit cone $(\kappa_{k} : \ell \to k)_{k \in \Phi}$.
  
  Now for any object $c \in \iC$,
  \begin{enumerate}
  \item if $c \in \Phi$, then choose $\kappa_{c} : \ell \to c$;
  \item otherwise, there is some $k \in \Phi$ and a morphism $f : k \to c$.
    Then choose $f \circ \kappa_{k} : \ell \to c$.
  \end{enumerate}
  This proves that $\ell$ is weakly initial.
  To show that $\ell$ is initial, it suffices to show that $\ell$ defines a limit cone $\lambda : \ell \to \id_{\iC}$ of the identity functor $\id_{\iC}$.
  First, for any $f : c \to c'$ in $\iC$, one can pull back $h_{c'}$ along $f \circ h_{c}$.
  This yields an object $p \in \iC$.
  Then there is some $k'' \in \Phi$ and a morphism $k'' \to p$.
  Composing this morphism with the legs of the pullback yields two morphisms $k'' \to k$ and $k'' \to k'$ living in the full subcategory spanned by $\Phi$.
  Since $\ell$ defines a cone over the inclusion functor.
  The top triangles commute.
  % https://q.uiver.app/#q=WzAsMTIsWzAsMCwiRGpfezF9Il0sWzQsMCwiRGpfezJ9Il0sWzIsMSwiXFxlbGwiXSxbMiwzLCJUXFxlbGwiXSxbMCwyLCJURGpfezF9Il0sWzQsMiwiVERqX3syfSJdLFsyLDUsIlReezJ9XFxlbGwiXSxbMCw0LCJUXnsyfURqX3sxfSJdLFs0LDQsIlReezJ9RGpfezJ9Il0sWzIsNywiXFxlbGwiXSxbMCw2LCJEal97MX0iXSxbNCw2LCJEal97Mn0iXSxbMCwxXSxbMCwyLCJcXGxhbWJkYV97al97MX19IiwyXSxbMSwyLCJcXGxhbWJkYV97al97Mn19Il0sWzQsMywiVFxcbGFtYmRhX3tqX3sxfX0iLDJdLFs1LDMsIlRcXGxhbWJkYV97al97Mn19Il0sWzQsNV0sWzcsOF0sWzcsNiwiVF57Mn1cXGxhbWJkYV97al97MX19IiwyXSxbOCw2LCJUXnsyfVxcbGFtYmRhX3tqX3syfX0iXSxbMCw0LCJcXGV0YV97RGpfezF9fSIsMl0sWzEsNSwiXFxldGFfe0RqX3syfX0iXSxbNCw3LCJUXFxldGFfe0RqX3sxfX0iLDJdLFs1LDgsIlRcXGV0YV97RGpfezJ9fSJdLFs3LDEwLCJcXGdhbW1hX3tEal97MX19IFxcY2lyYyBcXG11X3tEal97MX19IiwyXSxbOCwxMSwiXFxnYW1tYV97RGpfezJ9fSBcXGNpcmMgXFxtdV97RGpfezJ9fSJdLFsxMCw5LCJcXGxhbWJkYV97al97MX19IiwyXSxbMTEsOSwiXFxsYW1iZGFfe2pfezJ9fSJdLFsyLDMsIiIsMSx7InN0eWxlIjp7ImJvZHkiOnsibmFtZSI6ImRhc2hlZCJ9fX1dLFszLDYsIlxceGkiLDAseyJsYWJlbF9wb3NpdGlvbiI6MzAsInN0eWxlIjp7ImJvZHkiOnsibmFtZSI6ImRhc2hlZCJ9fX1dLFs2LDksIlxcemV0YSIsMCx7InN0eWxlIjp7ImJvZHkiOnsibmFtZSI6ImRhc2hlZCJ9fX1dXQ==
\[\begin{tikzcd}
	{Dj_{1}} &&&& {Dj_{2}} \\
	&& \ell \\
	{TDj_{1}} &&&& {TDj_{2}} \\
	&& T\ell \\
	{T^{2}Dj_{1}} &&&& {T^{2}Dj_{2}} \\
	&& {T^{2}\ell} \\
	{Dj_{1}} &&&& {Dj_{2}} \\
	&& \ell
	\arrow[from=1-1, to=1-5]
	\arrow["{\lambda_{j_{1}}}"', from=1-1, to=2-3]
	\arrow["{\lambda_{j_{2}}}", from=1-5, to=2-3]
	\arrow["{T\lambda_{j_{1}}}"', from=3-1, to=4-3]
	\arrow["{T\lambda_{j_{2}}}", from=3-5, to=4-3]
	\arrow[from=3-1, to=3-5]
	\arrow[from=5-1, to=5-5]
	\arrow["{T^{2}\lambda_{j_{1}}}"', from=5-1, to=6-3]
	\arrow["{T^{2}\lambda_{j_{2}}}", from=5-5, to=6-3]
	\arrow["{\eta_{Dj_{1}}}"', from=1-1, to=3-1]
	\arrow["{\eta_{Dj_{2}}}", from=1-5, to=3-5]
	\arrow["{T\eta_{Dj_{1}}}"', from=3-1, to=5-1]
	\arrow["{T\eta_{Dj_{2}}}", from=3-5, to=5-5]
	\arrow["{\gamma_{Dj_{1}} \circ \mu_{Dj_{1}}}"', from=5-1, to=7-1]
	\arrow["{\gamma_{Dj_{2}} \circ \mu_{Dj_{2}}}", from=5-5, to=7-5]
	\arrow["{\lambda_{j_{1}}}"', from=7-1, to=8-3]
	\arrow["{\lambda_{j_{2}}}", from=7-5, to=8-3]
	\arrow[dashed, from=2-3, to=4-3]
	\arrow["\xi"{pos=0.3}, dashed, from=4-3, to=6-3]
	\arrow["\zeta", dashed, from=6-3, to=8-3]
\end{tikzcd}\]
  This proves that $\ell$ defines a cone over $\id_{\iC}$.
  To prove that it is the limit cone, it suffices to show that $\lambda_{\ell}$ is the identity since, in this case, it guarantees a unique morphism into $\ell$.
  % https://q.uiver.app/#q=WzAsNCxbMCwzLCJcXGVsbCJdLFsyLDMsImMiXSxbMSwyLCJcXGVsbCJdLFsxLDAsIngiXSxbMCwxLCJmIiwyXSxbMiwwLCJcXGlkIiwyXSxbMiwxXSxbMywwLCJcXG11X3tcXGVsbH0iLDIseyJjdXJ2ZSI6Mn1dLFszLDEsIiIsMCx7ImN1cnZlIjotMn1dLFszLDIsIlxcbXVfe1xcZWxsfSIsMCx7InN0eWxlIjp7ImJvZHkiOnsibmFtZSI6ImRhc2hlZCJ9fX1dXQ==
\[\begin{tikzcd}
	& x \\
	\\
	& \ell \\
	\ell && c
	\arrow["f"', from=4-1, to=4-3]
	\arrow["\id"', from=3-2, to=4-1]
	\arrow[from=3-2, to=4-3]
	\arrow["{\mu_{\ell}}"', curve={height=12pt}, from=1-2, to=4-1]
	\arrow[curve={height=-12pt}, from=1-2, to=4-3]
	\arrow["{\mu_{\ell}}", dashed, from=1-2, to=3-2]
\end{tikzcd}\]
  The cone condition demands that the triangle commutes:
  % https://q.uiver.app/#q=WzAsMyxbMSwwLCJcXGVsbCJdLFswLDIsIlxcZWxsIl0sWzIsMiwiayJdLFsxLDIsIlxca2FwcGFfe2t9IiwyXSxbMCwyLCJcXGxhbWJkYV97a30gPSBcXGthcHBhX3trfSJdLFswLDEsIlxcbGFtYmRhX3tcXGVsbH0iLDJdXQ==
\[\begin{tikzcd}
	& \ell \\
	\\
	\ell && k
	\arrow["{\kappa_{k}}"', from=3-1, to=3-3]
	\arrow["{\lambda_{k} = \kappa_{k}}", from=1-2, to=3-3]
	\arrow["{\lambda_{\ell}}"', from=1-2, to=3-1]
\end{tikzcd}\]
  Thus, $\lambda_{\ell}$ must be $\id_{\ell}$.
\end{proof}

Here's a list of what we know so far
\begin{enumerate}
\item \cref{lem:left-adjoint-initial}: a functor $U : \iA \to \iS$ admits a left adjoint if and only if $s \dn U$ has an initial object for all $s \in \iS$.
\item \cref{lem:forgetful-create}: for any functor $U : \iA \to \iS$, if $\iA$ is complete and $U$ is continuous, then $s \dn U$ is complete for all $s \in \iS$.
\item \cref{lem:joint-initial}: if $s \dn U$ is complete, locally small, and has a jointly weakly initial set of objects, then $s \dn U$ has an initial object.
\end{enumerate}

\begin{thm}[General Adjoint Functor Theorem]
  Let $U : \iA \to \iS$ be a continuous functor whose domain is locally small and complete.
  Suppose that $U$ satisfies the following \emph{solution set condition}:
  \begin{itemize}
  \item for every $s \in \iS$ there exists a set of morphisms $\Phi_{s} = \{f_{i} : s \to Ua_{i}\}$ so that any $f : s \to Ua$ factors through some $f_{i} \in \Phi_{s}$ along a morphism $a_{i} \to a$ in $\iA$.
  \end{itemize}
  Then $U$ admits a left adjoint.
\end{thm}
\begin{proof}
  Since $\iA$ is complete and $U$ is continuous, by \cref{lem:forgetful-create}, $s \dn U$ is complete for all $s \in \iS$.
  Because $\iA$ is locally small, $s \dn U$ is locally small as well.
  The solution set condition says that for each $s \in \iS$, $s \dn U$ has a jointly weakly initial set of objects.
  Thus, by \cref{lem:joint-initial}, $s \dn U$ has an initial object.
  \cref{lem:left-adjoint-initial} concludes the proof.
\end{proof}

\begin{defn}
  A \emph{generating set} for a category $\iC$ is a set $\Phi$ of objects that can distinguish between distinct parallel morphisms.
  Given $f,g : x \toto y$, if $f \neq g$ then there is some $h : c \to x$ with $c \in \Phi$ so that $fh \neq gh$.
  A \emph{cogenerating set} in $\iC$ is a generating set in $\iC\op$.
\end{defn}

\begin{defn}
  A \emph{subobject} of an object $c \in \iC$ is a monomorphism $c' \mono c$.
  Isomorphic subobjects are typically identified.
\end{defn}

\begin{defn}
  The \emph{intersection} of a family of subobjects of $c$ is the limit of the diagram of monomorphisms with codomain $c$.
\end{defn}
In the poset of sets ordered by inclusion, the intersection of a family of sets is the largest set contained in each set of the family.

\begin{lem}\label{lem:cogen-initial}
  Suppose $\iC$ is locally small, complete, has a small cogenerating set $\Phi$, and has the property that every collection of subobjects has an intersection.
  Then $\iC$ has an initial object.
\end{lem}
\begin{proof}
  Let $p = \Pi_{k \in \Phi}k$ be the product of objects in the cogenerating set $\Phi$.
  Let $i$ be the intersection of the subobjects of $p$.

  Since $\iC$ is locally small and complete, the power $k^{\iC(c,k)}$ exists for each $c \in \iC$.
  And since there's only a set worth of cogenerating objects, the product
  \[
    \prod_{k\in\Phi}k^{\iC(c,k)}
  \]
  exists.
  For each $c \in \iC$, the canonical map from $c$ to this product corresponds to a family $\{c \to k^{\iC(c,k)}\}_{k\in\Phi}$ of morphisms, each of which corresponds to a set of morphisms $c \mono k$.
  Thus, the canonical map is a monomorphism
  \[
    c \mono \prod_{k\in\Phi}k^{\iC(c,k)}
  \]
  For each $k \in \Phi$, there a morphism
  \[
    k \to k^{\iC(c,k)}
  \]
  which picks out the identity map $\id_{k}$ for each component of the power.
  The product over $\Phi$ yields
  \[
    \Pi_{k\in\Phi}k \to \Pi_{k\in\Phi}k^{\iC(c,k)}
  \]
  Then the pullback
  % https://q.uiver.app/#q=WzAsNCxbMiwxLCIoQSxcXGFscGhhKSJdLFsyLDAsIihUKFxcY29wcm9kX3tJfUFfe2l9KSxcXG11X3tcXGNvcHJvZF97SX1BX3tpfX0pIl0sWzEsMSwiKFReezJ9KFxcY29wcm9kX3tJfUFfe2l9KSxcXG11X3tUKFxcY29wcm9kX3tJfUFfe2l9KX0pIl0sWzAsMCwiKFQoXFxjb3Byb2Rfe0l9VEFfe2l9KSxcXG11X3tcXGNvcHJvZF97SX1UQV97aX19KSJdLFszLDEsIlQoXFxjb3Byb2Rfe0l9XFxhbHBoYV97aX0pIl0sWzMsMiwiVFxca2FwcGEiLDJdLFsyLDEsIlxcbXVfe1xcY29wcm9kX3tJfUFfe2l9fSIsMl0sWzEsMCwiIiwyLHsic3R5bGUiOnsiaGVhZCI6eyJuYW1lIjoiZXBpIn19fV1d
\begin{tikzcd}
	{(T(\coprod_{I}TA_{i}),\mu_{\coprod_{I}TA_{i}})} && {(T(\coprod_{I}A_{i}),\mu_{\coprod_{I}A_{i}})} \\
	& {(T^{2}(\coprod_{I}A_{i}),\mu_{T(\coprod_{I}A_{i})})} & {(A,\alpha)}
	\arrow["{T(\coprod_{I}\alpha_{i})}", from=1-1, to=1-3]
	\arrow["T\kappa"', from=1-1, to=2-2]
	\arrow["{\mu_{\coprod_{I}A_{i}}}"', from=2-2, to=1-3]
	\arrow[two heads, from=1-3, to=2-3]
\end{tikzcd}
  yields a subobject $p_{c} \mono \Pi_{k\in\Phi}k$ and a map that can be composed $i \to p_{c} \to c$.
  This proves that $i$ is weakly initial.
  For any other morphism $i \to c$, the equalizer defines a subobject $i' \mono p$ with $i' \mono i$, since $i$ is the intersection of all subobjects of $p$, the equalizer must be the identity, proving that $i$ is initial.
\end{proof}

\begin{thm}[Special Adjoint Functor Theorem]
  Let $U : \iA \to \iS$ be a continuous functor whose domain is complete and whose domain and codomain are locally small.
  If $\iA$ has a small cogenerating set and every collection of subobjects of a fixed object in $\iA$ admits an intersection, then $U$ admits a left adjoint.
\end{thm}
\begin{proof}
  Since monomorphisms are preserved, reflected, and strictly created by $\Pi : s \dn U \to \iA$, $s \dn U$ has intersections.
  Let $\Phi$ be a cogenerating set of $\iA$.
  Then
  \[
    \Phi' = \{s \to Ua \mid a \in \Phi\}
  \]
  is cogenerating for $s \dn U$.
  Since $\iS$ is locally small, $\Phi'$ is a set.
  Thus, $s \dn U$ has an initial object, proving that $U$ admits a left adjoint.
\end{proof}

\begin{cor}
  Suppose $\iC$ is locally small, complete, has small cogenerating set, and has the property that every collection of subobjects of a fixed object has an intersection.
  Then $\iC$ is also cocomplete.
\end{cor}
\begin{proof}
  For any small indexing category $\iJ$, the diagonal functor $\Delta : \iC \to \iC^{\iJ}$ is continuous.
  Since $\iC$ is locally small and $\iJ$ is small, $\iC^{\iJ}$ is locally small.
  Then by the special adjoint functor theorem, $\Delta$ admits a left adjoint.
  Thus, $\iC$ has all small colimits.
\end{proof}

\begin{cor}
  Suppose $\iC$ is locally small, complete, has small cogenerating set, and has the property that every collection of subobjects of a fixed object has an intersection.
  Then any continuous functor $F : \iC \to \mathsf{Set}$ is representable.
\end{cor}
\begin{proof}
  By the special adjoint functor theorem, $F$ admits a left adjoint $L : \mathsf{Set} \to \iC$.
  Then
  \[
    \iC(L(\star),\blank) \iso \mathsf{Set}(\star,F\blank) \iso F
  \]
\end{proof}

\begin{thm}[Freyd's Representability Theorem]
  Let $F : \iC \to \mathsf{Set}$ be a continuous functor and suppose that $\iC$ is complete and locally small.
  If $F$ satisfies the solution set condition:
  \begin{quote}
    there exists a set $\Phi$ of objects of $\iC$ so that for any $c \in \iC$ and any element $x \in Fc$, there exists an $s \in \Phi$, an element $y \in Fs$, and a morphism $f : s \to c$ so that $Ff(y) = x$
  \end{quote}
  then $F$ is representable.
\end{thm}
\begin{proof}
  The solution set condition defines a weakly initial set in $\star \dn F \iso \int F$.
  This category is complete, and so it has an initial object, which defines a representation for $F$.
\end{proof}

\begin{defn}
  Let $\kappa$ be a regular cardinal.
  A locally small category $\iC$ is \emph{locally $\kappa$-representable} if it is cocomplete and if it has a set of objects $S$ so that
  \begin{enumerate}
  \item every object in $\iC$ can be written as a colimit of a diagram valued in the subcategory spanned by the objects in $S$.
  \item for each object $s \in \iS$, the functor $\iC(s, \blank) : \iC \to \mathsf{Set}$ preserves $\kappa$-filtered colimits.
  \end{enumerate}
\end{defn}

\begin{defn}
  A functor between $\kappa$-presentable categories is \emph{accessible} if it preserves $\kappa$-filtered colimits.
\end{defn}

\begin{thm}
  A functor $F : \iC \to \iD$ between locally presentable categories
  \begin{enumerate}
  \item admits a right adjoint if and only if it is cocontinuous;
  \item admits a left adjoint if and only if it is continuous and accessible.
  \end{enumerate}
\end{thm}

\begin{lem}
  Suppose $\iC$ is a locally small category with coproducts.
  A functor $F : \iC \to \mathsf{Set}$ is representable if and only if it admits a left adjoint.
\end{lem}

\end{document}
