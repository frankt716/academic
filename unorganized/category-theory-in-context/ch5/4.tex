\documentclass{amsart}
\input{decls}
\title{Canonical Presentations via Free Algebras}
\author{Frank Tsai}
\date{\today}
%\thanks{}
\begin{document}
\maketitle
\tableofcontents

\section{Introduction}
\label{sec:introduction}

\begin{defn}
  A \emph{split coequalizer} is a diagram
  % https://q.uiver.app/#q=WzAsMyxbMCwwLCJcXGlkX3tcXGlDfSJdLFsyLDAsIkdGIl0sWzIsMiwiR0YnIl0sWzEsMiwiR1xcdGhldGEiXSxbMCwxLCJcXGV0YSJdLFswLDIsIlxcZXRhJyIsMl1d
\begin{tikzcd}
	{\id_{\iC}} && GF \\
	\\
	&& {GF'}
	\arrow["G\theta", from=1-3, to=3-3]
	\arrow["\eta", from=1-1, to=1-3]
	\arrow["{\eta'}"', from=1-1, to=3-3]
\end{tikzcd}
  so that
  \begin{enumerate}
  \item $(f,g,h)$ defines a \emph{fork}, i.e., $h$ defines a cocone under the parallel morphisms $x \toto y$.
  \item $h$ is a retraction for $s$, i.e., $hs = \id_{z}$.
  \item $g$ is a retraction for $t$, i.e., $gt = \id_{y}$.
  \item $ft = sh$.
  \end{enumerate}
\end{defn}

\begin{lem}\label{lem:absolute-coequalizer}
  Any split coequalizer is an \emph{absolute coequalizer}, i.e., any functor preserves it.
\end{lem}
\begin{proof}
  For any fork $(f,g,u)$, we wish to show that there is a unique way to factorize $u$.
  Since $h$ is a split epimorphism, the uniqueness of any such factorization is immediate.
  It remains to show that a factorization always exists.
  A natural candidate is $u \circ s$.
  % https://q.uiver.app/#q=WzAsNCxbMCwwLCJ4Il0sWzIsMCwieSJdLFs0LDAsInoiXSxbNCwyLCJhIl0sWzAsMSwiZiIsMCx7Im9mZnNldCI6LTF9XSxbMCwxLCJnIiwyLHsib2Zmc2V0IjoxfV0sWzEsMiwiaCJdLFsxLDAsInQiLDAseyJjdXJ2ZSI6LTN9XSxbMiwxLCJzIiwwLHsiY3VydmUiOi0zfV0sWzEsMywidSIsMix7ImN1cnZlIjozfV0sWzIsMywidXMiLDAseyJzdHlsZSI6eyJib2R5Ijp7Im5hbWUiOiJkYXNoZWQifX19XV0=
\[\begin{tikzcd}
	x && y && z \\
	\\
	&&&& a
	\arrow["f", shift left, from=1-1, to=1-3]
	\arrow["g"', shift right, from=1-1, to=1-3]
	\arrow["h", from=1-3, to=1-5]
	\arrow["t", curve={height=-18pt}, from=1-3, to=1-1]
	\arrow["s", curve={height=-18pt}, from=1-5, to=1-3]
	\arrow["u"', curve={height=18pt}, from=1-3, to=3-5]
	\arrow["us", dashed, from=1-5, to=3-5]
\end{tikzcd}\]
  It remains to show that $ush = u$.
  \begin{align}
    ush &= uft\\
        &= ugt\\
        &= u
  \end{align}
  Since functors preserve commutative diagram, the image of a split coequalizer under any functor is a split coequalizer.
\end{proof}

\begin{defn}
  Given a functor $U : \iD \to \iC$:
  \begin{enumerate}
  \item A \emph{$U$-split coequalizer} is a parallel pair $f,g : x \toto y$ together with an extension of the pair $Uf,Ug : Ux \toto Uy$ to a split coequalizer.
    % https://q.uiver.app/#q=WzAsOCxbMCwwLCJGYyJdLFswLDIsIkZjJyJdLFsyLDAsIkYnYyJdLFsyLDIsIkYnYyciXSxbNCwwLCJjIl0sWzQsMiwiYyciXSxbNiwwLCJHRidjIl0sWzYsMiwiR0YnYyciXSxbMCwyLCJcXHRoZXRhX3tjfSJdLFsxLDMsIlxcdGhldGFfe2MnfSIsMl0sWzAsMSwiRmYiLDJdLFsyLDMsIkYnZiJdLFs0LDUsImYiLDJdLFs2LDcsIkdGJ2YiXSxbNCw2LCJcXGV0YV97Y30nIl0sWzUsNywiXFxldGFfe2MnfSciLDJdLFsxMSwxMiwiXFxsZWZ0cmlnaHRzcXVpZ2Fycm93IiwxLHsic2hvcnRlbiI6eyJzb3VyY2UiOjIwLCJ0YXJnZXQiOjIwfSwic3R5bGUiOnsiYm9keSI6eyJuYW1lIjoibm9uZSJ9LCJoZWFkIjp7Im5hbWUiOiJub25lIn19fV1d
\[\begin{tikzcd}
	Fc && {F'c} && c && {GF'c} \\
	\\
	{Fc'} && {F'c'} && {c'} && {GF'c'}
	\arrow["{\theta_{c}}", from=1-1, to=1-3]
	\arrow["{\theta_{c'}}"', from=3-1, to=3-3]
	\arrow["Ff"', from=1-1, to=3-1]
	\arrow[""{name=0, anchor=center, inner sep=0}, "{F'f}", from=1-3, to=3-3]
	\arrow[""{name=1, anchor=center, inner sep=0}, "f"', from=1-5, to=3-5]
	\arrow["{GF'f}", from=1-7, to=3-7]
	\arrow["{\eta_{c}'}", from=1-5, to=1-7]
	\arrow["{\eta_{c'}'}"', from=3-5, to=3-7]
	\arrow["\leftrightsquigarrow"{description}, draw=none, from=0, to=1]
\end{tikzcd}\]
    in $\iC$.
  \item $U$ \emph{creates coequalizers of $U$-split pairs} if every $U$-split coequalizer $(f,g,h,s,t)$ in $\iC$ admits a lifting: $f,g$ admit a coequalizer in $\iD$ whose image under $U$ is isomorphic to the fork $(Uf,Ug,h)$.
  \end{enumerate}
\end{defn}

\begin{lem}\label{lem:forgetful-strictly-creates}
  For any monad $(T, \eta, \mu)$ acting on $\iC$, the monadic forgetful functor $U^{T} : \iC^{T} \to \iC$ strictly creates coequalizers of $U^{T}$-split pairs.
\end{lem}
\begin{proof}
  Given any parallel pair $f,g : (A,\alpha) \toto (B,\beta)$ in $\iC^{T}$ that admits a $U^{T}$-splitting:
  % https://q.uiver.app/#q=WzAsNCxbMSwwLCJGIl0sWzMsMCwiRkdGJyJdLFs1LDAsIkYnIl0sWzAsMCwiXFx0aGV0YSJdLFswLDEsIkZcXGV0YSciXSxbMSwyLCJcXGVwc2lsb24gRiciXSxbMywwLCI6PSIsMSx7InN0eWxlIjp7ImJvZHkiOnsibmFtZSI6Im5vbmUifSwiaGVhZCI6eyJuYW1lIjoibm9uZSJ9fX1dXQ==
\[\begin{tikzcd}
	\theta & F && {FGF'} && {F'}
	\arrow["{F\eta'}", from=1-2, to=1-4]
	\arrow["{\epsilon F'}", from=1-4, to=1-6]
	\arrow["{:=}"{description}, draw=none, from=1-1, to=1-2]
\end{tikzcd}\]
  in $\iC$.
  We need to show that $C$ can be given a unique algebra structure $\gamma$ so that $h : (B,\beta) \to (C,\gamma)$ is a coequalizer.
  A $U^{T}$-splitting is an absolute coequalizer.
  Thus, the image of the $U^{T}$-splitting under $U$ is another absolute coequalizer.
  Since $f$ and $g$ are $T$-homomorphisms, the left-hand square commutes.
  % https://q.uiver.app/#q=WzAsNixbMCwwLCJUQSJdLFsyLDAsIlRCIl0sWzQsMCwiVEMiXSxbMCwyLCJBIl0sWzIsMiwiQiJdLFs0LDIsIkMiXSxbMyw0LCJmIiwwLHsib2Zmc2V0IjotMX1dLFszLDQsImciLDIseyJvZmZzZXQiOjF9XSxbNCw1LCJoIiwyXSxbMCwxLCJUZiIsMCx7Im9mZnNldCI6LTF9XSxbMCwxLCJUZyIsMix7Im9mZnNldCI6MX1dLFsxLDIsIlRoIl0sWzAsMywiXFxhbHBoYSIsMl0sWzEsNCwiXFxiZXRhIiwxXSxbMiw1LCJcXGdhbW1hIiwwLHsic3R5bGUiOnsiYm9keSI6eyJuYW1lIjoiZGFzaGVkIn19fV1d
\[\begin{tikzcd}
	TA && TB && TC \\
	\\
	A && B && C
	\arrow["f", shift left, from=3-1, to=3-3]
	\arrow["g"', shift right, from=3-1, to=3-3]
	\arrow["h"', from=3-3, to=3-5]
	\arrow["Tf", shift left, from=1-1, to=1-3]
	\arrow["Tg"', shift right, from=1-1, to=1-3]
	\arrow["Th", from=1-3, to=1-5]
	\arrow["\alpha"', from=1-1, to=3-1]
	\arrow["\beta"{description}, from=1-3, to=3-3]
	\arrow["\gamma", dashed, from=1-5, to=3-5]
\end{tikzcd}\]
  Clearly, $(Tf,Tg,h\beta)$ is a fork.
  Thus, there is a unique map $\gamma$ that makes the right-hand square commute.
  If $\gamma$ is an algebra structure, then this demonstrates that $h$ is a $T$-homomorphism, and $\gamma$ is the only algebra structure that makes $h$ a $T$-homomorphism.

  We need to check that the following diagrams commute:
  \begin{mathpar}
    % https://q.uiver.app/#q=WzAsNixbMCwwLCJcXGlkX3tcXGlDfSIsWzMwLDYwLDYwLDFdXSxbMCwyLCJHRiIsWzMwLDYwLDYwLDFdXSxbMiwyLCJHRkdGJyIsWzMwLDYwLDYwLDFdXSxbNCwyLCJHRiciXSxbNCw0LCJHRidHRiJdLFs0LDYsIkdGIl0sWzAsMSwiXFxldGEiLDIseyJjb2xvdXIiOlszMCw2MCw2MF19LFszMCw2MCw2MCwxXV0sWzEsMiwiR0ZcXGV0YSciLDAseyJjb2xvdXIiOlszMCw2MCw2MF19LFszMCw2MCw2MCwxXV0sWzIsMywiR1xcZXBzaWxvbiBGJyJdLFszLDQsIkdGJ1xcZXRhIl0sWzQsNSwiR1xcZXBzaWxvbicgRiJdXQ==
\[\begin{tikzcd}
	\textcolor{rgb,255:red,214;green,153;blue,92}{\id_{\iC}} \\
	\\
	\textcolor{rgb,255:red,214;green,153;blue,92}{GF} && \textcolor{rgb,255:red,214;green,153;blue,92}{GFGF'} && {GF'} \\
	\\
	&&&& {GF'GF} \\
	\\
	&&&& GF
	\arrow["\eta"', color={rgb,255:red,214;green,153;blue,92}, from=1-1, to=3-1]
	\arrow["{GF\eta'}", color={rgb,255:red,214;green,153;blue,92}, from=3-1, to=3-3]
	\arrow["{G\epsilon F'}", from=3-3, to=3-5]
	\arrow["{GF'\eta}", from=3-5, to=5-5]
	\arrow["{G\epsilon' F}", from=5-5, to=7-5]
\end{tikzcd}\] \and % https://q.uiver.app/#q=WzAsNCxbMCwwLCJUXnsyfUMiXSxbMiwwLCJUQyJdLFswLDIsIlRDIl0sWzIsMiwiQyJdLFswLDIsIlRcXGdhbW1hIiwyXSxbMCwxLCJcXG11X3tDfSJdLFsxLDMsIlxcZ2FtbWEiXSxbMiwzLCJcXGdhbW1hIiwyXV0=
\begin{tikzcd}
	{T^{2}C} && TC \\
	\\
	TC && C
	\arrow["T\gamma"', from=1-1, to=3-1]
	\arrow["{\mu_{C}}", from=1-1, to=1-3]
	\arrow["\gamma", from=1-3, to=3-3]
	\arrow["\gamma"', from=3-1, to=3-3]
\end{tikzcd}
  \end{mathpar}
  
  By \cref{lem:absolute-coequalizer}, $\gamma = h\beta\-Ts$.
  Thus,
  \begin{align}
    \gamma\eta_{C} &= h\beta\-(Ts)\eta_{C}\\
           &= h\beta\eta_{B}s\\
           &= hs\\
           &= \id_{C}\\
    \gamma\mu_{C} &= h\beta(Ts)\mu_{C}\\
           &= h\beta\mu_{B}T^{2}s\\
    \gamma\-T\gamma &= h\beta(Ts)(Th)(T\beta)T^{2}s\\
           &= h\beta(Tf)(Tt)(T\beta)T^{2}s\\
           &= h\beta(Tg)(Tt)(T\beta)T^{2}s\\
           &= h\beta(T\beta)T^{2}s\\
           &= h\beta\mu_{B}T^{2}s
  \end{align}

  It remains to show that $h : (B,\beta) \to (C,\gamma)$ is a coequalizer.
  That is, for any $T$-homomorphism $k : (B,\beta) \to (D,\delta)$ creating a fork $(f,g,h)$, $k$ admits a unique factorization.
  Note that $k$ admits a unique factorization in $\iC$ by the universal property of coequalizer.
  % https://q.uiver.app/#q=WzAsNCxbMCwwLCJBIl0sWzIsMCwiQiJdLFs0LDAsIkMiXSxbNCwyLCJEIl0sWzAsMSwiZiIsMCx7Im9mZnNldCI6LTF9XSxbMCwxLCJnIiwyLHsib2Zmc2V0IjoxfV0sWzEsMiwiaCJdLFsyLDMsImoiLDAseyJzdHlsZSI6eyJib2R5Ijp7Im5hbWUiOiJkYXNoZWQifX19XSxbMSwzLCJrIiwyXV0=
\[\begin{tikzcd}
	A && B && C \\
	\\
	&&&& D
	\arrow["f", shift left, from=1-1, to=1-3]
	\arrow["g"', shift right, from=1-1, to=1-3]
	\arrow["h", from=1-3, to=1-5]
	\arrow["j", dashed, from=1-5, to=3-5]
	\arrow["k"', from=1-3, to=3-5]
\end{tikzcd}\]
  Thus, it suffices to show that $j$ is a $T$-homomorphism.
  That is,
  % https://q.uiver.app/#q=WzAsNCxbMCwwLCJcXGlkX3tcXGlDfSJdLFs0LDAsIkdGJ0dGIixbMzAsNjAsNjAsMV1dLFs2LDAsIkdGIixbMzAsNjAsNjAsMV1dLFsyLDAsIkdGIl0sWzEsMiwiR1xcZXBzaWxvbicgRiIsMCx7ImNvbG91ciI6WzMwLDYwLDYwXX0sWzMwLDYwLDYwLDFdXSxbMywxLCJcXGV0YScgR0YiLDAseyJjb2xvdXIiOlszMCw2MCw2MF19LFszMCw2MCw2MCwxXV0sWzAsMywiXFxldGEiXV0=
\[\begin{tikzcd}
	{\id_{\iC}} && GF && \textcolor{rgb,255:red,214;green,153;blue,92}{GF'GF} && \textcolor{rgb,255:red,214;green,153;blue,92}{GF}
	\arrow["{G\epsilon' F}", color={rgb,255:red,214;green,153;blue,92}, from=1-5, to=1-7]
	\arrow["{\eta' GF}", color={rgb,255:red,214;green,153;blue,92}, from=1-3, to=1-5]
	\arrow["\eta", from=1-1, to=1-3]
\end{tikzcd}\]
  \begin{align}
    j\gamma &= jh\beta\-Ts\\
       &= k\beta\-Ts\\
       &= \delta\-TkTs\\
       &= \delta\-Tj
  \end{align}
\end{proof}

\begin{lem}
  Let $(T,\eta,\mu)$ be a monad on $\iC$, and let $(A,\alpha)$ be a $T$-algebra.
  Then
  % https://q.uiver.app/#q=WzAsMixbMCwwLCJcXGlkX3tcXGlDfSJdLFsyLDAsIkdGIl0sWzAsMSwiXFxldGEiXV0=
\[\begin{tikzcd}
	{\id_{\iC}} && GF
	\arrow["\eta", from=1-1, to=1-3]
\end{tikzcd}\]
  is a coequalizer in $\iC^{T}$.
\end{lem}
\begin{proof}
  By \cref{lem:forgetful-strictly-creates}, it suffices to show that $\alpha : TA \to A$ and $T\alpha,\mu_{A} : T^{2}A \toto TA$ can be extended to a split coequalizer.
  Consider
  % https://q.uiver.app/#q=WzAsNixbMCwwLCJcXGlkX3tcXGlDfSIsWzMwLDYwLDYwLDFdXSxbMCwyLCJHRiciLFszMCw2MCw2MCwxXV0sWzIsMiwiR0YnR0YiLFszMCw2MCw2MCwxXV0sWzQsMiwiR0YiXSxbNCw0LCJHRkdGJyJdLFs0LDYsIkdGJyJdLFswLDEsIlxcZXRhJyIsMix7ImNvbG91ciI6WzMwLDYwLDYwXX0sWzMwLDYwLDYwLDFdXSxbMSwyLCJHRidcXGV0YSIsMCx7ImNvbG91ciI6WzMwLDYwLDYwXX0sWzMwLDYwLDYwLDFdXSxbMiwzLCJHXFxlcHNpbG9uJyBGIl0sWzMsNCwiR0ZcXGV0YSciXSxbNCw1LCJHXFxlcHNpbG9uIEYnIl1d
\[\begin{tikzcd}
	\textcolor{rgb,255:red,214;green,153;blue,92}{\id_{\iC}} \\
	\\
	\textcolor{rgb,255:red,214;green,153;blue,92}{GF'} && \textcolor{rgb,255:red,214;green,153;blue,92}{GF'GF} && GF \\
	\\
	&&&& {GFGF'} \\
	\\
	&&&& {GF'}
	\arrow["{\eta'}"', color={rgb,255:red,214;green,153;blue,92}, from=1-1, to=3-1]
	\arrow["{GF'\eta}", color={rgb,255:red,214;green,153;blue,92}, from=3-1, to=3-3]
	\arrow["{G\epsilon' F}", from=3-3, to=3-5]
	\arrow["{GF\eta'}", from=3-5, to=5-5]
	\arrow["{G\epsilon F'}", from=5-5, to=7-5]
\end{tikzcd}\]
  Clearly, $\eta_{A}$ (resp., $\eta_{TA}$) is a retraction for $\alpha$ (resp., $\mu_{A}$).
  The required equation $\eta_{A} \circ \alpha = T\alpha \circ \eta_{TA}$ is just the naturality of $\eta$.
\end{proof}

\begin{cor}
  If $F,U : \iC \toot \iD$ is a monadic adjunction, then
  \begin{enumerate}
  \item $U : \iD \to \iC$ creates coequalizers of $U$-split pairs;
  \item For any $D \in \iD$, there is a coequalizer
    % https://q.uiver.app/#q=WzAsMyxbMCwwLCJGVUZVRCJdLFsyLDAsIkZVRCJdLFs0LDAsIkQiXSxbMCwxLCJGVVxcZXBzaWxvbl97RH0iLDAseyJvZmZzZXQiOi0xfV0sWzAsMSwiXFxlcHNpbG9uX3tGVUR9IiwyLHsib2Zmc2V0IjoxfV0sWzEsMiwiXFxlcHNpbG9uX3tEfSJdXQ==
\[\begin{tikzcd}
	FUFUD && FUD && D
	\arrow["{FU\epsilon_{D}}", shift left, from=1-1, to=1-3]
	\arrow["{\epsilon_{FUD}}"', shift right, from=1-1, to=1-3]
	\arrow["{\epsilon_{D}}", from=1-3, to=1-5]
\end{tikzcd}\]
  \end{enumerate}
\end{cor}
\begin{proof}
  Since the adjunction is monadic, the canonical comparison functor $K : \iD \to \iC^{T}$ defines an equivalence.
  % https://q.uiver.app/#q=WzAsNCxbMCwwLCJcXGlkX3tcXGlDfSIsWzMwLDYwLDYwLDFdXSxbMiwwLCJHRiIsWzMwLDYwLDYwLDFdXSxbNCwwLCJHRkdGJyIsWzMwLDYwLDYwLDFdXSxbNiwwLCJHRiciXSxbMSwyLCJHRlxcZXRhJyIsMCx7ImNvbG91ciI6WzMwLDYwLDYwXX0sWzMwLDYwLDYwLDFdXSxbMiwzLCJHXFxlcHNpbG9uIEYnIl0sWzAsMSwiXFxldGEiLDAseyJjb2xvdXIiOlszMCw2MCw2MF19LFszMCw2MCw2MCwxXV1d

\begin{equation}\label{eqn:eta-to-eta'}
  \begin{tikzcd}
    \textcolor{rgb,255:red,214;green,153;blue,92}{\id_{\iC}} && \textcolor{rgb,255:red,214;green,153;blue,92}{GF} && \textcolor{rgb,255:red,214;green,153;blue,92}{GFGF'} && {GF'}
    \arrow["{GF\eta'}", color={rgb,255:red,214;green,153;blue,92}, from=1-3, to=1-5]
    \arrow["{G\epsilon F'}", from=1-5, to=1-7]
    \arrow["\eta", color={rgb,255:red,214;green,153;blue,92}, from=1-1, to=1-3]
  \end{tikzcd}
\end{equation}

  Since $U = U^{T}K$, for any $U$-split pair $f,g : x \toto y$ in $\iD$, $Kf,Kg : Kx \toto Ky$ is a $U^{T}$-split pair.
  By \cref{lem:forgetful-strictly-creates}, $Kf,Kg$ can be extended to a coequalizer in $\iC^{T}$, which transports to a coequalizer in $\iD$ via any inverse equivalence of $K$.

  For (ii), it suffices to show that $FU\epsilon_{D}, \epsilon_{FUD}$ is a $U$-split pair.
  Indeed, they are.
  % https://q.uiver.app/#q=WzAsMyxbMCwwLCJVRlVGVUQiXSxbMiwwLCJVRlVEIl0sWzQsMCwiVUQiXSxbMCwxLCJVRlVcXGVwc2lsb25fe0R9IiwwLHsib2Zmc2V0IjotMX1dLFswLDEsIlVcXGVwc2lsb25fe0ZVRH0iLDIseyJvZmZzZXQiOjF9XSxbMSwyLCJVXFxlcHNpbG9uX3tEfSJdLFsyLDEsIlxcZXRhX3tVRH0iLDAseyJjdXJ2ZSI6LTN9XSxbMSwwLCJcXGV0YV97VUZVRH0iLDAseyJjdXJ2ZSI6LTN9XV0=
\[\begin{tikzcd}
	UFUFUD && UFUD && UD
	\arrow["{UFU\epsilon_{D}}", shift left, from=1-1, to=1-3]
	\arrow["{U\epsilon_{FUD}}"', shift right, from=1-1, to=1-3]
	\arrow["{U\epsilon_{D}}", from=1-3, to=1-5]
	\arrow["{\eta_{UD}}", curve={height=-18pt}, from=1-5, to=1-3]
	\arrow["{\eta_{UFUD}}", curve={height=-18pt}, from=1-3, to=1-1]
\end{tikzcd}\]
  All the equations hold due to the triangle equalities.
\end{proof}

\bibliographystyle{alpha}
\bibliography{all}

\end{document}
