\documentclass{amsart}
\input{decls}
\title{Pointwise Kan extension}
\author{Frank Tsai}
\date{\today}
%\thanks{}
\begin{document}
\maketitle
\tableofcontents

\section{Introduction}
\label{sec:introduction}

\begin{defn}
  A functor $L : \iE \to \iF$ \emph{preserves} a left Kan extension if the whiskered composition depicted as follows is a left Kan extension of $LF$ along $K$.
  \[% https://q.uiver.app/#q=WzAsMyxbMCwwLCJcXGlEIl0sWzQsMCwiXFxpQ157TH0iXSxbMiwyLCJcXGlDIl0sWzAsMiwiIiwyLHsib2Zmc2V0IjoyLCJzdHlsZSI6eyJ0YWlsIjp7Im5hbWUiOiJob29rIiwic2lkZSI6InRvcCJ9fX1dLFsyLDAsIkwiLDIseyJvZmZzZXQiOjJ9XSxbMiwxLCJGXntMfSIsMCx7Im9mZnNldCI6LTJ9XSxbMSwyLCJVXntMfSIsMCx7Im9mZnNldCI6LTJ9XSxbMCwxLCJLIl0sWzQsMywiIiwwLHsibGV2ZWwiOjEsInN0eWxlIjp7Im5hbWUiOiJhZGp1bmN0aW9uIn19XSxbNSw2LCIiLDIseyJsZXZlbCI6MSwic3R5bGUiOnsibmFtZSI6ImFkanVuY3Rpb24ifX1dXQ==
\[\begin{tikzcd}
	\iD &&&& {\iC^{L}} \\
	\\
	&& \iC
	\arrow[""{name=0, anchor=center, inner sep=0}, shift right=2, hook, from=1-1, to=3-3]
	\arrow[""{name=1, anchor=center, inner sep=0}, "L"', shift right=2, from=3-3, to=1-1]
	\arrow[""{name=2, anchor=center, inner sep=0}, "{F^{L}}", shift left=2, from=3-3, to=1-5]
	\arrow[""{name=3, anchor=center, inner sep=0}, "{U^{L}}", shift left=2, from=1-5, to=3-3]
	\arrow["K", from=1-1, to=1-5]
	\arrow["\dashv"{anchor=center, rotate=-135}, draw=none, from=1, to=0]
	\arrow["\dashv"{anchor=center, rotate=-45}, draw=none, from=2, to=3]
\end{tikzcd}\]\]
\end{defn}

\begin{lem}
  Left adjoints preserve left Kan extensions.
\end{lem}
\begin{proof}
  Let $(\lan_{K}F, \eta)$ be a Kan extension.
  We have the following isomorphisms:
  \begin{align}
    \iF^{\iD}(L \circ \lan_{K}F, H) &\iso \iE^{\iD}(\lan_{K}F, R \circ H)\\
                                &\iso \iE^{\iC}(F, R \circ H \circ K)\\
                                &\iso \iF^{\iC}(L \circ F, H \circ K)\\
                                &\iso \iF^{\iD}(\lan_{K} LF, H)
  \end{align}
  These isomorphisms take the identity natural transformations $\id_{\lan_{K}LF}$ to the whiskered composition $L\eta : LF \to \lan_{K}LF \circ K$.
  Thus, $(L\lan_{K}F, L\eta)$ is a left Kan extension.
\end{proof}

\begin{defn}
  When $\iE$ is locally small, a right Kan extension is a \emph{pointwise right Kan extension} if it is preserved by all representable functors $\iE(e,\blank)$.
  Dually, a left Kan extension is a \emph{pointwise left Kan extension} if its opposite is a pointwise right Kan extension, i.e.,
  \[% https://q.uiver.app/#q=WzAsNCxbMCwwLCJhIl0sWzAsMiwiYiJdLFsyLDAsIkxhIl0sWzIsMiwiTGIiXSxbMCwyLCJcXGV0YV97YX0iXSxbMSwzLCJcXGV0YV97Yn0iLDJdLFswLDEsImYiLDJdLFsyLDMsIkxmIl1d
\[\begin{tikzcd}
	a && La \\
	\\
	b && Lb
	\arrow["{\eta_{a}}", from=1-1, to=1-3]
	\arrow["{\eta_{b}}"', from=3-1, to=3-3]
	\arrow["f"', from=1-1, to=3-1]
	\arrow["Lf", from=1-3, to=3-3]
\end{tikzcd}\]\]
\end{defn}

\begin{lem}
  Given functors $F : \iC \to \iE$ and $K : \iC \to \iD$ with $\iD$ and $\iE$ locally small and an object $d \in \iD$, there is a natural isomorphism
  \[
    \mathsf{Cone}(e, F\Pi_{d}) \iso \mathsf{Set}^{\iC}(\iD(d,K\blank),\iE(e,F\blank))
  \]
\end{lem}
\begin{proof}
  The comma category $d \dn K$ consists of objects $(c \in \iC, f : d \to Kc)$ and morphisms $h : c \to c'$ such that $Kh \circ f = f'$.
  A cone over $F\Pi_{d}$ with submit $e$ is a natural transformation $e \to F\Pi_{d}$:
  \[% https://q.uiver.app/#q=WzAsMyxbMCwwLCJcXGlDX3tUfShhLGIpIl0sWzAsMiwiXFxpQyhhLFRiKSJdLFsyLDAsIlxcaUMoVGEsVGIpIl0sWzAsMiwiVV97VH0iXSxbMCwxLCJcXGlzbyIsMSx7InN0eWxlIjp7ImhlYWQiOnsibmFtZSI6Im5vbmUifX19XSxbMSwyLCJVIiwyXV0=
\[\begin{tikzcd}
	{\iC_{T}(a,b)} && {\iC(Ta,Tb)} \\
	\\
	{\iC(a,Tb)}
	\arrow["{U_{T}}", from=1-1, to=1-3]
	\arrow["\iso"{description}, no head, from=1-1, to=3-1]
	\arrow["U"', from=3-1, to=1-3]
\end{tikzcd}\]\]
  For each $c$, we can define a function $\lambda_{c} : \iD(d,Kc) \to \iE(e,Fc)$ by $f \mapsto \lambda_{(c,f)}$ making the following diagram commute:
  \[% https://q.uiver.app/#q=WzAsNCxbMCwwLCJcXGlEKGQsS2MpIl0sWzAsMiwiXFxpRChkLEtjJykiXSxbMiwwLCJcXGlFKGUsRmMpIl0sWzIsMiwiXFxpRShlLEZjJykiXSxbMCwyLCJcXGxhbWJkYV97Y30iXSxbMSwzLCJcXGxhbWJkYV97Yyd9IiwyXSxbMCwxLCIoS2gpX3sqfSIsMl0sWzIsMywiKEZoKV97Kn0iXV0=
\begin{tikzcd}
	{\iD(d,Kc)} && {\iE(e,Fc)} \\
	\\
	{\iD(d,Kc')} && {\iE(e,Fc')}
	\arrow["{\lambda_{c}}", from=1-1, to=1-3]
	\arrow["{\lambda_{c'}}"', from=3-1, to=3-3]
	\arrow["{(Kh)_{*}}"', from=1-1, to=3-1]
	\arrow["{(Fh)_{*}}", from=1-3, to=3-3]
\end{tikzcd}\]
  This is precisely the data of a natural transformation $\iD(d,K\blank) \to \iE(e,F\blank)$.
\end{proof}

\begin{thm}
  If $\iD$ and $\iE$ are locally small, a right Kan extension of $F : \iC \to \iE$ along $K : \iC \to \iD$ is pointwise if and only if it is computed by the limit formula.
  Dually, the left Kan extension of $F : \iC \to \iE$ along $K : \iC \to \iD$ is pointwise if and only if it is computed by the colimit formula.
\end{thm}
\begin{proof}
  For any $e \in \iE$, $\iE(e,\ran_{K}F\blank)$ is the right Kan extension of $\iE(e,F\blank)$.
  \begin{align}
    \iE(e, \ran_{K}F(d)) &\iso \mathsf{Set}^{\iD}(\iD(d, \blank), \iE(e, \ran_{K}F\blank))\\
                         &\iso \mathsf{Set}^{\iC}(\iD(d, K\blank), \iE(e, F\blank))\\
                         &\iso \mathsf{Cone}(e,F\Pi_{d})
  \end{align}
  Thus, $\ran_{K}F(d) \iso \lim F\Pi_{d}$.
\end{proof}

\bibliographystyle{alpha}
\bibliography{all}

\end{document}
